\RequirePackage{plautopatch}
\documentclass[uplatex, a4paper, 14Q, dvipdfmx]{jsarticle}
\usepackage{docmute}
\usepackage{../mypackage}

\title{\texorpdfstring{$\Ainf$}{Ainf}圏のホモトピー論}
\author{よの}
\date{\today}

\begin{document}

\maketitle

\begin{abstract}
  dg圏のホモトピー論と$\Ainf$圏のホモトピー論が1圏として圏同値であるだけでなく, $\infty$圏として圏同値であることを示す. 
  本章は\cite{CNS20}と\cite{Pas23}のまとめである. 
  dg圏のなす圏に入るモデル構造についてもまとめる. (執筆中)

  $\bbK$を体とし, 圏はsmallかつ$\bbK$線形であるとする.
\end{abstract}

\tableofcontents


% \section{バー構成とコバー構成}

% \begin{notation}
%   以下のような記号を用いる.
%   \begin{itemize}
%     \item 箙と箙の射のなす圏を$\Qu$
%     \item 次数付き箙と次数付き箙の射のなす圏を$\gQu$
%     \item dg箙とdg箙の射のなす圏を$\dgQu$
%     \item 恒等射を持たないdg圏と恒等射を考えないdg関手を$\dgcat^n$
%     \item c恒等射を持つdg圏とc恒等射を保つdg関手を$\dgcat^c$
%     \item s恒等射を持つdg圏とs恒等射を保つdg関手を$\dgcat$
%     \item 恒等射を持たない$\Ainf$圏と恒等射を考えない$\Ainf$関手のなす圏を$\Ainfcat^n$
%     \item c恒等射を持つ$\Ainf$圏とc恒等射を保つ$\Ainf$関手のなす圏を$\Ainfcat^c$
%     \item s恒等射を持つ$\Ainf$圏とs恒等射を保つ$\Ainf$関手のなす圏を$\Ainfcat$
%   \end{itemize}
% \end{notation}

% \begin{definition}
%   集合$O$に対して, 自明な微分を持つdg圏$\bfE_O$を次のように定義する.
%   \begin{itemize}
%     \item $\bfE_O$の対象は集合$O$.
%     \item 任意の$A_1,A_2 \in \bfE_O$に対して, 
%     \begin{align*}
%       \Hom_{\bfE_O}(A_1,A_2) := 
%       \begin{cases}
%         \bbK \cdot \id_{A_1} & (A_1=A_2) \\
%         0 & (A_1 \neq A_2)
%       \end{cases}
%     \end{align*}
%   \end{itemize}
% \end{definition}

% \begin{definition}[恒等射を持たない余圏]
%   恒等射を持たない余圏は以下のデータから構成される. 
%   \begin{itemize}
%     \item 対象の集合$\bfC$
%     \item 任意の$X,Y \in \bfC$に対して, 集合$\Hom_\bfC(X,Y)$
%     \item 任意の$X,Y \in \bfC$に対して, 対応
%     \begin{align*}
%       \Delta_\bfC : \Hom_\bfC(X,Y) \to \bigoplus_{Z \in \bfC} \Hom_\bfC(Z,Y) \otimes \Hom_\bfC(X,Z) 
%     \end{align*}
%     が与えられていて, 余結合則を満たす. 
%   \end{itemize}
% \end{definition}

% \begin{definition}[余完備な余圏]
%   恒等射を持たない余圏$\bfC$の任意の射が$\Delta_\bfC$のhigh itarateな核で表されるとき, $\bfC$は余完備(cocomplete)であるという. 
% \end{definition}

% \begin{definition}[恒等射を考えない余関手]
  
% \end{definition}

% \begin{notation}
%   余完備な恒等射を持たない余圏と恒等射を考えない余関手のなす圏を$\cocat^n$と表す. 
% \end{notation}

% \begin{definition}[テンソル余圏]
%   箙$\bfV$に対して, 余完備な恒等射を持たない余圏$\bar{T}^c(\bfV)$を次のように定義する.
%   \begin{itemize}
%     \item $\Ob\bar{T}^c(\bfV) := \Ob\bfV$
%     \item 任意の$X,Y \in \Ob\bar{T}^c(\bfV)$に対して, 
%     \begin{align*}
%       &\Hom_{\bar{T}^c(\bfV)}(X,Y)
%       &:= \Hom_\bfV(X,Y) \bigoplus_{n>0} \bigoplus_{Z_1,\cdots,Z_n \in \bfV} \Hom_bfV(Z_n,Y) \otimes \Hom_bfV(Z_{n-1},Z_n) \otimes \cdots \otimes \Hom_bfV(Z_1,Z_2) \otimes \Hom_bfV(X,Z_1) 
%     \end{align*}
%     \item 任意の$(f_n,\cdots,f_1) \in \Hom_{\bar{T}^c(\bfV)}(X,Y)$に対して, 
%     \begin{align*}
%       \Delta_{\bar{T}^c(\bfV)} := \sum_{i=1}^{n-1} (f_n,\cdots,f_{i+1}) \otimes (f_i\cdots,f_1)
%     \end{align*}
%   \end{itemize}
%   $\bar{T}^c(\bfV)$を$\bfV$上のテンソル余圏(tensor cocategory)という. 
%   双対的に, 通常の$\bar{T}(\bfV)$を$\bfV$上のテンソル圏(tensor category)という.
% \end{definition}

% \begin{theorem}
%   忘却関手$U : \cocat^n \to \Qu$は対応$\bfV \mapsto \bar{T}^c(\bfV)$による関手$\bar{T}^c : \cocat^n \to \Qu$を左随伴に持つ. 
% \end{theorem}

% \begin{definition}[余微分]
%   任意の$F_1,F_2 \in \Hom_{\cocat^n}(\bfC_1,\bfC_2)$に対して, $(F_1,F_2)$余微分($(F_1,F_2)$-coderivation)は以下のデータから構成される.
%   \begin{itemize}
%     \item 任意の$C_1,C_2 \in \bfC_1$に対して, 対応$D : \Hom_{\bfC_1}(C_1,C_2) \to \Hom_{\bfC_2}(F_1(C_1),F_2(C_2))$
%     が与えられていて, 
%     \begin{align*}
%       \Delta_{\bfC_2} \circ D = (F_2 \otimes D + D \otimes F_1) \circ \Delta_{\bfC_1}
%     \end{align*}
%     を満たす. 
%   \end{itemize}
% \end{definition}

% \begin{theorem}
%   任意の$\bfV \in \Qu$と$F_1,F_2 \in \cocat^n(\bfC,\bar{T}^c(\bfV))$に対して, 
% \end{theorem}

% \begin{definition}[バー構成]
%   任意の$\A \in \Ainfcat^n$に対して, 次数付き恒等射を持たないdg余圏$(B(\A), d_{B(\A)})$を次のように定義する.
%   \begin{itemize}
%     \item $B(\A) := \bar{T}^c(\A[1])$
%     \item $d_{B(\A)}$は次数$1$の$(\id_{\bar{T}^c(\A[1])},\id_{\bar{T}^c(\A[1])})$余微分
%   \end{itemize}
% \end{definition}

% \begin{theorem}
%   対応$\A \mapsto B(\A)$は関手
%   \begin{align*}
%     B : \Ainfcat^n \to \dgcocat^n
%   \end{align*}
%   を定め, $B$は忠実充満である. 
% \end{theorem}

% \begin{definition}[コバー構成]
%   任意の$\C \in \dgcocat^n$に対して, 恒等射を持たないdg圏$\Omega(\C)$を次のように定義する.
%   \begin{align*}
%     \Omega(\C) := \bar{T}(\C[-1])
%   \end{align*}
% \end{definition}


\section{\texorpdfstring{$\Ainf$}{Ainf}圏のホモトピー論とdg圏のホモトピー論} \label{section_non_unital_Ainf_cat}

\begin{notation}
  以下のような記号を用いる.
  \begin{itemize}
    \item 箙と箙の射のなす圏を$\Qu$
    \item 次数付き箙と次数付き箙の射のなす圏を$\gQu$
    \item dg箙とdg箙の射のなす圏を$\dgQu$
    \item 恒等射を持たないdg圏と恒等射を考えないdg関手を$\dgcat^n$
    \item c恒等射を持つdg圏とc恒等射を保つdg関手を$\dgcat^c$
    \item s恒等射を持つdg圏とs恒等射を保つdg関手を$\dgcat$
    \item 恒等射を持たない$\Ainf$圏と恒等射を考えない$\Ainf$関手のなす圏を$\Ainfcat^n$
    \item c恒等射を持つ$\Ainf$圏とc恒等射を保つ$\Ainf$関手のなす圏を$\Ainfcat^c$
    \item s恒等射を持つ$\Ainf$圏とs恒等射を保つ$\Ainf$関手のなす圏を$\Ainfcat$
  \end{itemize}
\end{notation}

dg圏は高次のホモトピーが自明な$\Ainf$圏であり, dg関手は高次の成分が自明な$\Ainf$関手なので, 埋め込み
\begin{align*}
  i : \dgcat \to \Ainfcat
\end{align*}
が存在する. 

$\Ainf$圏$\A$に対して, バー構成によりdg余圏$B\A$が得られ, コバー構成によりdg圏$\Omega(B\A)$が得られる. 
$\Ainf$関手に対しても同様にdg関手が得られる. 
よって, rectification関手と呼ばれる関手
\begin{align*}
  U : \Ainfcat \to \dgcat
\end{align*}
が存在する.

\begin{notation}
  dg圏とdg擬同値のなす圏を$\Wdgqeq \subset \dgcat$, $\Ainf$圏と$\Ainf$擬同値のなす圏を$\WAinfqeq \subset \Ainfcat$と表す.   
\end{notation}

\begin{theorem} \label{prop_U_and_i_are_adjoint}
  関手$U : \Ainfcat \to \dgcat, i : \dgcat \to \Ainfcat$に対して, 次が成立する.
  \begin{enumerate}
    \item $U$は$i$の左随伴である.
    \item $\Ainfcat$において$i(\Wdgqeq) \subset \WAinfqeq$である. 
    \item $\dgcat$において$U(\WAinfqeq) \subset \Wdgqeq$である. 
    \item 単位$\eta : \id_{\Ainfcat} \to iU$の成分は$\WAinfqeq$に属する.
    \item 余単位$\varepsilon : Ui \to \id_{\dgcat}$の成分は$\Wdgqeq$に属する. 
  \end{enumerate}
\end{theorem}

\begin{notation}
  $\dgcat$の$\Wdgqeq$による局所化を$\dgcat[(\Wdgqeq)^{-1}]$, $\Ainfcat$の$\WAinfqeq$による局所化を$\Ainfcat[(\WAinfqeq)^{-1}]$と表す. 
  それぞれ, dg圏のホモトピー圏, $\Ainf$圏のホモトピー圏という.
\end{notation}

\begin{theorem} \label{prop:Ainfcat_and_dgcat}
  随伴$U \dashv i$はホモトピー圏の擬圏同値
  \begin{align*}
    \Ainfcat[(\WAinfqeq)^{-1}] \cong \dgcat[(\Wdgqeq)^{-1}]
  \end{align*}
  を定める. 
\end{theorem}

\cref{prop:Ainfcat_and_dgcat}は「$\Ainf$圏のホモトピー論」と「dg圏のホモトピー論」が1圏として等しいことを示している.
しかし, 「$\Ainf$圏のホモトピー論」と「dg圏のホモトピー論」は$(\infty,1)$圏のレベルで等しいのだろうか. 
これは\cref{prop_U_and_i_are_adjoint}から直ちに従うことが分かる. 

\section{相対圏のDwyer-Kan対応}

\begin{definition}[相対圏]
  $\bfC$を圏とする.
  $\bfW$を$\bfC$の全ての対象(と恒等射を含む)$\bfC$の部分圏とする. 
  このとき, 組$(\bfC,\bfW)$を相対圏(relative category)といい, $\bfW$に属する射を弱同値(weak equivalence)という.
\end{definition}

\begin{definition}[相対関手]
  $(\bfC_1,\bfW_1), (\bfC_2,\bfW_2)$を相対圏とする. 
  関手$F : \bfC_1 \to \bfC_2$が$F(\bfW_1) \subset \bfW_2$を満たすとき, $F : (\bfC_1,\bfW_1) \to (\bfC_2,\bfW_2)$を相対関手(relative functor)という. 
\end{definition}

\begin{definition}[相対圏のDwyer-Kan随伴]
  $(\bfC_1,\bfW_1), (\bfC_2,\bfW_2)$を相対圏, $L : \bfC_1 \to \bfC_2, R : \bfC_2 \to \bfC_1$を相対関手とする.
  次の条件を満たすとき, 組$(L,R,\eta,\varepsilon)$を相対圏のDwyer-Kan随伴(Dwyer-Kan adjuction)という.
  \begin{enumerate}
    \item 任意の$X \in \bfC_1$に対して, 単位の成分$\eta_X : X \to RLX$は$\bfW_1$に属する.
    \item 任意の$Y \in \bfC_2$に対して, 余単位の成分$\varepsilon_X : LRY \to Y$は$\bfW_2$に属する. 
  \end{enumerate}
\end{definition}

\begin{remark}
  $i=1,2$に対して, $C_i=W_i$であるとき, 相対圏のDwyer-Kan随伴は通常の随伴である.
  $W_i$が$C_i$における同型射を全て含むとき, 相対圏のDwyer-Kan随伴は随伴同値である. 
\end{remark}

\cref{prop_U_and_i_are_adjoint}は相対圏のDwyer-Kan随伴を用いて次のように表される. 

\begin{theorem}
  $(\Ainfcat, \WAinfqeq), (\dgcat, \Wdgqeq)$は相対圏である. 
  $U : \Ainfcat \to \dgcat, i : \dgcat \to \Ainfcat$に対して, $(U,i,\WAinfqeq, \Wdgqeq)$は相対圏のDwyer-Kan随伴である. 
\end{theorem}

相対圏$(\bfC,\bfW)$に対して, 局所化により, 圏$\bfC[\bfW^{-1}]$を得る. 

\begin{definition}[hammock局所化]
  相対圏$(\bfC,\bfW)$に対して, $\sset$豊穣圏$L^H(\bfC,\bfW)$をhammock局所化を用いて構成する.
\end{definition}


\section{モデル構造}

\begin{notation}
  相対圏と相対関手のなす圏を$\relcat$, 単体的空間のなす圏を$\sspace$と表す. 
\end{notation}

$\relcat$にはBarwick-Kanモデル構造, $\sspace$にはRezk完備Segal空間モデル構造を入れる. 

\begin{theorem}
  $\relcat$と$\sspace$には次のような関係がある.
  \begin{enumerate}
    \item 関手$K_\xi : \sspace \to \relcat$と$N_\xi : \relcat \to \sspace$が存在して, $K_\xi$は$N_\xi$の左随伴である.
    また, この随伴は$\sspace$と$\relcat$のQuillen同値である.
    \item $\relcat$の射$F$が弱同値であることと, $N_\xi(F)$が$\sspace$の弱同値であることは同値である.
    \item $\relcat$の射$F$が弱同値であることと, $L^H(F)$が$\sset$豊穣圏のDwyer-Kan同値であることは同値である. 
  \end{enumerate}
\end{theorem}

\begin{corollary}
  相対関手$U : \Ainfcat \to \dgcat, i : \dgcat \to \Ainfcat$は$\relcat$の弱同値である. 
  また, $N_\xi(U)$と$N_\xi(i)$は$\sspace$上の弱同値である. 
\end{corollary}

$\relcat$におけるファイブラント対象を考える. 

\begin{theorem}
  $(\dgcat, \Wdgqeq)$は$\relcat$におけるファイブラント対象である.
\end{theorem}

\begin{remark}
  $(\Ainfcat, \WAinfqeq)$が$\relcat$におけるファイブラント対象であるかは分かっていない.
  $\relcat$におけるファイブラント置換$j_{\Ainf} : (\Ainfcat, \WAinfqeq) \to (\Ainfcat, \WAinfqeq)^\fib$を固定する.
  この対応から, 射$U^\fib : (\Ainfcat, \WAinfqeq)^\fib \to (\dgcat, \Wdgqeq)$と$i^\fib : (\dgcat, \Wdgqeq) \to (\Ainfcat, \WAinfqeq)^\fib$が存在して, 次の図式を可換にする.
  % https://q.uiver.app/#q=WzAsMyxbMCwwLCIoXFxkZ2NhdCwgXFxXZGdxZXEpIl0sWzAsMSwiKFxcQWluZmNhdCwgXFxXQWluZnFlcSkiXSxbMSwxLCIoXFxBaW5mY2F0LCBcXFdBaW5mcWVxKV5cXGZpYiAiXSxbMCwxLCJpIiwyXSxbMSwyLCJqX3tcXEFpbmZ9IiwyXSxbMCwyLCJpXlxcZmliIl1d
  \[\begin{tikzcd}
    {(\dgcat, \Wdgqeq)} \\
    {(\Ainfcat, \WAinfqeq)} & {(\Ainfcat, \WAinfqeq)^\fib }
    \arrow["i"', from=1-1, to=2-1]
    \arrow["{j_{\Ainf}}"', from=2-1, to=2-2]
    \arrow["{i^\fib}", from=1-1, to=2-2]
  \end{tikzcd}\]
  % https://q.uiver.app/#q=WzAsMyxbMSwxLCIoXFxkZ2NhdCwgXFxXZGdxZXEpIl0sWzAsMCwiKFxcQWluZmNhdCwgXFxXQWluZnFlcSkiXSxbMSwwLCIoXFxBaW5mY2F0LCBcXFdBaW5mcWVxKV5cXGZpYiAiXSxbMSwwLCJVIiwyXSxbMSwyLCJqX3tcXEFpbmZ9Il0sWzIsMCwiVV5cXGZpYiJdXQ==
  \[\begin{tikzcd}
    {(\Ainfcat, \WAinfqeq)} & {(\Ainfcat, \WAinfqeq)^\fib } \\
    & {(\dgcat, \Wdgqeq)}
    \arrow["U"', from=1-1, to=2-2]
    \arrow["{j_{\Ainf}}", from=1-1, to=1-2]
    \arrow["{U^\fib}", from=1-2, to=2-2]
  \end{tikzcd}\]
  ここで, $U^\fib$と$i^\fib$は$\relcat$におけるファイブラント対象の弱同値である.
\end{remark}

右Quillen関手はファイブラント対象と, ファイブラント対象の弱同値を保つ.

\begin{corollary}
  $N_\xi(\Ainfcat, \WAinfqeq)$と$N_\xi(\dgcat, \Wdgqeq)$は完備Segal空間である.
  また, $N_\xi(U^\fib)$と$N_\xi(i^\fib)$はそれらの間の弱同値である.
\end{corollary}

$\sset$にはJoyalモデル構造を入れる. 
Joyal-Tierneyの定理を紹介する.

\begin{theorem}
  関手$p_1^\ast : \sset \to \sspace$と$i_1^\ast : \sspace \to \sset$が存在して, $p_1^\ast$は$i_1^\ast$の左随伴である.
  また, この随伴は$\sset$と$\sspace$のQuillen同値である. 
\end{theorem}

今までの議論で用いた随伴関係をまとめる. (上が左随伴, 下が右随伴)
% https://q.uiver.app/#q=WzAsMyxbMSwwLCJcXHNzcGFjZSJdLFsyLDAsIlxccmVsY2F0Il0sWzAsMCwiXFxzc2V0Il0sWzAsMSwiS19cXHhpIiwwLHsib2Zmc2V0IjotMX1dLFsyLDAsInBfMV5cXGFzdCIsMCx7Im9mZnNldCI6LTF9XSxbMCwyLCJpXzFeXFxhc3QiLDAseyJvZmZzZXQiOi0xfV0sWzEsMCwiTl9cXHhpIiwwLHsib2Zmc2V0IjotMX1dXQ==
\[\begin{tikzcd}
	\sset & \sspace & \relcat
	\arrow["{K_\xi}", shift left, from=1-2, to=1-3]
	\arrow["{p_1^\ast}", shift left, from=1-1, to=1-2]
	\arrow["{i_1^\ast}", shift left, from=1-2, to=1-1]
	\arrow["{N_\xi}", shift left, from=1-3, to=1-2]
\end{tikzcd}\]

右Quillen関手はファイブラント対象と, ファイブラント対象の弱同値を保つことを再び用いる.

\begin{corollary} \label{prop:quasi_Ainf_and_dg}
  $i_1^\ast N_\xi(\Ainfcat, \WAinfqeq)$と$i_1^\ast N_\xi(\dgcat, \Wdgqeq)$は擬圏である.
  また, $i_1^\ast N_\xi(U^\fib)$と$i_1^\ast N_\xi(i^\fib)$はそれらの間の弱同値である.
\end{corollary}

\begin{remark}
  \cref{prop:quasi_Ainf_and_dg}は\cref{prop:Ainfcat_and_dgcat}を擬圏まで拡張したものである.
  実際, $i_1^\ast N_\xi(\Ainfcat, \WAinfqeq)$のホモトピー圏は$\Ainfcat[(\WAinfqeq)^{-1}]$であり, $i_1^\ast N_\xi(\dgcat, \Wdgqeq)$のホモトピー圏は$\dgcat[(\Wdgqeq)^{-1}]$である. 
  つまり, \cref{prop:quasi_Ainf_and_dg}においてホモトピー圏をとると, \cref{prop:Ainfcat_and_dgcat}が導かれる. 
\end{remark}

$\sset$豊穣圏$\ssetenriched$にはBergnerモデル構造を入れる. 

$L^H(\bfC,\bfW)$は$\ssetenriched$におけるファイブラント対象ではないので, ファイブラント置換$(L^H(\bfC,\bfW))^\fib$をとる必要がある.

\begin{theorem}
  関手$F : \sset \to \ssetenriched$とhomotopy coherence nerve $N_c : \ssetenriched \to \sset$が存在して, $F$は$N_c$の左随伴である.
  また, この随伴は$\sset$と$\ssetenriched$のQuillen同値である. 
\end{theorem}

右Quillen関手はファイブラント対象と, ファイブラント対象の弱同値を保つことを再び用いる.

\begin{corollary}
  $N_c((L^H(\bfC,\bfW))^\fib)$は擬圏である.
\end{corollary}

今までの議論で用いた随伴関係をまとめる. (上が左随伴, 下が右随伴)
% https://q.uiver.app/#q=WzAsNCxbMSwyLCJcXHNzcGFjZSJdLFsyLDEsIlxccmVsY2F0Il0sWzAsMSwiXFxzc2V0Il0sWzEsMCwiXFxzc2V0ZW5yaWNoZWQiXSxbMCwxLCJLX1xceGkiLDAseyJvZmZzZXQiOi0xfV0sWzIsMCwicF8xXlxcYXN0IiwwLHsib2Zmc2V0IjotMX1dLFswLDIsImlfMV5cXGFzdCIsMCx7Im9mZnNldCI6LTF9XSxbMSwwLCJOX1xceGkiLDAseyJvZmZzZXQiOi0xfV0sWzMsMiwiTl9jIiwwLHsib2Zmc2V0IjotMX1dLFsyLDMsIkYiLDAseyJvZmZzZXQiOi0xfV1d
\[\begin{tikzcd}
	& \ssetenriched \\
	\sset && \relcat \\
	& \sspace
	\arrow["{K_\xi}", shift left, from=3-2, to=2-3]
	\arrow["{p_1^\ast}", shift left, from=2-1, to=3-2]
	\arrow["{i_1^\ast}", shift left, from=3-2, to=2-1]
	\arrow["{N_\xi}", shift left, from=2-3, to=3-2]
	\arrow["{N_c}", shift left, from=1-2, to=2-1]
	\arrow["F", shift left, from=2-1, to=1-2]
\end{tikzcd}\]

\begin{theorem}
  $i_1^\ast N_\xi(\bfC,\bfW)$と$N_c((L^H(\bfC,\bfW))^\fib)$は$(\infty,1)$圏同値である. 
\end{theorem}

\begin{corollary}
  $N_c((L^H(\Ainfcat, \WAinfqeq))^\fib)$と$N_c((L^H(\dgcat, \Wdgqeq))^\fib)$は$(\infty,1)$圏同値である. 
\end{corollary}


\section{dg圏の圏に入るモデル構造}

$\C, \D$をdg圏とする. 
dg圏の圏$\dgcat$に入る2種類の組み合わせ論的モデル構造を説明する. (途中)

\begin{definition}[DK同値]
  dg関手$\F : \C \to \D$が次の条件を満たすとき, $\F$をDK同値(Dwyer-Kan equivalence)という.
  \begin{itemize}
    \item 任意の$x_0,x_1 \in \C$に対して, 複体の射$\Hom_\C(x_0,x_1) \to \Hom_\D(\F x_0,\F x_1)$は擬同型
    \item $H^0(\F) : H^0(\C) \to H^0(\D)$は通常の圏同値
  \end{itemize}
\end{definition}

\begin{definition}[DKファイブレーション]
  dg関手$\F : \C \to \D$が次の条件を満たすとき, $\F$をDKファイブレーション(Dwyer-Kan fibration)という.
  \begin{itemize}
    \item 任意の$x_0,x_1 \in \C$に対して, 複体の射$\Hom_\C(x_0,x_1) \to \Hom_\D(\F x_0,\F x_1)$は複体のファイブレーション(つまり, 任意の次数において全射)
    \item 任意の同型射$f' \in \Hom_{H^0(\D)}(x_0',x_1')$と$\F x_1 = x_1'$を満たす$x_1 \in H^0(\C)$に対して, ある同型射$u \in \Hom_{H^0(\C)}(x_0,x_1)$が存在して, $H^0(\F) u = u'$を満たす. 
  \end{itemize}
\end{definition}

\begin{definition}[Dwyer-Kanモデル構造]
  $\dgcat$に次のモデル構造を定義する.
  \begin{itemize}
    \item 弱同値はDK同値
    \item ファイブレーションはDKファイブレーション
  \end{itemize}
  このモデル構造を$\dgcat$上のDKモデル構造(Dwyer-Kanモデル構造)という. 
\end{definition}

\begin{remark}
  DKモデル構造を入れた$\dgcat$において, 任意の$\C \in \dgcat$はファイブラントである. 
\end{remark}

$\C$上の右加群のなす圏を$\hat{\C} := \fundg(\C^\myop,\chk)$と表す. 
$\chk$には組み合わせ論的モデル構造が入る. 
モデル圏論の一般論により, $\hat{\C}$にも組み合わせ論的モデル構造が入る. 

\begin{definition}[$\hat{\C}$のモデル構造]
  $\hat{\C}$に次のモデル構造を定義する.
  \begin{itemize}
    \item 弱同値は$\chk$におけるpointwiseの弱同値
    \item ファイブレーションは$\chk$におけるpointwiseのファイブレーション
  \end{itemize}
\end{definition}

\begin{remark}
  表現可能な右加群はコファイブラントかつコンパクトである. 
\end{remark}

\begin{remark}
  ファイブラントかつコファイブラントな右加群のなす圏を$\hat{\C}^{cf}$, $\C$の導来圏を$D(\C)$と表す.
  このとき, 三角圏同値$H^0(\hat{\C}^{cf}) \cong D(\C)$が成立する.
  よって, $\hat{\C}^{cf}$は$D(\C)$のdg増強である. 
\end{remark}

\begin{definition}[DK埋め込み]
  dg関手$\F : \C \to \D$が次の条件を満たすとき, $\F$をDK埋め込み(Dwyer-Kan embedding)という.
  \begin{itemize}
    \item 任意の$x_0,x_1 \in \C$に対して, 複体の射$\Hom_\C(x_0,x_1) \to \Hom_\D(\F x_0,\F x_1)$は擬同型
  \end{itemize}
\end{definition}

\begin{theorem}
  任意のdg圏は前三角的dg圏にDK埋め込みすることができる. 
\end{theorem}

\begin{proof}
  任意のdg圏$\C$に対して, $\hat{\C}$は前三角的である.
  $U$を集合とする.
  $\chk$の台集合$U$に値をとる右加群(関手)のなす$\hat{\C}$の充満部分圏を$U\hat{\C}$とする. 
  $U\hat{\C}$はdg圏である.
  また, $U$が十分大きい基数の冪集合であるとき, $U\hat{\C}$のモデル構造は$\C$と同じものをとれる.
  \footnote{
    small object argumentの話で必要となる仮定だと思われる. 
  }
  % また, $U\hat{\C}$上の右加群の圏上の(閉)モデル構造も得られる.
  $U\hat{\C}$のファイブラントかつコファイブラント対象のなす充満部分圏を$\tilde{\C}$とする. 
  $\tilde{\C}$は$\C$とDK埋め込みな前三角的dg圏である. 
\end{proof}


\bibliographystyle{jalpha}
\bibliography{../A_infty_cf}


\end{document}