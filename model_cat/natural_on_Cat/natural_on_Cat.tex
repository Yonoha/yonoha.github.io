\RequirePackage{plautopatch}
\documentclass[uplatex, a4paper, 14Q, dvipdfmx]{jsarticle}
\usepackage{docmute}
\usepackage{../new_mypackage2}

\title{$\Cat$上のnaturalモデル構造}
\author{よの}
\date{\today}

\begin{document}

\maketitle

\begin{abstract}
  小圏の圏にモデル構造を入れることにより, 圏論(の一部)をホモトピー論の枠組みで考えられるようになる. 
  まず, 圏同値(categorical equivalence)をweak equivalenceとするようなnaturalモデル構造が存在する. 
  他にも, 圏の脈体の間の射がKan weak equivalenceであるようなThomasonモデル構造が存在する. 
  前者の存在は\cite{Rezk96}で, 後者の存在は\cite{Thom80}でそれぞれ証明された. 
  本稿では, naturalモデル構造を定義し, 単体的集合の圏上のJoyalモデル構造とQuillen随伴であることを示す.  
\end{abstract}

\tableofcontents

\section{Naturalモデル構造}

小圏の圏にモデル構造を入れるとき, まずweak equivalenceとして圏同値(categorical equivalence)が考えられる. 
選択公理を仮定すると, weak equivalenceを圏同値とするような$\Cat$上のモデル構造は一意である. 
(Chris Schommer-Priesの\href{https://sbseminar.wordpress.com/2012/11/16/the-canonical-model-structure-on-cat/}{The canonical model structure on Cat}を参照)

\begin{definition}[擬ファイブレーション]
  $\C,\D$を圏, $F : \C \to \D$を関手とする. 
  任意の対象$c \in \Ob(\C)$とdomainが$F(c)$の同型射$g \in \D$に対して, domainが$c$のある同型射$f \in \C$が存在して, $F(f)=g$を満たすとき, $F$を擬ファイブレーション(quasi-fibration)という. 
\end{definition}

擬ファイブレーションはリフトを用いて特徴づけることができる. 

\begin{remark} \label{rem:quasi_fibration_has_RLP}
  $\C,\D$を圏, $F : \C \to \D$を関手とする. 
  このとき, $F$が擬ファイブレーションであることと, 次の四角がリフトを持つことは同値である. 
  % https://q.uiver.app/#q=WzAsNCxbMCwwLCIwIl0sWzEsMCwiXFxDIl0sWzEsMSwiXFxEIl0sWzAsMSwiSSJdLFsxLDIsIkYiXSxbMCwzXSxbMywyXSxbMCwxXSxbMywxLCIiLDEseyJzdHlsZSI6eyJib2R5Ijp7Im5hbWUiOiJkYXNoZWQifX19XV0=
  \[\begin{tikzcd}
    {\{0\}} & \C \\
    I & \D
    \arrow["F", from=1-2, to=2-2]
    \arrow[from=1-1, to=2-1]
    \arrow[from=2-1, to=2-2]
    \arrow[from=1-1, to=1-2]
    \arrow[dashed, from=2-1, to=1-2]
  \end{tikzcd}\]
  ここで, $\{0\}$は1点圏, $I$は2点対象とその間の一意な同型射からなる圏とする. 
\end{remark}

\begin{definition}[対象上のmono関手]
  $\C,\D$を圏, $F : \C \to \D$を関手とする. 
  $\Ob(F) : \Ob(\C) \to \Ob(\D)$がmono射のとき, $F$を対象上のmono関手(monic on objects)という. 
\end{definition}

\begin{definition}[naturalモデル構造]
  $\Cat$には次のモデル構造が存在する. 
  これを$\Cat$上のnaturalモデル構造
  \footnote{
    自明な(trivial)モデル構造や圏的(categorical)モデル構造と呼ばれることもある.
    このとき, $\Catnat$におけるfibrationをisofibraion, cofibrationをisocofibrationということもある. 
  }
  といい, $\Catnat$と表す. 
  \begin{itemize}
    \item weak equivalenceは通常の圏同値
    \item fibrationは擬ファイブレーション
    \item cofibrationは対象上のmono関手
  \end{itemize}
\end{definition}

\begin{remark}
  $\Catnat$において, 任意の対象(小圏)はファイブラントかつコファイブラントである. 
\end{remark}

\begin{remark}
  $\Catnat$は
  \begin{align*}
    I &:= \{\emptyset \to \{0\}, \{0\} \sqcup \{1\} \to \{0 \to 1\}, \{0 \rightrightarrows 1\} \to \{0 \to 1\}\} \\
    J &:= \{\{0\} \to \{0 \leftrightarrow 1\}\}
  \end{align*}
  をそれぞれgenerating cofibration, generating trivial cofibrationの集合とするコファイブラント生成なモデル圏である. 
\end{remark}

\begin{proof}
  まず, $I$に関して考える. 
  $u : \emptyset \to \{0\}, v : \{0\} \sqcup \{1\} \to \{0 \to 1\}, w : \{0 \rightrightarrows 1\} \to \{0 \to 1\}$とする. 
  まず, $u,v,w$がcofibrationであることは明らかである. 
  よって, trivial fibrationは$u,v,w$に対してRLPを持つ. 

  逆に, 関手$F : \C \to \D$が$u,v,w$に対してRLPを持つとする. 
  $u$に対してRLPを持つとき, $F$は対象上のepi射である. 
  $v$に対してRLPを持つとき, $G$は充満である. 
  $w$に対してRLPを持つとき, $G$は忠実である.
  \cref{rem:trifib_is_in_Cat}より, $F$はtrivial fibrationである. 
  
  $J$に関しては\cref{rem:quasi_fibration_has_RLP}より従う. 
\end{proof}

$\Catnat$におけるtrivial fibrationとtrivial cofibrationは簡単に表すことができる.

\begin{remark} \label{rem:trifib_is_in_Cat}
  $\C,\D$を圏, $F : \C \to \D$を関手とする. 
  このとき, $F$がtrivial fibrationであることと, $F$が圏同値かつ対象上のepi関手
  \footnote{
    対象上のepi関手は対象上のmono関手と同様に定義される. 
    $\C,\D$を圏, $F : \C \to \D$を関手とする. 
    $\Ob(\C) \to \Ob(\D)$がepi射のとき, $F$を対象上のepi関手(epic on objects)という. 
  }
  であることは同値である.
\end{remark}

\begin{remark} \label{rem:tricof_is_in_Cat}
  $\C,\D$を圏, $F : \C \to \D$を関手とする. 
  このとき, $F$がtrivial cofibrationであることと, $\C$が$\D$と圏同値であるような$\D$の部分圏であることは同値である. 
\end{remark}

$\Cat$上にnaturalモデル構造が存在することを示すために, いくつか準備をする.

まず, $\Catnat$がリフト性質を満たすことを示す. 

\begin{lemma} \label{prop:Cat_has_lift}
  次の図式において, $F$を対象上のmono関手, $G$を擬ファイブレーションとする. 
  % https://q.uiver.app/#q=WzAsNCxbMCwwLCJcXEMiXSxbMCwxLCJcXEQiXSxbMSwwLCJcXEUiXSxbMSwxLCJcXEYiXSxbMCwxLCJGIiwyXSxbMCwyLCJVIl0sWzIsMywiRyJdLFsxLDMsIlYiLDJdLFsxLDIsIkgiLDIseyJzdHlsZSI6eyJib2R5Ijp7Im5hbWUiOiJkYXNoZWQifX19XV0=
  \[\begin{tikzcd}
    \C & \E \\
    \D & \F
    \arrow["F"', from=1-1, to=2-1]
    \arrow["U", from=1-1, to=1-2]
    \arrow["G", from=1-2, to=2-2]
    \arrow["V"', from=2-1, to=2-2]
    \arrow["H"', dashed, from=2-1, to=1-2]
  \end{tikzcd}\]
  更に, $F$か$G$が圏同値のとき, この四角はリフト$H$を持つ. 
\end{lemma}

\begin{proof}
  まず, $G$が圏同値のときを考える. 
  このとき, $\Ob(F) : \Ob(\C) \to \Ob(\D)$はmono射, $\Ob(G) : \Ob(\E) \to \Ob(\F)$はepi射である. 
  $\D$の対象$d$が$F(\C)$に属するとき, $d=F(c)$を満たす一意な$\C$の対象$c$を用いて, $H(d) := U(c)$とする. 
  $d$が$F(\C)$に属さないとき, $V(d)=G(e)$を満たす$e \in \Ob(\E)$を用いて, $H(d):=e$とする.
  \footnote{
    ここで選択公理を用いている.
    実は, $F$が圏同値のときも同様の議論で示すことができる. 
    nlabの\href{https://ncatlab.org/nlab/show/canonical+model+structure+on+Cat}{Canonical model structure on Cat}のPropositin 1.2を参照. 
    このとき, $\Ob(F)$はepi射なので, 選択公理は用いない. 
    本文中の証明は\cite{Rezk96} Theorem 3.1を参考にした. 
  }
  $G$は圏同値かつ擬ファイブレーションなので, \cref{rem:trifib_is_in_Cat}より, 任意の$f : d \to d' \in \D$に対して, 
  \begin{align*}
    G : \Hom_\E(H(d),H(d')) \to  \Hom_\N(GH(d),GH(d')) = \Hom_\N(V(d),V(d'))
  \end{align*}
  は同型である. 
  よって, 射の対応$H : \C \to \D$はこの同型を用いて定める. 
  このとき, 求める四角の可換性はすぐに示すことができる. 

  次に, $F$が圏同値のときを考える. 
  \cref{rem:tricof_is_in_Cat}より, ある関手$F' : \D \to \C$が存在して, $F'F=\Id_\C$かつ$\alpha : FF' \to \Id_\D$は自然同型である. 
  更に, $\alpha$を$F$の像に制限すると, $\alpha|_{F(\C)} = \Id_{F(\C)}$である. 
  まず, 対象の対応$\Ob(H) : \Ob(\C) \to \Ob(\D)$を次のように定義する.
  $G$は擬ファイブレーションなので, $UF'(d) \in \Ob(\E)$とdomainが$GUF'(d) = VFF'(d)$である同型射$V(\alpha_d) : VFF'(d) \to V(d)$に対して, ある同型射$\beta_d : UF'(d) \to x$が存在して, $G(\beta_d) = V(\alpha_d)$かつ$G(x) = V(d)$となる. 
  よって, 任意の$d \in \Ob(D)$に対して, $\Ob(H)(d) := x$とする. 
  % https://q.uiver.app/#q=WzAsNCxbMCwwLCJVRicoZCkiXSxbMCwxLCJcXE9iKEgpKGQpIDo9eCJdLFsyLDAsIlZGRicoZCk9R1VGJyhkKSJdLFsyLDEsIlYoZCkiXSxbMCwxLCJcXGJldGFfZCIsMl0sWzIsMywiVihcXGFscGhhX2QpIl0sWzQsNSwiRyIsMCx7InNob3J0ZW4iOnsic291cmNlIjoyMCwidGFyZ2V0IjoyMH0sImxldmVsIjoxLCJzdHlsZSI6eyJ0YWlsIjp7Im5hbWUiOiJtYXBzIHRvIn19fV1d
  \[\begin{tikzcd}
    {UF'(d)} && {VFF'(d)=GUF'(d)} \\
    {\Ob(H)(d) :=x} && {V(d)}
    \arrow[""{name=0, anchor=center, inner sep=0}, "{\beta_d}"', from=1-1, to=2-1]
    \arrow[""{name=1, anchor=center, inner sep=0}, "{V(\alpha_d)}", from=1-3, to=2-3]
    \arrow["G", shorten <=22pt, shorten >=22pt, maps to, from=0, to=1]
  \end{tikzcd}\]
  次に, 射の対応$H : \C \to \D$を次のように定める. 
  任意の$\D$の射$f : d \to d'$に対して, 
  \begin{align*}
    H(f) := \beta_{d'} \cdot UF'(f) \cdot \beta_d^{-1} : H(d) \xrightarrow{\beta_d^{-1}} UF'(d) \xrightarrow{UF'(f)} UF'(d') \xrightarrow{\beta_{d'}} H(d')
  \end{align*}
  とする. 
  また, $d \in \Ob(\D)$が$F(\C)$に属する, つまりある対象$c \in \Ob(\C)$が一意に存在して$d=F(c)$と表せるときを考える. 
  $\alpha|_{F(c)} = \Id_{F(c)}$なので, $HF(c) = U(c)$かつ$\beta_{F(c)} = \id_{U(c)}$である. 
  これらのことから, 求める四角の可換性はすぐに示すことができる. 
\end{proof}

$\Catnat$が分解系を持つことを示す. 

\begin{lemma} \label{prop:Cat_has_fs}
  任意の関手$F : \C \to \D$は圏同値かつ対象上のmono射$U : \C \to \C'$と擬ファイブレーション$V : \C' \to \D$を用いて$F=VU$と分解できる. 
  また, 任意の関手$F : \C \to \D$は対象上のmono射$U : \C \to \D'$と圏同値かつ擬ファイブレーション$V : \D' \to \D$を用いて$F=VU$と分解できる. 
\end{lemma}

\begin{proof}
  まず, 任意の関手$F : \C \to \D$が圏同値かつ対象上のmono射$U : \C \to \C'$と擬ファイブレーション$V : \C' \to \D$を用いて$F=VU$と分解できることを示す. 
  圏$\C'$を次のように定義する. 
  まず, $\C'$の対象は 
  \begin{align*}
    \Ob(\C') := \{(c,d,\alpha) ~|~ c \in \Ob(\C), d \in \Ob(\D), \alpha : F(c) \cong d \in \D\}
  \end{align*}
  $\C'$の任意の対象$(c,d,\alpha), (c',d',\alpha')$に対して, 
  \begin{align*}
    \Hom_{\C'}((c,d,\alpha), (c',d',\alpha')) 
    := \Hom_\C(c,c')
  \end{align*}
  このとき, 関手$U : \C \to \C'$を任意の$c \in \Ob(\C)$と$f : c \to c' \in \C$に対して
  \begin{align*}
    U(c) := (c,F(c),\id_{F(c)}) ,~ U(f) := f
  \end{align*}
  とする. 
  関手$V : \C' \to \C$を任意の$(c,d,\alpha) \in \Ob(\C')$と$f : (c,d,\alpha) \to (c',d',\alpha') \in \C'$に対して
  \begin{align*}
    V((c,d,\alpha)) := d ,~ V(f) := \alpha^{-1} \cdot F(f) \cdot \alpha'
  \end{align*}
  とする. 
  このとき, $U$は圏同値かつ対象上のmono射, $V$は擬ファイブレーションである.

  次に, 任意の関手$F : \C \to \D$が対象上のmono射$U : \C \to \D'$と圏同値かつ擬ファイブレーション$V : \D' \to \D$を用いて$F=VU$と分解できることを示す. 
  圏$\D'$を$D' = \C \sqcup \D$で定義する. 
  このとき, 関手$U : \C \to \D'$を任意の$c \in \Ob(\C)$と$f : c \to c' \in \C$に対して
  \begin{align*}
    U(c) := c, ~ U(f) := F(f)
  \end{align*}
  とする. 
  関手$V : \D' \to \D$を任意の$(c,d) \in \Ob(\C \sqcup \D)$に対して, 
  \begin{align*}
    V((c,d)) := (F(c),d)
  \end{align*}
  とする. 
  このとき, $U$は対象上のmono射, $V$は圏同値かつ擬ファイブレーションである. 
\end{proof}

$\Cat$上にnaturalモデル構造が存在することを示す.

\begin{proof}
  まず, $\Cat$は任意の(有限)極限と(有限)余極限を持つ. 

  次に, weak equivalenceが2-out-of-3を満たすことは明らかである. 

  また, weak equivalenceとcofibrationがretractで閉じることは簡単に示すことができる.
  fibrationがretractで閉じることは\cref{rem:quasi_fibration_has_RLP}より, リフトの一般論から示すことができる. 

  リフト性質を満たすことは\cref{prop:Cat_has_lift}で, 分解系を持つことは\cref{prop:Cat_has_fs}で既に示した. 
\end{proof}

\section{$\sSetJoyal$と$\Catnat$のQuillen随伴} \label{sec:quillen_adj_sSetJoyal_Catnat}

\cref{sec:quillen_adj_sSetJoyal_Catnat}の目標は次の\cref{prop:quillen_adj_sSetJoyal_Catnat}を示すことである. 

\begin{proposition}[\cite{Joy08} Proposition 6.14] \label{prop:quillen_adj_sSetJoyal_Catnat}
  基本圏をとる関手$\tau_1 : \sSetJoyal \to \Catnat$と脈体$N : \Catnat \to \sSetJoyal$は, $\sSetJoyal$と$\Catnat$のQuillen随伴を定める.
  \begin{align*}
    \tau_1 : \sSetJoyal \rightleftarrows \Catnat : N
  \end{align*}
\end{proposition}

\begin{proof}
  まず, 左随伴がcofibrationを保つことを示す.
  任意の単体的集合$X$に対して, $\Ob(\tau_1(X))=X_0$である.
  また, 単体的集合の射$f : X \to Y$がmono射($\sSetJoyal$におけるcofibration)のとき, 特に$f_0 : X_0 \to Y_0$は対象上のmono射($\Catnat$におけるcofibration)である. 
  よって, 左随伴はcofibrationを保つ.

  次に, 左随伴がweak equivalenceを保つことを示す.
  $X$を単体的集合, $\C$を圏とする.
  \cite{Joy08} B.0.16より, 
  \begin{align*}
    \Fun(X,N(\C)) = \Fun(\tau_1(X),N(\C))
  \end{align*}
  である. 
  $\sSet^{\tau_0}$の定義より, 
  \begin{align*}
    \tau_0(X,N(\C))=\tau_0(\tau_1(X),N(\C))
  \end{align*}
  である. 
  従って, 任意の単体的集合の射$f : X \to Y$に対して, 
  \begin{align*}
    \tau_0(f,N(\C))=\tau_0(\tau_1(f),N(\C))
  \end{align*}
  である. 
  任意の圏$\C$に対して$N(\C)$は擬圏である. 
  よって, $f$が弱圏同値($\sSetJoyal$におけるweak equivalence)のとき, $\tau_0(f,N(\C))$は同型である. 
  つまり, $\tau_0(\tau_1(f),N(\C))$も同型である. 
  Yonedaの補題より, $\tau_1(f)$は$\Cat^{\tau_0}$における同型射である. 
  つまり, $\tau_1(f)$は圏同値($\Catnat$におけるweak equivalence)である. 
\end{proof}

\bibliographystyle{alpha}
\bibliography{../model_reference}

\end{document}