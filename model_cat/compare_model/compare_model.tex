\RequirePackage{plautopatch}
\documentclass[uplatex, a4paper, 14Q, dvipdfmx]{jsarticle}
\usepackage{docmute}
\usepackage{../new_mypackage}

\title{$(\infty,1)$圏のモデルについて}
\author{よの}
\date{\today}

\begin{document}

\maketitle

\begin{abstract}
  任意の$n > 1$に対して$n$射が可逆であるような$(\infty,1)$圏の異なるモデルを比較する.
  ホモトピー論の視点から見ると, $(\infty,1)$圏はホモトピー論のホモトピー論のモデルであると考えられる. 
  本稿では, $(\infty,1)$圏のモデルとして考えられる擬圏, 完備Segal空間, 相対圏の3つが$(\infty,1)$圏のモデルとして同値であることを見る. 
\end{abstract}

\tableofcontents

\newpage
\setcounter{section}{-1}

\section{準備}

$(\infty,1)$圏($(\infty,1)$-category)は任意の$n>1$に対して$n$射が可逆であるような弱$\infty$圏(weak $\infty$-category)であると言われる.
この$(\infty,1)$圏の情報をうまく組み込む方法が多く考えられてきた. 
実際, $(\infty,1)$圏のモデルとして単体的圏(simplicial category), Segal圏(Segal category), 完備Segal圏(complete Segal space), 擬圏(quasi-category), 相対圏(relative category)などがある.
\footnote{
  他にも, dg圏(dg-category), $A_\infty$圏($A_\infty$-category), ホモトピカル圏(homotopical category)などがある. 
} 
これらの圏はホモトピー論と(高次)圏論の両方の異なる動機から考えられた.
そして, Bergner \cite{Ber07b}やJoyal-Tierney \cite{JT07}により, これらの$(\infty,1)$圏のモデルが等価であることが示された. 

まず, 「ホモトピー論」というと位相空間のホモトピー論が挙げられる. 
ここでは, ホモトピー同値(homotopy equivalence)や弱ホモトピー同値(weak homotopy equivalence)である位相空間は同一視して扱いたい. 
これを考えるために, CW複体やCW近似が定義され研究された. 
この流れはホモロジー代数における複体のprojective resolutionなどにも見られる. 

このような操作を一般的に行う枠組みとして, Quillen \cite{Qui67}は1960年代にモデル圏(model category)を定義した. 
モデル圏はweak equivalence, fibration, cofibrationの3つ組を用いて定義される. 
モデル圏の局所化は集合論的な問題を避けることができる利点がある. 
モデル圏の同値としてQuillen同値(Quillen equivalence)がある. 
このQuillen同値は圏とホモトピーの両方を保つような同値である. 
よって, $(\infty,1)$圏のモデルの等価であるかは, 2つのモデル圏がQuillen同値であるかで判断することができる. 

単体的圏は1980年代にDwyerとKanにより「ホモトピー論」を考えるために深く研究された. 
weak equivalenceを持つ圏に対して, 単体的局所化(simplicial localization)やハンモック局所化(hammock localization)により単体的圏が得られる. 
更に, 任意の単体的圏がDK同値(Dwyer-Kan equivalence)の違いを除いてこの方法により得られることが示された. 
よって, 「ホモトピー論」がweak equivalenceを持つ圏であると思うと, 単体的圏はホモトピー論のモデルとして考えられる. 

更に, 単体的圏の圏自体もDK同値をweak equivalenceとして持つ圏となる. 
つまり, 単体的圏の圏も「ホモトピー論」を表すと考えられる. 
よって, 単体的圏の圏は「ホモトピー論のホモトピー論」のモデルと思える. 
この単体的圏の圏上にはBergnerモデル構造 \cite{Ber07}が存在する.
しかし, このモデル構造は単体的モデル圏(simplicial model category)でなかったり, weak equivalenceの判定が難しかったりなど扱いづらい. 
そのため, 単体的圏の圏上のBergnerモデル構造とQuillen同値である, より良い構造を持つモデル圏が考えられるようになった. 

$(\infty,1)$圏の他のモデルとしてSegal圏がある.
Segal圏は1990年代にDywer, Kan, Smithにより「up to homotopyで合成が定義されるような単体的圏」として定義された. 
HirschowitzとSimpsonは圏論的な視点からSegal圏を調べ, PellissierはSegal前圏上のモデル構造を与えた. 

完備Segal空間は2000年代始めにRezk \cite{Rezk00}により「ホモトピー論のホモトピー論」のモデルとして考案された.
完備Segal空間はいくつかの条件を満たす単体的空間である. 
単体的空間の圏上のReedyモデル構造におけるweak equivalenceはlevelwiseに定義されるために扱いやすい. 
Reedyモデル構造のBousfield局所化(left Bousfield localization)で得られるRezkモデル構造はCartesianモデル圏であるなど, 単体的圏の圏上のBergnerモデル構造より良い性質を多く持つ. 

擬圏は1970年代にBoardmanとVogtにより定義され, JoyalやLurie \cite{Lur09}などにより深く調べられた. 
Segal圏や完備Segal空間は単体的空間上で定義されるが, 擬圏は単体的集合上で定義される. 
このため, 擬圏は他のモデルよりも扱いやすいという利点がある. 
Quillen \cite{Qui67}により単体的集合の圏上にQuillenモデル構造が定義され, Joyal \cite{JoyXX}によりJoyalモデル構造が定義された. 
Quillenモデル構造は位相空間の圏上のKan-Quillenモデル構造とQuillen同値であるため, $(\infty,0)$圏のモデルと見れる. 
一方, Joyalモデル構造は単体的圏の圏上のBergnerモデル構造とQuillen同値であるため, $(\infty,1)$圏のモデルと思える. 

BarwickとKan \cite{BK11}は2010年代始めにweak equivalenceを持つ圏を相対圏として定義し, 「ホモトピー論のホモトピー論」のモデルの枠組みとして考えた. 
彼らは単体的空間の圏上のRezkモデル構造とQuillen同値になるように, 相対圏の圏上のBarwickモデル構造を与えた. 

\subsection*{本稿の目的}

第一の目的は, $(\infty,0)$圏のモデルと$(\infty,1)$圏のモデルについてまとめることである. 

主目的は, 次の図式にある$(\infty,1)$圏のモデルとその間のQuillen同値を示すことである. 
% https://q.uiver.app/#q=WzAsNixbMSwxLCJcXFByZVNlQ2F0X2MiXSxbMSwwLCJcXHNTcGFjZUNTUyJdLFswLDAsIlxcc1NldEpveWFsIl0sWzAsMSwiXFxzU2V0Q2F0QmVyZyJdLFsyLDAsIlxcUmVsQ2F0QmFyIl0sWzEsMiwiXFxQcmVTZUNhdF9mIl0sWzAsMSwiKEkgXFxkYXNodiBSKSIsMix7InN0eWxlIjp7InRhaWwiOnsibmFtZSI6ImFycm93aGVhZCJ9fX1dLFsxLDIsIihpIFxcZGFzaHYgXFxldl8wKSIsMix7InN0eWxlIjp7InRhaWwiOnsibmFtZSI6ImFycm93aGVhZCJ9fX1dLFswLDMsIihGIFxcZGFzaHYgUikiLDIseyJzdHlsZSI6eyJ0YWlsIjp7Im5hbWUiOiJhcnJvd2hlYWQifX19XSxbMywyLCIoXFxtYXRoZnJha3tDfSBcXGRhc2h2IFxcaGF0e059KSIsMCx7InN0eWxlIjp7InRhaWwiOnsibmFtZSI6ImFycm93aGVhZCJ9fX1dLFsxLDQsIihLX1xceGkgXFxkYXNodiBOX1xceGkpIiwwLHsic3R5bGUiOnsidGFpbCI6eyJuYW1lIjoiYXJyb3doZWFkIn19fV0sWzAsNSwiKFxcSWQgXFxkYXNodiBcXElkKSIsMCx7InN0eWxlIjp7InRhaWwiOnsibmFtZSI6ImFycm93aGVhZCJ9fX1dLFszLDUsIihGIFxcZGFzaHYgUikiLDIseyJzdHlsZSI6eyJ0YWlsIjp7Im5hbWUiOiJhcnJvd2hlYWQifX19XV0=
\[\begin{tikzcd}
	\sSetJoyal & \sSpaceCSS & \RelCatBar \\
	\sSetCatBerg & {\PreSeCat_c} \\
	& {\PreSeCat_f}
	\arrow["{(I \dashv R)}"', tail reversed, from=2-2, to=1-2]
	\arrow["{(i^\ast_1 \dashv p^\ast_1)}"', tail reversed, from=1-2, to=1-1]
	\arrow["{(F \dashv R)}"', tail reversed, from=2-2, to=2-1]
	\arrow["{(\mathfrak{C} \dashv \hat{N})}", tail reversed, from=2-1, to=1-1]
	\arrow["{(K_\xi \dashv N_\xi)}", tail reversed, from=1-2, to=1-3]
	\arrow["{(\Id \dashv \Id)}", tail reversed, from=2-2, to=3-2]
	\arrow["{(F \dashv R)}"', tail reversed, from=2-1, to=3-2]
\end{tikzcd}\]

本稿で示すQuillen随伴とQuillen同値をまとめておく. (随時更新予定)
\footnote{
  同じモデル圏の複数のQuillen随伴については, 特徴的ものを図にまとめている. 
  例えば, Bousfield局所化で得られるQuillen随伴はこの図に載せていない. 
}
% https://q.uiver.app/#q=WzAsOSxbMywxLCJcXFByZVNlQ2F0X2MiXSxbMywwLCJcXHNTcGFjZUNTUyJdLFsyLDAsIlxcc1NldEpveWFsIl0sWzIsMSwiXFxzU2V0Q2F0QmVyZyJdLFs0LDAsIlxcUmVsQ2F0QmFyIl0sWzMsMiwiXFxQcmVTZUNhdF9mIl0sWzEsMCwiXFxzU2V0S2FuIl0sWzAsMCwiXFxUb3BRdWlsbGVuIl0sWzAsMSwiXFxUb3BTdHJvbSJdLFswLDEsIihJIFxcZGFzaHYgUikiLDIseyJzdHlsZSI6eyJ0YWlsIjp7Im5hbWUiOiJhcnJvd2hlYWQifX19XSxbMSwyLCIoaSBcXGRhc2h2IFxcZXZfMCkiLDIseyJzdHlsZSI6eyJ0YWlsIjp7Im5hbWUiOiJhcnJvd2hlYWQifX19XSxbMCwzLCIoRiBcXGRhc2h2IFIpIiwyLHsic3R5bGUiOnsidGFpbCI6eyJuYW1lIjoiYXJyb3doZWFkIn19fV0sWzMsMiwiKFxcbWF0aGZyYWt7Q30gXFxkYXNodiBcXGhhdHtOfSkiLDAseyJzdHlsZSI6eyJ0YWlsIjp7Im5hbWUiOiJhcnJvd2hlYWQifX19XSxbMSw0LCIoS19cXHhpIFxcZGFzaHYgTl9cXHhpKSIsMCx7InN0eWxlIjp7InRhaWwiOnsibmFtZSI6ImFycm93aGVhZCJ9fX1dLFswLDUsIihcXElkIFxcZGFzaHYgXFxJZCkiLDAseyJzdHlsZSI6eyJ0YWlsIjp7Im5hbWUiOiJhcnJvd2hlYWQifX19XSxbMiw2LCJrXiEiLDAseyJvZmZzZXQiOi0xfV0sWzYsMiwia18hIiwwLHsib2Zmc2V0IjotMX1dLFs3LDYsInwtfCIsMCx7Im9mZnNldCI6LTF9XSxbNiw3LCJcXFNpbmciLDAseyJvZmZzZXQiOi0xfV0sWzgsNywiXFxJZCIsMCx7Im9mZnNldCI6LTF9XSxbNyw4LCJcXElkIiwwLHsib2Zmc2V0IjotMX1dLFszLDUsIihGIFxcZGFzaHYgUikiLDIseyJzdHlsZSI6eyJ0YWlsIjp7Im5hbWUiOiJhcnJvd2hlYWQifX19XV0=
\[\begin{tikzcd}
	\TopQuillen & \sSetKan & \sSetJoyal & \sSpaceCSS & \RelCatBar \\
	\TopStrom && \sSetCatBerg & {\PreSeCat_c} \\
	&&& {\PreSeCat_f}
	\arrow["{(I \dashv R)}"', tail reversed, from=2-4, to=1-4]
	\arrow["{(i^\ast_1 \dashv p^\ast_1)}"', tail reversed, from=1-4, to=1-3]
	\arrow["{(F \dashv R)}"', tail reversed, from=2-4, to=2-3]
	\arrow["{(\mathfrak{C} \dashv \hat{N})}", tail reversed, from=2-3, to=1-3]
	\arrow["{(K_\xi \dashv N_\xi)}", tail reversed, from=1-4, to=1-5]
	\arrow["{(\Id \dashv \Id)}", tail reversed, from=2-4, to=3-4]
	\arrow["{k^!}", shift left, from=1-3, to=1-2]
	\arrow["{k_!}", shift left, from=1-2, to=1-3]
	\arrow["{|-|}", shift left, from=1-1, to=1-2]
	\arrow["\Sing", shift left, from=1-2, to=1-1]
	\arrow["\Id", shift left, from=2-1, to=1-1]
	\arrow["\Id", shift left, from=1-1, to=2-1]
	\arrow["{(F \dashv R)}"', tail reversed, from=2-3, to=3-4]
\end{tikzcd}\]

% \subsection*{本稿の構成}

% \cref{sec:model_cat}では, モデル圏論の基本的なことについて説明する. 

% \cref{sec:model_stru_in_Top}から\cref{sec:model_stru_in_RelCat}はそれぞれの圏に入るモデル構造を述べる. 
% それぞれの圏の基本的な事柄については, 対応する\cref{sec:topological_space}から\cref{sec:relative_category}を参照してほしい. 
% 付録では, 本稿を読むために必要な定義や命題を最低限ではあるが紹介する. 

% \cref{sec:quillen_equiv_sSetKan_and_TopQuillen}から\cref{sec:quillen_equiv_sSpaceReedy_and_RelCatBar}はそれぞれのモデル構造の間にQuillen同値やQuillen随伴が存在することを証明する. 

% 圏は特に断らない限りsmallであるとする. 

\subsection*{記法}

本稿で登場する圏の記法をまとめる. 
\begin{itemize}
  \item 位相空間のなす圏を$\Top$, コンパクト生成弱Hausdorff空間のなす圏を$\CGWH$
  \item 小圏のなす圏を$\Cat$
  \item 単体的集合のなす圏を$\sSet$
  \item 単体的空間のなす圏を$\sSpace$
  % \item Segal前圏のなす圏を$\PreSeCat$
  % \item 単体的圏のなす圏を$\sSetCat$, $O$対象固定な単体的圏のなす圏を$\sSetCat_O$
  \item 小相対圏のなす圏を$\RelCat$
\end{itemize}

\newpage


\section{モデル圏論} \label{sec:model_cat}

モデル圏(model category)はホモトピー論を抽象的に行うための枠組みとして, Quillen \cite{Qui67}により導入された. 
ホモトピー論を抽象的に行う方法として, 弱同値(weak equivalence)をもつような圏の構造を公理化して扱うことが考えられる. 
動機としては, 弱ホモトピー同値をweak equivalenceとするような位相空間の圏がある. 
この位相空間の圏に対して定義されるホモトピー圏(homotopy category)を, 一般のモデル圏に対して定義することができる. 

\subsection{properなモデル圏} \label{subsec:additional_model_stru}

モデル圏において, fibrationはpullbackで閉じ, cofibrationはpushoutで閉じる. 
しかし, weak equivalenceがpullbackやpushoutで閉じるとは限らない. 
ここでは, weak equivalenceがfibration (cofibration)に沿ったpullback (pushout)で閉じるようなモデル圏を考える. 

\begin{definition}[properなモデル圏]
  $\M$をモデル圏とする. 
  \begin{itemize}
    \item cofibrationに沿ったweak equivalenceのpushoutがweak equivalenceになるとき, $\M$は左proper (left proper)であるという.   
    つまり, weak equivalence $f : A \to X$とcofibration $i : A \to B$に対して$g : B \to Y$がweak equivelenceになるときである. 
    % https://q.uiver.app/#q=WzAsNCxbMCwwLCJBIl0sWzAsMSwiQiJdLFsxLDAsIlgiXSxbMSwxLCJZIl0sWzAsMSwiaSIsMl0sWzAsMiwiZiJdLFsyLDNdLFsxLDMsImciLDJdLFszLDAsIiIsMSx7InN0eWxlIjp7Im5hbWUiOiJjb3JuZXIifX1dXQ==
    \[\begin{tikzcd}
      A & X \\
      B & Y
      \arrow["i"', from=1-1, to=2-1]
      \arrow["f", from=1-1, to=1-2]
      \arrow[from=1-2, to=2-2]
      \arrow["g"', from=2-1, to=2-2]
      \arrow["\lrcorner"{anchor=center, pos=0.125, rotate=180}, draw=none, from=2-2, to=1-1]
    \end{tikzcd}\]
    \item fibrationに沿ったweak equivalenceのpullbackがweak equivalenceになるとき, $\M$は右proper (right proper)であるという.   
    つまり, weak equivalence $g : B \to Y$とfibration $p : X \to Y$に対して$f: A \to X$がweak equivelenceとなるときである. 
    % https://q.uiver.app/#q=WzAsNCxbMCwwLCJBIl0sWzAsMSwiQiJdLFsxLDAsIlgiXSxbMSwxLCJZIl0sWzAsMV0sWzAsMiwiZiJdLFsyLDMsInAiXSxbMSwzLCJnIiwyXSxbMCwzLCIiLDEseyJzdHlsZSI6eyJuYW1lIjoiY29ybmVyIn19XV0=
    \[\begin{tikzcd}
      A & X \\
      B & Y
      \arrow[from=1-1, to=2-1]
      \arrow["f", from=1-1, to=1-2]
      \arrow["p", from=1-2, to=2-2]
      \arrow["g"', from=2-1, to=2-2]
      \arrow["\lrcorner"{anchor=center, pos=0.125}, draw=none, from=1-1, to=2-2]
    \end{tikzcd}\]
    \item $\M$が左properかつ右properであるとき, $\M$はproperであるという. 
  \end{itemize}
\end{definition}

次の命題はモデル圏がproperであるかを確かめる方法の1つである. 
これは\cref{prop:proper_hold_locally}の系として得られる. 

\begin{proposition} \label{prop:every_object_is_cofibrant_means_left_proper}
  $\M$をモデル圏とする. 
  \begin{enumerate}
    \item $\M$の任意の対象がコファイブラントのとき, $\M$は左properである. 
    \item $\M$の任意の対象がファイブラントのとき, $\M$は右properである. 
  \end{enumerate}
\end{proposition}

Reedyモデル構造はproperを保つ. 

\begin{proposition}[\cite{Hir03} Theorem 15.3.4]
  $\R$をReedy圏, $\M$を左(右) properなモデル圏とする.
  このとき, $\Fun(\R,\M)_\Reedy$は左(右) properなモデル圏である. 
\end{proposition}

Bousfield局所化は左properを保つ. 

\begin{proposition}
  左properな組み合わせ論的モデル圏のBousfield局所化は左properである.
\end{proposition}

本稿で登場する多くのモデル圏は左properである.

\begin{example}
  左properなモデル圏の例を挙げる. 
  \begin{itemize}
    \item \cref{prop:every_object_is_cofibrant_means_left_proper}の(1)より, $\sSetKan$と$\sSetJoyal$は左properである.  
    \item $\TopQuillen$において任意の対象はコファイブラントではないが, $\TopQuillen$は左properである.  
    \item $\RelCatBar$は左properである. 
  \end{itemize}
  
  右properなモデル圏の例を挙げる.
  \begin{itemize}
    \item \cref{prop:every_object_is_cofibrant_means_left_proper}の(2)より, $\TopQuillen$は右properである. 
    \item $\sSetKan$において任意の対象はファイブラントではないが, $\sSetKan$は右properである. 
  \end{itemize}
  
  properなモデル圏の例を挙げる. 
  \begin{itemize}
    \item \cref{prop:every_object_is_cofibrant_means_left_proper}の(1)と(2)より, $\TopStrom$はproperである.
    \item \cref{prop:every_object_is_cofibrant_means_left_proper}の(1)と(2)より, $\Catnat$はproperである.
    \item 上の2つの例より, $\TopQuillen$と$\sSetKan$はproperである. 
  \end{itemize}
\end{example}

左(右) properでないモデル圏も存在する. 

\begin{counter}
  左(右) properでないモデル圏の例を挙げる. 
  \begin{itemize}
    \item simplicial associative unital algebraの圏上のstandard model structureは左properではない. (\cite{Rezk00} Example 2.11)
    \item $\sSetJoyal$と$\sSpaceCSS$は右properではない. 
  \end{itemize}
\end{counter}

コファイブラント(ファイブラント)の間の左(右) proper性は一般のモデル圏で成立する. 

\begin{proposition} \label{prop:proper_hold_locally}
  一般のモデル圏に対して, 次が成立する. 
  \begin{itemize}
    \item cofibrationに沿ったコファイブラントの間のweak equivalenceのpushoutはweak equivalenceである. 
    \item fibraionに沿ったファイブラントの間のweak equivalenceのpullbackはweak equivalenceである. 
  \end{itemize}
\end{proposition}

% properなモデル圏において, homotopy (co)limitは一般のモデル圏より簡単に計算できる. 
% (途中)

\subsection{単体的モデル圏}

単体的モデル圏は\cite{Qui67}で導入された.

\begin{definition}[単体的圏]
  $\sSet$豊穣圏を単体的圏(simplicial category)という. 
  % \footnote{
  %   $\Cat$の単体的対象もsimplicial categoryということも多いので, $\sSet$で豊穣された圏を単体的豊穣圏(simplicially enriched category)ということも多い. 
  %   本稿では, $\Cat$の単体的対象は(おそらく)登場しないので, $\sSet$で豊穣された圏を単体的圏(simplicial category)という. 
  % }
\end{definition}

\begin{definition}[単体的モデル圏]
  $\sSetKan$豊穣圏を単体的モデル圏(simplicial model category)という.
\end{definition}

単体的モデル圏の定義を具体的に書き下すと次のようになる. 

\begin{definition}[単体的モデル圏]
  $\M$を単体的圏かつモデル圏とする. 
  \footnote{
    $\M$のunderlying categoryが単体圏であるとき, という意味である. 
    このような省略は今後でも用いる.
  }
  $\M$が次の条件を満たすとき, $\M$は単体的(simplicial)であるという.
  \begin{enumerate}
    \item $\M$の任意の対象$X,Y$と単体的集合$K$に対して, $\M$のある対象$X \otimes Y$と$Y^K$と単体的集合の同型
    \begin{align*}
      \Map_\M(X \otimes K, Y) \cong \Map_\M(K, \Map_\M(X,Y)) \cong \Map_\M(X,Y^K)
    \end{align*}
    が存在して, この同型は$X,Y,K$に関して自然である. 
    \item $\M$の任意のcofibration $i : A \to B$とfibration $p : X \to Y$に対して, 単体的集合の射
    \begin{align*}
      \Map_\M(B,X) \to \Map_\M(A,X) \times_{\Map_\M(A,Y)} \Map_\M(B,Y)
    \end{align*}
    はfibrationである. 
    更に, $i$か$p$のいずれかがweak equivalenceのとき, この射もweak equivalenceである. 
  \end{enumerate}
\end{definition}

Bousfield局所化は単体的であることを保つ. 

\begin{proposition}
  左properな単体的かつ組み合わせ論的モデル圏のBousfield局所化は単体的モデル圏である.
\end{proposition}

\begin{example}
  単体的モデル圏の例を挙げる. 
  \begin{itemize}
    \item $\sSetKan$は自明な単体的モデル圏である. 
    \item $\sSpaceReedy$は単体的モデル圏である. 
  \end{itemize}
\end{example}

単体的モデル圏でないモデル圏も存在する. 

\begin{counter}
  $\sSetJoyal$は$\sSet$豊穣圏(つまり単体的圏)ではあるが, $\sSetKan$豊穣圏ではなく$\sSetJoyal$豊穣圏なので, 単体的モデル圏ではない.
\end{counter}

単体的モデル圏の条件(2)より, 次の命題が従う. 

\begin{corollary} \label{prop:cofibration_induce_fibraion_in_simplicial_model_cat}
  $\M$を単体的モデル圏, $f : A \to B$を$\M$におけるcofibration, $X$をファイブラントとする. 
  このとき, 誘導される射 
  \begin{align*}
    \Map_\M(B,X) \to \Map_\M(A,X)
  \end{align*}
  は単体的集合のfibrationである. 
  更に, $f$がweak equivalenceのとき, この射もweak equivalenceである. 
\end{corollary}

\begin{proof}
  単体的モデル圏の条件(2)において, $Y$を$\M$における終対象とすればよい. 
\end{proof}

\subsection{組み合わせ論的モデル圏} \label{subsec:combinatorial}

モデル圏の扱いやすいクラスの1つに組み合わせ論的(combinatorial)モデル圏がある. 
組み合わせ論的モデル圏はJ.H.Smithにより導入された. 
組み合わせ論的モデル圏はsmall objectの間の(trivial) generating cofibrationから生成されるような圏といえる.
多くのモデル圏は組み合わせ論的であるか, 組み合わせ論的モデル圏とQuillen同値である. 
% また, 組み合わせ論的単体的モデル圏は局所表現可能($\infty$,1)圏のモデルであることがLurieにより示された. 

\begin{definition}[局所表現可能圏]
  $\C$を通常の圏とする. 
  $\C$が次の条件を満たすとき, $\C$は局所表現可能(locally presentable)であるという. 
  \begin{itemize}
    \item $\C$は任意の小余極限を持つ. 
    \item ある対象の集まり$S$が存在して, $\C$の任意の対象は$S$のある対象の有限余極限で表せる.  
  \end{itemize}
\end{definition}

\begin{definition}[組み合わせ論的モデル圏]
  $\M$が局所表現可能圏かつコファイブラント生成なモデル圏のとき, $\M$は組み合わせ論的(combinatorial)であるという. 
\end{definition}

Bousfield局所化組み合わせ論的であることを保つ. 

\begin{proposition} \label{prop:left_Bousfield_is_combinatorial}
  左properな単体的かつ組み合わせ論的モデル圏のBousfield局所化は組み合わせ論的モデル圏である.
\end{proposition}

組み合わせ論的モデル圏の例を挙げる. 

\begin{example}
  $\sSetKan$と$\sSetJoyal$は組み合わせ論的モデル圏である. 
\end{example}

コファイブラント生成であるが, 組み合わせ論的ではないモデル圏も存在する.

\begin{counter} \label{counter:TopQuillen_isnt_combinatorial}
  $\TopQuillen$はコファイブラント生成であるが, 組み合わせ論的ではない. 
\end{counter}

\begin{remark}
  \cref{counter:TopQuillen_isnt_combinatorial}で見たように, $\TopQuillen$は組み合わせ論的ではない. 
  しかし, \cref{prop:quillen_equiv_sSetKan_and_TopQuillen}より, $\TopQuillen$は組み合わせ論的モデル圏である$\sSetKan$とQuillen同値である.
  そこで, 任意のコファイブラント生成なモデル圏はある組み合わせ論的モデル圏とQuillen同値となるか, という疑問が生じる. 
  Vopenka principleを仮定すると, 任意のコファイブラント生成なモデル圏はある組み合わせ論的モデル圏とQuillen同値となることが証明されている. 
  しかし, Vopenka principleはZFC上で証明することができないことには注意が必要である. 
\end{remark}

組み合わせ論的モデル圏を特徴づけるSmithの定理を紹介する. 
Smithの定理は組み合わせ論的なモデル構造をより小さいデータから決定できることを示している. 
例えば, \cref{prop:left_Bousfield_is_combinatorial}などで有用である. 

\begin{proposition}[Smithの定理]
  $\M$を局所表現可能圏, $W$を$\M$の部分圏, $I \subset \Mor(\M)$を$\M$の射の集合とし, 次の条件を満たすとする. 
  \begin{itemize}
    \item $W$はレトラクトで閉じ, 2-out-of-3を満たす. 
    \item $\RLP(I) \subset W$である. 
    \item $\LLP(\RLP(I)) \cap W$はpushoutと超限合成で閉じる.
  \end{itemize}
  このとき, weak equivalenceの集まりを$W$, cofibrationの集まりを$\LLP(\RLP(I))$とする組み合わせ論的モデル構造$\M$が存在する. 
\end{proposition}

% \begin{definition}[到達可能圏]
%   $\C$を通常の圏とする. 
%   $\C$が次の条件を満たすとき, $\C$は到達可能(accessible)であるという. 
%   \begin{itemize}
%     \item $\C$は任意のフィルター付き余極限を持つ. 
%     \item コンパクト対象の集まりが存在して, $\C$の任意の対象はあるコンパクト対象のフィルター付き余極限で表せる. 
%   \end{itemize}
% \end{definition}

\subsection{Cartesianモデル圏}

圏のCartesian構造とモデル構造がある整合性を持つとき, モデル圏はCartesianであるという. 

\begin{definition}[Cartesianモデル圏]
  $\M$をCartesian閉なモデル圏とする. 
  $\M$が次の条件を満たすとき, $\M$はCartesianであるという. 
  \begin{itemize}
    \item $\M$の終対象はコファイブラントである. 
    \item $\M$の任意のcofibration $f : A \to A', g : B \to B'$に対して, 誘導される射 
    \begin{align*}
      h : A \times B' \coprod_{A \times B} A' \times B \to A' \times B'
    \end{align*}
    はcofibrationである. 
    更に, $f$か$g$のいずれかがweak equivalenceのとき, $h$もweak equivalenceである.
    \item $\M$の任意のcofibration $f : A \to A'$とfibration $p : X' \to X$に対して, 誘導される射
    \begin{align*}
      q : (X')^{A'} \to (X')^{A} \times_{X^A} X^{A'}
    \end{align*}
    はfibrationである. 
    更に, $f$か$p$のいずれかがweak equivalenceのとき, $q$もweak equivalenceである.
  \end{itemize}
\end{definition}

\begin{example}
  Cartesianモデル圏の例を挙げる. 
  \begin{itemize}
    \item $\CGWH$の圏上のQuillenモデル構造はCartesianモデル圏である. 
    \item $\Catnat$はCartesianモデル圏である.
    \item $\sSetKan$はCartesianモデル圏である. 
    \item $\sSpaceproj, \sSpaceReedy, \sSpaceCSS$はCartesianモデル圏である. 
  \end{itemize}
\end{example}

\subsection{Bousfield局所化} \label{subsec:left_bousfield_localization}

$\C$を圏, $W$を$\C$の射の集まりとする. % [Ber18 1.2]
$W$の元は$\C$における同型射ではないが, 何かしらの意味で同型射であると思いたいものとする. 
このような射はweak equivalenceと呼ばれる. 
通常の圏の局所化(localization)は, 圏$\C$からweak equivalenceが実際に同型射となるような圏$\C[W^{-1}]$を構成する方法である. 
この局所化をモデル圏に対して考える.

\begin{notation}
  $\M$をモデル圏とする. 
  $\M$の対象$X$のファイブラント置換を$X^f$, コファイブラント置換を$X^c$と表す. 
\end{notation}

\begin{definition}[ホモトピー射空間]
  $\M$をモデル圏とする. 
  $\M$の任意の対象$X,Y$に対して, 
  \begin{align*}
    \Map^h_\M(X,Y) := \Map_\M(X^c,Y^f)
  \end{align*}
  を$X$と$Y$のホモトピー射空間(homotopy mapping space)という. 
\end{definition}

\begin{definition}[$T$局所対象と$T$局所同値] \label{def:t_local}
  $\M$をモデル圏, $T$を$\M$のある射の集合とする. 
  \begin{itemize}
    \item $W$を$\M$のファイブラントとする. 
    $T$の任意の射$f : A \to B$に対して,  
    \begin{align*}
      \Map^h_\M(B,W) \to \Map^h(A,W)
    \end{align*}
    がweak equivalenceのとき, $W$を$T$局所対象($T$-local object)という. 
    \item $g : X \to Y$を$\M$の任意の射とする. 
    任意の$T$局所対象$W$に対して, 
    \begin{align*}
      \Map^h_\M(Y,W) \to \Map^h_\M(X,W)
    \end{align*}
    がweak equivalenceのとき, $g$を$T$局所同値($T$-local equivalence)という.
  \end{itemize}
\end{definition}

% \begin{definition}[$f$局所対象と$f$局所同値]
%   \cref{def:t_local}において, $T$が1つの射$f$からなるとする. 
%   このとき, $T$局所対象を$f$局所($f$-local object), $T$局所同値を$f$局所同値($f$-local equivalence)という. 
% \end{definition}

$T$局所対象と$T$局所同値を用いて, モデル圏から新しいモデル圏を構成することができる.

\begin{theorem}[\cite{Hir03} Proposition 4.1.1] \label{prop:left_bousfield_localization}
  $\M$を左properな, cellularまたは組み合わせ論的モデル圏, $T$を$\M$のある射の集合とする.  
  このとき, 同じunderlying category上に次の条件を満たすモデル構造$\L_T\M$が存在する. 
  \begin{itemize}
    \item $\L_T\M$のweak equivalenceは$\M$の$T$局所同値
    \item $\L_T\M$のcofibrationは$\M$のcofibration 
  \end{itemize}
  このとき, $\L_T\M$も左properな, cellularまたは組み合わせ論的モデル圏である. 
  更に, $\M$が単体的モデル圏のとき, $\L_T\M$も単体的モデル圏である. 
\end{theorem}

\begin{definition}[Bousfield局所化]
  \cref{prop:left_bousfield_localization}で得られる局所化$\L_T\M$を$T$による$\M$のBousfield局所化(Bousfield localization)
  \footnote{
    同様に, モデル圏のBousfield余局所化(Bousfield colocalization)も定義される.
  }
  という. 
\end{definition}

Bousfield局所化におけるweak equivalenceは$T$局所同値を用いて特徴づけられる. 
また, 次の命題からBousfield局所化が「局所化」であることが分かる. 

\begin{lemma}[\cite{Hir03} Proposition 3.1.5] 
  $\M$をモデル圏, $\L_T\M$をBousfield局所化とする. 
  このとき, $\M$のweak equivalenceは$\M$の$T$局所同値である. 
  つまり, $W_\M \subset W_{\L_T\M}$が成立する. 
\end{lemma}

Bousfield局所化におけるファイブラントは$T$局所対象を用いて特徴づけられる. 

\begin{remark}
  $\M$をモデル圏, $\L_T\M$をBousfield局所化とする. 
  このとき, $\L_T\M$におけるファイブラントは$T$局所対象である. 
\end{remark}

元のモデル圏とBousfield局所化におけるfibrationなどは次のような関係がある. 

\begin{remark} \label{rem:fibration_in_Bousfield_localization}
  $\M$をモデル圏, $\L_T\M$を$\M$のBousfield局所化とする. 
  \begin{itemize}
    \item $\L_T\M$のfibrationの集まりは$\M$のfibrationの部分クラスである. 
    \begin{align*}
      \Fib_{\L_T\M} = \RLP(\Cof_{\L_T\M} \cap W_{\L_T\M}) \subset \RLP(\Cof_{\L_T\M} \cap W_\M) = \Fib_\M
    \end{align*}
    \item $\L_T\M$のtrivial fibrationの集まりは$\M$のtrivial fibrationの集まりと一致する.
    \begin{align*}
      \Fib_{\L_T\M} \cap W_{\L_T\M} = \RLP(\Cof_{\L_T\M}) = \RLP(\Cof_\M) = \Fib_\M \cap W_\M
    \end{align*}
  \end{itemize}
\end{remark}

元のモデル圏とBousfield局所化の間には恒等関手によるQuillen随伴が存在する. 

\begin{proposition} \label{prop:quillen_adj_Bousfield_localization}
  恒等関手$\Id : \M \to \L_t\M$と恒等関手$\Id : \L_T\M \to \M$は, $\M$と$\L_T\M$のQuillen随伴を定める. 
  \begin{align*}
    \Id : \M \rightleftarrows \L_t\M : \Id
  \end{align*}
\end{proposition}

\begin{proof}
  Bousfield局所化の定義より, $\Id : \M \to \L_t\M$はcofibrationを保つ. 
  \cref{rem:fibration_in_Bousfield_localization}より, $\Id : \L_T\M \to \M$はfibrationを保つ. 
\end{proof}

\begin{proposition}
  $\M_1,\M_2$を同じunderlying category上のモデル圏とする. 
  $T$をunderlying categoryにおける射の集合とする. 
  $\M_1$と$\M_2$におけるweak equivalenceが一致するとき, $\L_T\M_1$と$\L_T\M_2$におけるweak equivalenceは一致する.
\end{proposition}

\begin{proof}
  一般のモデル圏$\M$の任意の対象$X,Y$に対して, 
  \begin{align*}
    \pi_0 \Map^h_\M(X,Y) = \Hom_{\Ho(\M)}(X,Y)
  \end{align*}
  が成立するので, ホモトピー射空間はunderlying categoryとweak equivalenceのみで決定され, fibrationやcofibrationによらない. 
  局所同値の定義より, 局所化$\L_T\M$のweak equivalenceは$T$とunderlying categoryのみによる. 
  局所化$\L_T\M$におけるweak equivalenceは$T$局所同値で定義されることから従う. 
\end{proof}

\begin{proposition} \label{prop:local_obj_equals_RLP}
  $\M$を単体的モデル圏, $T$をコファイブラントの間のcofibration, $\L_T\M$を$T$による$\M$のBousfield局所化とする. 
  このとき, $\M$のファイブラント$Z$が$T$局所対象であることと, $Z \to 0$が次の射の集合に対してRLPを持つことは同値である. 
  \begin{align*}
    \{A \otimes \Delta[m] \cup_{A \otimes \partial \Delta[m]} B \otimes \partial \Delta[m] \to B \otimes \Delta[m] ~|~ A \to B \in T, m \geq 0\}
  \end{align*}
\end{proposition}

\subsection{Reedyモデル構造}

$\R$を圏, $\M$をモデル圏とする. 
このとき, 関手圏$\Fun(\R,\M)$にどのようなモデル構造が入るかを考える. 
射影的/入射的モデル構造では, $\M$に更に条件を課すことで定義することができた. 
ここでは, $\R$に条件を課し, $\M$に全く条件を課さずに定まる関手圏上のReedyモデル構造(Reedy model structure)を定義する. 
このReedyモデル構造は射影的モデル構造と入射的モデル構造の中間的なモデル構造である. 

\begin{definition}[Reedy圏]
  $\R$を圏, $\R_+,\R_-$を$\R$の全ての対象を含む$\R$の部分圏, $\alpha$を基数, $d : \Ob(\R) \to \alpha$を関数とする. 
  この$d$を次数関数(degree function)という. 
  4つ組$(\R,\R_+,\R_-,d)$が次の条件を満たすとき, $\R$をReedy圏(Reedy category)という. 
  \begin{itemize}
    \item $\R_+$における恒等射以外の射は次数を上げる.
    \item $\R_-$における恒等射以外の射は次数を下げる. 
    \item $\R$の任意の射$f$はある$\R_-$の射$f'$と$\R_+$の射$f''$を用いて, $f=f'f''$と一意に分解できる. 
  \end{itemize}
\end{definition}

Reedy圏は単体圏の構造を抽出して公理化された構造である.

\begin{example}
  $\Delta$を単体圏とする. 
  次数関数$d : \Ob(\R) \to \alpha$は$[k] \mapsto k$, $\Delta_+$の射はmono射$[k] \mapsto [n]$, $\Delta_-$の射はepi射$[n] \mapsto [k]$とすると, $\Delta$はReedy圏である. 
\end{example}

\begin{example}
  基数$\alpha$を半順序集合とみなす.
  次数関数$d :\alpha \to \alpha$は恒等関数, $\alpha_+$は$\alpha$, $\alpha_-$を$\alpha$上の離散圏とすると, $\alpha$はReedy圏である. 
\end{example}

Reedy構造は双対的である. 

\begin{remark}
  $(\R,\R_+,\R_-,d)$をReedy圏とする. 
  このとき, $(\R^\myop,\R_-,\R_+,d)$はReedy圏である.
\end{remark}

関手圏上に定まるReedyモデル構造を定義するためにいくつか準備する.

\begin{definition}[latching and matching object]
  $\R$をReedy圏, $\M$をモデル圏とする. 
  $X : \R \to \M$を関手, $r$を$\R$の対象とする. 
  このとき, スライス圏$R_+/r$の$\id_r$を含まない部分圏における余極限
  \begin{align*}
    L_r(X) := \colim X(s)
  \end{align*}
  を$r$における$X$のlatching objectという.

  双対的に, コスライス圏$r/R_-$の$\id_r$を含まない部分圏における余極限
  \begin{align*}
    M_r(X) := \lim X(s)
  \end{align*}
  を$r$における$X$のmatching objectという.
\end{definition} 

関手圏上に定まるReedyモデル構造を定義する.

\begin{definition}[Reedyモデル構造]
  $\R$をReedy圏, $\M$をモデル圏, $f : X \to Y$を$\Fun(\R,\M)$の射とする.
  このとき, $\Fun(\R,\M)$には次のモデル構造が存在する. 
  これを$\Fun(\R,\M)$上のReedyモデル構造といい, $\Fun(\R,\M)_\Reedy$と表す.  
  \begin{itemize}
    \item weak equivalenceは$f$の各成分$f_r$が$\M$におけるweak equivalenceである自然変換$f$
    \item fibrationは$\R$の任意の対象$r$に対して, 
    \begin{align*}
      X(r) \to M_r(X) \times_{M_r(Y)} Y(r)
    \end{align*}
    が$\M$におけるfibrationである自然変換$f$
    \item cofibrationは$\R$の任意の対象$r$に対して, 
    \begin{align*}
      L_r(Y) \sqcup_{L_r(X)} X(r) \to Y(r)
    \end{align*}
    が$\M$におけるcofibrationである自然変換$f$
  \end{itemize}
\end{definition}

Reedyモデル構造における(コ)ファイブラントは次のように特徴づけることができる. 

\begin{remark}
  $\Fun(\R,\M)_\Reedy$における対象$X$がファイブラントであることと, $\R$の任意の対象$r$に対して射$X(r) \to M_r(X)$がfibrationであることは同値である. 

  双対的に, $\Fun(\R,\M)_\Reedy$における対象$X$がコファイブラントであることと, $\R$の任意の対象$r$に対して射$L_r(X) \to X(r)$がcofibrationであることは同値である. 
\end{remark}

% \begin{proposition}
%   $\R$をReedy圏, $\M$を左(右) properなモデル圏とする. 
%   このとき, $\Fun(\R,\M)_\Reedy$は左(右) properである. 
% \end{proposition}

Reedyモデル構造はコファイブラント生成であることを保つ. 

\begin{proposition}
  $\R$をReedy圏, $\M$をコファイブラント生成なモデル圏とする. 
  このとき, $\Fun(\R,\M)_\Reedy$はコファイブラント生成である. 
\end{proposition}

% \subsection{補足 : Quillen双関手について}

% \begin{definition}[Quillen双関手]
%   $\C,\D,\E$をモデル圏, $F : \C \times \D \to \E$を関手とする. 
%   $F$が次の条件を満たすとき, $F$をQuillen双関手(Quillen bifunctor)という.
%   \begin{itemize}
%     \item $F$は両変数について余極限を保つ. 
%     \item 
%   \end{itemize}
% \end{definition}

\newpage


\section{単体的集合の圏に入るモデル構造} \label{sec:model_stru_in_sSet}

単体的集合は$(\infty,0)$圏のなす$(\infty,1)$圏のモデルや, $(\infty,1)$圏のなす$(\infty,1)$圏のモデルとして考えられてきた. 
前者は$\sSet$上のKan-Quillenモデル構造, 後者は$\sSet$上のJoyalモデル構造として表される. 
Kan-Quillenモデル構造は\cite{Qui67_2}で, Joyalモデル構造は\cite{JoyXX}でそれぞれ証明された. 

\subsection{Kan-Quillenモデル構造}

Kan-Quillenモデル構造は, ファイブラントがKan複体(Kan complex)であり, weak equivalenceが弱ホモトピー同値(weak homotopy equivalence)であるようなモデル構造である. 

\begin{definition}[単体的集合の弱ホモトピー同値]
  $f : A \to B$を単体的集合の射とする. 
  任意のKan複体$X$に対して, $f$の合成から定まる射
  \begin{align*}
    \pi_0(f,X) : \pi_0(B,X) \to \pi_0(A,X)
  \end{align*}
  が集合の同型射のとき, $f$を単体的集合の弱ホモトピー同値(weak homotopy equivalence)という.
\end{definition}

\begin{definition}[Kanファイブレーション]
  $f : X \to S$を単体的集合の射とする.
  任意の$0 \leq i \leq n$に対して, 次の四角がリフト$\sigma : \Delta[n] \to X$をもつとき, $f$をKanファイブレーション(Kan fibration)という. 
  % https://q.uiver.app/#q=WzAsNCxbMCwwLCJcXExhbWJkYV5uX2kiXSxbMSwwLCJYIl0sWzAsMSwiXFxEZWx0YV5uIl0sWzEsMSwiUyJdLFswLDEsIlxcc2lnbWFfMCJdLFswLDIsIiIsMix7InN0eWxlIjp7InRhaWwiOnsibmFtZSI6Imhvb2siLCJzaWRlIjoidG9wIn19fV0sWzIsMSwiXFxzaWdtYSIsMix7InN0eWxlIjp7ImJvZHkiOnsibmFtZSI6ImRhc2hlZCJ9fX1dLFsyLDMsIlxcYmFye1xcc2lnbWF9IiwyXSxbMSwzLCJmIl1d
  \[\begin{tikzcd}
    {\Lambda[n,i]} & X \\
    {\Delta[n]} & S
    \arrow["{\sigma_0}", from=1-1, to=1-2]
    \arrow[hook, from=1-1, to=2-1]
    \arrow["\sigma"', dashed, from=2-1, to=1-2]
    \arrow[from=2-1, to=2-2]
    \arrow["f", from=1-2, to=2-2]
  \end{tikzcd}\]
\end{definition}

\begin{definition}[単体的集合のmono射]
  $f : X \to Y$を単体的集合の射とする. 
  任意の$n \geq 0$に対して, $f_n : X_n \to Y_n$が集合のmono射のとき, $f$を単体的集合のmono射(mono morphism)という.
\end{definition}

\begin{definition}[Kan-Quillenモデル構造]
  $\sSet$には次のモデル構造が存在する.
  これを$\sSet$上のKan-Quiilenモデル構造
  \footnote{
    古典的(classical)モデル構造やQuillenモデル構造と呼ばれることもある. 
  }
  といい, $\sSetKan$と表す. 
  \begin{itemize}
    \item weak equivalenceは単体的集合の弱ホモトピー同値
    \item fibrationはKanファイブレーション
    \item cofibrationは単体的集合のmono射
  \end{itemize}
\end{definition}

\begin{remark}
  $\sSetKan$において, 任意のKan複体はファイブラントであり, 任意の対象(単体的集合)はコファイブラントである. 
\end{remark}

\begin{remark} \label{rem:sSetKan_is_cof_gen}
  $\sSetKan$は
  \begin{align*}
    I &:= \{\partial \Delta[n] \hookrightarrow \Delta[n] ~|~ n \geq 0\}, \\
    J &:= \{\Lambda[n,i] \hookrightarrow \Delta[n] ~|~ n \geq 0, 0 \leq i \leq n\}
  \end{align*}
  をそれぞれgenerating cofibration, generating trivial cofibrationの集合とするコファイブラント生成なモデル圏である. 
\end{remark}

\begin{theorem}
  $\sSetKan$はproperである.
\end{theorem}

\begin{proof}
  左properであることは, $\sSetKan$において任意の対象がコファイブラントであることと, \cref{prop:every_object_is_cofibrant_means_left_proper}より従う. 
  右properであることの証明は\cite{GJ09} II.8.6-7を参照. 
\end{proof}

\begin{theorem}
  $\sSetKan$は単体的モデル圏である.
\end{theorem}

\begin{proof}
  任意の$X,Y \in \sSetKan$と$n \geq 0$に対して, 単体的集合$\Map_{\sSetKan}(X,Y)$の$n$単体を$\Hom_\sSet(X \times \Delta[n],Y)$とする. 
  任意の単体的集合$K$に対して, $X \otimes K$を通常の単体的集合の直積$X \times K$, $Y^K$を通常のべき$\Map(K,Y)$とする.
  このとき, 次の集合の同型を示せばよい.
  \begin{align*}
    \Hom_\sSet((X \times K) \times \Delta[n],Y) 
    &\cong \Hom_\sSet(K ,\Hom_\sSet(X \times \Delta[n], Y)) \\
    &\cong \Hom_\sSet(X \times \Delta[n], \Hom_\sSet(K \times \Delta[n], Y)) 
  \end{align*}
  これは単体的集合の直積とHom空間の随伴性より従う. 
\end{proof}

\begin{theorem}
  $\sSetKan$は組み合わせ論的モデル圏である. 
\end{theorem}

\begin{proof}
  \cref{rem:sSetKan_is_cof_gen}より, $\sSetKan$はコファイブラント生成なモデル圏である.
  任意の単体的集合は標準$n$単体$\Delta[n]$の余極限で表せるので, $\sSet$は局所表現可能圏である. 
  よって, $\sSetKan$は組み合わせ論的モデル圏である. 
\end{proof}

\begin{theorem}
  $\sSetKan$はCartesianモデル圏である. 
\end{theorem}

\begin{proof}
  $\sSetKan$において任意の対象はコファイブラントなので, 特に$\sSetKan$における終対象$\Delta[0]$もコファイブラントである. 
  Cartesianモデル圏の条件(2)の前半は単体的集合のmono射の定義から, 後半は\cite[\href{https://kerodon.net/tag/014D}{Tag 014D}]{kerodon}より従う. 
  Cartesianモデル圏の条件(3)は\cite[\href{https://kerodon.net/tag/00TK}{Tag 00TK}]{kerodon}と\cite[\href{https://kerodon.net/tag/014E}{Tag 014E}]{kerodon}より従う. 
\end{proof}

$\sSetKan$におけるtrivial fibrationとtrivial cofibrationはリフトを用いて表すことができる.

\begin{lemma}[\cite{GJ09} Proposition I.11.1] \label{prop:trifib_is_in_sSetKan}
  $f : X \to Y$を単体的集合の射とする. 
  このとき, $f$がtrivial fibrationであることと, $f$が自明なKanファイブレーションであることは同値である. 
\end{lemma}

\begin{lemma} \label{prop:tricof_is_in_sSetKan}
  $f : X \to Y$を単体的集合の射とする. 
  このとき, $f$がtrivial cofibrationであることと, $f$が緩射であることは同値である. 
\end{lemma}

$\sSet$上にKan-Quillenモデル構造が存在することを示すために, いくつか準備をする.

まず, $\sSetKan$が分解系を持つことを示す. 
Quillenの小対象論法(small object argument)の系として得られるが, 本稿では分解を具体的に構成することで(片方を)証明する. 

\begin{proposition} \label{prop:sSetKan_has_fs}
  $f : X \to Y$を単体的集合の射とする.
  このとき, ある緩射$f' : X \to Q(f)$とKan ファイブレーション$f'' : Q(f) \to Y$が存在して, 次の図式は可換である. 
  % https://q.uiver.app/#q=WzAsMyxbMCwwLCJYIl0sWzIsMCwiWSJdLFsxLDEsIlEoZikiXSxbMCwxLCJmIl0sWzAsMiwiZiciLDJdLFsyLDEsImYnJyIsMl1d
  \[\begin{tikzcd}
    X && Y \\
    & {Q(f)}
    \arrow["f", from=1-1, to=1-3]
    \arrow["{f'}"', from=1-1, to=2-2]
    \arrow["{f''}"', from=2-2, to=1-3]
  \end{tikzcd}\]
  更に, 単体的集合$Q(f)$と$f',f''$は関手的である. 
  % このとき, 関手 
  % \begin{align*}
  %   \sSet^I \to \sSet : (f : X \to Y) \mapsto Q(f)
  % \end{align*}
  % はフィルター付き余極限と交換する.
\end{proposition}

\begin{proof}
  単体的集合の列$\{X(m)\}_{m \geq 0}$と単体的集合の射$f(m) : X(m) \to Y$を帰納的に定義する. 
  まず$X(0) := X, f(0) := f$とする. 
  定義された単体的集合の射$f(m) : X(m) \to Y$に対して, 任意の$0 \leq i \leq n, n > 0$において, $S(m)$を次の可換図式$\sigma$の集まりとする. 
  % https://q.uiver.app/#q=WzAsNCxbMCwwLCJcXExhbWJkYV5uX2kiXSxbMCwxLCJcXERlbHRhXm4iXSxbMSwwLCJYKG0pIl0sWzEsMSwiWSJdLFswLDEsIiIsMSx7InN0eWxlIjp7InRhaWwiOnsibmFtZSI6Imhvb2siLCJzaWRlIjoidG9wIn19fV0sWzAsMl0sWzIsMywiZihtKSJdLFsxLDMsInVfXFxzaWdtYSIsMl1d
  \[\begin{tikzcd}
    {\Lambda[n,i]} & {X(m)} \\
    {\Delta[n]} & Y
    \arrow[hook, from=1-1, to=2-1]
    \arrow[from=1-1, to=1-2]
    \arrow["{f(m)}", from=1-2, to=2-2]
    \arrow["{u_\sigma}"', from=2-1, to=2-2]
  \end{tikzcd}\]
  図式$\sigma \in S(m)$に対して, $X(m+1)$を次のpushoutで定義する. 
  % https://q.uiver.app/#q=WzAsNCxbMCwwLCJcXGNvcHJvZF97XFxzaWdtYSBcXGluIFMobSl9IFxcTGFtYmRhXm5faSJdLFswLDEsIlxcY29wcm9kX3tcXHNpZ21hIFxcaW4gUyhtKX0gXFxEZWx0YV5uIl0sWzEsMCwiWChtKSJdLFsxLDEsIlgobSsxKSJdLFswLDEsIiIsMSx7InN0eWxlIjp7InRhaWwiOnsibmFtZSI6Imhvb2siLCJzaWRlIjoidG9wIn19fV0sWzAsMl0sWzIsM10sWzEsM10sWzMsMCwiIiwxLHsic3R5bGUiOnsibmFtZSI6ImNvcm5lciJ9fV1d
  \[\begin{tikzcd}
    {\coprod_{\sigma \in S(m)} \Lambda[n,i]} & {X(m)} \\
    {\coprod_{\sigma \in S(m)} \Delta[n]} & {X(m+1)}
    \arrow[hook, from=1-1, to=2-1]
    \arrow[from=1-1, to=1-2]
    \arrow[from=1-2, to=2-2]
    \arrow[from=2-1, to=2-2]
    \arrow["\lrcorner"{anchor=center, pos=0.125, rotate=180}, draw=none, from=2-2, to=1-1]
  \end{tikzcd}\]
  また, $f(m+1) : X(m+1) \to Y$を$X(m)$への制限が$f(m)$に等しく, 各$\Delta[n]$への制限が$u_\sigma$に等しい一意な射とする. 
  $X(m)$の定義より, 次の緩射の列が存在する.
  \begin{align*}
    X=X(0) \hookrightarrow X(1) \hookrightarrow X(2) \hookrightarrow \cdots
  \end{align*}
  ここで, $Q(f) := \colim_{m} X(m)$とする. 
  緩射の集まりは超限合成で閉じるので, 自然な射$f' : X \to Q(f)$も緩射である. 
  また, ある射$f'' : Q(f) \to Y$が一意に存在して, 次の図式を可換にする. 
  % https://q.uiver.app/#q=WzAsNSxbMCwwLCJYPVgoMCkiXSxbMSwwLCJYKDEpIl0sWzIsMCwiXFxjZG90cyJdLFszLDAsIlEoZik9IFxcY29saW1fbSBYKG0pIl0sWzIsMSwiWSJdLFswLDEsIiIsMix7InN0eWxlIjp7InRhaWwiOnsibmFtZSI6Imhvb2siLCJzaWRlIjoidG9wIn19fV0sWzEsMl0sWzIsM10sWzAsNCwiZiIsMl0sWzMsNCwiZicnIl0sWzAsMywiZiciLDAseyJjdXJ2ZSI6LTJ9XV0=
  \[\begin{tikzcd}
    {X=X(0)} & {X(1)} & \cdots & {Q(f)= \colim_m X(m)} \\
    && Y
    \arrow[hook, from=1-1, to=1-2]
    \arrow[from=1-2, to=1-3]
    \arrow[from=1-3, to=1-4]
    \arrow["f"', from=1-1, to=2-3]
    \arrow["{f''}", from=1-4, to=2-3]
    \arrow["{f'}", curve={height=-12pt}, from=1-1, to=1-4]
  \end{tikzcd}\]
  % 構成より, $f \mapsto Q(f)$は関手的であり, フィルター付き余極限と交換する. 
  構成より, $f \mapsto Q(f)$は関手的である. 
  後は, $f'' : Q(f) \to Y$がKanファイブレーションであることを示せばよい. 
  つまり, 次の図式がリフトを持つことを示せばよい. 
  % https://q.uiver.app/#q=WzAsNCxbMCwwLCJcXExhbWJkYV5uX2kiXSxbMSwwLCJRKGYpIl0sWzEsMSwiWSJdLFswLDEsIlxcRGVsdGFebiJdLFswLDEsInYiXSxbMSwyLCJmJyciXSxbMCwzLCIiLDAseyJzdHlsZSI6eyJ0YWlsIjp7Im5hbWUiOiJob29rIiwic2lkZSI6InRvcCJ9fX1dLFszLDJdLFszLDEsIiIsMSx7InN0eWxlIjp7ImJvZHkiOnsibmFtZSI6ImRhc2hlZCJ9fX1dXQ==
  \[\begin{tikzcd}
    {\Lambda[n,i]} & {Q(f)} \\
    {\Delta[n]} & Y
    \arrow["v", from=1-1, to=1-2]
    \arrow["{f''}", from=1-2, to=2-2]
    \arrow[hook, from=1-1, to=2-1]
    \arrow[from=2-1, to=2-2]
    \arrow[dashed, from=2-1, to=1-2]
  \end{tikzcd}\]
  $\Lambda[n,i]$は有限単体的集合なので, $v$の像は十分大きな$m \gg 0$に対して, $X(m)$の像に含まれる. 
  つまり, 次の図式は可換である. 
  % https://q.uiver.app/#q=WzAsMyxbMCwwLCJcXExhbWJkYV5uX2kiXSxbMiwwLCJRKGYpIl0sWzEsMSwiWChtKSJdLFswLDEsInYiXSxbMCwyLCJ2JyIsMl0sWzIsMV1d
  \[\begin{tikzcd}
    {\Lambda[n,i]} && {Q(f)} \\
    & {X(m)}
    \arrow["v", from=1-1, to=1-3]
    \arrow["{v'}"', from=1-1, to=2-2]
    \arrow[from=2-2, to=1-3]
  \end{tikzcd}\]
  次の図式
  % https://q.uiver.app/#q=WzAsNCxbMCwwLCJcXExhbWJkYV5uX2kiXSxbMSwwLCJYKG0rMSkiXSxbMSwxLCJZIl0sWzAsMSwiXFxEZWx0YV5uIl0sWzAsMSwidiJdLFsxLDIsImYobSsxKSJdLFswLDMsIiIsMCx7InN0eWxlIjp7InRhaWwiOnsibmFtZSI6Imhvb2siLCJzaWRlIjoidG9wIn19fV0sWzMsMl0sWzMsMSwiIiwxLHsic3R5bGUiOnsiYm9keSI6eyJuYW1lIjoiZGFzaGVkIn19fV1d
  \[\begin{tikzcd}
    {\Lambda[n,i]} & {X(m)} \\
    {\Delta[n]} & Y
    \arrow["{v'}", from=1-1, to=1-2]
    \arrow["{f(m)}", from=1-2, to=2-2]
    \arrow[hook, from=1-1, to=2-1]
    \arrow[from=2-1, to=2-2]
    \arrow[dashed, from=2-1, to=1-2]
  \end{tikzcd}\]
  は$S(m)$に属するので, 次の図式は可換かつリフトを持つ. 
  % https://q.uiver.app/#q=WzAsNyxbMCwwLCJcXExhbWJkYV5uX2kiXSxbMSwwLCJcXGNvcHJvZF97XFxzaWdtYSBcXGluIFMobSl9IFxcTGFtYmRhXm5faSJdLFswLDIsIlxcRGVsdGFebiJdLFsyLDAsIlgobSkiXSxbMywwLCJYKG0rMSkiXSxbMiwxLCJcXGNvcHJvZF97XFxzaWdtYSBcXGluIFMobSl9IFxcRGVsdGFebiJdLFszLDIsIlkiXSxbMCwxLCIiLDAseyJzdHlsZSI6eyJ0YWlsIjp7Im5hbWUiOiJob29rIiwic2lkZSI6InRvcCJ9fX1dLFswLDIsIiIsMix7InN0eWxlIjp7InRhaWwiOnsibmFtZSI6Imhvb2siLCJzaWRlIjoidG9wIn19fV0sWzEsMywiXFxjb3Byb2QgdiciXSxbMyw0LCIiLDAseyJzdHlsZSI6eyJ0YWlsIjp7Im5hbWUiOiJob29rIiwic2lkZSI6InRvcCJ9fX1dLFsyLDUsIiIsMix7InN0eWxlIjp7InRhaWwiOnsibmFtZSI6Imhvb2siLCJzaWRlIjoidG9wIn19fV0sWzUsNF0sWzQsNiwiZihtKzEpIl0sWzIsNl0sWzEsNSwiIiwwLHsic3R5bGUiOnsidGFpbCI6eyJuYW1lIjoiaG9vayIsInNpZGUiOiJ0b3AifX19XV0=
  \[\begin{tikzcd}
    {\Lambda^n_i} & {\coprod_{\sigma \in S(m)} \Lambda^n_i} & {X(m)} & {X(m+1)} \\
    && {\coprod_{\sigma \in S(m)} \Delta^n} \\
    {\Delta^n} &&& Y
    \arrow[hook, from=1-1, to=1-2]
    \arrow[hook, from=1-1, to=3-1]
    \arrow["{\coprod v'}", from=1-2, to=1-3]
    \arrow[hook, from=1-3, to=1-4]
    \arrow[hook, from=3-1, to=2-3]
    \arrow[from=2-3, to=1-4]
    \arrow["{f(m+1)}", from=1-4, to=3-4]
    \arrow[from=3-1, to=3-4]
    \arrow[hook, from=1-2, to=2-3]
  \end{tikzcd}\]
\end{proof}

次に, $\sSetKan$がリフト性質を満たすことを示す. 

\begin{lemma} \label{prop:sSetKan_has_lift}
  次の図式において, $i$を単体的集合のmono射, $p$をKanファイブレーションとする. 
  % https://q.uiver.app/#q=WzAsNCxbMCwwLCJVIl0sWzAsMSwiViJdLFsxLDAsIlgiXSxbMSwxLCJZIl0sWzAsMSwiaSIsMl0sWzAsMl0sWzIsMywicCJdLFsxLDNdLFsxLDIsIiIsMSx7InN0eWxlIjp7ImJvZHkiOnsibmFtZSI6ImRhc2hlZCJ9fX1dXQ==
  \[\begin{tikzcd}
    U & X \\
    V & Y
    \arrow["i"', from=1-1, to=2-1]
    \arrow["j", from=1-1, to=1-2]
    \arrow["p", from=1-2, to=2-2]
    \arrow[from=2-1, to=2-2]
    \arrow[dashed, from=2-1, to=1-2]
  \end{tikzcd}\]
  $i$か$p$が弱ホモトピー同値のとき, この四角はリフトを持つ. 
\end{lemma}

\begin{proof}
  まず, $p$が自明なKanファイブレーションのときを考える. 
  \cref{prop:trifib_is_in_sSetKan}より, 次の図式はリフトを持つ.
  % https://q.uiver.app/#q=WzAsNCxbMCwwLCJcXHBhcnRpYWwgXFxEZWx0YVtuXSJdLFswLDEsIlxcRGVsdGFbbl0iXSxbMSwwLCJYIl0sWzEsMSwiWSJdLFswLDEsIiIsMix7InN0eWxlIjp7InRhaWwiOnsibmFtZSI6Imhvb2siLCJzaWRlIjoidG9wIn19fV0sWzAsMl0sWzIsMywicCJdLFsxLDNdLFsxLDIsIiIsMSx7InN0eWxlIjp7ImJvZHkiOnsibmFtZSI6ImRhc2hlZCJ9fX1dXQ==
  \[\begin{tikzcd}
    {\partial \Delta[n]} & X \\
    {\Delta[n]} & Y
    \arrow[hook, from=1-1, to=2-1]
    \arrow[from=1-1, to=1-2]
    \arrow["p", from=1-2, to=2-2]
    \arrow[from=2-1, to=2-2]
    \arrow[dashed, from=2-1, to=1-2]
  \end{tikzcd}\] 
  単体的集合のmono射の集まりは$\{\partial \Delta[n] \hookrightarrow \Delta[n]\}_{n \geq 0}$で生成される弱飽和なクラスである. 
  RLPの性質は弱飽和なクラスについて閉じるので, 求める四角はリフトを持つ.
  
  次に, $i$が緩射のときを考える. 
  \cref{prop:sSetKan_has_fs}より, $i$は緩射$j : U \to X$とKanファイブレーション$p : X \to V$を用いて$i=pj$と分解できる. 
  2-out-of-3より$p$も弱ホモトピー同値なので, $p$は自明なKanファイブレーションである. 
  上述の議論より, この図式はリフト$s : V \to X$を持つ. 
  % https://q.uiver.app/#q=WzAsNCxbMCwwLCJVIl0sWzAsMSwiViJdLFsxLDAsIlgiXSxbMSwxLCJWIl0sWzAsMSwiaSIsMl0sWzAsMiwiaiJdLFsyLDMsInAiXSxbMSwzLCJcXGlkX1YiLDIseyJsZXZlbCI6Miwic3R5bGUiOnsiaGVhZCI6eyJuYW1lIjoibm9uZSJ9fX1dLFsxLDIsInMiLDIseyJzdHlsZSI6eyJib2R5Ijp7Im5hbWUiOiJkYXNoZWQifX19XV0=
  \[\begin{tikzcd}
    U & X \\
    V & V
    \arrow["i"', from=1-1, to=2-1]
    \arrow["j", from=1-1, to=1-2]
    \arrow["p", from=1-2, to=2-2]
    \arrow["{\id_V}"', Rightarrow, no head, from=2-1, to=2-2]
    \arrow["s"', dashed, from=2-1, to=1-2]
  \end{tikzcd}\]
  次の図式を考えると, $i$は$j$のretractであり, 緩射の集まりはretractで閉じるので, $i$も緩射である.
  % https://q.uiver.app/#q=WzAsNixbMCwwLCJVIl0sWzAsMSwiViJdLFsxLDEsIlgiXSxbMiwxLCJWIl0sWzEsMCwiVSJdLFsyLDAsIlUiXSxbMCwxLCJpIl0sWzIsMywicCJdLFsxLDIsInMiXSxbMCw0LCIiLDAseyJsZXZlbCI6Miwic3R5bGUiOnsiaGVhZCI6eyJuYW1lIjoibm9uZSJ9fX1dLFs0LDUsIiIsMCx7ImxldmVsIjoyLCJzdHlsZSI6eyJoZWFkIjp7Im5hbWUiOiJub25lIn19fV0sWzUsMywiaSJdLFs0LDIsImoiXSxbMSwzLCJcXGlkX1YiLDIseyJjdXJ2ZSI6Mn1dXQ==
  \[\begin{tikzcd}
    U & U & U \\
    V & X & V
    \arrow["i", from=1-1, to=2-1]
    \arrow["p", from=2-2, to=2-3]
    \arrow["s", from=2-1, to=2-2]
    \arrow[Rightarrow, no head, from=1-1, to=1-2]
    \arrow[Rightarrow, no head, from=1-2, to=1-3]
    \arrow["i", from=1-3, to=2-3]
    \arrow["j", from=1-2, to=2-2]
    \arrow["{\id_V}"', curve={height=12pt}, from=2-1, to=2-3]
  \end{tikzcd}\] 
  緩射は任意のkanファイブレーションに対してLLPを持つので, 求める四角はリフトを持つ.
\end{proof}

$\sSet$上にKan-Quillenモデル構造が存在することを示す. 

\begin{proof}
  まず, $\sSet$は任意の(有限)極限と(有限)余極限を持つ. 

  次に, weak equivalenceが2-out-of-3を満たすことと, retractで閉じることは位相空間の一般論より従う. 
  fibrationはRLPで定義されているので, retractで閉じることはリフトの一般論から示すことができる.
  cofibrationがretractで閉じることは簡単に示すことができる. 

  分解系を持つことは\cref{prop:sSetKan_has_fs}で, リフト性質を満たすことは\cref{prop:sSetKan_has_lift}で既に示した. 
\end{proof}

\subsection{Joyalモデル構造}

Joyalモデル構造は, ファイブラントが擬圏(quasi-category)であり, weak equivalenceが弱圏同値(weak categorical equivalence)であるようなモデル構造である. 

\begin{definition}[弱圏同値]
  $f : A \to B$を単体的集合の射とする. 
  任意の擬圏$X$に対して, $f$の合成から定まる射
  \begin{align*}
    \tau_0(f,X) : \tau_0(B,X) \to \tau_0(A,X)
  \end{align*}
  が集合の同型射のとき, $f$を弱圏同値(weak categorical equivalence)という.
\end{definition}

\begin{definition}[擬ファイブレーション]
  任意の弱圏同値と単体的集合のmono射に対してRLPを持つ単体的集合の射を擬ファイブレーション(quasi fibration)という.
\end{definition}

\begin{definition}[Joyalモデル構造]
  $\sSet$には次のモデル構造が存在する. 
  これを$\sSet$上のJoyalモデル構造
  \footnote{
    擬圏のモデル(model structure for quasi-categories)と呼ばれることもある. 
  }
  といい, $\sSetJoyal$と表す. 
  \begin{itemize}
    \item weak equivalenceは弱圏同値
    \item fibrationは擬ファイブレーション
    \footnote{
      $\Catnat$におけるfibraitonである擬ファイブレーションと区別するために, これをgeneral quasi-fibrationと呼ぶこともある. 
    }
    \item cofibrationは単体的集合のmono射    
  \end{itemize}
\end{definition}

\begin{remark}
  $\sSetJoyal$において, 任意の擬圏はファイブラントであり, 任意の対象(単体的集合)はコファイブラントである.
\end{remark}

\begin{remark} \label{rem:sSetJoyal_is_cof_gen}
  $\sSetJoyal$は 
  \begin{align*}
    I := \{\partial \Delta[n] \hookrightarrow \Delta[n] ~|~ n \geq 0\}
  \end{align*}
  をgenerating cofibrationの集合とするコファイブラント生成なモデル圏である. 
  $\sSetJoyal$におけるgenerating trivial cofibrationの集合の明示的な表示は知られていない. 
\end{remark}


\begin{theorem}
  $\sSetJoyal$は左properである.
\end{theorem}

\begin{proof}
  $\sSetJoyal$において任意の対象がコファイブラントであることと, \cref{prop:every_object_is_cofibrant_means_left_proper}より従う. 
\end{proof}

\begin{theorem}
  $\sSetJoyal$は組み合わせ論的モデル圏である. 
\end{theorem}

\begin{proof}
  \cref{rem:sSetJoyal_is_cof_gen}より, $\sSetJoyal$はコファイブラント生成なモデル圏である.
  任意の単体的集合は標準$n$単体$\Delta[n]$の余極限で表せるので, $\sSet$は局所表現可能圏である. 
  よって, $\sSetJoyal$は組み合わせ論的モデル圏である. 
\end{proof}

\begin{theorem}
  $\sSetJoyal$はCartesianモデル圏である. 
\end{theorem}

% $\sSet$上にJoyalモデル構造が存在することを示すために, いくつか準備をする.
% % 証明は\cite{Joy08} section 6.2に従う. 

% まず, $\sSetJoyal$がリフト性質を満たすことを示す. (途中)

% \begin{lemma}
  
% \end{lemma}

% \begin{proof}
%   fibrationに対してLLPを持つ射がweak equivalenceかつcofibrationであることと示す. 
%   $u : A \to B$をfibrationに対してLLPを持つ射とする. 

%   まず, $u$がcofibrationであることを示す. 
%   $X$を擬圏, $i : A \to X$を内緩射とする. 
%   次の図式において, 射$X \to 0$はfibrationなのでリフト$d : B \to X$を持つ. 
%   % https://q.uiver.app/#q=WzAsNCxbMCwxLCJCIl0sWzAsMCwiQSJdLFsxLDAsIlgiXSxbMSwxLCIwIl0sWzEsMCwidSIsMl0sWzEsMiwiaSJdLFsyLDNdLFswLDNdLFswLDIsImQiLDIseyJzdHlsZSI6eyJib2R5Ijp7Im5hbWUiOiJkYXNoZWQifX19XV0=
%   \[\begin{tikzcd}
%     A & X \\
%     B & 0
%     \arrow["u"', from=1-1, to=2-1]
%     \arrow["i", from=1-1, to=1-2]
%     \arrow[from=1-2, to=2-2]
%     \arrow[from=2-1, to=2-2]
%     \arrow["d"', dashed, from=2-1, to=1-2]
%   \end{tikzcd}\]
%   $i(=du)$はmono射なので$u$はmono射, つまりcofibrationである.

%   次に, $u$がweak equivalenceであることを示す. 

% \end{proof}

\subsection{$\sSetJoyal$と$\sSetKan$のQuillen随伴1}

$\sSet$上のKan-Quillenモデル構造とJoyal構造におけるweak equivalenceなどを区別するために, 次のような呼び方を用いる. 

\begin{notation}
  $\sSetKan$, $\sSetJoyal$におけるweak equivalenceをそれぞれKan weak equivalence, Joyal weak equivalenceと呼ぶ.
  fibrationに対しても同様である. 
  cofibraionはともに単体的集合のmono射なので特に区別しない. 
\end{notation}

恒等関手による$\sSetJoyal$と$\sSetKan$のQuillen随伴を示す. (\cref{prop:quillen_adj_sSetJoyal_sSetKan})

\begin{lemma}[\cite{Rie08} Proposition 5.8] \label{prop:tau0X_equal_pi0X}
  $X$をKan複体とする. 
  このとき, $\tau_0(X)=\pi_0(X)$である. 
\end{lemma}

\begin{proposition} \label{prop:Joyal_weq_is_Kan_weq}
  任意のJoyal weak equivalenceはKan weak equivalenceである. 
\end{proposition}

\begin{proof}
  $f : A \to B$をJoyal weak equivalenceとする. 
  つまり, 任意のKan複体$X$に対して, $\pi_0(f,X) : \pi_0(B,X) \to \pi_0(A,X)$が集合の同型射であるとする. 
  \cref{prop:tau0X_equal_pi0X}より, $\tau_0(f,X)$も集合の同型射である. 
  よって, $f$はKan weak equivalenceである. 
\end{proof}

\begin{corollary} \label{prop:sSetKan_is_left_Bousfield_localization_of_sSetJoyal} % [\cite{Rie08} Proposition 5.9]
  $\sSetKan$は$\sSetJoyal$のBousfield局所化である.
\end{corollary}

\begin{proof}
  $\sSetKan$と$\sSetJoyal$におけるcofibrationはともに単体的集合のmono射である. 
  \cref{prop:Joyal_weq_is_Kan_weq}より, $\sSetJoyal$におけるweak equivalenceは$\sSetKan$におけるweak equivalenceである. 
\end{proof}

\begin{corollary} \label{prop.kan_fib_is_joyal_fib}
  任意のKan fibraionはJoyal fibraionである. 
\end{corollary}

\begin{corollary} \label{prop:quillen_adj_sSetJoyal_sSetKan}
  $\Id : \sSetJoyal \to \sSetKan$と$\Id : \sSetKan \to \sSetJoyal$は, $\sSetKan$と$\sSetJoyal$のQuillen随伴を定める. 
  % \begin{align*}
  %   \Id : \sSetJoyal \rightleftarrows \sSetKan : \Id
  % \end{align*}
  % https://q.uiver.app/#q=WzAsMixbMCwwLCJcXHNTZXRKb3lhbCJdLFsyLDAsIlxcc1NldEthbiJdLFswLDEsIlxcSWQiLDAseyJvZmZzZXQiOi0yLCJjdXJ2ZSI6LTF9XSxbMSwwLCJcXElkIiwwLHsib2Zmc2V0IjotMiwiY3VydmUiOi0xfV0sWzIsMywiIiwyLHsibGV2ZWwiOjEsInN0eWxlIjp7Im5hbWUiOiJhZGp1bmN0aW9uIn19XV0=
  \[\begin{tikzcd}
    \sSetJoyal && \sSetKan
    \arrow[""{name=0, anchor=center, inner sep=0}, "\Id", shift left, curve={height=-6pt}, from=1-1, to=1-3]
    \arrow[""{name=1, anchor=center, inner sep=0}, "\Id", shift left, curve={height=-6pt}, from=1-3, to=1-1]
    \arrow["\dashv"{anchor=center, rotate=-90}, draw=none, from=0, to=1]
  \end{tikzcd}\]
\end{corollary}

\begin{proof}
  \cref{prop:sSetKan_is_left_Bousfield_localization_of_sSetJoyal}と\cref{prop:quillen_adj_Bousfield_localization}より従う. 
\end{proof}

\begin{remark}
  Quillen随伴$\Id : \sSetJoyal \rightleftarrows \sSetKan : \Id$はQuillen同値ではない. 
\end{remark}

\subsection{$\sSetKan$と$\sSetJoyal$のQuillen随伴2}

恒等関手ではない$\sSetKan$と$\sSetJoyal$のQuillen随伴を示す. (\cref{prop:quillen_adj_sSetKan_sSetJoyal})
% 証明は\cite{Joy08} Proposition 6.22に従う. 

\begin{remark}
  任意の$n \geq 0$に対して, 圏$[n]$で自由生成された亜群の脈体を$\Delta'[n]$と表すと, 構成$[n] \mapsto \Delta'[n]$は関手$k : \Delta \to \sSet$を定める.
\end{remark}

\begin{definition}
  $X$を単体的集合とする. 
  このとき, 単体的集合$k^!(X)$を任意の$n \geq 0$に対して次のように定義する. 
  \begin{align*}
    k^!(X)_n := \Hom_\sSet(\Delta'[n],X)
  \end{align*}
\end{definition}

\begin{remark}
  構成$X \mapsto k^!(X)$は関手$k^! : \sSet \to \sSet$を定める. 
\end{remark}

\begin{remark}
  包含$\Delta[n] \hookrightarrow \Delta'[n]$は任意の$n \geq 0$に対して, $k^!(X)_n \to X_n$を定める. 
  これから, 単体的集合の射$\beta_X : k^!(X) \to X$が定まる. 
  この構成は自然変換$\beta : k^! \to \Id$を定める.
\end{remark}

\begin{remark}
  普遍随伴の一般論より, $k^! : \sSet \to \sSet$は左随伴$k_! : \sSet \to \sSet$を持つ. 
  また, 自然変換$\alpha : \Id \to k_!$を定める. 
  % https://q.uiver.app/#q=WzAsMyxbMCwyLCJcXERlbHRhIl0sWzAsMCwiXFxzU2V0Il0sWzIsMiwiXFxzU2V0Il0sWzAsMSwi44KIIl0sWzAsMiwiayIsMl0sWzIsMSwia18hIiwwLHsib2Zmc2V0IjotMSwiY3VydmUiOi0xfV0sWzEsMiwia18hIiwwLHsib2Zmc2V0IjotMSwiY3VydmUiOi0xfV0sWzYsNSwiIiwxLHsibGV2ZWwiOjEsInN0eWxlIjp7Im5hbWUiOiJhZGp1bmN0aW9uIn19XV0=
  \[\begin{tikzcd}
    \sSet \\
    \\
    \Delta && \sSet
    \arrow["{よ}", from=3-1, to=1-1]
    \arrow["k"', from=3-1, to=3-3]
    \arrow[""{name=0, anchor=center, inner sep=0}, "{k^!}", shift left, curve={height=-6pt}, from=3-3, to=1-1]
    \arrow[""{name=1, anchor=center, inner sep=0}, "{k_!}", shift left, curve={height=-6pt}, from=1-1, to=3-3]
    \arrow["\dashv"{anchor=center, rotate=-135}, draw=none, from=1, to=0]
  \end{tikzcd}\]
  また, 任意の$n \geq 0$に対して次が成立する.
  \begin{align*}
    k_!(\Delta[n]) = \Delta'[n]
  \end{align*}
\end{remark}

\begin{proposition}[\cite{Joy08} Proposition 6.22] \label{prop:quillen_adj_sSetKan_sSetJoyal}
  $k_! : \sSetKan \to \sSetJoyal$と$k^! : \sSetJoyal \to \sSetKan$は, $\sSetKan$と$\sSetJoyal$のQuillen随伴を定める. 
  % \begin{align*}
  %   k_! : \sSetKan \rightleftarrows \sSetJoyal : k^!
  % \end{align*}
  % https://q.uiver.app/#q=WzAsMixbMCwwLCJcXHNTZXRLYW4iXSxbMiwwLCJcXHNTZXRKb3lhbCJdLFswLDEsImtfISIsMCx7Im9mZnNldCI6LTEsImN1cnZlIjotMX1dLFsxLDAsImteISIsMCx7Im9mZnNldCI6LTEsImN1cnZlIjotMX1dLFsyLDMsIiIsMix7ImxldmVsIjoxLCJzdHlsZSI6eyJuYW1lIjoiYWRqdW5jdGlvbiJ9fV1d
  \[\begin{tikzcd}
    \sSetKan && \sSetJoyal
    \arrow[""{name=0, anchor=center, inner sep=0}, "{k_!}", shift left, curve={height=-6pt}, from=1-1, to=1-3]
    \arrow[""{name=1, anchor=center, inner sep=0}, "{k^!}", shift left, curve={height=-6pt}, from=1-3, to=1-1]
    \arrow["\dashv"{anchor=center, rotate=-90}, draw=none, from=0, to=1]
  \end{tikzcd}\]
\end{proposition}

\begin{proof}
  まず, 左随伴がcofibrationを保つことを示す.
  $k_!$は包含$\partial \Delta[1] \hookrightarrow \Delta[1]$を包含$\partial \Delta[1] \hookrightarrow \Delta'[1]$にうつす. 
  \cite{Joy08} B.0.17より, $k_!$は任意の単体的集合のmono射を保つ.
  つまり, $k_!$はcofibrationを保つ.

  次に, 左随伴がweak equivalenceを保つことを示す. 
  $k_!(\Delta[n])= \Delta'[n]$より, 単体的集合の射$\alpha_{\Delta[n]}$は包含$\Delta[n] \hookrightarrow \Delta'[n]$に一致する.
  この射はKan weak equivalenceである. 
  \cite{Joy08} B.0.18より, 任意の単体的集合$X$に対して$\alpha_X$はKan weak equivalenceである.
  自然変換から定まる次の図式を考える. 
  % https://q.uiver.app/#q=WzAsNCxbMCwwLCJYIl0sWzAsMSwiWSJdLFsxLDAsImtfIShYKSJdLFsxLDEsImtfIShZKSJdLFswLDEsImYiLDJdLFswLDIsIlxcYWxwaGFfWCJdLFsyLDMsImtfIShmKSJdLFsxLDMsIlxcYWxwaGFfWSIsMl1d
  \[\begin{tikzcd}
    X & {k_!(X)} \\
    Y & {k_!(Y)}
    \arrow["f"', from=1-1, to=2-1]
    \arrow["{\alpha_X}", from=1-1, to=1-2]
    \arrow["{k_!(f)}", from=1-2, to=2-2]
    \arrow["{\alpha_Y}"', from=2-1, to=2-2]
  \end{tikzcd}\]
  $f$がKan weak equivalenceのとき, 2-out-of-3より$k_!(f)$もKan weak equivalenceである. 
  よって, $k_!$はweak equivalenceを保つ. 
  \footnote{
    と\cite{Joy08} Proposition 6.22にあるが, この証明は不十分に思える. 
    $k_!(f)$はKan weak equivalenceではあるが, Joyal weak equivalenceであるとは限らないからである.
    Kan weak equivalenceならばJoyal weak equivalenceであるが, 逆は一般には成立しない. 
    Kan複体に対しては, 逆が成立する. 
    この行間について分かる方はご連絡ください. 
  }
\end{proof}

\newpage


\section{単体的空間の圏に入るモデル構造}

単体的空間の圏において, levelwiseで$\sSet$においてweak equivalenceであるような射とするようなモデル構造が考えられる. 
このとき, fibrationやcofibrationもlevelwiseで定義したいが, リフト条件を考えるとこれは不可能であることが分かる. 
よって, weak equivalenceとfibrationがlevelwiseで定義されるモデル構造と, weak equivalenceとcofibrationがlevelswiseで定義されるモデル構造が考えられる.
前者を射影的モデル構造, 後者を入射的モデル構造という. (射影的/入射的モデル構造はより一般に定義される.)

Reedyモデル構造の一般論から, 単体的集合の圏上のKan-Quillenモデル構造, Joyalモデル構造から定まるReedyモデル構造がそれぞれ存在する. 
つまり, 前者は各列がKan weak equivalenceであるような単体的空間上のモデル構造である. 
後者は各行がJoyal weak equivalenceであるような単体的空間上のモデル構造である. 

更に, 前者のReedyモデル構造のBousfield局所化により, Segal空間モデル構造とRezkモデル構造が得られる.
また, Rezkモデル構造はJoyalモデル構造から定まるReedyモデル構造のBousfield局所化でもある. 

% 実は,  単体的空間の圏における入射的モデル構造とReedyモデル構造は一致する. (\cref{prop:inj_and_Reedy_is_equal})

\subsection{射影的モデル構造}

関手圏上の射影的モデル構造の一般論から, $\sSpace$上の射影的モデル構造が得られる. 

\begin{definition}[射影的モデル構造]
  $\sSpace$には次のモデル構造が存在する. 
  これを$\sSpace$上の射影的モデル構造といい, $\sSpaceproj$と表す.
  \begin{itemize}
    \item weak equivalenceは各次数でKan weak equivalenceである単体的空間の射
    \item fibrationは各次数でKan fibrationである単体的空間の射
  \end{itemize} 
\end{definition}

\begin{remark}
  $\sSpaceproj$は
  \begin{align*}
    &I := \{\partial \Delta[m] \times \Delta[n]^t \hookrightarrow \Delta[m] \times \Delta[n]^t ~|~ m,n \geq 0\}, \\
    &J := \{\Lambda[m,k] \times \Delta[n]^t \hookrightarrow \Delta[m] \times \Delta[n]^t ~|~ m \geq 1, 0 \leq k \leq m, n \geq 0\}
  \end{align*}
  をそれぞれgenerating cofibration, generating trivial cofibrationの集合とするコファイブラント生成なモデル圏である. 
\end{remark}

\begin{proof}
  $\sSpace$上の射影的モデル構造において, cofibrationはtrivial fibrationに対してLLPを持つ射として定義される. 
  射影的モデル構造におけるtrivial fibrationは各次数でtrivial Kan fibrationとなる射である. 
  つまり, 単体的空間の射$X \to Y$がtrivial fibrationであることは, 任意の$n \geq 0$に対して単体的集合の射$X_n \to Y_n$がtrivial Kan fibrationであることと同値である. 
  つまり, 任意の$m,n \geq 0$に対して, 次の四角がリフトを持つような射である.
  % https://q.uiver.app/#q=WzAsNCxbMCwwLCJcXHBhcnRpYWwgXFxEZWx0YV5tIl0sWzEsMCwiWF9uIl0sWzEsMSwiWV9uIl0sWzAsMSwiXFxEZWx0YV5tIl0sWzAsMV0sWzEsMl0sWzMsMl0sWzAsMywiIiwwLHsic3R5bGUiOnsidGFpbCI6eyJuYW1lIjoiaG9vayIsInNpZGUiOiJ0b3AifX19XSxbMywxLCIiLDEseyJzdHlsZSI6eyJib2R5Ijp7Im5hbWUiOiJkYXNoZWQifX19XV0=
  \[\begin{tikzcd}
    {\partial \Delta^m} & {X_n} \\
    {\Delta^m} & {Y_n}
    \arrow[from=1-1, to=1-2]
    \arrow[from=1-2, to=2-2]
    \arrow[from=2-1, to=2-2]
    \arrow[hook, from=1-1, to=2-1]
    \arrow[dashed, from=2-1, to=1-2]
  \end{tikzcd}\]
  自然な同型$X_n \cong \Map_\sSpace(\Delta[n]^t,X)$より, 次の図式のリフト性と同値である. 
  % https://q.uiver.app/#q=WzAsNCxbMCwwLCJcXHBhcnRpYWwgXFxEZWx0YV5tIl0sWzEsMCwiXFxNYXAoXFxEZWx0YV5uLFgpIl0sWzEsMSwiXFxNYXAoXFxEZWx0YV5uLFkpIl0sWzAsMSwiXFxEZWx0YV5tIl0sWzAsMV0sWzEsMl0sWzMsMl0sWzAsM10sWzMsMSwiIiwxLHsic3R5bGUiOnsiYm9keSI6eyJuYW1lIjoiZGFzaGVkIn19fV1d
  \[\begin{tikzcd}
    {\partial \Delta[m]} & {\Map(\Delta[n]^t,X)} \\
    {\Delta[m]} & {\Map(\Delta[n]^t,Y)}
    \arrow[from=1-1, to=1-2]
    \arrow[from=1-2, to=2-2]
    \arrow[from=2-1, to=2-2]
    \arrow[hook, from=1-1, to=2-1]
    \arrow[dashed, from=2-1, to=1-2]
  \end{tikzcd}\]
  直積と写像空間の随伴性より, 次の図式のリフト性と同値である. 
  % https://q.uiver.app/#q=WzAsNCxbMCwwLCJcXHBhcnRpYWwgXFxEZWx0YV5tIFxcdGltZXMgXFxEZWx0YV5uIl0sWzEsMCwiWCJdLFsxLDEsIlkiXSxbMCwxLCJcXERlbHRhXm0gXFx0aW1lcyBcXERlbHRhXm4iXSxbMCwxXSxbMSwyXSxbMywyXSxbMCwzXSxbMywxLCIiLDEseyJzdHlsZSI6eyJib2R5Ijp7Im5hbWUiOiJkYXNoZWQifX19XV0=
  \[\begin{tikzcd}
    {\partial \Delta[m] \times \Delta[n]^t} & X \\
    {\Delta[m] \times \Delta[n]^t} & Y
    \arrow[from=1-1, to=1-2]
    \arrow[from=1-2, to=2-2]
    \arrow[from=2-1, to=2-2]
    \arrow[hook, from=1-1, to=2-1]
    \arrow[dashed, from=2-1, to=1-2]
  \end{tikzcd}\]
  よって, 
  \begin{align*}
    \{\partial \Delta[m] \times \Delta[n]^t \hookrightarrow \Delta[m] \times \Delta[n]^t ~|~ m,n \geq 0\}
  \end{align*}
  をgenerating cofibrationの集合とすればよい. 
  generating trivial cofibrationについても同様である. 
\end{proof}

% \begin{remark}
%   $\sSpaceproj$はproperな単体的モデル圏である.
% \end{remark}

\subsection{入射的モデル構造}

関手圏上の入射的モデル構造の一般論から, $\sSpace$上の入射的モデル構造が得られる. 

\begin{definition}[単体的空間のmono射]
  $f : X \to Y$を単体的空間の射とする.
  任意の$n \geq 0$に対して, $f_{n,-} : X_{x,-} \to Y_{n,-}$が単体的集合のmono射のとき, $f$を単体的空間のmono射(mono morphism)という. 
\end{definition}

\begin{definition}[入射的モデル構造]
  $\sSpace$には次のモデル構造が存在する. 
  これを$\sSpace$上の入射的モデル構造といい, $\sSpaceinj$と表す. 
  \begin{itemize}
    \item weak equivalenceは各次数でKan weak equivalenceである単体的空間の射
    \item cofibrationは単体的空間のmono射
  \end{itemize}
\end{definition}

\begin{remark}
  $\sSpaceinj$はコファイブラント生成なモデル圏である.
\end{remark}

\subsection{垂直Reedyモデル構造}

weak equivalenceが各列でKan weak equivalenceであるような垂直Reedyモデル構造が存在する. 
つまり, 単体的空間の圏上の垂直Reedyモデル構造は単体的集合の圏上のKan-Quillenモデル構造から誘導されるようなモデル構造である. 

\begin{definition}[各列Kan weak equivalence]
  $f : X \to Y$を単体的空間の射とする.
  任意の$m \geq 0$に対して, $\Delta[m] \backslash f = f_m : X_{m,-} \to Y_{m,-}$がKan weak equivalenceのとき, $f$を各列Kan weak equivalence (column-wise weak homotopy equivalence)という.
\end{definition}

\begin{definition}[垂直ファイブレーション]
  任意の$m,n \geq 0$において, $\delta_m \Box' h^k_n$に対してRLPを持つ射を単体的空間の垂直ファイブレーション(v-fibration)という.
\end{definition}

\begin{definition}[垂直Reedyモデル構造]
  $\sSpace$には次のモデル構造が存在する.
  これを$\sSpace$上の垂直Reedyモデル構造といい, $\sSpace_v$と表す. 
  \begin{itemize}
    \item weak equivalenceは各列Kan weak equivalence
    \item fibraionは垂直ファイブレーション
    \item cofibraitionは単体的空間のmono射
  \end{itemize}
\end{definition}

$\sSpace_v$は$\sSetKan$から定まるReedyモデル構造である.

% \begin{proof} % {\cite{JT07} Proposition 2.10}
%   $\sSpace_v$が$\sSetKan$から定まるReedyモデル構造であることを示す. 

%   まず, $\sSpace_v$におけるcofibrationは単体的空間のmono射, つまり各次数でKan cofibrationである. 

%   次に, $\sSpace_v$におけるweak equivalenceは各列Kan weak equivalence, つまり各次数でKan weak equivalenceである. 

%   後は, $\sSpace_v$におけるfibrationが各次数でKan fibrationであることを示せばよい.
%   $f$を垂直ファイブレーション($\sSpace_v$におけるfibration)とする. 
%   \cref{prop:v-fibration}より, これは任意の$m \geq 0$に対して$\angle{\delta_m \backslash f}$はKan fibrationであることと同値である. 
%   \cref{prop:vfib_is_angle_Kanfib}より, 各次数で$f$はKan fibraionのとき, $\angle{\delta_m \backslash f}$はKan fibrationである.
% \end{proof}

\subsection{水平Reedyモデル構造}

weak equivalenceが各行でJoyal weak equivalenceであるような水平Reedyモデル構造が存在する. 
つまり, 単体的空間の圏上の水平Reedyモデル構造は単体的集合の圏上のJoyalモデル構造から誘導されるようなモデル構造である. 

\begin{definition}[各行Joyal weak equivalence]
  $f : X \to Y$を単体的空間の射とする. 
  任意の$n \geq 0$に対して, $\Delta[n] \backslash f = f_n : X_{-,n} \to Y_{-,n}$がJoyal weak equivalenceのとき, $f$を各行Joyal weak equivalence (row-wise Joyal weak equivalence)という. 
\end{definition}

\begin{definition}[水平ファイブレーション]
  $f : X \to Y$を単体的空間の射とする. 
  任意の$n \geq 0$に対して, $\angle{f / \delta_n}$がJoyal fibrationのとき, $f$を水平ファイブレーション(horizontal fibration)という. 
\end{definition}

\begin{definition}[水平Reedyモデル構造]
  $\sSpace$には次のモデル構造が存在する.
  これを$\sSpace$上の垂直Reedyモデル構造といい, $\sSpace_h$と表す. 
  \begin{itemize}
    \item weak equivalenceは各行Joyal weak equivalence
    \item fibraionは水平ファイブレーション
    \item cofibraitionは単体的空間のmono射
  \end{itemize}
\end{definition}

$\sSpace_v$は$\sSetJoyal$から定まるReedyモデル構造である.

% \subsection{Reedyモデル構造}

% Reedyモデル構造の一般論から, $\sSpace$上のReedyモデル構造が得られる. 

% \begin{definition}[Reedyモデル構造]
%   $\sSpace$には次のモデル構造が存在する. 
%   これを$\sSpace$上のReedyモデル構造といい, $\sSpaceReedy$と表す. 
%   \begin{itemize}
%     \item weak equivalenceは各次数でKan weak equivalenceである単体的空間の射
%     \item fibrationは次の射
%     \begin{align*}
%       \Map_\sSpace(\Delta[n]^t,X) \to \Map_\sSpace(\partial \Delta[n]^t,X) \times_{\Map_\sSpace(\partial \Delta[n]^t,Y)} \Map_\sSpace(\Delta[n]^t,Y)
%     \end{align*}
%     がKan fibrationである単体的空間の射$X \to Y$
%   \end{itemize}
% \end{definition}

% \begin{remark}
%   $\sSpaceReedy$において, 任意のSegal空間はファイブラントであり, 任意の対象(単体的空間)はコファイブラントである.  
% \end{remark}

% \begin{remark}
%   $\sSpaceReedy$は 
%   \begin{align*}
%     I &:= \{\partial \Delta[m] \times \Delta[n]^t \cup \Delta[m] \times \partial \Delta[n]^t \to \Delta[m] \times \Delta[n]^t ~|~ m,n \geq 0\} \\
%     J &:= \{\Lambda[m,k] \times \Delta[n]^t \cup \Delta[m] \times \partial \Delta[n]^t \to \Delta[m] \times \Delta[n]^t ~|~ m \geq 1, 0 \leq k \leq m, n \geq 0\}
%   \end{align*}
%   をそれぞれgenerating cofibration, generating trivial cofibrationの集合とするコファイブラント生成なモデル圏である. 
% \end{remark}

% \begin{proof}
%   $\sSpace$上のReedyモデル構造において, trivial cofibrationはfibrationに対してLLPを持つ射として定義される. 

%   $n=0$のときを考える.
%   fibrationの定義より, 任意の$m \geq 1, 0 \leq k \leq m$に対して, 次の四角はリフトを持つ. 
%   % https://q.uiver.app/#q=WzAsNCxbMCwwLCJcXExhbWJkYV5tX2siXSxbMSwwLCJYXzAiXSxbMSwxLCJZXzAiXSxbMCwxLCJcXERlbHRhXm0iXSxbMCwxXSxbMSwyXSxbMywyXSxbMCwzXSxbMywxLCIiLDEseyJzdHlsZSI6eyJib2R5Ijp7Im5hbWUiOiJkYXNoZWQifX19XV0=
%   \[\begin{tikzcd}
%     {\Lambda[m,k]} & {X} \\
%     {\Delta[m]} & {Y}
%     \arrow[from=1-1, to=1-2]
%     \arrow[from=1-2, to=2-2]
%     \arrow[from=2-1, to=2-2]
%     \arrow[from=1-1, to=2-1]
%     \arrow[dashed, from=2-1, to=1-2]
%   \end{tikzcd}\]
%   自然な同型$X_0 \cong \Map_\sSpace(\Delta[0],X)$と直積と写像空間の随伴性より, 次の図式のリフト性と同値である. 
%   % https://q.uiver.app/#q=WzAsNCxbMCwwLCJcXExhbWJkYV5tX2sgXFx0aW1lcyBcXERlbHRhXjAiXSxbMSwwLCJYXzAiXSxbMSwxLCJZXzAiXSxbMCwxLCJcXERlbHRhXm0gXFx0aW1lcyBcXERlbHRhXjAiXSxbMCwxXSxbMSwyXSxbMywyXSxbMCwzXSxbMywxLCIiLDEseyJzdHlsZSI6eyJib2R5Ijp7Im5hbWUiOiJkYXNoZWQifX19XV0=
%   \[\begin{tikzcd}
%     {\Lambda[m,k] \times \Delta[0]^t} & {X_0} \\
%     {\Delta[m] \times \Delta[0]^t} & {Y_0}
%     \arrow[from=1-1, to=1-2]
%     \arrow[from=1-2, to=2-2]
%     \arrow[from=2-1, to=2-2]
%     \arrow[from=1-1, to=2-1]
%     \arrow[dashed, from=2-1, to=1-2]
%   \end{tikzcd}\]
%   よって, $n=0$のとき, generating trivial cofibrationの集合は次のように表せる. 
%   \begin{align*}
%     J := \{\Lambda[m,k] \times \Delta[0]^t \to \Delta[m] \times \Delta[0]^t ~|~ m \geq 1, 0 \leq k \leq m\}
%   \end{align*}

%   $n \geq 1$のときを考える.
%   $P := \Map_\sSpace(\partial \Delta[n]^t,X) \times_{\Map_\sSpace(\partial \Delta[n]^t,Y)} \Map_\sSpace(\Delta[n]^t,Y)$とする. 
%   fibrationの定義より, 任意の$m \geq 1, 0 \leq k \leq m, n \geq 1$に対して, 次の四角はリフトを持つ.
%   % https://q.uiver.app/#q=WzAsNyxbMCwwLCJcXExhbWJkYV5tX2siXSxbMSwwLCJYX24iXSxbMSwyLCJZX24iXSxbMCwyLCJcXERlbHRhXm0iXSxbMSwxLCJQIl0sWzIsMSwiXFxNYXAoXFxwYXJ0aWFsIFxcRGVsdGFebixYKSJdLFsyLDIsIlxcTWFwKFxccGFydGlhbCBcXERlbHRhXm4sWSkiXSxbMCwxXSxbMywyXSxbMCwzXSxbMywxLCIiLDEseyJzdHlsZSI6eyJib2R5Ijp7Im5hbWUiOiJkYXNoZWQifX19XSxbMSw0XSxbNCwyXSxbNCw1XSxbNSw2XSxbMiw2XSxbNCw2LCIiLDAseyJzdHlsZSI6eyJuYW1lIjoiY29ybmVyIn19XV0=
%   \[\begin{tikzcd}
%     {\Lambda[m,k]} & {X_n} \\
%     & P & {\Map_\sSpace(\partial \Delta[n]^t,X)} \\
%     {\Delta[m]} & {Y_n} & {\Map_\sSpace(\partial \Delta[n]^t,Y)}
%     \arrow[from=1-1, to=1-2]
%     \arrow[from=3-1, to=3-2]
%     \arrow[from=1-1, to=3-1]
%     \arrow[dashed, from=3-1, to=1-2]
%     \arrow[from=1-2, to=2-2]
%     \arrow[from=2-2, to=3-2]
%     \arrow[from=2-2, to=2-3]
%     \arrow[from=2-3, to=3-3]
%     \arrow[from=3-2, to=3-3]
%     \arrow["\lrcorner"{anchor=center, pos=0.125}, draw=none, from=2-2, to=3-3]
%   \end{tikzcd}\]
%   直積と写像空間の随伴性より, 左の四角のリフト性は次の図式のリフト性と同値である. 
%   % https://q.uiver.app/#q=WzAsNCxbMCwwLCJcXExhbWJkYV5tX2sgXFx0aW1lcyBcXERlbHRhXm4iXSxbMSwwLCJYIl0sWzEsMSwiWSJdLFswLDEsIlxcRGVsdGFebSBcXHRpbWVzIFxcRGVsdGFebiJdLFswLDFdLFsxLDJdLFszLDJdLFswLDNdLFszLDEsIiIsMSx7InN0eWxlIjp7ImJvZHkiOnsibmFtZSI6ImRhc2hlZCJ9fX1dXQ==
%   \[\begin{tikzcd}
%     {\Lambda[m,k] \times \Delta[n]^t} & X \\
%     {\Delta[m] \times \Delta[n]^t} & Y
%     \arrow[from=1-1, to=1-2]
%     \arrow[from=1-2, to=2-2]
%     \arrow[from=2-1, to=2-2]
%     \arrow[from=1-1, to=2-1]
%     \arrow[dashed, from=2-1, to=1-2]
%   \end{tikzcd}\]
%   pullbackの普遍性より, 右の四角についても考える必要がある. 
%   まず, $\Delta[m] \to Y_n$は$\Delta[m] \times \Delta[n]^t \to Y$と対応する. 
%   次に, $\Delta[m] \to \Map(\partial \Delta[n]^t,Y)$は$\Delta[m] \times \partial \Delta[n]^t \to X$と対応する.
%   よって, 全体の四角のリフト性は次の図式のリフト性と同値である. 
%   % https://q.uiver.app/#q=WzAsNCxbMCwwLCJcXExhbWJkYV5tX2sgXFx0aW1lcyBcXERlbHRhXm4gXFxjdXAgXFxEZWx0YV5tIFxcdGltZXMgXFxwYXJ0aWFsIFxcRGVsdGFebiJdLFsxLDAsIlgiXSxbMSwxLCJZIl0sWzAsMSwiXFxEZWx0YV5tIFxcdGltZXMgXFxEZWx0YV5uIl0sWzAsMV0sWzEsMl0sWzAsM10sWzMsMl0sWzMsMSwiIiwxLHsic3R5bGUiOnsiYm9keSI6eyJuYW1lIjoiZGFzaGVkIn19fV1d
%   \[\begin{tikzcd}
%     {\Lambda[m,k] \times \Delta[n]^t \cup \Delta[m] \times \partial \Delta[n]^t} & X \\
%     {\Delta[m] \times \Delta[n]^t} & Y
%     \arrow[from=1-1, to=1-2]
%     \arrow[from=1-2, to=2-2]
%     \arrow[from=1-1, to=2-1]
%     \arrow[from=2-1, to=2-2]
%     \arrow[dashed, from=2-1, to=1-2]
%   \end{tikzcd}\]
%   よって, generating trivial cofibrationの集合を
%   \begin{align*}
%     J := \{\Lambda[m,k] \times \Delta[n]^t \cup \Delta[m] \times \partial \Delta[n]^t \to \Delta[m] \times \Delta[n]^t ~|~ m \geq 1, 0 \leq k \leq m, n \geq 0\}
%   \end{align*}
%   とすればよい. (これは$n=0$のときも含んでいる.)
%   同様に, generating cofibrationの集合を
%   \begin{align*}
%     I := \{\partial \Delta[m] \times \Delta[n]^t \cup \Delta[m] \times \partial \Delta[n]^t \to \Delta[m] \times \Delta[n]^t ~|~ m,n \geq 0\}
%   \end{align*}
%   とすればよい. 
% \end{proof}

% % \begin{remark}
% %   $\sSpaceReedy$はproperかつCartesianな単体的モデル圏である. 
% % \end{remark}

% $\sSpace$上のReedyモデル構造の最も重要な性質として, 次の定理がある. 

% \begin{theorem}[\cite{Hir03} Prop 15.8.7] \label{prop:inj_and_Reedy_is_equal}
%   $\sSpace$上の入射的モデル構造とReedyモデル構造は一致する. 
%   特に, Reedyモデル構造におけるcofibrationは単体的空間のmono射(入射的モデル構造におけるcofibration)である. 
% \end{theorem}

% \begin{notation}
%   $\sSpace$上のReedyモデル構造におけるweak equivalence, fibration, cofibrationをそれぞれvertical weak equivalence, Reedy fibration, Reedy cofibrationと呼ぶこともある. 
% \end{notation}

\subsection{Segal空間モデル構造}

垂直Reedyモデル構造のBousfield局所化により, Segal空間モデル構造が得られる. 

\begin{definition}[Segal weak equivalence]
  $f : X \to Y$を単体的空間の射とする. 
  任意のSegal空間$W$に対して, 
    \begin{align*}
      \Map_\sSpace(f,W) : \Map_\sSpace(X,W) \to \Map_\sSpace(Y,W)
    \end{align*}
    が各次数でKan weak equivalenceのとき, $f$をSegal weak equivalenceという. 
\end{definition}

\begin{definition}[Segal空間モデル構造]
  $\sSpace$には次のモデル構造が存在する. 
  これを$\sSpace$上のSegal空間モデル構造といい, $\sSpaceSegal$と表す. 
  \begin{itemize}
    \item weak equivalenceはSegal weak equivalence
    \item cofibrationは単体的空間のmono射
  \end{itemize}
\end{definition}

\begin{remark}
  $\sSpaceSegal$において, 任意のSegal空間はファイブラントであり, 任意の対象(単体的空間)はコファイブラントである. 
\end{remark}

\begin{remark}
  $\sSpaceSegal$はコファイブラント生成なモデル圏である. 
\end{remark}

Segal空間モデル構造は垂直Reedyモデル構造のBousfield局所化である. 

\begin{proposition} \label{prop:sSpaceCSS_is_left_Bousfield_localization_of_sSpacev}
  $\sSpaceCSS$は$\sSpace_v$のBousfield局所化である.
\end{proposition}

\begin{proof}
  \cref{rem:InX_is}とSegal weak equivalenceの定義より従う. 
\end{proof}

% \begin{remark}
%   $\sSpaceSegal$は左properかつCartesianな単体的かつ組み合わせ論的モデル圏である. 
% \end{remark}

\subsection{Rezkモデル構造}

Segal空間構造モデル構造のBousfield局所化により, Rezkモデル構造が得られる. 

\begin{definition}[Rezk weak equivalence]
  $f : X \to Y$を単体的空間の射とする. 
  任意の完備Segal空間$W$に対して, 
    \begin{align*}
      \Map_\sSpace(f,W) : \Map_\sSpace(Y,W) \to \Map_\sSpace(X,W)
    \end{align*}
    が各次数でKan weak equivalenceのとき, $f$をRezk weak equivalenceという. 
\end{definition}

\begin{definition}[Rezkモデル構造]
  $\sSpace$には次のモデル構造が存在する. 
  これを$\sSpace$上のRezkモデル構造といい, $\sSpaceCSS$と表す. 
  \begin{itemize}
    \item weak equivalenceはRezk weak equivalence
    \item cofibrationは単体的空間のmono射
  \end{itemize}
\end{definition}

\begin{remark}
  $\sSpaceCSS$において, 任意の完備Segal空間はファイブラントであり, 任意の対象(単体的空間)はコファイブラントである. 
\end{remark}

\begin{remark}
  $\sSpaceCSS$はコファイブラント生成なモデル圏である. 
\end{remark}

% \begin{remark}
%   $\sSpaceCSS$は左properかつCartesianな単体的かつ組み合わせ論的モデル圏である. 
% \end{remark}

Rezkモデル構造はSegal空間構造モデル構造のBousfield局所化により得られた. 
ここでは, Rezkモデル構造が水平Reedyモデル構造のBousfield局所化でもあることを見る. (\cref{prop:sSpaceCSS_is_left_Bousfield_localization_of_sSpaceh})

\begin{lemma} \label{prop:rezkfib_of_fibrant_is}
  $f : X \to Y$を$\sSpaceCSS$におけるファイブラントの間のRezk fibrationとする. 
  このとき, 任意の単体的集合のmono射$v : S \to T$に対して, 
  \begin{align*}
    \angle{f / v} : X/T \to Y/T \times_{Y/S} X/S
  \end{align*}
  はJoyal fibrationである.
\end{lemma}

\begin{lemma}
  $X$を単体的空間とする. 
  $X$が完備Segal空間であることと, 次を満たすことは同値である. 
  \begin{enumerate}
    \item 任意の$n \geq 0$に対して, $X / \delta_n$はJoyal fibrationである.
    \item $X$は圏的定値である.
  \end{enumerate}
\end{lemma}

\begin{proposition} \label{prop:sSpaceCSS_is_left_Bousfield_localization_of_sSpaceh}
  $\sSpaceCSS$は$\sSpace_h$のBousfield局所化である.
\end{proposition}

\begin{proof}
  $\sSpaceCSS$と$\sSpace_h$におけるcofibrationはともに単体的空間のmono射である. 
  $f : X \to Y$を$\sSpaceCSS$におけるファイブラントの間のRezk fibrationとする. 
  \cref{prop:rezkfib_of_fibrant_is}より, 任意の$n \geq 0$に対して$\angle{f/\delta_n}$はJoyal fibrationである. 
  つまり, 完備Segal空間の間のRezk fibrationは水平ファイブレーションである. 
  (途中)
\end{proof}

\newpage


% \section{Segal前圏の圏に入るモデル構造}

% 完備Segal空間の圏がモデル構造を持たないように, Segal空間の圏もモデル構造を持たない. 
% しかし, Segal前圏の圏はモデル構造を持つ. 

% 単体的空間の圏上には, weak equivalenceが各次数でweak equivalenceであり, cofibrationが各次数でcofibration (つまりmono射)であるようなモデル構造が存在する. 
% しかし, Segal前圏の圏上にこのようなモデル構造は存在しない. 
% 例えば, 単体的空間の射$\Delta[0] \sqcup \Delta[0] \to \Delta[0]$を考えると, この射はSegal前圏の圏において, mono射とweak equivalenceの合成で表すことができない. 

% このようにSegal前圏の圏は単体的空間の圏とかなり異なるが, 単体的空間の圏上のモデル構造から誘導されるSegal前圏上のモデル構造が存在する. 
% Segal前圏上のモデル構造は\cite{Ber07b}で証明された. 
% % ファイブラントがReedyファイブラントSegal圏であるようなモデル圏は$(\infty,1)$圏のなす$(\infty,1)$圏を表していると考えられる. 

% \subsection{$\sSpaceReedy$から誘導されるモデル構造}

% 簡約関手$(-)_r : \sSpace \to \PreSeCat$を用いて, 単体的空間やその射をSegal前圏やその射にうつすことができる. 
% 実際, $\sSpace$上のReedyモデル構造におけるgenerating (trivial) cofibrationを簡約関手でうつすことで, $\PreSeCat$上のモデル構造におけるgenerating (trivial) cofibrationが定義できないか考える. 
% $\sSpace$上のReedyモデル構造におけるcofibrationはmono射であったので, cofibrationがmono射であるような$\PreSeCat$上のモデル構造を考える. 

% $\sSpaceReedy$におけるcofibraionはmono射であり, $\sSpaceReedy$におけるgenerating cofibrationは次の形で表される. 
% \begin{align*}
%   \partial \Delta[m] \times \Delta[n]^t \cup \Delta[m] \times \partial \Delta[n]^t \to \Delta[m] \times \Delta[n]^t  ~~(m,n \geq 0)
% \end{align*}
% しかし, $\partial \Delta[0], \partial \Delta[1], \Delta[0]$を除いて, $\partial \Delta[m]$や$\Delta[m]$はSegal前圏ではない. 
% つまり一般には, このgenerating cofibrationは$\PreSeCat$における射ではない. 
% これらの簡約関手をとると, 次のような射を得る. 
% \begin{align*}
%   (\partial \Delta[m] \times \Delta[n]^t \cup \Delta[m] \times \partial \Delta[n]^t)_r \to (\Delta[m] \times \Delta[n]^t)_r  ~~(m,n \geq 0)
% \end{align*}
% $\PreSeCat$のモデル構造としてcofibrationがmono射であるようなものを考えたいが, mono射でないようなものが存在する. 
% 例えば, $n=0$のときである.  
% \begin{align*}
%   \partial \Delta[m]_r \to \Delta[m]_r 
% \end{align*}
% \begin{description}
%   \item[($m=0$)] 上の射は$\emptyset \to \Delta[0]$となるので, mono射である. 
%   \item[($m=1$)] $\partial \Delta[1]$とその簡約$\partial \Delta[1]_r$は一致し, 2点からなる離散単体的空間である.
%   しかし, $\Delta^1$とその簡約$\Delta[1]_r$は一致せず, $\Delta[1]_r$は 1点からなる離散単体的空間である.
%   よって, $\partial \Delta[1]_r \to \Delta[1]_r$はmono射ではない. 
%   \item[($m \geq 2$)] $\partial \Delta[m]_r$や$\Delta[m]_r$は連結な単体的集合なので, ともに1点からなる離散単体的空間である. 
%   よって, 上の射は同型射である. 
% \end{description}

% 以上より, ($\sSpaceReedy$から誘導される) $\PreSeCat$のモデル構造として, generating cofibrationは次のようにすればよい. 
% \begin{align*}
%   I_c := \{(\partial \Delta[m] \times \Delta[n]^t \cup \Delta[m] \times \partial \Delta[n]^t)_r \to (\Delta[m] \times \Delta[n]^t)_r  ~|~ m \geq 0 \text{かつ} n \geq 1, n=m=0\}
% \end{align*}

% \begin{proposition}
%   $X$をSegal前圏とする. 
%   このとき, あるSegal空間$L_cX$が存在して, $X \to L_cX$は$\sSpace$上のSegal空間モデル構造におけるweak equivalenceである. 
% \end{proposition}

% % \begin{proof}
% %   $\sSpace$上のSegal空間モデル構造$\sSpaceSegal$におけるファイブラント置換を考えたのちに, 簡約関手を用いて$\PreSeCat$において考える. 
% %   \cref{prop:local_obj_equals_RLP}
% % \end{proof}

% 単体的空間のDK同値と同様に, Segal空間のDK同値を定義する. 

% \begin{definition}[Segal空間のDK同値]
%   $X,Y$をSegal空間, $f : X \to Y$を単体的空間の射とする. 
%   $f$が次の条件を満たすとき, $f$をSegal空間のDK同値(Dwyer-Kan equivalence)という. 
%   \begin{itemize}
%     \item $X$の任意の対象$x,y$に対して, $\map_X(x,y) \to \map_Y(fx,fy)$は単体的集合のweak equivalenceである.
%     \item 誘導される関手$\Ho(f) : \Ho(X) \to \Ho(Y)$は圏同値である.
%   \end{itemize}
% \end{definition}

% \begin{definition}[Segal前圏のDK同値]
%   $X,Y$をSegal前圏, $f : X \to Y$を単体的空間の射とする. 
%   誘導されるSegal空間の射$L_cX \to L_cY$がSegal空間のDK同値であるとき, $f$をSegal前圏のDK同値(Dwyer-Kan equivalence)という. 
% \end{definition}

% \begin{definition}[Segal前圏モデル構造]
%   $\PreSeCat$には次のモデル構造が存在する. 
%   これを$\PreSeCat$上のSegal前圏モデル構造といい, $\PreSeCat_c$と表す. 
%   \begin{itemize}
%     \item weak equivalenceはSegal前圏のDK同値
%     \item cofibrationは各次元で$\sSet$におけるcofibrationである単体的空間の射
%   \end{itemize}
% \end{definition}

% $\PreSeCat$上にSegal前圏モデル構造が存在することを示すために, いくつか準備をする.

% \begin{notation}
%   $\PreSeCat_c$における次の条件を満たす同型射のクラスを$J_c := \{i : A \to B\}$と表す. 
%   \begin{itemize}
%     \item $i : A \to B$はweak equivalenceかつcofibration
%     \item 任意の$n \geq 0$に対して, $A_n$と$B_n$は加算個の単体を含む. 
%   \end{itemize}
% \end{notation}

% \begin{proposition}[\cite{Ber07b} Proposition 5.7]
%   $\PreSeCat_c$における任意のtrivial cofibrationは$J_c$に属する射に沿ったpushoutのdirected colimtで表せる.
% \end{proposition}

% $I_c$が実際にgenerating cofibrationであることを確かめる. 
% このことを示すために必要な補題を用意する. 

% \begin{lemma}
%   $f$をSegal前圏の射とする. 
%   $f$が$I_c$-fibrationのとき, $f$はDK同値である.
% \end{lemma}

% \begin{proposition}
%   $I_c$に対してRLPを持つ射は$\PreSeCat_c$におけるtrivial fibrationである. 
% \end{proposition}

% \subsection{$\sSpaceproj$から誘導されるモデル構造}

% 一方, 簡約をとってgenerating cofibrationを与える操作は$\sSpaceproj$に対してはうまくいかない. 
% $\sSpaceproj$のgenerating cofibrationは次の形で表される.
% \begin{align*}
%   \partial \Delta[m] \times \Delta[n]^t \to \Delta[m] \times \Delta[n]^t ~~ (m,n \geq 0)
% \end{align*}
% $m=1$のときを考える. 
% \begin{align*}
%   \partial \Delta[1] \times \Delta[n]^t \to \Delta[1] \times \Delta[n]^t ~~ (n \geq 0)
% \end{align*}
% $\Delta[1] \times \Delta[n]^t$の簡約$(\Delta[1] \times \Delta[n]^t)_r$は次数$0$に$n+1$点を持つSegal前圏である. 
% しかし, $\partial \Delta[1] \times \Delta[n]^t \to \Delta[1] \times \Delta[n]^t$の簡約$(\partial \Delta[1] \times \Delta[n]^t \to \Delta[1] \times \Delta[n]^t)_r$は次数$0$に$2(n+1)$点を持つSegal前圏である.
% つまり, 簡約をとった後の射はmono射ではない. 
% $\sSpaceReedy$におけるgenerating cofibrationで考えた簡約とは異なり, この射を($\sSpaceproj$から誘導される)$\PreSeCat$のモデル構造におけるcofibrationにこのような射を単に含まないことはできない. 
% $n>0$のとき, 次数$n$における非退化な$1$単体を生成するために必要であるからである.(らしい)

% この問題を解決するために, 次の単体的空間$P_{m,n}$と$Q_{m,n}$を考える. 
% \begin{description}
%   \item[($m=0,n \geq 0$)] $P_{0,n}$は空単体的空間
%   \item[($m \geq 1, n \geq 0$)] $P_{m,n}$は次のpushoutで定める. 
%    % https://q.uiver.app/#q=WzAsNCxbMCwwLCJcXHBhcnRpYWwgXFxEZWx0YVttXSBcXHRpbWVzIFxcRGVsdGFbbl1edF8wIl0sWzEsMCwiXFxwYXJ0aWFsIFxcRGVsdGFbbV0gXFx0aW1lcyBcXERlbHRhW25dXnQiXSxbMSwxLCJQX3ttLG59Il0sWzAsMSwiXFxEZWx0YVtuXV50XzAiXSxbMCwxLCIiLDAseyJzdHlsZSI6eyJ0YWlsIjp7Im5hbWUiOiJob29rIiwic2lkZSI6InRvcCJ9fX1dLFsxLDJdLFswLDNdLFszLDJdLFsyLDAsIiIsMSx7InN0eWxlIjp7Im5hbWUiOiJjb3JuZXIifX1dXQ==
%   \[\begin{tikzcd}
%     {\partial \Delta[m] \times \Delta[n]^t_0} & {\partial \Delta[m] \times \Delta[n]^t} \\
%     {\Delta[n]^t_0} & {P_{m,n}}
%     \arrow[hook, from=1-1, to=1-2]
%     \arrow[from=1-2, to=2-2]
%     \arrow[from=1-1, to=2-1]
%     \arrow[from=2-1, to=2-2]
%     \arrow["\lrcorner"{anchor=center, pos=0.125, rotate=180}, draw=none, from=2-2, to=1-1]
%   \end{tikzcd}\]
% \end{description}
% \begin{description}
%   \item[($m \geq 0,n=0$)] $Q_{m,0}$は空単体的空間
%   \item[($m \geq 0,n \geq 1$)] $Q_{m,n}$は次のpushoutで定める. 
%   % https://q.uiver.app/#q=WzAsNCxbMCwwLCJcXHBhcnRpYWwgXFxEZWx0YVttXSBcXHRpbWVzIFxcRGVsdGFbbl1edF8wIl0sWzEsMCwiXFxEZWx0YVttXSBcXHRpbWVzIFxcRGVsdGFbbl1edCJdLFsxLDEsIlFfe20sbn0iXSxbMCwxLCJcXERlbHRhW25dXnRfMCJdLFswLDEsIiIsMCx7InN0eWxlIjp7InRhaWwiOnsibmFtZSI6Imhvb2siLCJzaWRlIjoidG9wIn19fV0sWzEsMl0sWzAsM10sWzMsMl0sWzIsMCwiIiwxLHsic3R5bGUiOnsibmFtZSI6ImNvcm5lciJ9fV1d
%   \[\begin{tikzcd}
%     {\partial \Delta[m] \times \Delta[n]^t_0} & {\Delta[m] \times \Delta[n]^t} \\
%     {\Delta[n]^t_0} & {Q_{m,n}}
%     \arrow[hook, from=1-1, to=1-2]
%     \arrow[from=1-2, to=2-2]
%     \arrow[from=1-1, to=2-1]
%     \arrow[from=2-1, to=2-2]
%     \arrow["\lrcorner"{anchor=center, pos=0.125, rotate=180}, draw=none, from=2-2, to=1-1]
%   \end{tikzcd}\]
% \end{description}
% 任意の$m,n \geq 0$に対して, 包含$\partial \Delta[m] \times \Delta[n]^t \hookrightarrow \Delta[m] \times \Delta[n]^t$は, pushoutの普遍性より単体的空間の射$i_{m,n} : P_{m,n} \to Q_{m,n}$を定める. 
% % https://q.uiver.app/#q=WzAsNixbMCwwLCJcXHBhcnRpYWwgXFxEZWx0YVttXSBcXHRpbWVzIFxcRGVsdGFbbl1edF8wIl0sWzEsMCwiXFxwYXJ0aWFsIFxcRGVsdGFbbV0gXFx0aW1lcyBcXERlbHRhW25dXnQiXSxbMSwxLCJQX3ttLG59Il0sWzAsMiwiXFxEZWx0YVtuXV50XzAiXSxbMiwwLCJcXERlbHRhW21dIFxcdGltZXMgXFxEZWx0YVtuXV50Il0sWzIsMiwiUV97bSxufSJdLFswLDEsIiIsMCx7InN0eWxlIjp7InRhaWwiOnsibmFtZSI6Imhvb2siLCJzaWRlIjoidG9wIn19fV0sWzEsMl0sWzAsM10sWzMsMl0sWzEsNCwiIiwwLHsic3R5bGUiOnsidGFpbCI6eyJuYW1lIjoiaG9vayIsInNpZGUiOiJ0b3AifX19XSxbNCw1XSxbMyw1XSxbMiw1LCJpX3ttLG59IiwwLHsic3R5bGUiOnsiYm9keSI6eyJuYW1lIjoiZGFzaGVkIn19fV1d
% \[\begin{tikzcd}
%   {\partial \Delta[m] \times \Delta[n]^t_0} & {\partial \Delta[m] \times \Delta[n]^t} & {\Delta[m] \times \Delta[n]^t} \\
%   & {P_{m,n}} \\
%   {\Delta[n]^t_0} && {Q_{m,n}}
%   \arrow[hook, from=1-1, to=1-2]
%   \arrow[from=1-2, to=2-2]
%   \arrow[from=1-1, to=3-1]
%   \arrow[from=3-1, to=2-2]
%   \arrow[hook, from=1-2, to=1-3]
%   \arrow[from=1-3, to=3-3]
%   \arrow[from=3-1, to=3-3]
%   \arrow["{i_{m,n}}", dashed, from=2-2, to=3-3]
% \end{tikzcd}\]
% ここで, 集合$I_f$を次で定める.
% \begin{align*}
%   I_f := \{i_{m,n} : P_{m,n} \to Q_{m,n} ~|~ m,n \geq 0\}
% \end{align*}

% \newpage


% \section{単体的圏の圏に入るモデル構造} \label{sec:model_stru_in_SC}

% $\sSet$はKan-Quillenモデル構造を考えることで$(\infty,0)$圏のモデルとして, Joyalモデル構造を考えることで$(\infty,1)$圏のモデルとして見ることができる.
% 同様に, $\sSetCat$には次のモデル構造が存在すると考えられる. 
% \begin{itemize}
%   \item ファイブラントが$(\infty,0)$圏豊穣圏で, $(\infty,1)$圏のモデルであるようなモデル構造
%   \item ファイブラントが$(\infty,1)$圏豊穣圏で, $(\infty,2)$圏のモデルであるようなモデル構造
% \end{itemize}
% 前者のモデル構造の存在は\cite{Ber07}で証明された. 

% \subsection{Bergnerモデル構造}

% $(\infty,1)$圏のなす$(\infty,1)$圏のモデルとみなせるような$\sSetCat$のモデル構造が考えられる. 

% \begin{definition}[Bergnerモデル構造]
%   $\sSetCat$には次のモデル構造が存在する. 
%   これを$\sSetCat$上のBergnerモデル構造といい, $\sSetCatBerg$と表す. 
%   \begin{itemize}
%     \item weak equivalenceはDwyer-Kan同値
%     \item fibrationはDwyer-Kanファイブレーション 
%   \end{itemize}
% \end{definition}

% \begin{remark}
%   $\sSetCatBerg$において, 任意のKan複体豊穣圏はファイブラントであり, 任意の対象(単体的圏)はコファイブラントである. 
% \end{remark}

% % \begin{remark}
% %   $\sSetCatBerg$はproperなモデル圏である
% % \end{remark}

% % \begin{remark}
% %   $\sSetCatBerg$はCartesianモデル圏ではない. 
% % \end{remark}

% \newpage


\section{相対圏の圏に入るモデル構造} \label{sec:model_stru_in_RelCat}

相対圏にfibrationやcofibrationの情報はない. 
\cite{BK11}では, 相対圏の圏上のモデル構造は単体的空間の圏上のReedyモデル構造とQuillen同値になるようにして定められた. 
このことは\cref{sec:quillen_equiv_sSpaceReedy_and_RelCatBar}で証明しているが, ほかの章と体裁を合わせるために, 相対圏の圏上のBarwickモデル構造を定義する. 

\subsection{Barwickモデル構造}

\begin{definition}[Barwickモデル構造]
  $\RelCat$には次のモデル構造が存在する. 
  これを$\RelCat$上のBarwickモデル構造といい, $\RelCatBar$と表す. 
  \begin{itemize}
    \item weak equivalenceは$N_\xi$での像がvertical weak equivalenceである相対関手
    \item fibrationは$N_\xi$での像がReedy fibrationである相対関手
    \item cofibrationはDwyer射
  \end{itemize}
\end{definition}

\begin{remark}
  $\RelCatBar$において, 任意の相対半順序集合はコファイブラントである.
  ファイブラントの明示的な表示は知られていない. 
\end{remark}

\begin{remark}
  $\RelCatBar$は相対半順序集合のDwyer射をgenerating cofibrationとするコファイブラント生成なモデル圏である.
  generating trivial cofibraitionの集合の明示的な表示は知られていない. 
\end{remark}

\newpage


\section{$\sSpace$上のモデル構造と$\sSetJoyal$のQuillen随伴, Quillen同値}

\subsection{全関手とその随伴}

\begin{definition}
  関手$t : \Delta \times \Delta \to \sSet$を次のように定義する. 
  \begin{align*}
    t : \Delta \times \Delta \to \sSet : ([m],[n]) \mapsto \Delta[m] \times \Delta'[n]
  \end{align*}
\end{definition}

\begin{definition}
  $X$を単体的集合とする. 
  このとき, 単体的空間$t^!(X)$を任意の$m,n \geq 0$に対して次のように定義する. 
  \begin{align*}
    t^!(X)_{m,n} := \Hom_\sSet(\Delta[m] \times \Delta'[n], X)
  \end{align*}
\end{definition}

\begin{remark}
  構成$X \mapsto t^!(X)$は関手$t^! : \sSet \to \sSpace$を定める. 
\end{remark}

\begin{remark}
  普遍随伴の一般論より, $t^! : \sSet \to \sSpace$は左随伴$t_! : \sSpace \to \sSet$を持つ. 
  % https://q.uiver.app/#q=WzAsMyxbMCwyLCJcXERlbHRhIFxcdGltZXMgXFxEZWx0YSJdLFswLDAsIlxcc1NwYWNlIl0sWzIsMiwiXFxzU2V0Il0sWzAsMiwidCIsMl0sWzIsMSwidF4hIiwwLHsib2Zmc2V0IjotMSwiY3VydmUiOi0xfV0sWzEsMiwidF8hIiwwLHsib2Zmc2V0IjotMSwiY3VydmUiOi0xfV0sWzAsMSwi44KIIFxcdGltZXMg44KIIl0sWzUsNCwiIiwxLHsibGV2ZWwiOjEsInN0eWxlIjp7Im5hbWUiOiJhZGp1bmN0aW9uIn19XV0=
  \[\begin{tikzcd}
    \sSpace \\
    \\
    {\Delta \times \Delta} && \sSet
    \arrow["t"', from=3-1, to=3-3]
    \arrow[""{name=0, anchor=center, inner sep=0}, "{t^!}", shift left, curve={height=-6pt}, from=3-3, to=1-1]
    \arrow[""{name=1, anchor=center, inner sep=0}, "{t_!}", shift left, curve={height=-6pt}, from=1-1, to=3-3]
    \arrow["{よ}", from=3-1, to=1-1]
    \arrow["\dashv"{anchor=center, rotate=-135}, draw=none, from=1, to=0]
  \end{tikzcd}\]
  特に, 任意の$m,n \geq 0$に対して次が成立する.
  \begin{align*}
    t_!(\Delta[m] \Box \Delta'[n]) = \Delta[m] \times \Delta'[n]
  \end{align*}
\end{remark}

\begin{lemma} \label{prop:k_and_t_and_box}
  $A,B,X$を単体的集合とする. 
  このとき, 次の単体的集合の同型が成立する. 
  \begin{align*}
    t_!(A \Box B) = A \times k_!(B), ~~
    A \backslash t^!(X) = k^!(X^A), ~~
    t^!(X) / B = X^{k_!(B)}
  \end{align*}
\end{lemma}

\subsection{$\sSpace_v, \sSpace_h, \sSpaceSegal$と$\sSetJoyal$のQuillen随伴}

$\sSpace$上の垂直Reedyモデル構造と$\sSet$上のJoyalモデル構造はQuillen随伴である. 

\begin{proposition}[\cite{JT07} Theorem 2.12] \label{prop:quillen_adj_sSpacev_sSetJoyal}
  $t_! : \sSpace_v \to \sSetJoyal$と$t^! : \sSetJoyal \to \sSpace_v$は, $\sSpace_v$と$\sSetJoyal$のQuillen随伴を定める. 
  % \begin{align*}
  %   t_! : \sSpace_v \rightleftarrows \sSetJoyal : t^!
  % \end{align*}
  % https://q.uiver.app/#q=WzAsMixbMCwwLCJcXHNTcGFjZV92Il0sWzIsMCwiXFxzU2V0Sm95YWwiXSxbMCwxLCJ0XyEiLDAseyJvZmZzZXQiOi0xLCJjdXJ2ZSI6LTF9XSxbMSwwLCJ0XiEiLDAseyJvZmZzZXQiOi0xLCJjdXJ2ZSI6LTF9XSxbMiwzLCIiLDIseyJsZXZlbCI6MSwic3R5bGUiOnsibmFtZSI6ImFkanVuY3Rpb24ifX1dXQ==
  \[\begin{tikzcd}
    {\sSpace_v} && \sSetJoyal
    \arrow[""{name=0, anchor=center, inner sep=0}, "{t_!}", shift left, curve={height=-6pt}, from=1-1, to=1-3]
    \arrow[""{name=1, anchor=center, inner sep=0}, "{t^!}", shift left, curve={height=-6pt}, from=1-3, to=1-1]
    \arrow["\dashv"{anchor=center, rotate=-90}, draw=none, from=0, to=1]
  \end{tikzcd}\]
\end{proposition}

\begin{proof}
  まず, 左随伴がcofibrationを保つことを示す.
  $t_!(u)$が単体的集合のmono射($\sSetJoyal$におけるcofibration)である$\sSpace$における射$u$の飽和クラスを$A$と表す. 
  \cref{prop:k_and_t_and_box}より, 任意の$m,n \geq 0$に対して, $t_!(\delta_m \Box' \delta_n) = \delta_m \times' k_!(\delta_n)$である.
  \cref{prop:quillen_adj_sSetKan_sSetJoyal}より, $k_!$はcofibrationを保つ. 
  よって, $k_!(\delta_n)$は単体的集合のmono射である. 
  つまり, $t_!(\delta_m \Box' \delta_n)$は単体的空間のmono射なので, $\delta_m \Box' \delta_n$は$A$に属する. 
  \cref{prop:mono_in_sSpace}より, $A$の任意の元$u$は単体的空間のmono射($\sSpace_v$におけるcofibration)である. 
  よって, $t_!$はcofibrationを保つ.

  次に, 右随伴がfibrationを保つことを示す. 
  つまり, Joyal fibration $f : X \to Y$に対して, $t^!(f) : t^!(X) \to t^!(Y)$が垂直ファイブレーション($\sSpace_v$におけるfibration)であることを示す. 
  \cref{prop:v-fibration}より, 任意の単体的集合のmono射$u$に対して, $\angle{u \backslash t^!(f)}$がKan fibrationであることを示せばよい. 
  \cref{prop:k_and_t_and_box}より, 次の図式
  % https://q.uiver.app/#q=WzAsNSxbMCwwLCJCIFxcYmFja3NsYXNoIHReIShYKSJdLFsxLDIsIkIgXFxiYWNrc2xhc2ggdF4hKFkpIl0sWzIsMSwiQSBcXGJhY2tzbGFzaCB0XiEoWCkiXSxbMiwyLCJBIFxcYmFja3NsYXNoIHReIShZKSJdLFsxLDEsIkEgXFxiYWNrc2xhc2ggdF4hKFgpIFxcdGltZXNfe0EgXFxiYWNrc2xhc2ggdF4hKFkpfSBCIFxcYmFja3NsYXNoIHReIShZKSJdLFsyLDNdLFsxLDNdLFs0LDFdLFs0LDJdLFswLDQsIlxcYW5nbGV7dSBcXGJhY2tzbGFzaCB0XiEoZil9Il0sWzQsMywiIiwwLHsic3R5bGUiOnsibmFtZSI6ImNvcm5lciJ9fV1d
  \[\begin{tikzcd}
    {B \backslash t^!(X)} \\
    & {A \backslash t^!(X) \times_{A \backslash t^!(Y)} B \backslash t^!(Y)} & {A \backslash t^!(X)} \\
    & {B \backslash t^!(Y)} & {A \backslash t^!(Y)}
    \arrow[from=2-3, to=3-3]
    \arrow[from=3-2, to=3-3]
    \arrow[from=2-2, to=3-2]
    \arrow[from=2-2, to=2-3]
    \arrow["{\angle{u \backslash t^!(f)}}", from=1-1, to=2-2]
    \arrow["\lrcorner"{anchor=center, pos=0.125}, draw=none, from=2-2, to=3-3]
  \end{tikzcd}\]
  は次の図式
  % https://q.uiver.app/#q=WzAsNSxbMCwwLCJrXiEoWF5CKSJdLFsxLDIsImteIShZXkIpIl0sWzIsMSwia14hKFheQSkiXSxbMiwyLCJrXiEoWV5BKSJdLFsxLDEsImteIShYXkEpIFxcdGltZXNfe2teIShZXkEpfSBrXiEoWV5CKSJdLFsyLDNdLFsxLDNdLFs0LDFdLFs0LDJdLFswLDQsImteIShcXGFuZ2xle3UsZn0pIl0sWzQsMywiIiwwLHsic3R5bGUiOnsibmFtZSI6ImNvcm5lciJ9fV1d
  \[\begin{tikzcd}
    {k^!(X^B)} \\
    & {k^!(X^A) \times_{k^!(Y^A)} k^!(Y^B)} & {k^!(X^A)} \\
    & {k^!(Y^B)} & {k^!(Y^A)}
    \arrow[from=2-3, to=3-3]
    \arrow[from=3-2, to=3-3]
    \arrow[from=2-2, to=3-2]
    \arrow[from=2-2, to=2-3]
    \arrow["{k^!(\angle{u,f})}", from=1-1, to=2-2]
    \arrow["\lrcorner"{anchor=center, pos=0.125}, draw=none, from=2-2, to=3-3]
  \end{tikzcd}\]
  と同一視できる. 
  よって, 射$\angle{u \backslash t^!(f)}$は射$k^!(\angle{u,f})$と同一視できる. 
  Joyalモデル構造はCartesianモデル圏なので, $\angle{u,f}$はJoyal fibrationである.
  \cref{prop:quillen_adj_sSetKan_sSetJoyal}より$k^!$はfibrationを保つので, $k^!(\angle{u,f})$はKan fibrationである. 
  つまり,  $\angle{u \backslash t^!(f)}$もKan fibrationである.
  \cref{prop.kan_fib_is_joyal_fib}より,  $\angle{u \backslash t^!(f)}$はJoyal fibrationである. 
  以上より, $t^!$はfibrationを保つ. 
\end{proof}

$\sSpace$上の水平Reedyモデル構造と$\sSet$上のJoyalモデル構造はQuillen随伴である. 

\begin{proposition}[\cite{JT07} Theorem 2.12] \label{prop:quillen_adj_sSpaceh_sSetJoyal}
  $t_! : \sSpace_h \to \sSetJoyal$と$t^! : \sSetJoyal \to \sSpace_h$は, $\sSpace_h$と$\sSetJoyal$のQuillen随伴を定める. 
  % \begin{align*}
  %   t_! : \sSpace_h \rightleftarrows \sSetJoyal : t^!
  % \end{align*}
  % https://q.uiver.app/#q=WzAsMixbMCwwLCJcXHNTcGFjZV9oIl0sWzIsMCwiXFxzU2V0Sm95YWwiXSxbMCwxLCJ0XyEiLDAseyJvZmZzZXQiOi0xLCJjdXJ2ZSI6LTF9XSxbMSwwLCJ0XiEiLDAseyJvZmZzZXQiOi0xLCJjdXJ2ZSI6LTF9XSxbMiwzLCIiLDIseyJsZXZlbCI6MSwic3R5bGUiOnsibmFtZSI6ImFkanVuY3Rpb24ifX1dXQ==
  \[\begin{tikzcd}
    {\sSpace_h} && \sSetJoyal
    \arrow[""{name=0, anchor=center, inner sep=0}, "{t_!}", shift left, curve={height=-6pt}, from=1-1, to=1-3]
    \arrow[""{name=1, anchor=center, inner sep=0}, "{t^!}", shift left, curve={height=-6pt}, from=1-3, to=1-1]
    \arrow["\dashv"{anchor=center, rotate=-90}, draw=none, from=0, to=1]
  \end{tikzcd}\]
\end{proposition}

\begin{proof}
  左随伴がcofibrationを保つことは\cref{prop:quillen_adj_sSpacev_sSetJoyal}ですでに示した. 

  右随伴がfibrationを保つことを示す. 
  つまり, Joyal fibration $f : X \to Y$に対して, $t^!(f) : t^!(X) \to t^!(Y)$が水平ファイブレーション($\sSpace_h$におけるfibration)であることを示す. 
  任意の単体的集合のmono射$u$に対して, $\angle{t^!(f) / u}$がJoyal fibrationであることを示せばよい. 
  \cref{prop:k_and_t_and_box}より, 次の図式
  % https://q.uiver.app/#q=WzAsNSxbMCwwLCJ0XiEoWCkgLyBCIl0sWzEsMiwidF4hKFkpIC8gQiJdLFsyLDEsInReIShYKSAvIEEiXSxbMiwyLCJ0XiEoWSkgLyBBIl0sWzEsMSwidF4hKFgpIC8gQSBcXHRpbWVzX3t0XiEoWSkgLyBBfSB0XiEoWSkgLyBCIl0sWzEsM10sWzIsM10sWzQsMV0sWzQsMl0sWzQsMywiIiwxLHsic3R5bGUiOnsibmFtZSI6ImNvcm5lciJ9fV0sWzAsNCwiXFxhbmdsZXt0XiEoZikgLyB1fSJdXQ==
  \[\begin{tikzcd}
    {t^!(X) / B} \\
    & {t^!(X) / A \times_{t^!(Y) / A} t^!(Y) / B} & {t^!(X) / A} \\
    & {t^!(Y) / B} & {t^!(Y) / A}
    \arrow[from=3-2, to=3-3]
    \arrow[from=2-3, to=3-3]
    \arrow[from=2-2, to=3-2]
    \arrow[from=2-2, to=2-3]
    \arrow["\lrcorner"{anchor=center, pos=0.125}, draw=none, from=2-2, to=3-3]
    \arrow["{\angle{t^!(f) / u}}", from=1-1, to=2-2]
  \end{tikzcd}\]
  は次の図式 
  % https://q.uiver.app/#q=WzAsNSxbMCwwLCJYXntrXyEoQil9Il0sWzEsMiwiWV57a18hKEIpfSJdLFsyLDEsIlhee2tfIShBKX0iXSxbMiwyLCJZXntrXyEoQSl9Il0sWzEsMSwiWF57a18hKEEpfSBcXHRpbWVzX3tZXntrXyEoQSl9fSBZXntrXyEoQil9Il0sWzIsM10sWzEsM10sWzQsMV0sWzQsMl0sWzAsNCwiXFxhbmdsZXtrXiEodSksZn0iXSxbNCwzLCIiLDAseyJzdHlsZSI6eyJuYW1lIjoiY29ybmVyIn19XV0=
  \[\begin{tikzcd}
    {X^{k_!(B)}} \\
    & {X^{k_!(A)} \times_{Y^{k_!(A)}} Y^{k_!(B)}} & {X^{k_!(A)}} \\
    & {Y^{k_!(B)}} & {Y^{k_!(A)}}
    \arrow[from=2-3, to=3-3]
    \arrow[from=3-2, to=3-3]
    \arrow[from=2-2, to=3-2]
    \arrow[from=2-2, to=2-3]
    \arrow["{\angle{k^!(u),f}}", from=1-1, to=2-2]
    \arrow["\lrcorner"{anchor=center, pos=0.125}, draw=none, from=2-2, to=3-3]
  \end{tikzcd}\]
  と同一視できる. 
  よって, 射$\angle{t^!(u) / u}$は射$\angle{k_!(u),f}$と同一視できる. 
  \cref{prop:quillen_adj_sSetKan_sSetJoyal}より$k_!$はcofibrationを保つので, $k_!(u)$は単体的集合のmono射である.
  Joyalモデル構造はCartesianモデル圏なので, $\angle{k_!(u),f}$はJoyal fibrationである. 
  つまり, $\angle{t^!(f) / u}$もJoyal fibrationである. 
\end{proof}

$\sSpace$上のSegal空間モデル構造と$\sSet$上のJoyalモデル構造はQuillen随伴である. 

\begin{proposition}[\cite{JT07} Theorem 3.3] \label{prop:quillen_adj_sSpceSegal_sSetJoyal}
  $t_! : \sSpaceSegal \to \sSetJoyal$と$t^! : \sSetJoyal \to \sSpaceSegal$は, $\sSpaceSegal$と$\sSetJoyal$のQuillen随伴を定める. 
  % \begin{align*}
  %   t_! : \sSpaceSegal \rightleftarrows \sSetJoyal : t^!
  % \end{align*}
  % https://q.uiver.app/#q=WzAsMixbMCwwLCJcXHNTcGFjZVNlZ2FsIl0sWzIsMCwiXFxzU2V0Sm95YWwiXSxbMCwxLCJ0XyEiLDAseyJvZmZzZXQiOi0xLCJjdXJ2ZSI6LTF9XSxbMSwwLCJ0XiEiLDAseyJvZmZzZXQiOi0xLCJjdXJ2ZSI6LTF9XSxbMiwzLCIiLDIseyJsZXZlbCI6MSwic3R5bGUiOnsibmFtZSI6ImFkanVuY3Rpb24ifX1dXQ==
  \[\begin{tikzcd}
    \sSpaceSegal && \sSetJoyal
    \arrow[""{name=0, anchor=center, inner sep=0}, "{t_!}", shift left, curve={height=-6pt}, from=1-1, to=1-3]
    \arrow[""{name=1, anchor=center, inner sep=0}, "{t^!}", shift left, curve={height=-6pt}, from=1-3, to=1-1]
    \arrow["\dashv"{anchor=center, rotate=-90}, draw=none, from=0, to=1]
  \end{tikzcd}\]
\end{proposition}

\begin{proof}
  \cref{prop:quillen_adj_sSpacev_sSetJoyal}より, $t_!$はcofibrationを保つ.
  次に, $t^!$がファイブラントを保つことを示す. 
  \cref{prop:quillen_adj_sSpacev_sSetJoyal}より, 任意の擬圏($\sSetJoyal$におけるファイブラント) $X$に対して, $t^!(X)$は垂直ファイブラント($\sSpace_v$におけるファイブラント)である. 
  よって, この$t^!(X)$がSegal空間($\sSpaceSegal$におけるファイブラント)であることを示す. 
  つまり, 任意の$n \geq 0$に対して, 
  \begin{align*}
    \varphi_n \backslash t^!(X) : \Delta[n] \backslash t^!(X) \to I_n \backslash t^!(X)
  \end{align*}
  がKan weak equivalenceであることを示せばよい. 
  \cref{prop:k_and_t_and_box}より, $\varphi_n \backslash t^!(X)$は$k^!(X^{\varphi_n})$と同一視できる. 
  $\varphi_n : I_n \hookrightarrow \Delta[n]$は内緩射なので, $\varphi_n$はJoyal weak equivalenceである. 
  $\varphi_n$は単体的集合のmono射かつ, $\sSetJoyal$はCartesian閉であることから, $X^{\varphi_n}$はJoyal fibrationである. 
  以上より, $X^{\varphi_n}$はJoyal trivial fibrationである. 
  \cref{prop:quillen_adj_sSetKan_sSetJoyal}より, $k^!$は右Quillen随伴である. 
  よって, $k^!(X^{\varphi_n})$はKan trivial fibrationである. 
  従って, $\varphi_n \backslash t^!(X)$もKan trivial fibrationである. 
  特に$\varphi_n \backslash t^!(X)$はKan weak equivalenceなので, $t^!(X)$はSegal空間である. 
\end{proof}

\subsection{$\sSpaceCSS$と$\sSetJoyal$のQuillen同値}

$\sSpace$上のRezkモデル構造と$\sSet$上のJoyalモデル構造はQuillen同値である. 

\begin{proposition}[\cite{JT07} Theorem 4.12] \label{prop:quillen_adj_sSpceCSS_sSetJoyal}
  $t_! : \sSpaceCSS \to \sSetJoyal$と$t^! : \sSetJoyal \to \sSpaceCSS$は, $\sSpaceCSS$と$\sSetJoyal$のQuillen同値を定める. 
  % \begin{align*}
  %   t_! : \sSpaceCSS \rightleftarrows \sSetJoyal : t^!
  % \end{align*}
  % https://q.uiver.app/#q=WzAsMixbMCwwLCJcXHNTcGFjZUNTUyJdLFsyLDAsIlxcc1NldEpveWFsIl0sWzAsMSwidF8hIiwwLHsib2Zmc2V0IjotMSwiY3VydmUiOi0xfV0sWzEsMCwidF4hIiwwLHsib2Zmc2V0IjotMSwiY3VydmUiOi0xfV0sWzIsMywiIiwyLHsibGV2ZWwiOjEsInN0eWxlIjp7Im5hbWUiOiJhZGp1bmN0aW9uIn19XV0=
  \[\begin{tikzcd}
    \sSpaceCSS && \sSetJoyal
    \arrow[""{name=0, anchor=center, inner sep=0}, "{t_!}", shift left, curve={height=-6pt}, from=1-1, to=1-3]
    \arrow[""{name=1, anchor=center, inner sep=0}, "{t^!}", shift left, curve={height=-6pt}, from=1-3, to=1-1]
    \arrow["\dashv"{anchor=center, rotate=-90}, draw=none, from=0, to=1]
  \end{tikzcd}\]
\end{proposition}

\begin{proof}
  まず, $(t_! \dashv t^!)$がQuillen随伴であることを示す. 
  \cref{prop:quillen_adj_sSpceSegal_sSetJoyal}より, 任意の擬圏($\sSetJoyal$におけるファイブラント) $X$に対して, $t^!(X)$はSegal空間($\sSpaceSegal$におけるファイブラント)である. 
  よって, この$t^!(X)$が完備Segal空間($\sSpaceCSS$におけるファイブラント)であることを示せばよい. 
  \cref{prop:complete_is}より, $u_0 \backslash X : J \backslash X \to \{0\} \backslash X$がKan trivial fibrationであることを示せばよい. 
  \cref{prop:k_and_t_and_box}より, 射$u_0 \backslash X$は射$k^!(X^{u_0})$と同一視できる. 
  $u_0 : \{0\} \to J$は通常の圏同値なので, $u_0$はJoyal weak equivalenceである. 
  また, $u_0$は単体的空間のmono射なので, $X^{u_0}$はJoyal trivial fibrationである.
  \cref{prop:quillen_adj_sSetKan_sSetJoyal}より$k^!$はtrivial fibrationを保つので,  $k^!(X^{u_0})$はKan trivial fibrationである. 
  よって, $u_0 \backslash X$もKan trivial fibrationである. 

  次に, $(t_! \dashv t^!)$がQuillen同値であることを示す.
  任意の単体的集合$A$に対して, \cref{prop:k_and_t_and_box}より, 
  \begin{align*}
    t_!p^\ast_1(A) = t_!(A \Box \Delta[0]) = A \times k_!(\Delta[0]) = A  
  \end{align*}
  なので, $t_!p^\ast_1$は$\Id_{\sSet}$と同型である.
  同様に, 
  \begin{align*}
    i^\ast_1t^!(A) = i^\ast_1(\Hom_\sSpace(\Delta[-] \times \Delta'[?], X)) = \Hom_\sSet(\Delta[-],X) = X
  \end{align*}
  なので, $i^\ast_1t^!$は$\Id_{\sSet}$と同型である.
  \footnote{
    随伴の一意性からも従う.
  }
  \cref{quillen_equiv_sSetJoyal_sSpaceCSS}より, $(p^\ast_1,i^\ast_1)$はQuillen同値である. 
  2-out-of-3より, $(t_! \dashv t^!)$もQuillen同値である.
\end{proof}

\subsection{$\sSetJoyal$と$\sSpaceCSS$のQuillen同値} \label{subsec:quillen_equiv_sSetJoyal_sSpaceCSS}

\cref{subsec:quillen_equiv_sSetJoyal_sSpaceCSS}の目標は次の\cref{quillen_equiv_sSetJoyal_sSpaceCSS}を示すことである. 

\begin{proposition} \label{quillen_equiv_sSetJoyal_sSpaceCSS}
  $p^\ast_1 : \sSetJoyal \to \sSpaceCSS$と$i^\ast_1 : \sSpaceCSS \to \sSetJoyal$は$\sSetJoyal$と$\sSpaceCSS$のQuillen同値を定める.
  % \begin{align*}
  %   p^\ast_1 : \sSetJoyal \rightleftarrows \sSpaceCSS : i^\ast_1
  % \end{align*}
  % https://q.uiver.app/#q=WzAsMixbMCwwLCJcXHNTZXRKb3lhbCJdLFsyLDAsIlxcc1NwYWNlQ1NTIl0sWzAsMSwicF5cXGFzdF8xIiwwLHsib2Zmc2V0IjotMSwiY3VydmUiOi0xfV0sWzEsMCwiaV5cXGFzdF8xIiwwLHsib2Zmc2V0IjotMSwiY3VydmUiOi0xfV0sWzIsMywiIiwyLHsibGV2ZWwiOjEsInN0eWxlIjp7Im5hbWUiOiJhZGp1bmN0aW9uIn19XV0=
  \[\begin{tikzcd}
    \sSetJoyal && \sSpaceCSS
    \arrow[""{name=0, anchor=center, inner sep=0}, "{p^\ast_1}", shift left, curve={height=-6pt}, from=1-1, to=1-3]
    \arrow[""{name=1, anchor=center, inner sep=0}, "{i^\ast_1}", shift left, curve={height=-6pt}, from=1-3, to=1-1]
    \arrow["\dashv"{anchor=center, rotate=-90}, draw=none, from=0, to=1]
  \end{tikzcd}\]
\end{proposition}

\begin{lemma} \label{prop:box_product_is_left_quillen}
  ボックス積を与える関手$\sSet \times \sSet \to \sSpace$はQuillen双関手
  \begin{align*}
    \sSetJoyal \times \sSetKan \to \sSpaceCSS
  \end{align*}
  を定める. 
\end{lemma}

% \begin{proof}
%   $\sSet$上のKan-Quillenモデル構造と$\sSet$上のJoyal構造におけるcofibrationはともに単体的集合のmono射である. 
%   また, $\sSpace$上のRezkモデル構造におけるcofibrationも各次数で単体的集合のmono射となるような単体的空間の射である. 

%   $u : A \to B, v : S \to T$を単体的集合のmono射とする. 
%   このとき, $u \Box' v$は単体的空間のmono射である. 
%   (途中)
% \end{proof}

\begin{remark} \label{prop:adj_p1_i1}
  第一射影
  \begin{align*}
    p_1 : \Delta \times \Delta \to \Delta : ([m],[n]) \mapsto [m]
  \end{align*}
  と第一入射
  \begin{align*}
    i_1 : \Delta \to \Delta \times \Delta : [m] \mapsto ([m],[0])
  \end{align*}
  は次の随伴を定める.
  % https://q.uiver.app/#q=WzAsMixbMCwwLCJcXERlbHRhIFxcdGltZXMgXFxEZWx0YSJdLFsyLDAsIlxcRGVsdGEiXSxbMCwxLCJwXzEiLDAseyJvZmZzZXQiOi0xLCJjdXJ2ZSI6LTF9XSxbMSwwLCJpXzEiLDAseyJvZmZzZXQiOi0xLCJjdXJ2ZSI6LTF9XV0=
  \[\begin{tikzcd}
    {\Delta \times \Delta} && \Delta
    \arrow["{p_1}", shift left, curve={height=-6pt}, from=1-1, to=1-3]
    \arrow["{i_1}", shift left, curve={height=-6pt}, from=1-3, to=1-1]
  \end{tikzcd}\] 
\end{remark}

\begin{proof}{(\cref{quillen_equiv_sSetJoyal_sSpaceCSS})}
  まず, \cref{prop:adj_p1_i1}より$(p^\ast_1 \dashv i^\ast_1)$は随伴である. 

  次に, $(p^\ast_1 \dashv i^\ast_1)$がQuillen随伴であることを示す. 
  まず, 左随伴がcofibrationを保つことを示す.
  任意の単体的集合$A$に対して, 
  \begin{align*}
    p^\ast_1: \sSetJoyal \to \sSpaceCSS : A \mapsto A \Box \Delta[0]
  \end{align*}
  である. 
  \cref{prop:box_product_is_left_quillen}より, ボックス積を与える関手は左Quillen関手である.
  よって, $p^\ast_1$も左Quillen関手である. 

  次の, $(p^\ast_1 \dashv i^\ast_1)$の余単位射がweak equivalenceであることを示す. 
  $\sSpaceCSS$におけるファイブラントは完備Segal空間である. 
  $\sSetJoyal$において, 任意の対象(単体的集合)はコファイブラントである. 
  よって, 任意の完備Segal空間$X$に対して, 
  \begin{align*}
    \varepsilon_X : p^\ast_1 i^\ast_1 (X) \to X
  \end{align*}
  がRezk weak fibraionであることを示せばよい.
  \cref{prop:sSpaceCSS_is_left_Bousfield_localization_of_sSpaceh}より, $\varepsilon$が各列弱圏同値であることを示せばよい. 
  ここで, 
  \begin{align*}
    p^\ast_1 i^\ast_1 (X) 
    = p^\ast_1 (X_{-,0})
    = X_{-,0} \Box \Delta[0]
  \end{align*}
  であり, $X_{-,0} \Box \Delta[0]$は任意の$n \geq 0$に対して, $X_{-,n}=X_{-,0}$であるような単体的空間である. 
  よって, 任意の$n \geq 0$に対して, 
  \begin{align*}
    \varepsilon_{-,n} : X_{-,0} \to X_{-,n}
  \end{align*}
  がJoyal weak equivalenceであることを示せばよい.
  $X$は完備Segal空間なので, $X$は圏的同値である. 
  よって, 任意の$n \geq 0$に対して, $\varepsilon_{-,n}$はJoyal weak equivalenceである.
\end{proof}

\newpage


% \section{$\sSetJoyal$と$\sSpaceCSS$のQuillen同値} \label{sec:quillen_equiv_sSetJoyal_and_sSpaceCSS}

% \cref{sec:quillen_equiv_sSetJoyal_and_sSpaceCSS}の目標は, 次の\cref{prop:quillen_equiv_sSetJoyal_and_sSpaceCSS}を示すことである. 

% \begin{proposition} \label{prop:quillen_equiv_sSetJoyal_and_sSpaceCSS}
%   水平埋め込み
%   \footnote{
%     \cite{Ber18}では, \cref{sec:CSS}と異なり「垂直」や「水平」という言葉を用いていない.
%     理由は\cite{Ber18}の7.9節を参照.   
%     本稿では, \cref{sec:CSS}と同様の言葉を用いる. 
%   }
%   $i : \sSetJoyal \to \sSpaceCSS$と評価射$\ev_0 : \sSpaceCSS \to \sSetJoyal$は, $\sSetJoyal$と$\sSpaceCSS$のQuillen同値 
%   \begin{align*}
%     i : \sSetJoyal \rightleftarrows \sSpaceCSS : \ev_0
%   \end{align*}
%   を定める. 
% \end{proposition}

% \subsection{$\sSpaceCSS$における完備Segal空間の性質}

% $\sSpaceCSS$において, ファイブラントは完備Segal空間である. 
% 完備Segal空間はReedy ファイブラント条件, Segal条件, 完備性を満たす単体的空間である. 
% 完備Segal空間を射のリフト条件で表す. 

% \begin{proposition} \label{prop:CSS_means_this_lift_property}
%   単体的空間$X$が完備Segal空間であることと, 単体的空間の射$X \to \Delta[0]$が次の射に対してRLPを持つことは同値である. 
%   \begin{itemize}
%     \item $\{\partial \Delta[n] \times \Delta[m]^t \cup \Delta[n] \times \Lambda[m,k]^t \to \Delta[n] \times \Delta[m]^t ~|~ n \geq 0, m \geq 1, 0 \leq k \leq m\}$
%     \item $\{\partial \Delta[m] \times \Delta[n]^t \cup \Delta[m] \times G(n)^t \to \Delta[m] \times \Delta[n]^t ~|~ m \geq 0, n \geq 2\}$
%     \item $\{\partial \Delta[m] \times E^t \cup \Delta[m] \times \Delta[0]^t \to \Delta[m] \times E^t ~|~ m \geq 0\}$
%   \end{itemize}
% \end{proposition}

% \begin{proof}
%   $W$がReedyファイブラントのとき, 単体的空間の射$X \to \Delta[0]$はReedy trivial cofibrationに対してRLPを持つ. 
%   特に, $\sSpaceCSS$におけるgenerating trivial cofibration
%   \begin{align*}
%     \{\partial \Delta[n] \times \Delta[m]^t \cup \Delta[n] \times \Lambda[m,k]^t \to \Delta[n] \times \Delta[m]^t ~|~ n \geq 0, m \geq 1, 0 \leq k \leq m\}
%   \end{align*}
%   に対してRLPを持つ. 

%   $W$が更にSegal空間のとき, 任意の$n \geq 2$に対して,
%   \begin{align*}
%     \Map_\sSpace(\Delta[n]^t,X) \to \Map_\sSpace(G(n)^t,X)
%   \end{align*}
%   は単体的集合のKan weak equivalenceである. 
%   $G(n)^t \hookrightarrow \Delta[n]^t$は$\sSpaceCSS$におけるcofibrationなので, 
%   \footnote{
%     $\sSpaceCSS$におけるcofibrationは各次数で単体的集合のcofibrationとなる射である. 
%   }
%   \cref{prop:cofibration_induce_fibraion_in_simplicial_model_cat}より, $\Map(\Delta[n]^t,X) \to \Map(G(n)^t,X)$はfibrationである. 
%   よって, この射はKan trivial fibrationである. 
%   \cref{prop:trifib_is_in_sSetKan}より, 任意の$m \geq 0$に対して, 次の四角はリフトを持つ. 
%   % https://q.uiver.app/#q=WzAsNCxbMCwwLCJcXHBhcnRpYWwgXFxEZWx0YVttXSJdLFswLDEsIlxcRGVsdGFbbl0iXSxbMSwwLCJcXE1hcChcXERlbHRhW25dXnQsVykiXSxbMSwxLCJcXE1hcChHKG4pXnQsVykiXSxbMCwxLCIiLDIseyJzdHlsZSI6eyJ0YWlsIjp7Im5hbWUiOiJob29rIiwic2lkZSI6InRvcCJ9fX1dLFswLDJdLFsyLDNdLFsxLDNdLFsxLDIsIiIsMSx7InN0eWxlIjp7ImJvZHkiOnsibmFtZSI6ImRhc2hlZCJ9fX1dXQ==
%   \[\begin{tikzcd}
%     {\partial \Delta[m]} & {\Map_\sSpace(\Delta[n]^t,X)} \\
%     {\Delta[n]} & {\Map_\sSpace(G(n)^t,X)}
%     \arrow[hook, from=1-1, to=2-1]
%     \arrow[from=1-1, to=1-2]
%     \arrow[from=1-2, to=2-2]
%     \arrow[from=2-1, to=2-2]
%     \arrow[dashed, from=2-1, to=1-2]
%   \end{tikzcd}\]
%   積と射空間の随伴性とpullbackの普遍性より, $X \to \Delta[0]$は次の射の集まりに対してRLPを持つことと同値である. 
%   \begin{align*}
%     \{\partial \Delta[m] \times \Delta[n]^t \cup \Delta[m] \times G(n)^t \to \Delta[m] \times \Delta[n]^t ~|~ m \geq 0, n \geq 2\}
%   \end{align*}

%   $X$が完備Segal空間のとき, 同様の議論より, $X \to \Delta[0]$は次の射の集まりに対してRLPを持つ. 
%   \begin{align*}
%     \{\partial \Delta[m] \times E^t \cup \Delta[m] \times \Delta[0]^t \to \Delta[m] \times E^t ~|~ m \geq 0\}
%   \end{align*}
% \end{proof}

% \subsection{$\sSpace$上の反対Joyalモデル構造と局所定値モデル構造}

% $\sSet$上のJoyalモデル構造とQuillen同値になるような$\sSpace$上のモデル構造を考える. 
% このモデル構造が実際に存在することは後で示す. 

% \begin{definition}[反対Joyalモデル構造]
%   $\sSpace$上には次のモデル構造が存在する. 
%   これを$\sSpace$上の反対Joyalモデル構造といい, $\sSetJoyal^\myop$と表す. 
%   \begin{itemize}
%     \item weak equivalenceは任意の$n \geq 0$に対して, $W_{-,n} \to Z_{-,n}$がJoyal weak equivalenceであるような単体的空間の射
%     \item cofibrationは各次数で$\sSet$においてcofibrationである単体的空間の射
%   \end{itemize}
% \end{definition}

% \cref{prop:CSS_means_this_lift_property}と同様に, $\sSetJoyal^\myop$におけるファイブラントをリフト条件で表す.
% (途中)

% \begin{definition}[局所定値な単体的空間]
%   $X$を$\sSetJoyal^\myop$における対象とする. 
%   任意の$m \geq 1$に対して, $\Delta$における一意な射$[n] \to [0]$が定める単体的集合の射$X_{-,0} \to X_{-,n}$がJoyal weak equivalenceのとき, $X$は局所定値(locally constant)であるという. 
% \end{definition}

% $\sSetJoyal^\myop$における$X$が局所定値なファイブラントをリフト条件で表す.

% \begin{proposition}
%   $X$を$\sSetJoyal^\myop$におけるファイブラントとする.
%   $X$が局所定値であることと, 単体的空間の射$W \to \Delta[0]$が次の射の集合に対してRLPを持つことは同値である.  
%   \footnote{
%     \cite{Ber18} Proposition 7.9.3では, 
%     \begin{align*}
%       \Lambda[m,k] \times \Delta[n]^t \cup \Delta[m] \times \partial \Delta[n]^t \to \Delta[m] \times \Delta[n]^t
%     \end{align*}
%     となっているが, 本稿で示した射の誤植であると思われる. 
%     \cite{Ber18}の誤植一覧は\href{https://sites.google.com/view/julie-bergner/home/publications}{Bergner's homepage}にある. 
%     (この箇所は載っていない.)
%   }
%   \begin{align*}
%     \{\partial \Delta[n] \times \Delta[n]^t \cup \Delta[m] \times \Lambda[m,k]^t \to \Delta[m] \times \Delta[n]^t ~|~ m \geq 1, 0 \leq k \leq m, n \geq 0\}
%   \end{align*}
% \end{proposition}

% % \begin{proof}
% %   \cref{prop:CSS_means_this_lift_property}と同様の議論より, $W \to \Delta[0]$が上の射の集合に対してRLPを持つことと, 
% %   \begin{align*}
% %     \Map_\sSpace(\Delta[m],W) \to \Map_\sSpace(\Lambda[m,k],W)
% %   \end{align*}
% %   がKan trivial fibrationであることは同値である. 
% %   つまり, この射が$\partial \Delta[n] \hookrightarrow \Delta[n]$に対してRLPを持つことと同値である. 

% %   まず, $m=1$のときを考える. 
% %   このとき, 任意の$0 \leq k \leq 1$に対して, $\lambda[1,k] \cong \Delta[0]$である. 
% %   よって, 
% %   \begin{align*}
% %     &\Map_\sSpace(\Delta[1],W) \cong X_{-,1} \\
% %     &\Map_\sSpace(\Lambda[1,k],W) \cong \Map_\sSpace(\Delta[0],W) = X_{-,0}
% %   \end{align*}
% %   である. 
% %   つまり, $X_{-,1} \to X_{-,0}$がJoyal weak equivalenceであることと同値である. 
% % \end{proof}

% \begin{proposition}
%   $\sSpace$上の局所定値モデル構造は$\sSpace$上のRezkモデル構造に一致する. 
% \end{proposition}

% \subsection{\cref{prop:quillen_equiv_sSetJoyal_and_sSpaceCSS}の証明}

% \cref{prop:quillen_equiv_sSetJoyal_and_sSpaceCSS}に出てくる関手を復習しておく. 

% \begin{definition}[水平埋め込み]
%   $i : \sSet \to \sSpace$を任意の単体的集合$K$を, 任意の$n \geq 0$に対して$Z_{-,n} = K$となるような単体的空間$Z$に送ることで定義する. 
%   この関手$i$を水平埋め込みという. 
% \end{definition}

% \begin{definition}[評価射]
%   $\ev_0 : \sSpace \to \sSet$を任意の単体的空間$W$を単体的集合$W_{-,n}$に送ることで定義する. 
%   この関手$\ev_0$を評価射という.
% \end{definition}

% \cref{prop:quillen_equiv_sSetJoyal_and_sSpaceCSS}を証明する. 

% \begin{proof}[\cref{prop:quillen_equiv_sSetJoyal_and_sSpaceCSS}の証明]
%   まず, $i$と$\ev_0$が随伴であることを示す. (途中)

%   次に, この随伴がQuillen随伴であることを示す. 
%   まず, 左随伴$i$がcofibtrationを保つことを示す. 
%   $\sSetJoyal$におけるcofibrationの定義は, 各次数でmono射となる単体的集合の射である. 
%   $\sSpaceCSS$におけるcofibraionの定義は, 各次数でcofibrationとなる単体的空間の射である. 
%   よって, $i$がcofibtrationを保つことは明らかである.

%   また, 左随伴$i$がtrivial cofibrationを保つことを示す.
%   まず, $\sSpace$上の局所定値モデル構造の定義より, $i$はweak equivalenceを保つ. 
%   特に, trivial cofibrationを保つことも分かる.

%   最後に, このQuillen随伴がQuillen同値であることを示す. 
%   まず, 任意の単体的集合$K$に対して, 単体的集合の射$K \to \ev_0 Z^f$がJoyal同値であることを示す. 
%   ここで, 単体的空間$Z$のファイブラント置換$Z^f$は各次数が擬圏である局所定値な単体的空間である. 
%   Joyalモデル構造において, 任意の擬圏はファイブラントである. 
%   よって, $\ev_0 Z^f$は$\sSetJoyal$におけるファイブラント置換とみなせるので, この射はJoyal同値(weak equivalence)である. 
%   また, 
% \end{proof}

% \newpage


% \section{$\sSetJoyal$と$\sSetCatBerg$のQuillen同値} \label{sec:quillen_equiv_sSetJoyal_and_SCBergner}

% \cref{sec:quillen_equiv_sSetJoyal_and_SCBergner}の目標は次の\cref{prop:quillen_equiv_sSetJoyal_and_SCBergner}を証明することである. 

% \begin{proposition} \label{prop:quillen_equiv_sSetJoyal_and_SCBergner}
%   剛化関手$\mathfrak{C} : \sSetJoyal \to \sSetCatBerg$とホモトピー連接脈体$\tilde{N} : \sSetCatBerg \to \sSetJoyal$は$\sSetCatBerg$と$\sSetJoyal$のQuillen同値
%   \begin{align*}
%     \mathfrak{C} : \sSetJoyal \rightleftarrows \sSetCatBerg : \tilde{N}
%   \end{align*}
%   を定める. 
% \end{proposition}

% \subsection{剛化関手とホモトピー連接脈体}

% ホモトピー連接脈体の定義を復習する.

% \begin{definition}[ホモトピー連接脈体]
%   $\C$を単体的圏とする. 
%   単体的集合$\tilde{N}(\C)$を任意の$n \geq 0$に対して次のように定義し, $\tilde{N}(\C)$を$\C$のホモトピー連接脈体(homotopy coherence nerve)という. 
%   \begin{align*}
%     \tilde{N}(\C)_n := \Hom_\sSetCat(F_\ast[n],\C)
%   \end{align*}
%   ここで, $F_\ast[n]$は圏$[n]$のfree resolutionである. 
%   この対応は関手$\tilde{N} : \sSetCat \to \sSet$を定め, 関手$\tilde{N}$もホモトピー連接脈体という. 
% \end{definition}

% \begin{definition}[剛化関手]
%   普遍随伴の一般論より, ホモトピー連接脈体は左随伴$\mathfrak{C} : \sSet \to \sSetCat$を持つ. 
%   この関手$\mathfrak{C}$を剛化関手(rigidification functor)という. 
%   % https://q.uiver.app/#q=WzAsMyxbMCwxLCJcXERlbHRhIl0sWzEsMSwiXFxTQyJdLFswLDAsIlxcc1NldCJdLFswLDFdLFswLDIsIuOCiCJdLFsxLDIsIlxcdGlsZGV7Tn0iLDAseyJvZmZzZXQiOi0xfV0sWzIsMSwiXFxtYXRoZnJha3tDfSIsMCx7Im9mZnNldCI6LTF9XV0=
%   \[\begin{tikzcd}
%     \sSet \\
%     \Delta & \sSetCat
%     \arrow[from=2-1, to=2-2]
%     \arrow["{よ}", from=2-1, to=1-1]
%     \arrow["{\tilde{N}}", shift left, from=2-2, to=1-1]
%     \arrow["{\mathfrak{C}}", shift left, from=1-1, to=2-2]
%   \end{tikzcd}\]
% \end{definition}

% \begin{remark}
%   擬圏はup to homotopyで合成が定義されるような単体的圏と考えられる. 
%   剛化関手$\mathfrak{C}$はこのような擬圏を単体的圏に送るので, 剛化(rigidification)とみなせる. 
% \end{remark}

% \begin{lemma} \label{prop:C(Delta^n)_is_F_ast[n]}
%   任意の$n \geq 0$に対して, 単体的圏の同値$\mathfrak{C}(\Delta[n]) \cong F_\ast[n]$が成立する. 
% \end{lemma}

% \begin{proof}
%   単体的圏$\C$に対して, ホモトピー連接脈体$\tilde{N}(\C)$は単体的集合である. 
%   Yonedaの補題より, 次の同型が存在する. 
%   \begin{align*}
%     \tilde{N}(\C)_n \cong \Hom_\sSet(\Delta[n],\tilde{N}(\C))
%   \end{align*}
%   つまり, 次の同型が存在する. 
%   \begin{align*}
%     \Hom_\sSetCat(F_\ast[n],\C) \cong \Hom_\sSet(\Delta[n],\tilde{N}(\C))
%   \end{align*}
%   ($\mathfrak{C},\tilde{N}$)は随伴なので, 
%   \begin{align*}
%     \mathfrak{C}(\Delta[n]) \cong F_\ast[n]
%   \end{align*}
%   が成立する. 
% \end{proof}

% よって, $\mathfrak{C}$を調べるために, まずは単体的圏$F_\ast[n]$について考える. 

% $0 \leq i,j \leq n$とする. 
% (途中)

% \begin{lemma}
%   任意の$n \geq 0$に対して, 単体的圏の同型$P = F_\ast[n]$が成立する. 
% \end{lemma}

% \cref{prop:C(Delta^n)_is_F_ast[n]}より, $P \cong \mathfrak{C}(\Delta[n])$が成立する.
% よって, 次に任意の単体的集合$K$に対する剛化$\mathfrak{C}(K)$を考える. 
% 任意の単体的集合は$\Delta[n]$の余極限で表せる.
% \begin{align*}
%   K \cong \colim_{\Delta[n] \to K} \Delta[n]
% \end{align*}
% $\mathfrak{C}$は左随伴なので, 余極限と交換する. 
% よって, 次の単体的圏の同型が成立する. 
% \begin{align*}
%   \mathfrak{C}(K) \cong \colim_{\Delta[n] \to K} \mathfrak{C}(\Delta[n])
% \end{align*}

% \subsection{単体的集合のネックレス}

% 剛化関手$\mathfrak{C}$の像である単体的圏の射空間を考えるために, 単体的集合のネックレスという概念を用いる. 
% ネックレスは$i$番目の単体的集合における最後の点と, $i+1$番目の単体的集合における最初の点を繋げるような単体的集合である. 

% \begin{definition}[ネックレス]
%   任意の$0 \leq i \leq k$に対して, $n_i \geq 0$とする. 
%   $\Delta[n_0],\Delta[n_1],\cdots,\Delta[n_k]$を単体的集合とする. 
%   各$i$に対して, $\Delta[n_i]$の終点と$\Delta[n_{i+1}]$の始点を同一視することで, 単体的集合 
%   \begin{align*}
%     \Delta[n_0] \vee \Delta[n_1] \vee \cdots \vee \Delta[n_k]
%   \end{align*}
%   が定まり, 単体的集合のネックレス(necklace)という. 
%   各$\Delta[n_i]$をネックレスのビーズ(bead)という. 
%   各ビーズの始点と終点をネックレスの節(joint)という.
% \end{definition}

% \begin{notation}
%   ネックレス$T$に対して, $V_T$を$T$の点の集合, $J_T (\subset V_T)$を$T$の節の集合とする. 
%   また, $\alpha_T$を$T$の始点, $\omega_T$を$T$の終点とする. 
% \end{notation}

% \begin{definition}[ネックレスの結合]
%   $S,T$を単体的集合のネックレスとする. 
%   このとき, $S$の終点$\omega_S$と$T$の始点$\alpha_T$を同一視することで, 単体的集合のネックレス
%   \begin{align*}
%     S \vee T 
%   \end{align*}
%   が定まり, この$S \vee T$を$S$と$T$の結合(connect)という.
% \end{definition}

% \begin{definition}[部分ネックレス]
  
% \end{definition}

% \begin{example}
%   任意の$n \geq 0$に対して, 単体的集合$\Delta[n]$はネックレス$\Delta[n]$自身のただ1つのビーズである. 
% \end{example}

% \begin{example} \label{example:necklace}
%   次のネックレス$T$を考える. 
%   % https://q.uiver.app/#q=WzAsNyxbMCwwLCJcXGJ1bGxldF97dl8xfSJdLFsxLDEsIlxcYnVsbGV0X3t2XzJ9Il0sWzMsMSwiXFxidWxsZXRfe3ZfNH0iXSxbMiwwLCJcXGJ1bGxldF97dl8zfSJdLFs1LDEsIlxcYnVsbGV0X3t2XzZ9Il0sWzQsMCwiXFxidWxsZXRfe3ZfNX0iXSxbNiwwLCJcXGJ1bGxldF97dl83fSJdLFswLDFdLFsxLDJdLFsxLDNdLFszLDJdLFsyLDRdLFsyLDVdLFs1LDRdLFs0LDZdXQ==
%   \[\begin{tikzcd}
%     {\bullet_{v_1}} && {\bullet_{v_3}} && {\bullet_{v_5}} && {\bullet_{v_7}} \\
%     & {\bullet_{v_2}} && {\bullet_{v_4}} && {\bullet_{v_6}}
%     \arrow[from=1-1, to=2-2]
%     \arrow[from=2-2, to=2-4]
%     \arrow[from=2-2, to=1-3]
%     \arrow[from=1-3, to=2-4]
%     \arrow[from=2-4, to=2-6]
%     \arrow[from=2-4, to=1-5]
%     \arrow[from=1-5, to=2-6]
%     \arrow[from=2-6, to=1-7]
%   \end{tikzcd}\]
%   ここで, 三角形で表される$2$単体の中は埋められていることに注意.  
%   このとき, ネックレスの点の集合と節の集合は
%   \begin{align*}
%     V_T &= \{v_1,v_2,v_3,v_4,v_5,v_6,v_7\} \\
%     J_T &= \{v_1,v_2,v_4,v_6,v_7\}
%   \end{align*}
%   であり, $T$の始点$\alpha_T$は$v_1$, 終点$\omega_T$は$v_7$である. 
% \end{example}

% 単体的集合のネックレスは始点と終点という2点を持った単体的集合と考えられる. 
% このような2点を持った単体的集合のなす圏を$\sSet_{\ast,\ast}$と表す. 

% \begin{definition}[ネックレスの圏]
%   $\sSet_{\ast,\ast}$の充満部分圏$\Nec$を次のように定義し, ネックレスの圏(category of necklaces)という. 
%   \begin{itemize}
%     \item $\Nec$の対象は単体的集合のネックレス
%     \item $\Nec$の任意の対象$S,T$に対して, $\Nec$の射$f : S \to T$は$f(\alpha_S) = \alpha_T$と$f(\omega_S) = \omega_T$を満たす単体的集合の射
%   \end{itemize}
% \end{definition}

% \begin{remark}
%   任意のビーズが$1$単体であるネックレスを椎(spine)という. 
%   これは通常の単体的集合の椎の定義と等しい. 
%   任意のネックレス$T$に対して, $1$単体を繋げることで得られる椎$G[T]$と表す. 
%   例えば, \cref{example:necklace}の椎$G[T]$は次のように表せる. 
%   % https://q.uiver.app/#q=WzAsNyxbMCwwLCJcXGJ1bGxldF97dl8xfSJdLFsxLDEsIlxcYnVsbGV0X3t2XzJ9Il0sWzMsMSwiXFxidWxsZXRfe3ZfNH0iXSxbMiwwLCJcXGJ1bGxldF97dl8zfSJdLFs1LDEsIlxcYnVsbGV0X3t2XzZ9Il0sWzQsMCwiXFxidWxsZXRfe3ZfNX0iXSxbNiwwLCJcXGJ1bGxldF97dl83fSJdLFswLDFdLFsxLDNdLFszLDJdLFsyLDVdLFs1LDRdLFs0LDZdXQ==
%   \[\begin{tikzcd}
%     {\bullet_{v_1}} && {\bullet_{v_3}} && {\bullet_{v_5}} && {\bullet_{v_7}} \\
%     & {\bullet_{v_2}} && {\bullet_{v_4}} && {\bullet_{v_6}}
%     \arrow[from=1-1, to=2-2]
%     \arrow[from=2-2, to=1-3]
%     \arrow[from=1-3, to=2-4]
%     \arrow[from=2-4, to=1-5]
%     \arrow[from=1-5, to=2-6]
%     \arrow[from=2-6, to=1-7]
%   \end{tikzcd}\]
%   しかし, この構成$T \mapsto G[T]$は関手的でない. 
%   例えば, 一意なネックレスの射$\Delta[1] \to \Delta[2]$に対して, $G[\Delta[1]] = \Delta[1]$の像は$G[\Delta[2]]$に含まれない. 
% \end{remark}

% \begin{remark}
%   ネックレス$T$に対して, $T$の点の集合$J_T$のなす単体を$\Delta[T]$と表す. 
%   例えば, \cref{example:necklace}の$J_T$は$7$点集合なので,$\Delta[T]$は$7$単体である. 
%   この構成$T \mapsto \Delta[T]$は関手的である. 
%   また, 任意のネックレス$T$に対して, 単体的集合の包含$G[T] \hookrightarrow T \hookrightarrow \Delta[T]$が得られる. 
% \end{remark}

% \begin{proposition}
%   $T$をネックレスとする. 
%   このとき, 単体的集合の射$G[T] \hookrightarrow T \hookrightarrow \Delta[T]$はJoyal同値である. 
% \end{proposition}

% \begin{proof}
%   任意の$n \geq 0$に対して, $T = \Delta[n] \vee \Delta[1]$の場合を考える. 
%   このとき, $\Delta[T] = \Delta[n+1]$である. 
%   まず, $T \hookrightarrow \Delta[T]$, つまり$\Delta[n] \vee \Delta[1] \hookrightarrow \Delta[n+1]$が内部角体の包含に沿ったpushoutでとることで得られる射の合成として表せることを示す. 
%   次のフィルトレーションを考える. 
%   \begin{align*}
%     \Delta[n] \vee \Delta[1] = \Delta[n+1]^0 \subset \Delta[n+1]^1 \subset \cdots \subset \Delta[n+1]^{n-2} \subset \Delta[n+1]^{n-1} = \Delta[n+1] 
%   \end{align*}
% \end{proof}

% \subsection{順序つき単体的集合}

% 単体的集合のネックレスをより理解するために, 順序付き単体的集合について考える. 
% 単体的集合のネックレスは順序付き単体的集合の例である. 

% \begin{remark} \label{rem:realation_on_simplicial_set}
%   $K$を単体的集合, $k,k'$を$K$の点とする. 
%   $k$から$k'$への向きづけられた$1$単体が存在する, つまり, あるネックレス$T$と単体的集合の射$f : T \to K$が存在して, $f(\alpha_T) = k$と$f(\omega_T)=k'$を満たすとき, $k \leq_K k'$と表す. 
%   この二項関係は対称律と推移律を満たすが, 反対称性を満たさない. 
%   例えば, 点$k,k'$の間の両方向に向きづけられた$1$単体が存在するような場合は, 反対称性を満たさない. 
% \end{remark}

% 上の二項関係$\leq_K$が反対称律を満たすような単体的集合のクラスを考える. 

% \begin{definition}[順序付き単体的集合]
%   $K$を単体的集合とする. 
%   $K$が次の条件を満たすとき, $K$は順序付き(ordered)であるという. 
%   このとき, $K$上の二項関係$\leq_K$を順序(order)という. 
%   \begin{itemize}
%     \item \cref{rem:realation_on_simplicial_set}で定義された$\leq_K$は反対称律を満たす. 
%     \item $K$の$n$単体$K_n$の任意の元$k$は次の点の列により定まる. 
%     \begin{align*}
%       k(0) \leq_K \cdots \leq_K k(n)
%     \end{align*}
%   \end{itemize}
% \end{definition}

% \begin{example} \label{example:Deltan_is_ordered}
%   任意の$n \geq 0$に対して, $\Delta[n]$は順序付きである.
% \end{example}

% \begin{example}
%   任意のネックレスは順序付き単体的集合である.
% \end{example}

% \begin{proof}
%   任意のネックレスは椎を持つ. 
%   ネックレスの任意の2点$k,k'$に対して, $k$と$k'$をつなぐ椎が存在するとき, $k \leq_K k'$と定義する. 
%   椎の定義より, この二項関係は反対称律を満たす. 
%   \cref{example:Deltan_is_ordered}より, ネックレスも順序付きである. 
% \end{proof}

% \begin{notation}
%   順序付き単体的集合と点における順序を保つ単体的集合の射のなす$\sSet$の部分圏を$\OrdsSet$と表す. 
% \end{notation}

% \begin{lemma}
%   $\OrdsSet$は有限直積で閉じる.
% \end{lemma}

% \begin{proof}
%   $\OrdsSet$が終対象を持ち, 任意のpullbackが存在することを示せばよい. 
%   まず, $\Delta[0]$は$\OrdsSet$における終対象である. 
%   次に, 順序付き単体的集合の図式$K \to M \gets L$を考える. 
%   $A$を$\sSet$におけるこの図式のpullbackとする. 
%   $A$における順序$(k,l) \leq_A (k',l')$を$k \leq_K k'$と$l \leq_L l'$を満たすときと定める.
%   $\leq_K$と$\leq_L$はともに反対象律を満たすので, $\leq_A$を反対象律を満たす. 
%   任意の$n \geq 0$に対して, $A$の$n$単体は
%   任意の$n \geq 0$に対して, $K$と$L$の$n$単体は点の集合により定まるので, $A$の$n$単体も点の集合により定まる.
% \end{proof}

% % ネックレスが順序付き単体的集合であることを用いて, $\Nec$におけるネックレスの射の持つ性質を見る.

% % \begin{remark}
% %   $f : S \to T$をネックレスの射とする. 
% %   ネックレスの射の定義より, $T$の節は$S$の節の像である. 
% %   $f$がepi射のとき, 
% % \end{remark}

% 次に, ネックレスの剛化を考える. 
% これは一般の単体的集合に対する剛化を考える上で役に立つ.

% \begin{notation}
%   $T$をネックレス, $a,b$を$T$の点とする. 
%   このとき, 
%   \begin{align*}
%     V_T(a,b) := \{v \in V_T ~|~ a \leq_T v \leq_T b\} \\
%     J_T(a,b) := \{v \in J_T ~|~ a \leq_T v \leq_T b\}
%   \end{align*}
%   とする.
% \end{notation}

% \begin{remark}
%   ネックレス$T$と$T$の2点$a,b$に対して, 点の集合を$V_T(a,b)$, 節の集合を$J_T(a,b)$とするような部分ネックレスが一意に存在する. 
%   この部分ネックレスのビーズを$B_0,\cdots,B_k$と表す. 
%   このとき, 任意の$0 \leq i \leq k$に対して, 包含$B_i \hookrightarrow T$が存在する. 
%   $a$が$T$の節でないとき, $B_0$は$T$のあるビーズの面となる. 
%   $b$が$T$の節でないときも同様である. 
% \end{remark}

% $j_i, j_{i+1}$を各$B_i$の節とする. 
% 包含$B_i \hookrightarrow T$と$\mathfrak{C}(T)$の射空間における合成は自然な射
% \begin{align*}
%   \Map_{\mathfrak{C}(B_0)}(a,j_1) \times \Map_{\mathfrak{C}(B_1)}(j_1,j_2) \times \cdots \Map_{\mathfrak{C}B_k}(j_k,b) \to \Map_{\mathfrak{C}(T)}(a,b)
% \end{align*}
% を定める.
% 各$B_i$は単体なので, \cref{prop:C(Delta^n)_is_F_ast[n]}より, $\Map_{\mathfrak{B_i}}(j_i,j_{i+1})$は具体的に表せる. 
% このとき, $\Map_{\mathfrak{C}(T)}(a,b)$を具体的に表すことを考える.

% \begin{remark}
%   $V_T(a,b)$の部分集合の集まりを包含によって半順序集合とみなした集合を$C_T(a,b)$と表す. 
%   $T$の任意の点$a,b,c$に対して, 部分集合の和集合をとる操作は
%   \begin{align*}
%     C_T(a,b) \times C_T(b,c) \to C_T(a,c)
%   \end{align*}
%   を定める. 
% \end{remark}

% この構成を$T_0$において全て考え, 半順序集合の脈体をとることで単体的圏が定まる. 

% \begin{definition}
%   $T$をネックレスとする. 
%   このとき, 単体的圏$NC_T$を次のように定義する. 
%   \begin{itemize}
%     \item $NC_T$の対象は$V_T$と同じ
%     \item $NC_T$の任意の対象$a,b$に対して, $\Map_{NC_T}(a,b)$は$(J_T \subset )T^0 \subset \cdots \subset T^n (\subset V_T)$により定まる. 
%   \end{itemize}
% \end{definition}

% \begin{proposition}
%   $T$をネックレスとする. 
%   このとき, 単体的圏の圏同型$\mathfrak{C}(T) = NC_T$が成立する.
% \end{proposition}

% \newpage


\section{$\sSpaceCSS$と$\RelCatBar$のQuillen同値} \label{sec:quillen_equiv_sSpaceCSS_and_RelCatBar}

\cref{sec:quillen_equiv_sSpaceCSS_and_RelCatBar}の目標は, 次の\cref{prop:quillen_equiv_sSpaceCSS_and_RelCatBar}を示すことである. 
これは\cite{BK11}で証明された. 

\begin{proposition} \label{prop:quillen_equiv_sSpaceCSS_and_RelCatBar}
  細分化相対化関手$K_\xi : \sSpaceCSS \to \RelCat$と細分化分類図式$N_\xi : \RelCat \to \sSpaceCSS$は, $\sSpaceCSS$と$\RelCatBar$のQuillen同値を定める.  
  % \begin{align*}
  %   K_\xi : \sSpaceCSS \rightleftarrows \RelCatBar : N_\xi
  % \end{align*}
  % https://q.uiver.app/#q=WzAsMixbMCwwLCJcXHNTcGFjZUNTUyJdLFsyLDAsIlxcc1NldEpveWFsIl0sWzAsMSwiS19cXHhpIiwwLHsib2Zmc2V0IjotMSwiY3VydmUiOi0xfV0sWzEsMCwiTl9cXHhpIiwwLHsib2Zmc2V0IjotMSwiY3VydmUiOi0xfV0sWzIsMywiIiwyLHsibGV2ZWwiOjEsInN0eWxlIjp7Im5hbWUiOiJhZGp1bmN0aW9uIn19XV0=
  \[\begin{tikzcd}
    \sSpaceCSS && \sSetJoyal
    \arrow[""{name=0, anchor=center, inner sep=0}, "{K_\xi}", shift left, curve={height=-6pt}, from=1-1, to=1-3]
    \arrow[""{name=1, anchor=center, inner sep=0}, "{N_\xi}", shift left, curve={height=-6pt}, from=1-3, to=1-1]
    \arrow["\dashv"{anchor=center, rotate=-90}, draw=none, from=0, to=1]
  \end{tikzcd}\]
\end{proposition}

この命題は次の命題の系として得られる. 

\begin{proposition} \label{prop:quillen_equiv_sSpaceReedy_and_RelCatBar}
  細分化相対化関手$K_\xi : \sSpace_n \to \RelCat$と細分化分類図式$N_\xi : \RelCat \to \sSpace_n$は, $\sSpaceReedy$と$\RelCatBar$のQuillen同値を定める.  
  % \begin{align*}
  %   K_\xi : \sSpaceReedy \rightleftarrows \RelCatBar : N_\xi
  % \end{align*}
  % https://q.uiver.app/#q=WzAsMixbMCwwLCJcXHNTcGFjZVJlZWR5Il0sWzIsMCwiXFxzU2V0Sm95YWwiXSxbMCwxLCJLX1xceGkiLDAseyJvZmZzZXQiOi0xLCJjdXJ2ZSI6LTF9XSxbMSwwLCJOX1xceGkiLDAseyJvZmZzZXQiOi0xLCJjdXJ2ZSI6LTF9XSxbMiwzLCIiLDIseyJsZXZlbCI6MSwic3R5bGUiOnsibmFtZSI6ImFkanVuY3Rpb24ifX1dXQ==
  \[\begin{tikzcd}
    \sSpace_n && \sSetJoyal
    \arrow[""{name=0, anchor=center, inner sep=0}, "{K_\xi}", shift left, curve={height=-6pt}, from=1-1, to=1-3]
    \arrow[""{name=1, anchor=center, inner sep=0}, "{N_\xi}", shift left, curve={height=-6pt}, from=1-3, to=1-1]
    \arrow["\dashv"{anchor=center, rotate=-90}, draw=none, from=0, to=1]
  \end{tikzcd}\]
\end{proposition}

\subsection{(細分化)分類図式と(細分化)相対化関手}

相対半順序集合を用いて分類図式を次のように定義する. 

\begin{definition}[分類図式] \label{def:classification_diagram_with_relative_poset}
  $\C$を相対圏とする. 
  単体的空間$N\C$を, 任意の$m,n \geq 0$に対して集合$(N\C)_{n,m}$を次のように定義することで定め, $N\C$を$\C$の分類図式(classification diagram)という.
  \footnote{
    分類図式は次のように表すことができる. (この定義で書かれることが多い.) 
    しかし, 後で見る細分化関手との合成を考えるときには, 上の定義の見方が重要である. 

    \begin{notation}
      $\C$を相対圏, $\D$を圏とする. 
      関手圏$\C^\D$において, 自然変換の各成分が$\C$のweak equivalenceである$\C^\D$の部分圏を$\we(\C^\D)$と表す. 
    \end{notation}

    \begin{definition}[分類図式]
      $\C$を相対圏とする. 
      単体的空間$N\C$を, 任意の$n \geq 0$に対して単体的集合を次のように定義することで定め, $N\C$を$\C$の分類図式(classification diagram)という.
      ここで右辺の$N$は通常の圏の脈体である. 
      \begin{align*}
        (N\C)_n := N(\we(\C^{[n]}))
      \end{align*}
    \end{definition}
  }
  \begin{align*}
    (N\C)_{n,m} := \Hom_\RelCat([n]_\min \times [m]_\max, \C)
  \end{align*}
\end{definition}

\begin{remark}
  構成$\C \mapsto N\C$は関手$N : \RelCat \to \sSpace$を定める. 
  この関手$N$も分類図式(classification diagram)という. 
\end{remark}

\begin{definition}[相対化関手]
  普遍随伴の一般論より, 分類図式$N$は左随伴$K : \sSpace \to \RelCat$を持つ. 
  この関手$K$を相対化関手(relativization functor)
  \footnote{
    この用語は一般的には(英語を含めて)用いられておらず, 本稿独自の言葉である.
    のちに定義される細分化分類図式や細分化相対関手も同様である. 
  }
  という. 
\end{definition}

相対化関手を具体的に書き下すと次のようになる. 

\begin{remark}
  任意の$m,n \geq 0$において, $\Delta[n]^t \times \Delta[m]$に対して次が成立する. 
  \begin{align*}
    K(\Delta[n]^t \times \Delta[m]) = [n]_\min \times [m]_\max
  \end{align*}
\end{remark}

細分化関手との合成をとることで, 細分化分類図式と細分化相対化関手が定まる. 

\begin{definition}[細分化分類図式]
  $\C$を相対圏とする. 
  単体的空間$N_\xi\C$を, 任意の$m,n \geq 0$に対して集合$N_\xi\C_{n,m}$を次のように定義することで定め, $N_\xi\C$を$\C$の細分化分類図式(subdivision classification diagram)という.
  \begin{align*}
    (N_\xi\C)_{n,m} :=  \Hom_\RelCat(\xi([n]_\min \times [m]_\max), \C)
  \end{align*}
\end{definition}

\begin{definition}[細分化相対化関手]
  普遍随伴の一般論より, 細分化分類図式$N_\xi$は左随伴$K_\xi : \sSpace \to \RelCat$を持つ.
  この関手$K_\xi$を細分化相対化関手(subdivision relativization functor)という. 
\end{definition}

細分化相対化関手を具体的に書き下すと次のようになる. 

\begin{remark} \label{rem:subdivision_relativization_functor}
  任意の$m,n \geq 0$において, $\Delta[n]^t \times \Delta[m] \in \sSpace$に対して, 
  \begin{align*}
    K_\xi(\Delta[n]^t \times \Delta[m]) := \xi([n]_\min \times [m]_\max)
  \end{align*}
\end{remark}

\subsection{分類図式と細分化分類図式}

分類図式と細分化分類図式の関係を見る.
 
\begin{remark}
  $\xi, \Id :\RelCat \to \RelCat$は自然変換$\pi : \xi \to \Id$を定める. 
  % ここで, $\pi$の各成分は射影関手で与えられる. 
\end{remark}

\begin{lemma} \label{prop:N_to_Nxi_is_Reedy_weak_equivalence}
  自然変換$\pi : \xi \to \Id$から定まる自然変換$\pi^\ast : N \to N_\xi$はvertical weak equivalenceである. 
\end{lemma}

\begin{proof}
  任意の相対圏$\C$と$n \geq 0$に対して, 単体的集合の射 
  \begin{align*}
    \pi^\ast_n : (N\C)_n \to (N_\xi\C)_n
  \end{align*}
  がKan weak equivalenceであることを示せばよい. 
  定義より, 
  \begin{align*}
    (N\C)_{n,m} &:= \Hom_\RelCat([n]_\min \times [m]_\max, \C) \\
    (N_\xi\C)_{n,m} &:= \Hom_\RelCat(\xi([n]_\min \times [m]_\max), \C)
  \end{align*}
  である. 
  ここで, 単体的空間$F_n\C$と$\overline{F}_n\C$をそれぞれ次のように定義する.
  \begin{align*}
    (F_n\C)_{m,p} &:= \Hom_\RelCat([n]_\min \times [m]_\max, \C^{[p]_\max}) \\
    (\overline{F}_n\C)_{m,p} &:= \Hom_\RelCat([n]_\min \times [m]_\max, \C^{[0]_\max}) 
  \end{align*}
  包含$[0]_\max \hookrightarrow [p]_\max$は単体的空間の射
  \begin{align*}
    \overline{F}_n\C \to F_n\C
  \end{align*}
  を定める. 
  同様に, 単体的空間$G_n\C$と$\overline{G}_n\C$をそれぞれ次のように定義する. 
  \begin{align*}
    (G_n\C)_{m,p} &:= \Hom_\RelCat(\xi([n]_\min \times [m]_\max), \C^{[p]_\max}) \\
    (\overline{G}_n\C)_{m,p} &:= \Hom_\RelCat(\xi([n]_\min \times [m]_\max), \C^{[0]_\max}) 
  \end{align*}
  同様に, 包含$[0]_\max \hookrightarrow [p]_\max$は単体的空間の射
  \begin{align*}
    \overline{F}_n\C \to F_n\C
  \end{align*}
  を定める. 
  積とべきの随伴性と自然変換の成分$\xi([n]_\min \times [m]_\max) \to [n]_\min \times [m]_\max$から, 次の射が定まる. 
  \begin{align*}
    (F_n\C)_{m,p} 
    &= \Hom_\RelCat([n]_\min \times [m]_\max, \C^{[p]_\max}) 
    \cong \Hom_\RelCat([p]_\max, \C^{[n]_\min \times [m]_\max}) \\
    &\to \Hom_\RelCat([p]_\max, \C^{\xi([n]_\min \times [m]_\max)}) 
    \cong \Hom_\RelCat(\xi([n]_\min \times [m]_\max), \C^{[p]_\max}) \\
    &= (G_n\C)_{m,p}
  \end{align*}
  \cref{prop:all_maps_is_homotopy_equivalence_in_RelCat}より, $\xi([n]_\min \times [m]_\max) \to [n]_\min \times [m]_\max$は強ホモトピックである. 
  \cref{prop:homotopic_compatible_power_in_RelCat}より, $\C^{[n]_\min \times [m]_\max} \to \C^{\xi([n]_\min \times [m]_\max)}$は強ホモトピックである. 
  よって, $m$を固定して$p$を動かすと, $(F_n\C)_{m} \to (G_n\C)_{m}$はKan weak equivalenceである. 
  また, 単体的集合の同型
  \begin{align*}
    (N\C)_n \cong \diag (\overline{F}_n\C), ~  (N_\xi\C)_n \cong \diag (\overline{G}_n\C)
  \end{align*}
  が成立する. 
  よって, $(N\C)_n \to (N_\xi\C)_n$は単体的集合のweak equivalenceである. 
\end{proof}

右随伴$N_\xi$がcofibrationを保つことを示す.

\begin{lemma}
  細分化分類図式$N_\xi : \RelCat \to \sSpace$はDwyer射を単体的空間のmono射にうつす. 
\end{lemma}

\begin{proof}
  $F : \C \to \D$をDwyer射とする. 
  まず, $F$が相対半順序集合の相対包含である場合を考える. 
  このとき, 任意の$m,n \geq 0$に対して, 集合の射
  \begin{align*}
    F^\ast: (N_\xi\C)_{n,m} \to (N_\xi\D)_{n,m}
  \end{align*}
  がmono射であることを示せばよい. 
  つまり, 次の射が集合のmono射であることを示せばよい.
  \begin{align*}
    F^\ast : \Hom_\RelCat(\xi([n]_\min \times [m]_\max), \C) \to  \Hom_\RelCat(\xi([n]_\min \times [m]_\max), \D)
  \end{align*}
  ここで, $F$は相対包含なので, $F^\ast$は集合のmono射である. 
\end{proof}

\begin{proposition}
  細分化分類図式$N_\xi : \RelCat \to \sSpace$は$\RelCat$の強ホモトピックな相対関手を$\sSpace$のホモトピックな単体的空間の射にうつす. 
\end{proposition}

\begin{proof}
  $F,G : \C \to \D$を相対関手, $H :  \C \times [1]_\max \to \D$を$F$から$G$への強ホモトピーとする. 
  $H$に$N_\xi$を作用させると, 単体的空間の射
  \begin{align*}
    N_\xi(\C \times [1]_\max) \to N_\xi(\D)
  \end{align*}
  を得る. 
  $N_\xi$は右随伴なので直積を保つ. 
  \begin{align*}
    N_\xi(\C) \times N_\xi[1]_\max \cong N_\xi(\C \times [1]_\max)
  \end{align*}
  また, 自然な単体的空間の射$\Delta[1]^t \to N_\xi[1]_\max$は単体的空間の射
  \begin{align*}
    N_\xi(\C) \times \Delta[1]^t \to N_\xi(\C) \times N_\xi[1]_\max
  \end{align*}
  を定める. 
  これらの合成
  \begin{align*}
    N_\xi(\C) \times \Delta[1]^t \to N_\xi(\D)
  \end{align*}
  は$N_\xi(\C)$から$N_\xi(\D)$へのホモトピーである.
\end{proof}  

単体的空間のホモトピー同値がvertical weak equivalenceであることと合わせると, 次の系を得る. 

\begin{corollary}
  細分化分類図式$N_\xi : \RelCat \to \sSpace$は$\RelCat$のホモトピー同値を$\sSpace$のvertical weak equivalenceにうつす. 
\end{corollary}

\subsection{細分化相対化関手について}

\begin{notation}
  単体的空間$\Delta^\myop \times \Delta^\myop \to \Set$に対して, 次の記号を用いる. 
  \begin{align*}
    \Delta[n,m] &:= \Delta[n]^t \times \Delta[m] \\
    \partial \Delta[n,m] &:= \partial \Delta[n] \times \Delta[m]^t \cup \Delta[n] \times \partial \Delta[m]^t 
  \end{align*}
\end{notation}

\begin{proposition}
  $K_\xi : \sSpace \to \RelCat$は包含$\partial \Delta[n,m] \hookrightarrow \Delta[n,m]$をDwyer射にうつす. 
\end{proposition}

\begin{proof}
  $K_\xi(\partial \Delta[n,m]) \to K_\xi(\Delta[n,m])$がDwyer射であることを示せばよい. 
  \cref{rem:subdivision_relativization_functor}より, $K_\xi(\Delta[n,m]) = \xi([n]_\min \times [m]_\max)$である. 
  始細分化関手$\xi_i$に対しても関手$K_{\xi_i} : \sSpace \to \RelCat$を同様に定義する. 
  \begin{align*}
    K_{\xi_i}(\Delta[n,m]) := \xi_i([n]_\min \times [m]_\max)
  \end{align*}
  この関手も同様に右随伴を持つので, $K_{\xi_i}$は余極限を保つ. 
  また, 定義より次が成立する. 
  \begin{align*}
    \xi_t K_{\xi_i}(\Delta[n,m]) 
    = \xi_t \xi_i([n]_\min \times [m]_\max) 
    = \xi([n]_\min \times [m]_\max)
    = K_\xi(\Delta[n,m])
  \end{align*}

  次に, 相対関手$K_{\xi_i}(\partial \Delta[n,m]) \to K_{\xi_i}(\Delta[n,m])$が\cref{prop:sieve_induce_dwyer_map}の仮定を満たすことを示す. 
  この仮定を満たすとき, 
  \begin{align*}
    \xi_t K_{\xi_i}(\partial \Delta[n,m]) \to \xi_t K_{\xi_i}(\Delta[n,m]) = K_\xi(\Delta[n,m])
  \end{align*}
  はDwyer射である. 
  まず, 相対半順序集合$[n]_\min \times [m]_\max$の部分相対半順序集合$\P$を次のように定義する. 
  \begin{itemize}
    \item $\P$の対象は$[a]_\min \subset [n]_\min$と$[b]_\max \subset [m]_\max$を満たす$[a]_\min$と$[b]_\max$の直積$[a]_\min \times [b]_\max$
    \item $\P$の任意の対象$[a_1]_\min \times [b_1]_\max, [a_2]_\min \times [b_2]_\max$に対して, 射はそれぞれの包含が定める射$[a_1]_\min \times [b_1]_\max \to [a_2]_\min \times [b_2]_\max$
  \end{itemize}
  $\P$の対象$[a_1]_\min \times [b_1]_\max, [a_2]_\min \times [b_2]_\max$が$[a_1]_\min \cap [a_2]_\min \neq \emptyset$かつ$[b_1]_\max \cap [b_2]_\max \neq \emptyset$を満たすとき, 
  \begin{align*}
    ([a_1]_\min \times [b_1]_\max) \cap ([a_2]_\min \times [b_2]_\max)
    = ([a_1]_\min \cap [a_2]_\min) \cap ([b_1]_\max \cap [b_2]_\max)
  \end{align*}
  である. 
  つまり, $\P$は共通部分をとる操作で閉じる. 
  $\xi_i$や$\xi$を作用させても同じように共通部分をとる操作で閉じる. 
  \begin{align*}
    \xi_i([a_1]_\min \times [b_1]_\max) \cap \xi_i([a_2]_\min \times [b_2]_\max)
    &= \xi_i([a_1]_\min \cap [a_2]_\min) \cap \xi_i([b_1]_\max \cap [b_2]_\max) \\
    \xi([a_1]_\min \times [b_1]_\max) \cap \xi([a_2]_\min \times [b_2]_\max)
    &= \xi([a_1]_\min \cap [a_2]_\min) \cap \xi([b_1]_\max \cap [b_2]_\max)
  \end{align*}
  よって, $K_{\xi_i}(\partial \Delta[n,m])$は次のように表せる.
  \begin{align*}
    K_{\xi_i}(\partial \Delta[n,m]) = \bigcup_{a \neq n, b \neq m} \xi_i([a]_\min \times [b]_\max)
  \end{align*}
  $\P$の射$f : [a_1]_\min \times [b_1]_\max \to [a_2]_\min \times [b_2]_\max$に対して, $\xi_if$は相対包含かつ, $\xi_i([a_1]_\min \times [b_1]_\max)$は$\xi_i([a_2]_\min \times [b_2]_\max)$における余ふるいである. 
  よって, $K_{\xi_i}(\partial \Delta[n,m])$は$K_{\xi_i}(\Delta[n,m])$の余ふるいなので, \cref{prop:sieve_induce_dwyer_map}の仮定を満たす.

  以上より, $\xi_t K_{\xi_i}(\partial \Delta[n,m]) \to  K_\xi(\Delta[n,m])$はDwyer射である. 
  最後に, $K_\xi(\partial \Delta[n,m]) = \xi_t K_{\xi_i}(\partial \Delta[n,m])$であることを示す.
  (途中)
\end{proof}

左随伴がcofibrationを保つことを示す. 

\begin{proposition} \label{prop:K_xi_preserve_cofibration}
  $K_\xi : \sSpace \to \RelCat$は単体的空間のmono射を相対半順序集合のDwyer射にうつす.
\end{proposition}

\begin{proof}
  $X \hookrightarrow Y$を単体的空間のmono射とする. 
  単体的空間$Y$に対して, $i+j \leq n$を満たす非退化な$(i,j)$両単体のなす単体的空間を$Y_n (\subset Y)$と表す. 
  このとき, $X \hookrightarrow Y$はmono射なので
  \begin{align*}
    Y &= \bigcup_{n \geq 1} (Y^n \cup X) \\
    K_\xi Y &= \bigcup_{n \geq 1} K_\xi(Y^n \cup X)
  \end{align*}
  である. (途中)
\end{proof}

\begin{proposition} \label{prop:X_to_NKX_is_Reedy_weak_equivalence}
  随伴$(K_\xi \dashv N_\xi)$の単位射$\eta_\xi : \Id \to N_\xi K_\xi$はvertical weak equivalenceである.
\end{proposition}

\begin{proof}
  任意の単体的空間$X$に対して, $\eta_\xi : X \to  N_\xi K_\xi(X)$がvertical weak equivalenceであることを示せばよい. 
  まず, $X = \Delta[n,m]$のときを示す. 
  次の図式を考える. 
  % https://q.uiver.app/#q=WzAsNixbMCwwLCJcXERlbHRhW24sbV0iXSxbMSwwLCJOX1xceGkgS19cXHhpKFxcRGVsdGFbbixtXSkiXSxbMSwxLCJOX1xceGkgSyhcXERlbHRhW24sbV0pIl0sWzAsMSwiTksoXFxEZWx0YVtuLG1dKSJdLFsyLDAsIk5fXFx4aSBcXHhpKFtuXV9cXG1pbiB0aW1lcyBbbV1fXFxtYXgpIl0sWzIsMSwiTl9cXHhpKFtuXV9cXG1pbiB0aW1lcyBbbV1fXFxtYXgpIl0sWzAsMSwiXFxldGFfXFx4aSJdLFsxLDJdLFszLDIsIlxccGleXFxhc3QiLDJdLFswLDMsIlxcZXRhIiwyXSxbMSw0LCIiLDAseyJsZXZlbCI6Miwic3R5bGUiOnsiaGVhZCI6eyJuYW1lIjoibm9uZSJ9fX1dLFsyLDUsIiIsMCx7ImxldmVsIjoyLCJzdHlsZSI6eyJoZWFkIjp7Im5hbWUiOiJub25lIn19fV0sWzQsNSwiXFxwaV9cXGFzdCJdXQ==
  \[\begin{tikzcd}
    {\Delta[n,m]} & {N_\xi K_\xi(\Delta[n,m])} & {N_\xi \xi([n]_\min \times [m]_\max)} \\
    {NK(\Delta[n,m])} & {N_\xi K(\Delta[n,m])} & {N_\xi([n]_\min \times [m]_\max)}
    \arrow["{\eta_\xi}", from=1-1, to=1-2]
    \arrow[from=1-2, to=2-2]
    \arrow["{\pi^\ast}"', from=2-1, to=2-2]
    \arrow["\eta"', from=1-1, to=2-1]
    \arrow[Rightarrow, no head, from=1-2, to=1-3]
    \arrow[Rightarrow, no head, from=2-2, to=2-3]
    \arrow["{\pi_\ast}", from=1-3, to=2-3]
  \end{tikzcd}\]
  ここで, $\eta$は随伴$(K \dashv N)$の単位射である. 
  $N$は右随伴で極限を保つので, 
  \begin{align*}
    NK(\Delta[n,m]) = N([n]_\min \times [m]_\max) \cong N([n]_\min) \times N([m]_\max)
  \end{align*}
  であり, $\Delta[n,m]$と$NK(\Delta[n,m])$はvertical weak equivalenceである(らしい). 
  \cref{prop:N_to_Nxi_is_Reedy_weak_equivalence}より, $\pi^\ast : NK(\Delta[n,m]) \to N_\xi K(\Delta[n,m])$はvertical weak equivalenceである.
  \cref{prop:all_maps_is_homotopy_equivalence_in_RelCat}より, $\xi([n]_\min \times [m]_\max) \to [n]_\min \times [m]_\max$はweak equivalenceである. 
  $\RelCatBar$におけるweak equivalenceの定義より, $\pi_\ast : N_\xi \xi([n]_\min \times [m]_\max) \to N_\xi([n]_\min \times [m]_\max)$はvertical weak equivalenceである. 
  2-out-of-3より, $\eta_\xi : \Delta[n,m] \to N_\xi K_\xi (\Delta[n,m])$はvertical weak equivalenceである.

  次に, 一般の単体的空間$X$に対して示す. 
  \cref{prop:K_xi_preserve_cofibration}の証明で用いた記号を用いる. 
  $X=\cup_n X^n$として, 任意の$n \geq 0$に対して 
  \begin{align*}
    \eta_\xi : X^n \to N_\xi K_\xi X^n 
  \end{align*}
  がweak equivalenceであることを示せばよい. 

\end{proof}

\begin{corollary}
  単体的空間の射$f$がvertical weak equivalenceであることと, $N_\xi K_\xi f$がvertical weak equivalenceであることは同値である. 
\end{corollary}

\subsection{\cref{prop:quillen_equiv_sSpaceReedy_and_RelCatBar}の証明}

\cref{prop:quillen_equiv_sSpaceReedy_and_RelCatBar}を示す. 

\begin{proof}{\cref{prop:quillen_equiv_sSpaceReedy_and_RelCatBar}}
  まず, $K_\xi : \sSpace \rightleftarrows \RelCat : N_\xi$が随伴であることは定義(と普遍随伴の一般論)から従う. 

  次に, $(K_\xi \dashv N_\xi)$がQuillen随伴であることを示す.
  \cref{prop:K_xi_preserve_cofibration}より, 左随伴$K_\xi$はcofibrationを保つ. 
  右随伴がfibrationを保つことは, $\RelCatBar$におけるfibrationの定義から従う. 

  最後に, $(K_\xi,N_\xi)$がQuillen同値であることを示す.
  任意の単体的空間$X$と相対圏$Y$に対して, $K_\xi(X) \to Y$が$\RelCatBar$におけるweak equivalenceであることと, $X \to N_\xi(Y)$は$\sSpaceReedy$におけるweak equivalenceであることが同値であることを示せばよい. 
  $\RelCatBar$におけるweak equivalenceの定義より, $N_\xi K_\xi(X) \to N_\xi(Y)$がvertical weak equivalenceであることと, $X \to N_\xi(Y)$がvertical weak equivalenceであることが同値であることを示せばよい. 
  \cref{prop:X_to_NKX_is_Reedy_weak_equivalence}より, 次の図式を考えるとこの同値性は明らかである. 
  % https://q.uiver.app/#q=WzAsMyxbMCwxLCJOX1xceGkgS19cXHhpKFgpIl0sWzIsMSwiWCJdLFsxLDAsIk5fXFx4aShZKSJdLFsxLDAsIlxcZXRhX1xceGkiXSxbMCwyXSxbMSwyXV0=
  \[\begin{tikzcd}
    & {N_\xi(Y)} \\
    {N_\xi K_\xi(X)} && X
    \arrow["{\eta_\xi}", from=2-3, to=2-1]
    \arrow[from=2-1, to=1-2]
    \arrow[from=2-3, to=1-2]
  \end{tikzcd}\]
\end{proof}

Quillen同値であることは次のように示すこともできる. 

\begin{remark}
  $\sSpaceReedy$において, 任意の対象(単体的空間)はコファイブラントである. 
  よって, 任意の単体的空間$X$に対して, 単体的空間の射$X \to N_\xi (K_\xi(X))^f$がvertical weak equivalenceであることを示す.
  \cref{prop:X_to_NKX_is_Reedy_weak_equivalence}より, $X \to N_\xi K_\xi(X)$はvertical weak equivalenceである.
  $\RelCatBar$におけるファイブラント置換$K_\xi(X) \to  (K_\xi(X))^f$はtrivial fibrationである. 
  $\RelCatBar$におけるweak equivalenceの定義より, $N_\xi K_\xi(X) \to  N_\xi(K_\xi(X))^f$はvertical weak equivalenceである.
  weak equivalenceは合成で閉じるので, $X \to N_\xi (K_\xi(X))^f$はvertical weak equivalenceである.

  もう一方も, \cref{prop:X_to_NKX_is_Reedy_weak_equivalence}を用いて示すことができる. 
\end{remark}

\newpage
\appendix


\section{擬圏について} \label{sec:quasi_category}

\subsection{単体的集合の同値}

単体的集合の同値として, ホモトピー同値や弱ホモトピー同値, 弱圏同値がある. 
実際, Kan-Quillenモデル構造におけるweak equivalenceは弱ホモトピー同値であり, Joyalモデル構造におけるweak equivalenceは弱圏同値である. 
まずは, この2つのweak equivalenceの定義を見る. 
それぞれ同値な定義がたくさんあるが, ここでは2つのweak equivalenceを対比するように表せる定義を用いる. 

$\sSet$はCartesian閉であり, ホモトピー圏の対象の同型類をとる関手$\pi_0 : \sSet \to \Set$は有限直積を保つ. 
単体的集合$A,B$に対して
\begin{align*}
  \pi_0(A,B) := \pi_0(\Fun(A,B))
\end{align*}
とおく. 
単体的集合$A,B,C$に対して, 合成から定まる射$\Fun(B,C) \times \Fun(A,B) \to \Fun(A,C)$に関手$\pi_0$を作用させると, 
\begin{align*}
  \pi_0(B,C) \times \pi_0(A,B) \to \pi_0(A,C)
\end{align*}
を得る. 
これらから自然に圏が定まる. 

\begin{definition}
  圏$\sSet^{\pi_0}$を次のように定義する. 
  \begin{itemize}
    \item 対象は$\sSet$と同じ
    \item $\sSet^{\pi_0}$の任意の対象(単体的集合)に対して, $\Hom_{\sSet^{\pi_0}}(A,B) := \pi_0(A,B)$
  \end{itemize}
\end{definition}

\begin{definition}[弱ホモトピー同値]
  $f : A \to B$を単体的集合の射とする. 
  任意のKan複体$X$に対して, $f$の合成から定まる射
  \begin{align*}
    \pi_0(f,X) : \pi_0(B,X) \to \pi_0(A,X)
  \end{align*}
  が集合の同型射のとき, $f$を弱ホモトピー同値(weak homotopy equivalence)という.
\end{definition}

同様に, 基本群の対象の同型類をとる関手$\tau_0 : \sSet \to \Set$に対しても圏が定まる. 

\begin{definition}
  圏$\sSet^{\tau_0}$を次のように定義する. 
  \begin{itemize}
    \item 対象は$\sSet$と同じ
    \item $\sSet^{\tau_0}$の任意の対象(単体的集合)に対して, $\Hom_{\sSet^{\tau_0}}(A,B) := \tau_0(A,B)$
  \end{itemize}
\end{definition}

\begin{definition}[弱圏同値]
  $f : A \to B$を単体的集合の射とする. 
  任意の擬圏$X$に対して, $f$の合成から定まる射
  \begin{align*}
    \tau_0(f,X) : \tau_0(B,X) \to \tau_0(A,X)
  \end{align*}
  が集合の同型射のとき, $f$を弱圏同値(weak categorical equivalence)という.
\end{definition}

% \subsection{Joyal同値}

% $\sSet$上のJoyalモデル構造に出てくる同値(Joyal equivalence)を定義する. 

% \begin{notation}
%   2点とその間の一意な同型射からなる圏の脈体を$E$と表す. 
% \end{notation}

% \begin{definition}[$E$ホモトピー圏]
%   $X,Y$を単体的集合とする. 
%   集合の圏において, 次のcoequalizerで与えられる集合$[X,Y]_E$を$E$ホモトピー圏($E$-homotopy category)という. 
%   \begin{align*}
%     \Hom_\sSet(X \times E,Y) \rightrightarrows \Hom_\sSet(X,Y) \to [X,Y]_E
%   \end{align*}
% \end{definition}

% \begin{definition}[弱圏同値]
%   $f : X \to Y$を単体的集合の射とする. 
%   任意の擬圏$K$に対して, 集合の射
%   \begin{align*}
%     [Y,K]_E \to [X,K]
%   \end{align*}
%   が同型のとき, $f$を弱圏同値(weak categorical equivalence)という. 
% \end{definition}

% 弱圏同値は次のように定義することもできる. 

% \begin{definition}[基本圏]
%   普遍随伴の一般論より, 通常の脈体$N : \Cat \to \sSet$の左随伴$\tau_1 : \sSet \to \Cat$が得られる. 
%   この関手$\tau_1$を基本圏(fundamental category)をとる関手といい, 単体的集合$X$に対して$\tau_1X$を$X$のホモトピー圏(fundamental category)という. 
% \end{definition}

% \begin{remark} \label{rem:composition_of_sSettau0}
%   単体的集合$X$のホモトピー圏$\tau_1X$における対象の同型類の集合を$\tau_0X$と表す. 
%   この対応は関手$\tau_0 : \sSet \to \Set$を定める. 
%   $\tau_1 : \sSet \to \Cat$は有限直積を保つので, $\tau_0 : \sSet \to \Set$も有限直積を保つ. 
%   任意の単体的集合$A,B$に対して, 
%   \begin{align*}
%     \tau_0(A,B) := \tau_0(B^A)
%   \end{align*}
%   と表す.
%   任意の単体的集合$A,B,C$に対して,定まる単体的集合の射$C^B \times B^A \to C^A$に$\tau_0$を作用させると, 集合の射
%   \begin{align*}
%     \tau_0(B,C) \times \tau_0(A,B) \to \tau_0(A,C)
%   \end{align*}
%   が定まる. 
% \end{remark}

% \begin{definition}
%   圏$\sSet^{\tau_0}$を次のように定義する. 
%   \begin{itemize}
%     \item $\sSet^{\tau_0}$の対象は$\sSet$と同じ
%     \item $\sSet^{\tau_0}$の任意の対象$A,B$に対して, 
%     \begin{align*}
%       \Hom_{\sSet^{\tau_0}} := \tau_0(A,B) = \tau_0(B^A)
%     \end{align*}
%     \item $\sSet^{\tau_0}$の合成は\cref{rem:composition_of_sSettau0}で得られる集合の射
%   \end{itemize}
% \end{definition}

% \begin{definition}[擬圏の圏同値]
%   $f : A \to B$を単体的集合の射とする.
%   $\tau_0 : \sSet \to \Set$で得られる射$\tau_0f : \tau_0X \to \tau_0Y$が$\sSet^{\tau_0}$における同型射のとき, $f$を単体的集合の圏同値(categorical equivalence)という.
%   $A,B$がともに擬圏のとき, 単体的集合の圏同値を擬圏の圏同値(categorical equivalence of quasi-categories)という. 
% \end{definition}

% \begin{definition}[Joyal同値]
%   $f : A \to B$を単体的集合の射とする.
%   任意の擬圏$X$に対して, 集合の射 
%   \begin{align*}
%     \tau_0(f,X) : \tau_0(B,X) \to \tau_0(A,X)
%   \end{align*}
%   が集合の同型射のとき, $f$をJoyal同値(Joyal equivalence)という.
%   \footnote{
%     Joyalは弱圏同値(weak categorical equivalence)と呼び, Lurieは単に圏同値(categorical equivalence)と呼んだ. 
%   }
% \end{definition}

% \begin{definition}[Joyalファイブレーション]
%   任意のmono射とJoyal同値に対してRLPを持つ単体的集合の射をJoyalファイブレーション(Joyal fibration)という.
%   \footnote{
%     Joyalは擬ファイブレーション(quasi-fibration)と呼んだ.
%   }
% \end{definition}

\newpage


% \section{完備Segal空間について} \label{sec:CSS}

% \subsection{単体的空間} \label{subsec:simplicial_space}

% 単体的集合には, 通常の圏の脈体(nerve)と位相空間の特異単体(singular functor)という2つの特別なクラスがある.
% 高次圏論では, この圏論とホモトピー論を同時に一般化した枠組みで考える必要がある. 
% \cref{subsec:simplicial_space}では, 2つの単体的集合のクラスを1つにまとめる方法として, 単体的空間(simplicial space)を考える. 

% \begin{definition}[単体的空間]
%   関手$\Delta^\myop \to \sSet$を単体的空間(simplicial space)という.
% \end{definition}

% \begin{notation}
%   単体的空間の圏$\Fun(\Delta^\myop,\sSet)$を$\sSpace$と表す. 
% \end{notation}

% \begin{remark} \label{rem:bisimplicial_set}
%   積-Hom随伴より, 次の圏同値が存在する. 
%   \begin{align*}
%     \sSpace = \Fun(\Delta^\myop,\sSet) = \Fun(\Delta^\myop, \Fun(\Delta^\myop,\Set)) \cong \Fun(\Delta^\myop \times \Delta^\myop, \Set)
%   \end{align*}
%   このことから, 単体的空間は両単体的集合(bisimplicial set)とも呼ばれる. 
% \end{remark}

% \begin{remark} \label{rem:diagram_of_simplicial_space}
%   \cref{rem:bisimplicial_set}より, 単体的空間は次のように表せる. 
%   ここで, $X_{i,j}$は集合であり, $X_{i,-}$や$X_{-,j}$は単体的集合である. 
%   % https://q.uiver.app/#q=WzAsMjQsWzAsMCwiWF97MCwwfSJdLFsxLDAsIlhfezEsMH0iXSxbMCwxLCJYX3swLDF9Il0sWzEsMSwiWF97MSwxfSJdLFswLDIsIlhfezAsMn0iXSxbMSwyLCJYX3sxLDJ9Il0sWzIsMCwiWF97MiwwfSJdLFsyLDEsIlhfezIsMX0iXSxbMiwyLCJYX3syLDJ9Il0sWzMsMCwiXFxjZG90cyJdLFs0LDAsIlhfey0sMH0iXSxbMywxLCJcXGNkb3RzIl0sWzQsMSwiWF97LSwxfSJdLFszLDIsIlxcY2RvdHMiXSxbNCwyLCJYX3stLDJ9Il0sWzAsMywiXFx2ZG90cyJdLFsxLDMsIlxcdmRvdHMiXSxbMiwzLCJcXHZkb3RzIl0sWzAsNCwiWF97MCwxfSJdLFsxLDQsIlhfezEsLX0iXSxbMiw0LCJYX3syLC19Il0sWzQsMywiXFx2ZG90cyJdLFszLDQsIlxcY2RvdHMiXSxbMywzLCJcXGRkb3RzIl0sWzAsMV0sWzEsMCwiIiwyLHsib2Zmc2V0IjoxfV0sWzEsMCwiIiwxLHsib2Zmc2V0IjotMX1dLFswLDJdLFsyLDNdLFsxLDNdLFsyLDAsIiIsMSx7Im9mZnNldCI6MX1dLFsyLDAsIiIsMSx7Im9mZnNldCI6LTF9XSxbMywxLCIiLDEseyJvZmZzZXQiOjF9XSxbMywxLCIiLDEseyJvZmZzZXQiOi0xfV0sWzMsMiwiIiwxLHsib2Zmc2V0IjoxfV0sWzMsMiwiIiwxLHsib2Zmc2V0IjotMX1dLFs0LDJdLFs0LDIsIiIsMSx7Im9mZnNldCI6LTJ9XSxbNSwzXSxbMiw0LCIiLDEseyJvZmZzZXQiOjF9XSxbMiw0LCIiLDEseyJvZmZzZXQiOi0xfV0sWzQsMiwiIiwxLHsib2Zmc2V0IjoyfV0sWzQsNV0sWzUsNCwiIiwxLHsib2Zmc2V0IjotMX1dLFszLDUsIiIsMSx7Im9mZnNldCI6MX1dLFszLDUsIiIsMSx7Im9mZnNldCI6LTF9XSxbNSwzLCIiLDEseyJvZmZzZXQiOjJ9XSxbNSwzLCIiLDEseyJvZmZzZXQiOi0yfV0sWzYsMV0sWzEsNiwiIiwxLHsib2Zmc2V0IjotMX1dLFsxLDYsIiIsMSx7Im9mZnNldCI6MX1dLFs2LDEsIiIsMSx7Im9mZnNldCI6Mn1dLFs2LDEsIiIsMSx7Im9mZnNldCI6LTJ9XSxbNywzXSxbMyw3LCIiLDEseyJvZmZzZXQiOjF9XSxbMyw3LCIiLDEseyJvZmZzZXQiOi0xfV0sWzcsMywiIiwxLHsib2Zmc2V0IjoyfV0sWzcsMywiIiwxLHsib2Zmc2V0IjotMn1dLFs2LDddLFs3LDYsIiIsMSx7Im9mZnNldCI6MX1dLFs3LDYsIiIsMSx7Im9mZnNldCI6LTF9XSxbOCw3XSxbNyw4LCIiLDEseyJvZmZzZXQiOjF9XSxbNyw4LCIiLDEseyJvZmZzZXQiOi0xfV0sWzgsNywiIiwxLHsib2Zmc2V0IjoyfV0sWzgsNywiIiwxLHsib2Zmc2V0IjotMn1dLFs4LDVdLFs1LDgsIiIsMSx7Im9mZnNldCI6MX1dLFs1LDgsIiIsMSx7Im9mZnNldCI6LTF9XSxbOCw1LCIiLDEseyJvZmZzZXQiOjJ9XSxbOCw1LCIiLDEseyJvZmZzZXQiOi0yfV0sWzE4LDE5XSxbMTksMTgsIiIsMSx7Im9mZnNldCI6MX1dLFsyMCwxOV0sWzE5LDIwLCIiLDEseyJvZmZzZXQiOjF9XSxbMTksMTgsIiIsMSx7Im9mZnNldCI6LTF9XSxbMTksMjAsIiIsMSx7Im9mZnNldCI6LTF9XSxbMjAsMTksIiIsMSx7Im9mZnNldCI6Mn1dLFsyMCwxOSwiIiwxLHsib2Zmc2V0IjotMn1dLFsxMCwxMl0sWzEyLDEwLCIiLDEseyJvZmZzZXQiOjF9XSxbMTIsMTAsIiIsMSx7Im9mZnNldCI6LTF9XSxbMTQsMTJdLFsxMiwxNCwiIiwxLHsib2Zmc2V0IjoxfV0sWzEyLDE0LCIiLDEseyJvZmZzZXQiOi0xfV0sWzE0LDEyLCIiLDEseyJvZmZzZXQiOjJ9XSxbMTQsMTIsIiIsMSx7Im9mZnNldCI6LTJ9XSxbNiw5XSxbOSw2LCIiLDEseyJvZmZzZXQiOjF9XSxbOSw2LCIiLDEseyJvZmZzZXQiOi0xfV0sWzYsOSwiIiwxLHsib2Zmc2V0IjoyfV0sWzYsOSwiIiwxLHsib2Zmc2V0IjotMn1dLFs5LDYsIiIsMSx7Im9mZnNldCI6M31dLFs5LDYsIiIsMSx7Im9mZnNldCI6LTN9XSxbNywxMV0sWzExLDcsIiIsMSx7Im9mZnNldCI6LTF9XSxbMTEsNywiIiwxLHsib2Zmc2V0IjoxfV0sWzcsMTEsIiIsMSx7Im9mZnNldCI6Mn1dLFs3LDExLCIiLDEseyJvZmZzZXQiOi0yfV0sWzExLDcsIiIsMSx7Im9mZnNldCI6M31dLFsxMSw3LCIiLDEseyJvZmZzZXQiOi0zfV0sWzgsMTNdLFsxMyw4LCIiLDEseyJvZmZzZXQiOjF9XSxbMTMsOCwiIiwxLHsib2Zmc2V0IjotMX1dLFs4LDEzLCIiLDEseyJvZmZzZXQiOjJ9XSxbOCwxMywiIiwxLHsib2Zmc2V0IjotMn1dLFsxMyw4LCIiLDEseyJvZmZzZXQiOjN9XSxbMTMsOCwiIiwxLHsib2Zmc2V0IjotM31dLFs5LDEwLCIiLDEseyJzdHlsZSI6eyJoZWFkIjp7Im5hbWUiOiJub25lIn19fV0sWzEwLDksIiIsMSx7Im9mZnNldCI6MSwic3R5bGUiOnsiaGVhZCI6eyJuYW1lIjoibm9uZSJ9fX1dLFsxMSwxMiwiIiwxLHsic3R5bGUiOnsiaGVhZCI6eyJuYW1lIjoibm9uZSJ9fX1dLFsxMiwxMSwiIiwxLHsib2Zmc2V0IjoxLCJzdHlsZSI6eyJoZWFkIjp7Im5hbWUiOiJub25lIn19fV0sWzEzLDE0LCIiLDEseyJzdHlsZSI6eyJoZWFkIjp7Im5hbWUiOiJub25lIn19fV0sWzE0LDEzLCIiLDEseyJvZmZzZXQiOjEsInN0eWxlIjp7ImhlYWQiOnsibmFtZSI6Im5vbmUifX19XSxbNCwxNV0sWzE1LDQsIiIsMSx7Im9mZnNldCI6MX1dLFsxNSw0LCIiLDEseyJvZmZzZXQiOi0xfV0sWzQsMTUsIiIsMSx7Im9mZnNldCI6Mn1dLFs0LDE1LCIiLDEseyJvZmZzZXQiOi0yfV0sWzE1LDQsIiIsMSx7Im9mZnNldCI6LTN9XSxbNSwxNl0sWzE2LDUsIiIsMSx7Im9mZnNldCI6MX1dLFsxNiw1LCIiLDEseyJvZmZzZXQiOi0xfV0sWzUsMTYsIiIsMSx7Im9mZnNldCI6Mn1dLFs1LDQsIiIsMSx7Im9mZnNldCI6MX1dLFs1LDE2LCIiLDEseyJvZmZzZXQiOi0yfV0sWzE2LDUsIiIsMSx7Im9mZnNldCI6M31dLFsxNiw1LCIiLDEseyJvZmZzZXQiOi0zfV0sWzE1LDQsIiIsMSx7Im9mZnNldCI6M31dLFsxNSwxOCwiIiwxLHsic3R5bGUiOnsiaGVhZCI6eyJuYW1lIjoibm9uZSJ9fX1dLFsxNSwxOCwiIiwxLHsib2Zmc2V0IjoxLCJzdHlsZSI6eyJoZWFkIjp7Im5hbWUiOiJub25lIn19fV0sWzE2LDE5LCIiLDEseyJzdHlsZSI6eyJoZWFkIjp7Im5hbWUiOiJub25lIn19fV0sWzE2LDE5LCIiLDEseyJvZmZzZXQiOjEsInN0eWxlIjp7ImhlYWQiOnsibmFtZSI6Im5vbmUifX19XSxbMTcsMjAsIiIsMSx7InN0eWxlIjp7ImhlYWQiOnsibmFtZSI6Im5vbmUifX19XSxbMTcsMjAsIiIsMSx7Im9mZnNldCI6MSwic3R5bGUiOnsiaGVhZCI6eyJuYW1lIjoibm9uZSJ9fX1dLFs4LDE3XSxbMTcsOCwiIiwxLHsib2Zmc2V0IjoxfV0sWzE3LDgsIiIsMSx7Im9mZnNldCI6LTF9XSxbOCwxNywiIiwxLHsib2Zmc2V0IjoyfV0sWzgsMTcsIiIsMSx7Im9mZnNldCI6LTJ9XSxbMTcsOCwiIiwxLHsib2Zmc2V0IjozfV0sWzE3LDgsIiIsMSx7Im9mZnNldCI6LTN9XSxbMjAsMjJdLFsyMywyMiwiIiwxLHsic3R5bGUiOnsiaGVhZCI6eyJuYW1lIjoibm9uZSJ9fX1dLFsyMywyMiwiIiwxLHsib2Zmc2V0IjoxLCJzdHlsZSI6eyJoZWFkIjp7Im5hbWUiOiJub25lIn19fV0sWzIzLDIxLCIiLDEseyJzdHlsZSI6eyJoZWFkIjp7Im5hbWUiOiJub25lIn19fV0sWzIzLDIxLCIiLDEseyJvZmZzZXQiOi0xLCJzdHlsZSI6eyJoZWFkIjp7Im5hbWUiOiJub25lIn19fV0sWzE0LDIxXSxbMjEsMTQsIiIsMSx7Im9mZnNldCI6MX1dLFsyMSwxNCwiIiwxLHsib2Zmc2V0IjotMX1dLFsxNCwyMSwiIiwxLHsib2Zmc2V0IjoyfV0sWzE0LDIxLCIiLDEseyJvZmZzZXQiOi0yfV0sWzIxLDE0LCIiLDEseyJvZmZzZXQiOjN9XSxbMjEsMTQsIiIsMSx7Im9mZnNldCI6LTN9XSxbMjIsMjAsIiIsMSx7Im9mZnNldCI6MX1dLFsyMiwyMCwiIiwxLHsib2Zmc2V0IjotMX1dLFsyMCwyMiwiIiwxLHsib2Zmc2V0IjoyfV0sWzIwLDIyLCIiLDEseyJvZmZzZXQiOi0yfV0sWzIyLDIwLCIiLDEseyJvZmZzZXQiOjN9XSxbMjIsMjAsIiIsMSx7Im9mZnNldCI6LTN9XV0=
%   \[\begin{tikzcd}
%     {X_{0,0}} & {X_{1,0}} & {X_{2,0}} & \cdots & {X_{-,0}} \\
%     {X_{0,1}} & {X_{1,1}} & {X_{2,1}} & \cdots & {X_{-,1}} \\
%     {X_{0,2}} & {X_{1,2}} & {X_{2,2}} & \cdots & {X_{-,2}} \\
%     \vdots & \vdots & \vdots & \ddots & \vdots \\
%     {X_{0,-}} & {X_{1,-}} & {X_{2,-}} & \cdots
%     \arrow[from=1-1, to=1-2]
%     \arrow[shift right, from=1-2, to=1-1]
%     \arrow[shift left, from=1-2, to=1-1]
%     \arrow[from=1-1, to=2-1]
%     \arrow[from=2-1, to=2-2]
%     \arrow[from=1-2, to=2-2]
%     \arrow[shift right, from=2-1, to=1-1]
%     \arrow[shift left, from=2-1, to=1-1]
%     \arrow[shift right, from=2-2, to=1-2]
%     \arrow[shift left, from=2-2, to=1-2]
%     \arrow[shift right, from=2-2, to=2-1]
%     \arrow[shift left, from=2-2, to=2-1]
%     \arrow[from=3-1, to=2-1]
%     \arrow[shift left=2, from=3-1, to=2-1]
%     \arrow[from=3-2, to=2-2]
%     \arrow[shift right, from=2-1, to=3-1]
%     \arrow[shift left, from=2-1, to=3-1]
%     \arrow[shift right=2, from=3-1, to=2-1]
%     \arrow[from=3-1, to=3-2]
%     \arrow[shift left, from=3-2, to=3-1]
%     \arrow[shift right, from=2-2, to=3-2]
%     \arrow[shift left, from=2-2, to=3-2]
%     \arrow[shift right=2, from=3-2, to=2-2]
%     \arrow[shift left=2, from=3-2, to=2-2]
%     \arrow[from=1-3, to=1-2]
%     \arrow[shift left, from=1-2, to=1-3]
%     \arrow[shift right, from=1-2, to=1-3]
%     \arrow[shift right=2, from=1-3, to=1-2]
%     \arrow[shift left=2, from=1-3, to=1-2]
%     \arrow[from=2-3, to=2-2]
%     \arrow[shift right, from=2-2, to=2-3]
%     \arrow[shift left, from=2-2, to=2-3]
%     \arrow[shift right=2, from=2-3, to=2-2]
%     \arrow[shift left=2, from=2-3, to=2-2]
%     \arrow[from=1-3, to=2-3]
%     \arrow[shift right, from=2-3, to=1-3]
%     \arrow[shift left, from=2-3, to=1-3]
%     \arrow[from=3-3, to=2-3]
%     \arrow[shift right, from=2-3, to=3-3]
%     \arrow[shift left, from=2-3, to=3-3]
%     \arrow[shift right=2, from=3-3, to=2-3]
%     \arrow[shift left=2, from=3-3, to=2-3]
%     \arrow[from=3-3, to=3-2]
%     \arrow[shift right, from=3-2, to=3-3]
%     \arrow[shift left, from=3-2, to=3-3]
%     \arrow[shift right=2, from=3-3, to=3-2]
%     \arrow[shift left=2, from=3-3, to=3-2]
%     \arrow[from=5-1, to=5-2]
%     \arrow[shift right, from=5-2, to=5-1]
%     \arrow[from=5-3, to=5-2]
%     \arrow[shift right, from=5-2, to=5-3]
%     \arrow[shift left, from=5-2, to=5-1]
%     \arrow[shift left, from=5-2, to=5-3]
%     \arrow[shift right=2, from=5-3, to=5-2]
%     \arrow[shift left=2, from=5-3, to=5-2]
%     \arrow[from=1-5, to=2-5]
%     \arrow[shift right, from=2-5, to=1-5]
%     \arrow[shift left, from=2-5, to=1-5]
%     \arrow[from=3-5, to=2-5]
%     \arrow[shift right, from=2-5, to=3-5]
%     \arrow[shift left, from=2-5, to=3-5]
%     \arrow[shift right=2, from=3-5, to=2-5]
%     \arrow[shift left=2, from=3-5, to=2-5]
%     \arrow[from=1-3, to=1-4]
%     \arrow[shift right, from=1-4, to=1-3]
%     \arrow[shift left, from=1-4, to=1-3]
%     \arrow[shift right=2, from=1-3, to=1-4]
%     \arrow[shift left=2, from=1-3, to=1-4]
%     \arrow[shift right=3, from=1-4, to=1-3]
%     \arrow[shift left=3, from=1-4, to=1-3]
%     \arrow[from=2-3, to=2-4]
%     \arrow[shift left, from=2-4, to=2-3]
%     \arrow[shift right, from=2-4, to=2-3]
%     \arrow[shift right=2, from=2-3, to=2-4]
%     \arrow[shift left=2, from=2-3, to=2-4]
%     \arrow[shift right=3, from=2-4, to=2-3]
%     \arrow[shift left=3, from=2-4, to=2-3]
%     \arrow[from=3-3, to=3-4]
%     \arrow[shift right, from=3-4, to=3-3]
%     \arrow[shift left, from=3-4, to=3-3]
%     \arrow[shift right=2, from=3-3, to=3-4]
%     \arrow[shift left=2, from=3-3, to=3-4]
%     \arrow[shift right=3, from=3-4, to=3-3]
%     \arrow[shift left=3, from=3-4, to=3-3]
%     \arrow[no head, from=1-4, to=1-5]
%     \arrow[shift right, no head, from=1-5, to=1-4]
%     \arrow[no head, from=2-4, to=2-5]
%     \arrow[shift right, no head, from=2-5, to=2-4]
%     \arrow[no head, from=3-4, to=3-5]
%     \arrow[shift right, no head, from=3-5, to=3-4]
%     \arrow[from=3-1, to=4-1]
%     \arrow[shift right, from=4-1, to=3-1]
%     \arrow[shift left, from=4-1, to=3-1]
%     \arrow[shift right=2, from=3-1, to=4-1]
%     \arrow[shift left=2, from=3-1, to=4-1]
%     \arrow[shift left=3, from=4-1, to=3-1]
%     \arrow[from=3-2, to=4-2]
%     \arrow[shift right, from=4-2, to=3-2]
%     \arrow[shift left, from=4-2, to=3-2]
%     \arrow[shift right=2, from=3-2, to=4-2]
%     \arrow[shift right, from=3-2, to=3-1]
%     \arrow[shift left=2, from=3-2, to=4-2]
%     \arrow[shift right=3, from=4-2, to=3-2]
%     \arrow[shift left=3, from=4-2, to=3-2]
%     \arrow[shift right=3, from=4-1, to=3-1]
%     \arrow[no head, from=4-1, to=5-1]
%     \arrow[shift right, no head, from=4-1, to=5-1]
%     \arrow[no head, from=4-2, to=5-2]
%     \arrow[shift right, no head, from=4-2, to=5-2]
%     \arrow[no head, from=4-3, to=5-3]
%     \arrow[shift right, no head, from=4-3, to=5-3]
%     \arrow[from=3-3, to=4-3]
%     \arrow[shift right, from=4-3, to=3-3]
%     \arrow[shift left, from=4-3, to=3-3]
%     \arrow[shift right=2, from=3-3, to=4-3]
%     \arrow[shift left=2, from=3-3, to=4-3]
%     \arrow[shift right=3, from=4-3, to=3-3]
%     \arrow[shift left=3, from=4-3, to=3-3]
%     \arrow[from=5-3, to=5-4]
%     \arrow[no head, from=4-4, to=5-4]
%     \arrow[shift right, no head, from=4-4, to=5-4]
%     \arrow[no head, from=4-4, to=4-5]
%     \arrow[shift left, no head, from=4-4, to=4-5]
%     \arrow[from=3-5, to=4-5]
%     \arrow[shift right, from=4-5, to=3-5]
%     \arrow[shift left, from=4-5, to=3-5]
%     \arrow[shift right=2, from=3-5, to=4-5]
%     \arrow[shift left=2, from=3-5, to=4-5]
%     \arrow[shift right=3, from=4-5, to=3-5]
%     \arrow[shift left=3, from=4-5, to=3-5]
%     \arrow[shift right, from=5-4, to=5-3]
%     \arrow[shift left, from=5-4, to=5-3]
%     \arrow[shift right=2, from=5-3, to=5-4]
%     \arrow[shift left=2, from=5-3, to=5-4]
%     \arrow[shift right=3, from=5-4, to=5-3]
%     \arrow[shift left=3, from=5-4, to=5-3]
%   \end{tikzcd}\]
% \end{remark}

% 単体的集合は単体的空間に2種類の方法で埋め込むことができる. 

% \begin{definition}[垂直埋め込み, 水平埋め込み]
%   関手
%   \begin{align*}
%     i_F : \Delta \times \Delta \to \Delta : ([n],[m]) \mapsto [n]
%   \end{align*}
%   は関手
%   \begin{align*}
%     i_F^\ast : \sSet \to \sSpace : i_F^\ast(X)_{k,l} \mapsto X_k
%   \end{align*}
%   を定める. 
%   この埋め込み$i_F^\ast$を垂直埋め込み(vertical embedding)という. 
  
%   関手
%   \begin{align*}
%     i_\Delta : \Delta \times \Delta \to \Delta : ([n],[m]) \mapsto [m]
%   \end{align*}
%   は関手
%   \begin{align*}
%     i_\Delta^\ast : \sSet \to \sSpace : i_F^\ast(X)_{k,l} \mapsto X_l
%   \end{align*}
%   を定める. 
%   この埋め込み$i_\Delta^\ast$を水平埋め込み(horizontal embedding)という. 
% \end{definition}

% 埋め込みを用いて, いくつかの代表的な単体的空間を定義する. 

% \begin{definition}[空間関手と標準的単体]
%   単体的空間$F(n)$を次のように定義し, $n$次空間関手($n$-th space functor)という. 
%   \begin{align*}
%     (F(n))_{k,l} := i_F^\ast(\Delta[n]) = \Hom_\Delta([k],[n])
%   \end{align*}
%   単体的空間$\Delta[n]$を次のように定義し, 標準的$n$単体(standard $n$-simplex)という.
%   \begin{align*}
%     (\Delta[n])_{k,l} := i_\Delta^\ast(\Delta[n]) = \Hom_\Delta([l],[n])
%   \end{align*}
% \end{definition}

% \begin{definition}[空間関手の境界]
%   単体的空間$\partial F(n)$を次のように定義し, $n$次空間関手の境界(boundary of the $n$-th space functor)という. 
%   \begin{align*}
%     \partial F(n) := i_F^\ast(\partial \Delta[n])
%   \end{align*}
% \end{definition}

% \begin{notation}
%   単体的集合$X_{n,-}$を単に$X_n$と表す. 
% \end{notation}

% \begin{remark} \label{rem:yoneda_for_sSpace}
%   Yonedaの補題より, 次の同型が存在する. 
%   \begin{align*}
%     \Map_\sSpace(F(n),X) \cong X_n
%   \end{align*}
%   特に, 次の2つの同型が存在する. 
%   \begin{align*}
%     &\Map_\sSpace(F(0),X)_l \cong \Map_\sSpace(\Delta^l,X) \cong X_{0,l} \\
%     &\Map_\sSpace(\partial F(0),X) \cong \Map_\sSpace(\emptyset,X) \cong \Delta[0] \cong \{\ast\}
%   \end{align*}
% \end{remark}

% \begin{definition}[べき乗]
%   単体的空間$X,Y \in \sSpace$に対して, 単体的空間$Y^X$を次のように定義する. 
%   \begin{align*}
%     (Y^X)_{n,l} := \Hom_\sSpace(F(n) \times \Delta^l \times X,Y)
%   \end{align*}
% \end{definition}

% \begin{lemma}
%   単体的空間の圏$\sSpace$はCartesian閉である. 
%   つまり, 次の同型が存在する. 
%   \begin{align*}
%     \Map_\sSpace(X \times Y,Z) \cong \Map_\sSpace(X,Z^Y)
%   \end{align*}
% \end{lemma}

% \subsection{Reedyファイブラント} \label{subsec:reedy_ファイブラント}

% 単体的空間の定義の動機づけにあるように, 無限圏のモデルには空間(特異単体)の性質を持つ必要がある.
% 単体的空間に空間の性質を特徴づけるものがReedyファイブラント条件である. 
% 実際, Reedyファイブラント単体的空間$X$に対して, $X_n$はKanファイブラント(つまり, 空間)となる. (\cref{lem:Xn_is_Kan_ファイブラント})

% \begin{definition}[Reedyファイブレーション]
%   $X \to Y$を単体的空間の射とする. 
%   任意の$n,l \geq 0$と$0 \leq i \leq n$に対して, 次の図式がリフトを持つとき, $f$をReedyファイブレーション(Reedy fibration)という. 
%   % https://q.uiver.app/#q=WzAsNCxbMCwwLCJcXHBhcnRpYWwgRihuKSBcXHRpbWVzIFxcRGVsdGFebCBcXGNvcHJvZF97XFxwYXJ0aWFsIEYobikgXFx0aW1lcyBcXExhbWJkYV5uX2l9IEYobikgXFx0aW1lcyBcXExhbWJkYV5uX2kiXSxbMSwwLCJYIl0sWzEsMSwiWSJdLFswLDEsIkYobikgXFx0aW1lcyBcXERlbHRhXmwiXSxbMCwxXSxbMSwyLCJmIl0sWzAsM10sWzMsMl0sWzMsMSwiIiwxLHsic3R5bGUiOnsiYm9keSI6eyJuYW1lIjoiZGFzaGVkIn19fV1d
%   \[\begin{tikzcd}
%     {\partial F(n) \times \Delta^l \coprod_{\partial F(n) \times \Lambda^l_i} F(n) \times \Lambda^l_i} & X \\
%     {F(n) \times \Delta^l} & Y
%     \arrow[from=1-1, to=1-2]
%     \arrow["f", from=1-2, to=2-2]
%     \arrow[from=1-1, to=2-1]
%     \arrow[from=2-1, to=2-2]
%     \arrow[dashed, from=2-1, to=1-2]
%   \end{tikzcd}\]
% \end{definition}

% \begin{definition}[Reedyファイブラント]
%   $X$を単体的空間とする.  
%   単体的空間の射$X \to \Delta[0]$がReedyファイブレーションのとき, $X$をReedyファイブラント(Reedy ファイブラント)という. 
% \end{definition}

% ReedyファイブラントはKanファイブレーションを用いて定義することができる. 

% \begin{definition}[Reedyファイブラント]
%   $X$を単体的空間とする.  
%   任意の$n,l \geq 0$と$0 \leq i \leq n$に対して, 次の単体的空間の射 
%   \begin{align*}
%     \Map_\sSpace(F(n),X) \to \Map_\sSpace(\partial F(n),X)
%   \end{align*}
%   がKanファイブレーションのとき, $X$をReedyファイブラントという.
% \end{definition}

% \begin{theorem}
%   任意の$n \geq 0$に対して, $F(n)$はReedyファイブラントである. 
% \end{theorem}

% \begin{proof}
%   次の図式のリフト$\alpha : F(k) \times \Delta^l \to F(n)$が存在することを言えばよい. 
%   % https://q.uiver.app/#q=WzAsMyxbMCwwLCJcXHBhcnRpYWwgRihrKSBcXHRpbWVzIFxcRGVsdGFebCBcXGNvcHJvZF97XFxwYXJ0aWFsIEYoaykgXFx0aW1lcyBcXExhbWJkYV5sX2l9IEYoaykgXFx0aW1lcyBcXExhbWJkYV5sX2kiXSxbMCwxLCJGKGspIFxcdGltZXMgXFxEZWx0YV5sIl0sWzEsMCwiRihuKSJdLFswLDFdLFsxLDIsIlxcYWxwaGEiLDIseyJzdHlsZSI6eyJib2R5Ijp7Im5hbWUiOiJkYXNoZWQifX19XSxbMCwyXV0=
%   \[\begin{tikzcd}
%     {\partial F(k) \times \Delta^l \coprod_{\partial F(k) \times \Lambda^l_i} F(k) \times \Lambda^l_i} & {F(n)} \\
%     {F(k) \times \Delta^l}
%     \arrow[from=1-1, to=2-1]
%     \arrow["\alpha"', dashed, from=2-1, to=1-2]
%     \arrow[from=1-1, to=1-2]
%   \end{tikzcd}\]
%   Yonedaの補題より, 
%   \begin{align*}
%     \Map_\sSpace(F(k) \times \Delta^l, F(n)) 
%     &\cong \Hom_{\Delta \times \Delta}({k} \times [l], [n] \times [0]) \\
%     &\cong \Hom_{\Delta \times \Delta}([k],[l]) \\
%     &\cong \Map_\sSpace(F(k),F(n))
%   \end{align*}
%   次の図式において, $\theta$と$\theta'$は一致するので, $\alpha := \theta \circ p_1$とすればよい. 
%   % https://q.uiver.app/#q=WzAsNSxbMSwwLCJcXHBhcnRpYWwgRihrKSBcXHRpbWVzIFxcRGVsdGFebCBcXGNvcHJvZF97XFxwYXJ0aWFsIEYoaykgXFx0aW1lcyBcXExhbWJkYV5sX2l9IEYoaykgXFx0aW1lcyBcXExhbWJkYV5sX2kiXSxbMSwxLCJGKGspIFxcdGltZXMgXFxEZWx0YV5sIl0sWzIsMCwiRihuKSJdLFsyLDEsIkYoaykiXSxbMCwwLCJGKGspIFxcdGltZXMgXFxMYW1iZGFebF9pIl0sWzAsMV0sWzEsMiwiXFxhbHBoYSIsMix7InN0eWxlIjp7ImJvZHkiOnsibmFtZSI6ImRhc2hlZCJ9fX1dLFswLDJdLFsxLDMsInBfMSIsMl0sWzMsMiwiXFx0aGV0YSIsMl0sWzQsMCwiIiwyLHsic3R5bGUiOnsidGFpbCI6eyJuYW1lIjoiaG9vayIsInNpZGUiOiJ0b3AifX19XSxbNCwyLCJcXHRoZXRhJyIsMCx7ImN1cnZlIjotM31dXQ==
%   \[\begin{tikzcd}
%     {F(k) \times \Lambda^l_i} & {\partial F(k) \times \Delta^l \coprod_{\partial F(k) \times \Lambda^l_i} F(k) \times \Lambda^l_i} & {F(n)} \\
%     & {F(k) \times \Delta^l} & {F(k)}
%     \arrow[from=1-2, to=2-2]
%     \arrow["\alpha"', dashed, from=2-2, to=1-3]
%     \arrow[from=1-2, to=1-3]
%     \arrow["{p_1}"', from=2-2, to=2-3]
%     \arrow["\theta"', from=2-3, to=1-3]
%     \arrow[hook, from=1-1, to=1-2]
%     \arrow["{\theta'}", curve={height=-18pt}, from=1-1, to=1-3]
%   \end{tikzcd}\]
% \end{proof}

% \begin{remark}
%   任意の$0 \leq i \leq n$における関手
%   \begin{align*}
%     i : [0] \to [n] : 0 \mapsto i
%   \end{align*}
%   は単体的空間$X$に対して, 単体的集合の射
%   \begin{align*}
%     i^\ast : X_n \to X_0
%   \end{align*}
%   を定める. 
%   また, $i$は次の単体的集合の射
%   \begin{align*}
%     (0^\ast, \cdots, n^\ast) : X_n \to X_0 \times \cdots \times X_0 \cong (X_0)^{n+1}
%   \end{align*}
%   を定める. 
% \end{remark}

% \begin{lemma} \label{lem:Xn_to_X0^n+1_is_Kan_fibration}
%   $X$をReedyファイブラントとする. 
%   任意の$n \geq 0$に対して, 単体的集合の射$X_n \to (X_0)^{n+1}$はKanファイブレーションである. 
% \end{lemma}

% \begin{proof}
%   次の図式のリフト$\Delta^k \to X_n$が存在することを言えばよい. 
%   % https://q.uiver.app/#q=WzAsNCxbMCwwLCJcXExhbWJkYV5rX2kiXSxbMSwwLCJYX24gXFxjb25nIFxcTWFwX1xcc1NwYWNlKEYobiksWCkiXSxbMCwxLCJcXERlbHRhXmsiXSxbMSwxLCIoWF8wKV57bisxfSBcXGNvbmcgXFxNYXBfXFxzU3BhY2UoXFxjb3Byb2Rfe24rMX1GKDApLFgpIl0sWzAsMV0sWzAsMiwiIiwyLHsic3R5bGUiOnsidGFpbCI6eyJuYW1lIjoiaG9vayIsInNpZGUiOiJ0b3AifX19XSxbMiwzXSxbMSwzXSxbMiwxLCIiLDEseyJzdHlsZSI6eyJib2R5Ijp7Im5hbWUiOiJkYXNoZWQifX19XV0=
%   \[\begin{tikzcd}
%     {\Lambda^k_i} & {X_n \cong \Map_\sSpace(F(n),X)} \\
%     {\Delta^k} & {(X_0)^{n+1} \cong \Map_\sSpace(\coprod_{n+1}F(0),X)}
%     \arrow[from=1-1, to=1-2]
%     \arrow[hook, from=1-1, to=2-1]
%     \arrow[from=2-1, to=2-2]
%     \arrow[from=1-2, to=2-2]
%     \arrow[dashed, from=2-1, to=1-2]
%   \end{tikzcd}\]
%   これは次の射
%   % https://q.uiver.app/#q=WzAsMixbMCwwLCJcXEhvbV9cXHNTZXQoXFxEZWx0YV5rLFxcTWFwX1xcc1NwYWNlKEYobiksWCkpIl0sWzAsMSwiXFxIb21fXFxzU2V0KFxcTGFtYmRhXmtfaSxcXE1hcF9cXHNTcGFjZShGKG4pLFgpKSBcXHRpbWVzX3tcXE1hcF9cXHNTcGFjZShcXExhbWJkYV5rX2ksXFxNYXBfXFxzU3BhY2UoXFxjb3Byb2Rfe24rMX1GKDApLFgpKX0gXFxIb21fXFxzU2V0KFxcRGVsdGFeayxcXE1hcF9cXHNTcGFjZShcXGNvcHJvZF97bisxfUYoMCksWCkpIl0sWzAsMV1d
%   \[\begin{tikzcd}
%     {\Hom_\sSet(\Delta^k,\Map_\sSpace(F(n),X))} \\
%     {\Hom_\sSet(\Lambda^k_i,\Map_\sSpace(F(n),X)) \times_{\Map_\sSpace(\Lambda^k_i,\Map_\sSpace(\coprod_{n+1}F(0),X))} \Hom_\sSet(\Delta^k,\Map_\sSpace(\coprod_{n+1}F(0),X))}
%     \arrow[from=1-1, to=2-1]
%   \end{tikzcd}\]
%   が全射であることと同値である. 
%   \cref{rem:bisimplicial_set}より, これは次の射
%   % https://q.uiver.app/#q=WzAsMixbMCwwLCJcXEhvbV9cXHNTcGFjZShGKG4pIFxcdGltZXMgXFxEZWx0YV5rLFgpIl0sWzAsMSwiXFxIb21fXFxzU3BhY2UoRihuKSBcXHRpbWVzIFxcTGFtYmRhXmtfaSxYKSBcXHRpbWVzX3tcXE1hcF9cXHNTcGFjZShcXGNvcHJvZF97bisxfUYoMCkgXFx0aW1lcyBcXExhbWJkYV5rX2ksWCl9IFxcSG9tX1xcc1NwYWNlKFxcY29wcm9kX3tuKzF9RigwKSBcXHRpbWVzIFxcRGVsdGFeayxYKSJdLFswLDFdXQ==
%   \[\begin{tikzcd}
%     {\Hom_\sSpace(F(n) \times \Delta^k,X)} \\
%     {\Hom_\sSpace(F(n) \times \Lambda^k_i,X) \times_{\Map_\sSpace(\coprod_{n+1}F(0) \times \Lambda^k_i,X)} \Hom_\sSpace(\coprod_{n+1}F(0) \times \Delta^k,X)}
%     \arrow[from=1-1, to=2-1]
%   \end{tikzcd}\]
%   が全射であることと同値である. 
%   これは次の図式のリフト$F(n) \times \Delta^k \to X$が存在することと同値である. 
%   % https://q.uiver.app/#q=WzAsMyxbMCwwLCIoRihuKSBcXHRpbWVzIFxcTGFtYmRhXmtfaSkgXFxjb3Byb2Rfe1xcY29wcm9kX3tuKzF9RigwKSBcXHRpbWVzIFxcTGFtYmRhXmtfaX0gXFx0aW1lcyBcXERlbHRhXmsiXSxbMCwxLCJGKG4pIFxcdGltZXMgXFxEZWx0YV5rIl0sWzEsMCwiWCJdLFswLDFdLFsxLDIsIiIsMCx7InN0eWxlIjp7ImJvZHkiOnsibmFtZSI6ImRhc2hlZCJ9fX1dLFswLDJdXQ==
%   \[\begin{tikzcd}
%     {(F(n) \times \Lambda^k_i) \coprod_{\coprod_{n+1}F(0) \times \Lambda^k_i} \times \Delta^k} & X \\
%     {F(n) \times \Delta^k}
%     \arrow[from=1-1, to=2-1]
%     \arrow[dashed, from=2-1, to=1-2]
%     \arrow[from=1-1, to=1-2]
%   \end{tikzcd}\]
%   (途中)
% \end{proof}

% \begin{lemma} \label{lem:Xn_is_Kan_ファイブラント}
%   $X$をReedyファイブラントとする. 
%   任意の$n \geq 0$に対して, $X_n$はKanファイブラントである. 
% \end{lemma}

% \begin{proof}
%   単体的空間$X$はReedyファイブラントなので, 
%   \begin{align*}
%     \Map_\sSpace(F(n),X) \to \Map_\sSpace(\partial F(n),X)
%   \end{align*}
%   はKanファイブレーションである. 
%   $n=0$のとき, \cref{rem:yoneda_for_sSpace}より, 
%   \begin{align*}
%     \Map_\sSpace(F(0),X) \cong X_0 \\
%     \Map_\sSpace(\partial F(0),X) \cong \Delta[0]
%   \end{align*}
%   なので, 
%   \begin{align*}
%     X_0 \to \Delta[0]
%   \end{align*}
%   はKanファイブレーションである. 
%   よって, $X_0$はKanファイブラントである.
%   \cref{lem:Xn_to_X0^n+1_is_Kan_fibration}より, $X_n \to (X_0)^{n+1}$はKanファイブレーションである.
%   Kanファイブラントは合成で閉じるので, 任意の$n \geq 0$に対して, $X_n$もKanファイブラントである.
% \end{proof}

% \subsection{Segal空間} \label{subsec:segal_space}

% 単体的空間のReedyファイブラント条件は単体的空間の図式における垂直成分が空間のようにふるまうことを意味している. 
% この章では, 単体的空間の図式における並行成分が圏の性質をもつようにふるまうための条件であるSegal空間を定義する. 
% Segal空間は$\sSet$におけるSegal条件の一般化である. 

% \begin{definition}[Segal空間]
%   $X$をReedyファイブラントとする. 
%   任意の$n \geq 2$に対して, 単体的集合の射
%   \begin{align*}
%     \varphi_n : X_n \to X_1 \times_{X_0} \cdots \times_{X_0} X_1
%   \end{align*}
%   が自明なKanファイブレーションのとき, $X$をSegal空間(Segal space)という. 
% \end{definition}

% \begin{remark}
%   Reedyファイブラント条件より, 単体的集合の射
%   \begin{align*}
%     \varphi_n : X_n \to X_1 \times_{X_0} \cdots \times_{X_0} X_1
%   \end{align*}
%   は自動的にKanファイブレーションである. 
% \end{remark}

% 単体的集合$X_1 \times_{X_0} \cdots \times_{X_0} X_1$は$F(n)$の単体的部分空間として表すことができる. 

% \begin{definition}[椎]
%   単体的空間$F(n)$に対して, $F(n)$の単体的部分空間
%   \begin{align*}
%     G(n) := F(1) \coprod_{F(0)} \cdots \coprod_{F(0)} F(1)
%   \end{align*}
%   を$F(n)$の椎(spine)という. 
% \end{definition}

% \begin{remark}
%   Yonedaの補題より, 次の同型が存在する. 
%   \begin{align*}
%     \Map_\sSpace(G(n),X) \cong X_1 \times_{X_0} \cdots \times_{X_0} X_1
%   \end{align*}
% \end{remark}

% Segal空間$X$は単体的空間なので, 空間$X_0,X_0,X_2,\cdots$を持つ. 
% Segal条件により, $X_0$を「点の空間」, $X_1$を「射の空間」, $X_2$を「合成の空間」のように思うことができる. 
% 具体的には, 次のように考えることができる. (\cref{sec:composition_in_Segal_space}でこの議論を厳密に考える.)

% \begin{remark} \label{rem:2_segal_condition}
%   $n=2$のSegal空間の条件は$\varphi_2 : X_2 \xrightarrow{\simeq} X_1 \times_{X_0} X_1$である. 
%   $X_2$は2単体の集まりなので, 2単体$\sigma \in X_2$は次のように表せる. 
%   \[
%     \begin{tikzpicture}[auto,->]
%       \node (x) at (0,0) {$x$};
%       \node (z) at (3,0) {$z$};
%       \node (y) at (1.5,2) {$y$};
%       \node (sigma) at (1.5,0.7) {$\sigma$};
%       \draw (x) -- node {$f$} (y);
%       \draw (y) -- node {$g$} (z);
%       \draw (x) -- (z);  
%     \end{tikzpicture}
%   \]
%   同様に, $X_1 \times_{X_0} X_1$は2つの合成可能な射の集まりとみなせて, 次のように表せる. 
%   \[
%     \begin{tikzpicture}[auto,->]
%       \node (x) at (0,0) {$x$};
%       \node (z) at (3,0) {$z$};
%       \node (y) at (1.5,2) {$y$};
%       \draw (x) -- node {$f$} (y);
%       \draw (y) -- node {$g$} (z);
%     \end{tikzpicture}
%   \]
%   Segal条件はこのような図式が次の2単体に拡張できることを意味している. 
%   \[
%     \begin{tikzpicture}[auto,->]
%       \node (x) at (0,0) {$x$};
%       \node (z) at (3,0) {$z$};
%       \node (y) at (1.5,2) {$y$};
%       \node (sigma) at (1.5,0.7) {$\sigma$};
%       \draw (x) -- node {$f$} (y);
%       \draw (y) -- node {$g$} (z);
%       \draw[dashed] (x) -- node[swap] {$h$} (z);  
%     \end{tikzpicture}
%   \]
%   よって, $h$を$gf$と表すことが多いが, $h$は$g$と$f$(と$\sigma$)に対して一意に定まらないことに注意. 
%   しかし, \cref{theorem:composition_is_homotopic}で$h$はホモトピーの違いを除いて一意であることが分かる.  
% \end{remark}

% \begin{remark} \label{rem:3_segal_condition}
%   $n=3$のSegal空間の条件は$\varphi_3 : X_3 \xrightarrow{\simeq} X_1 \times_{X_0} X_1 \times_{X_0} X_1$である. 
%   $X_3$は3単体の集まりで, 3単体は次のように表せる. 
%   \[
%     \begin{tikzpicture}[auto,->]
%       \node (x) at (0,0) {$x$};
%       \node (y) at (4,0.5) {$y$};
%       \node (z) at (3,-1) {$z$};
%       \node (w) at (2,3) {$w$};
%       \node (sigma) at (2.5,-0.3) {$\sigma$};
%       \node (gamma) at (3,1) {$\gamma$};
%       \draw (x) -- node {$f$} (y);
%       \draw (y) -- node {$g$} (z);
%       \draw (z) -- node {$h$} (w);
%       \draw (x) -- (z);
%       \draw (x) -- (w);
%       \draw (y) -- (w);
%     \end{tikzpicture}
%   \]
%   同様に, $X_1 \times_{X_0} X_1 \times_{X_0} X_1$は3つの合成可能な射の集まりとみなせて, 次のように表せる.
%   \[
%     \begin{tikzpicture}[auto,->]
%       \node (x) at (0,0) {$x$};
%       \node (y) at (4,0.5) {$y$};
%       \node (z) at (3,-1) {$z$};
%       \node (w) at (2,3) {$w$};
%       \draw (x) -- node {$f$} (y);
%       \draw (y) -- node {$g$} (z);
%       \draw (z) -- node {$h$} (w);
%     \end{tikzpicture}
%   \]
%   Segal条件はこのような図式が次の3単体に拡張できることを意味している.
%   \[
%     \begin{tikzpicture}[auto,->]
%       \node (x) at (0,0) {$x$};
%       \node (y) at (4,0.5) {$y$};
%       \node (z) at (3,-1) {$z$};
%       \node (w) at (2,3) {$w$};
%       \node (sigma) at (2.5,-0.3) {$\sigma$};
%       \node (gamma) at (3,1) {$\gamma$};
%       \draw (x) -- node {$f$} (y);
%       \draw (y) -- node {$g$} (z);
%       \draw (z) -- node {$h$} (w);
%       \draw[dashed] (x) -- node[swap] {$gf$} (z);
%       \draw[dashed] (x) -- node {$h(gf),~ (hg)f$}  (w);
%       \draw[dashed] (y) -- node[swap] {$hg$} (w);
%     \end{tikzpicture}
%   \]
%   よって, この3単体は合成可能な射の結合性を表している. 
%   2単体において合成は一意ではなかったが, 3単体において$(hg)f$と$h(gf)$は等しいことを意味している. 
% \end{remark}

% \cref{rem:2_segal_condition}と\cref{rem:3_segal_condition}で見たように, Segal条件は圏の脈体における状況と非常に似ている. 
% \cref{eg:NC_is_Segal_space}で見るように, 圏の脈体はSegal空間であることが分かる. 

% \begin{example} \label{eg:NC_is_Segal_space}
%   任意の圏$\C$に対して, 脈体$N\C$はSegal空間である. 
% \end{example}

% \begin{proof}
%   $N\C$が単体的集合のSegal条件を満たすことから従う.
% \end{proof}

% \begin{theorem}
%   $X$をReedyファイブラントとする. 
%   $X$がhomotopically constantのとき, $X$はSegal空間である.
% \end{theorem}

% % \begin{proof}
% %   $X$はReedyファイブラントなので, 次の単体的集合の射
% %   \begin{align*}
% %     \varphi_n : X_n \to X_1 \times_{X_0} \cdots \times_{X_0} X_1
% %   \end{align*}
% %   は自明なKanファイブレーションである.

% % \end{proof}

% \subsection{Segal空間における射の合成} \label{subsec:composition_in_Segal_space}

% Segal空間とはReedyファイブラント条件とSegal条件を満たすような単体的空間であった. 
% \cref{sec:segal_space}で見たように, Segal空間における射の合成は一意ではなかった (\cref{rem:2_segal_condition}).
% しかし, 射の合成に関する結合律は成立していた (\cref{rem:3_segal_condition}).
% この理由を考える.
% $X$をSegal空間とする. 

% \begin{definition}[単体的空間における対象]
%   $X$をSegal空間とする. 
%   $X_0$の点を$X$の対象(object)という. 
%   $X$の対象の集まりを$\Ob(X)$と表すと, $\Ob(X) = T_{0,0}$である. 
% \end{definition}

% \begin{definition}[合成の空間]
%   $X$の対象$x_0,\cdots,x_n$に対して, 単体的集合$\map_X(x_0,\cdots,x_n)$を次のpullbackで定義し, $\map_X(x_0,\cdots,x_n)$を合成の空間(space of composition)という.  
%   % https://q.uiver.app/#q=WzAsNCxbMCwwLCJcXG1hcF9YKHhfMCxcXGNkb3RzLHhfbikiXSxbMSwwLCJYX24iXSxbMSwxLCIoWF8wKV57bisxfSJdLFswLDEsIlxcRGVsdGFeMCJdLFswLDFdLFswLDNdLFszLDJdLFsxLDJdLFswLDIsIiIsMSx7InN0eWxlIjp7Im5hbWUiOiJjb3JuZXIifX1dXQ==
%   \[\begin{tikzcd}
%     {\map_X(x_0,\cdots,x_n)} & {X_n} \\
%     {\Delta[0]} & {(X_0)^{n+1}=X_0 \times \cdots \times X_0}
%     \arrow[from=1-1, to=1-2]
%     \arrow[from=1-1, to=2-1]
%     \arrow["{(x_0,\cdots,x_n)}"', from=2-1, to=2-2]
%     \arrow[from=1-2, to=2-2]
%     \arrow["\lrcorner"{anchor=center, pos=0.125}, draw=none, from=1-1, to=2-2]
%   \end{tikzcd}\]
% \end{definition}

% \begin{remark}
%   \cref{lem:Xn_to_X0^n+1_is_Kan_fibration}より, $X_n \to (X_0)^{n+1}$はKanファイブレーションである.
%   Kanファイブレーションはpullbackで閉じるので, $\map_X(x_0,\cdots,x_n) \to \Delta[0]$はKanファイブレーションである.
%   つまり, $\map_X(x_0,\cdots,x_n)$はKanファイブラントである.
% \end{remark}

% \begin{definition}[単体的空間における射]
%   合成の空間において, $n=1$のとき, 単体的集合$\map_X(x_0,x_1)$を射空間(mapping space)という. 
%   $\map_X(x_0,x_1)$の点を$X$の射(morphism)といい, $f : x_0 \to x_1$と表す. 
% \end{definition}

% \begin{remark} \label{rem:map_is_map_times_map}
%   $X$はSegal空間なので, $ X_n \to X_1 \times_{X_0} \cdots \times_{X_0} X_1$は自明なKanファイブレーションである.
%   また, 合成の空間の図式は次の図式に拡張できる. 
%   % https://q.uiver.app/#q=WzAsNSxbMCwwLCJcXG1hcF9YKHhfMCxcXGNkb3RzLHhfbikiXSxbMSwwLCJYX24iXSxbMSwxLCIoWF8wKV57bisxfSJdLFswLDEsIlxcRGVsdGFeMCJdLFsyLDAsIlhfMSBcXHRpbWVzX3tYXzB9IFxcdGltZXMgXFx0aW1lc197WF8wfSBYXzEiXSxbMCwxXSxbMCwzXSxbMywyLCIoeF8wLFxcY2RvdHMseF9uKSIsMl0sWzEsMl0sWzAsMiwiIiwxLHsic3R5bGUiOnsibmFtZSI6ImNvcm5lciJ9fV0sWzEsNCwiXFxjb25nIl0sWzQsMl1d
%   \[\begin{tikzcd}
%     {\map_X(x_0,\cdots,x_n)} & {X_n} & {X_1 \times_{X_0} \cdots \times_{X_0} X_1} \\
%     {\Delta[0]} & {(X_0)^{n+1}}
%     \arrow[from=1-1, to=1-2]
%     \arrow[from=1-1, to=2-1]
%     \arrow["{(x_0,\cdots,x_n)}"', from=2-1, to=2-2]
%     \arrow[from=1-2, to=2-2]
%     \arrow["\lrcorner"{anchor=center, pos=0.125}, draw=none, from=1-1, to=2-2]
%     \arrow["\simeq", from=1-2, to=1-3]
%     \arrow[from=1-3, to=2-2]
%   \end{tikzcd}\]
% \end{remark}

% \begin{lemma} 
%   単体的集合の射
%   \begin{align*}
%     \map_X(x_0,\cdots,x_n) \to \map_X(x_0,x_1) \times \cdots \times \map_X(x_{n-1},x_n)
%   \end{align*}
%   は自明なKanファイブレーションである.
% \end{lemma}

% \begin{proof}
%   次の図式において, 下と全体の四角はpullbackである. 
%   % https://q.uiver.app/#q=WzAsNixbMCwwLCJcXGJ1bGxldFxcbWFwX1goeF8wLFxcY2RvdHMseF9uKSJdLFsxLDAsIlhfbiJdLFsxLDEsIlhfMSBcXHRpbWVzX3tYXzB9IFxcY2RvdHMgXFx0aW1lc197WF8wfSBYXzEiXSxbMSwyLCIoWF8wKV57bisxfSJdLFswLDEsIlxcbWFwX1goeF8wLHhfMSkgXFx0aW1lcyBcXGNkb3RzIFxcdGltZXMgXFxtYXBfWCh4X3tuLTF9LHhfbikiXSxbMCwyLCJcXERlbHRhXjAiXSxbMCwxXSxbMSwyLCJcXHNpbWVxIl0sWzIsM10sWzAsNF0sWzQsNV0sWzUsM10sWzQsMl1d
%   \[\begin{tikzcd}
%     {\bullet\map_X(x_0,\cdots,x_n)} & {X_n} \\
%     {\map_X(x_0,x_1) \times \cdots \times \map_X(x_{n-1},x_n)} & {X_1 \times_{X_0} \cdots \times_{X_0} X_1} \\
%     {\Delta[0]} & {(X_0)^{n+1}}
%     \arrow[from=1-1, to=1-2]
%     \arrow["\simeq", from=1-2, to=2-2]
%     \arrow[from=2-2, to=3-2]
%     \arrow[from=1-1, to=2-1]
%     \arrow[from=2-1, to=3-1]
%     \arrow[from=3-1, to=3-2]
%     \arrow[from=2-1, to=2-2]
%   \end{tikzcd}\]
%   よって, 上の四角もpullbackである. 
%   自明なKanファイブレーションはpullbackで閉じるので, 単体的集合の射
%   \begin{align*}
%     \map_X(x_0,\cdots,x_n) \to \map_X(x_0,x_1) \times \cdots \times \map_X(x_{n-1},x_n)
%   \end{align*}
%   は自明なKanファイブレーションである.
% \end{proof}

% \begin{remark}
%   $x,y,z$を$X$の任意の対象とする. 
%   合成の空間の定義から, 次の図式が得られる. 
%   % https://q.uiver.app/#q=WzAsMyxbMCwwLCJcXG1hcF9YKHgseSx6KSJdLFsxLDAsIlxcbWFwX1goeCx6KSJdLFswLDEsIlxcbWFwX1goeCx5KSBcXHRpbWVzIFxcbWFwX1goeSx6KSJdLFswLDEsImRfMSJdLFswLDIsIlxcY29uZyIsMl1d
%   \[\begin{tikzcd}
%     {\map_X(x,y,z)} & {\map_X(x,z)} \\
%     {\map_X(x,y) \times \map_X(y,z)}
%     \arrow["{d_1}", from=1-1, to=1-2]
%     \arrow["\simeq"', from=1-1, to=2-1]
%   \end{tikzcd}\]
%   2つの射$f \in \map_X(x,y), g \in \map_X(y,z)$に対して, $\map_X(x,y,z)$の単体的部分集合$\Comp(f,g)$を次のpullbackで定義する. 
%   % https://q.uiver.app/#q=WzAsNSxbMSwwLCJcXG1hcF9YKHgseSx6KSJdLFsyLDAsIlxcbWFwX1goeCx6KSJdLFsxLDEsIlxcbWFwX1goeCx5KSBcXHRpbWVzIFxcbWFwX1goeSx6KSJdLFswLDAsIlxcQ29tcChmLGcpIl0sWzAsMSwiXFxEZWx0YV4wIl0sWzAsMSwiZF8xIl0sWzAsMiwiXFxjb25nIiwyXSxbMywwLCIiLDIseyJzdHlsZSI6eyJ0YWlsIjp7Im5hbWUiOiJob29rIiwic2lkZSI6InRvcCJ9fX1dLFszLDRdLFs0LDJdLFszLDIsIiIsMSx7InN0eWxlIjp7Im5hbWUiOiJjb3JuZXIifX1dXQ==
%   \[\begin{tikzcd}
%     {\Comp(f,g)} & {\map_X(x,y,z)} & {\map_X(x,z)} \\
%     {\Delta[0]} & {\map_X(x,y) \times \map_X(y,z)}
%     \arrow["{d_1}", from=1-2, to=1-3]
%     \arrow["\simeq", from=1-2, to=2-2]
%     \arrow[hook, from=1-1, to=1-2]
%     \arrow["\simeq", from=1-1, to=2-1]
%     \arrow[from=2-1, to=2-2]
%     \arrow["\lrcorner"{anchor=center, pos=0.125}, draw=none, from=1-1, to=2-2]
%   \end{tikzcd}\]
%   自明なKanファイブレーションはpullbackで閉じるので, $\Comp(f,g) \to \Delta[0]$は自明なKanファイブレーションである. 
%   つまり, $\Comp(f,g)$は可縮なKanファイブラントである. 
%   よって, 射の合成は可縮な空間の違いを除いて一意に定めることができる.
% \end{remark}

% \begin{example}
%   $X$を任意の圏$\C$の脈体とする. 
%   $X$の任意の対象$x,y,z$と射$f:x \to y, g: y \to z$に対して, $\Comp(f,g) \cong \Delta[0]$である. 
%   % このとき, 2つの射がホモトピックであることと, 2つの射が等しいことは同値である. 
%   % 更に, 射がホモトピー同値であることと, 射が同型であることは同値である. 
% \end{example}

% \begin{proof}
%   次の図式において, 同型がpullbackで閉じることから従う. 
%   % https://q.uiver.app/#q=WzAsNCxbMCwwLCJcXENvbXAoZixnKSJdLFsxLDAsIk4oXFxDKV8yIl0sWzEsMSwiTihcXEMpXzEgXFx0aW1lc197TihcXEMpXzB9IE4oXFxDKV8xIl0sWzAsMSwiXFxEZWx0YV4wIl0sWzAsMV0sWzEsMiwiXFxjb25nIl0sWzAsM10sWzMsMl0sWzAsMiwiIiwxLHsic3R5bGUiOnsibmFtZSI6ImNvcm5lciJ9fV1d
%   \[\begin{tikzcd}
%     {\Comp(f,g)} & {N\C_2} \\
%     {\Delta[0]} & {N\C_1 \times_{N\C_0} N\C_1}
%     \arrow[from=1-1, to=1-2]
%     \arrow["\cong", from=1-2, to=2-2]
%     \arrow[from=1-1, to=2-1]
%     \arrow[from=2-1, to=2-2]
%     \arrow["\lrcorner"{anchor=center, pos=0.125}, draw=none, from=1-1, to=2-2]
%   \end{tikzcd}\]
% \end{proof}

% \begin{definition}[単体的空間における恒等射]
%   任意の$X$の対象$x$に対して, $s_0 : X_0 \to X_1$の$x$の像$s_0(x)$を$x$の恒等射(identity map)といい, $\id_x$と表す.
%   \begin{align*}
%     s_0 : X_0 \to X_1 : x \mapsto \id_x
%   \end{align*}
% \end{definition}

% \subsection{ホモトピー同値}

% \cref{sec:composition_in_Segal_space}では主にSegal空間の圏論的な側面を見た. 
% この章では, Segal空間のもつホモトピー論的な性質に着目する.
% $X$をSegal空間とする. 

% \begin{definition}[単体的空間におけるホモトピック]
%   $x,y$を$X$の対象, $f,g \in \map_X(x,y)$を$X$の射とする.
%   単体的集合の射$f,g : \Delta[0] \to \map_X(x,y)$が単体的集合の意味でホモトピックであるとき, $f$と$g$はホモトピック(homotopic)であるといい, $f \sim g$と表す.  
% \end{definition}

% 射の合成はホモトピーの違いを除いて結合的かつ単位を持つ. 

% \begin{theorem} \label{theorem:composition_is_homotopic}
%   $x,y,z,w$を$X$の対象, $f \in \map_X(x,y), g \in \map_X(y,z), h \in \map_X(z,w)$を$X$の射とする. 
%   このとき, $h(gf) \sim (hg)f$かつ$f\id_x \sim \id_yf \sim f$である. 
% \end{theorem}

% \cref{theorem:composition_is_homotopic}より, 単体的空間の対して通常の圏が定まる. 

% \begin{definition}[単体的空間のホモトピー圏]
%   単体的空間$X$に対して, 通常の圏$\Ho(X)$を次のように定義し, $X$のホモトピー圏(homotopy category)という. 
%   \begin{itemize}
%     \item $\Ho(X)$の対象は$X$の対象と同じ
%     \item 任意の$x,y \in \Ho(X)$に対して, $\Hom_{\Ho(X)}(x,y)$は$x$から$y$への射のホモトピー類の空間
%     \begin{align*}
%       \Hom_{\Ho(X)}(x,y) := \pi_0(\map_X(x,y))
%     \end{align*}
%     \item 任意の$x,y,z \in \Ho(X)$に対して, 合成
%     \begin{align*}
%       \Hom_{\Ho(X)}(x,y) \times \Hom_{\Ho(X)}(y,z) \to \Hom_{\Ho(X)}(x,z)
%     \end{align*}
%     は
%     \begin{align*}
%       \map_X(x,y) \times \map_X(y,z) \to \map_X{\Ho(X)}(x,z)
%     \end{align*}
%     から定まる自然な対応
%   \end{itemize}
% \end{definition}

% \begin{theorem} \label{thrm:C_is_HoNC}
%   任意の圏$\C$は$\Ho(N\C)$と圏同型である.
% \end{theorem}

% \begin{proof}
%   $\Ho(N\C)$の対象は
%   \begin{align*}
%     \Ob\Ho(N\C) = \Ob(N\C) = N\C_0 = \Ob(\C)
%   \end{align*}
%   任意の$x,y \in \Ho(N\C)$に対して, 
%   \begin{align*}
%     \Hom_{\Ho(N\C)} = \pi_0(\Map_{N\C}(x,y)) = \pi_0(N\C_1 \times_{N\C_0 \times N\C_0} \ast)
%   \end{align*}
%   $N\C_1$と$N\C_0 \times N\C_0$は離散的なので, 
%   \begin{align*}
%     \pi_0(N\C_1 \times_{N\C_0 \times N\C_0} \ast) = N\C_1 \times_{N\C_0 \times N\C_0} \ast = \Hom_\C(x,y)
%   \end{align*}
%   よって, $\C$と$\Ho(N\C)$は圏同型である.
% \end{proof}

% \begin{definition}[ホモトピー同値]
%   $X$の射$f \in \map_X(x,y)$に対して, ある射$g,h \in \map_X(y,x)$が存在して, 
%   \begin{align*}
%     gf \sim \id_x,~ fh \sim \id_y
%   \end{align*}
%   を満たすとき, $f$をホモトピー同値(homotopy equivalence)という.  
%   このとき, $g$を$f$の左ホモトピー逆射(left homotopy inverse), $h$を$f$の右ホモトピー逆射(right homotopy inverse)という. 
% \end{definition}

% ホモトピー逆射はホモトピーの違いを除いて一意である. 

% \begin{remark}
%   $g$を$f$の左ホモトピー逆射, $h$を$f$の右ホモトピー逆射とする. 
%   \cref{theorem:composition_is_homotopic}より, 
%   \begin{align*}
%     g \sim g \id_y \sim g f h \sim \id_x h \sim h
%   \end{align*}
%   なので, ホモトピー逆射はホモトピーの違いを除いて一意である. 
% \end{remark}

% \begin{example} \label{eg:id_is_homotopy_equivalence}
%   $X$の任意の対象$x$に対して, $x$の恒等射$\id_x$はホモトピー逆射である.
% \end{example}

% ホモトピー同値がホモトピーに関して保たれることを見るために, いくつか準備をする.

% \begin{definition}
%   $Z(3)$を次の図式の余極限で定まる$F(3)$の単体的部分空間とする. 
%   % https://q.uiver.app/#q=WzAsNSxbMCwwLCJGKDEpIl0sWzEsMSwiRigwKSJdLFsyLDAsIkYoMSkiXSxbMywxLCJGKDApIl0sWzQsMCwiRigxKSJdLFsxLDAsIjEiXSxbMSwyLCIxIiwyXSxbMywyLCIwIl0sWzMsNCwiMCIsMl1d
%   \[\begin{tikzcd}
%     {F(1)} && {F(1)} && {F(1)} \\
%     & {F(0)} && {F(0)}
%     \arrow["1", from=2-2, to=1-1]
%     \arrow["1"', from=2-2, to=1-3]
%     \arrow["0", from=2-4, to=1-3]
%     \arrow["0"', from=2-4, to=1-5]
%   \end{tikzcd}\]
% \end{definition}

% \begin{remark}
%   単体的空間$Z(3)$に対して, $\Map_\sSpace(Z(3),X)$は次の極限
%   \begin{align*}
%     \Map_\sSpace(Z(3),X) \cong X_1 \times_{X_0} X_1 \times_{X_0} X_1
%   \end{align*}
%   で表され, 次の図式の極限として表せる. 
%   % https://q.uiver.app/#q=WzAsNSxbMCwwLCJYXzEiXSxbMSwxLCJYXzAiXSxbMiwwLCJYXzEiXSxbMywxLCJYXzAiXSxbNCwwLCJYXzEiXSxbMCwxLCJkXzEiLDJdLFsyLDEsImRfMSJdLFsyLDMsImRfMCIsMl0sWzQsMywiZF8wIl1d
%   \[\begin{tikzcd}
%     {X_1} && {X_1} && {X_1} \\
%     & {X_0} && {X_0}
%     \arrow["{d_1}"', from=1-1, to=2-2]
%     \arrow["{d_1}", from=1-3, to=2-2]
%     \arrow["{d_0}"', from=1-3, to=2-4]
%     \arrow["{d_0}", from=1-5, to=2-4]
%   \end{tikzcd}\]

% \end{remark}

% \begin{lemma}
%   $f,g \in X_1$を$X$の射とする. 
%   ある$\gamma : \Delta^1 \to X_1$が存在して, $\gamma(0)=f, \gamma(1)=g$を満たすとする. 
%   $g$がホモトピー同値のとき, $f$もホモトピー同値である. 
% \end{lemma}

% \begin{definition}[ホモトピー同値の空間]
%   $X$のホモトピー同値で生成される$X_1$の単体的部分空間を$X_\hoequiv$と表し, $X$のホモトピー同値の空間(space of homotopy equivalences)という. 
% \end{definition}

% \begin{remark}
%   $X_\hoequiv$の点は$X$の射である.
%   具体的には, $X_\hoequiv$はホモトピー逆射が存在するような$X$の射のなす$X_1$の充満部分空間である.  
% \end{remark}

% ホモトピー同値の空間は可縮である. 
% つまり, 射がホモトピー同値となるような逆射の選択はホモトピーの違いを除いて一意である. 

% \begin{lemma}
%   $X$のホモトピー同値$f$, $f$の左ホモトピー逆射$g$, $f$の右ホモトピー逆射$h$の3つ組$(f,g,h)$で生成される$X_3$の単体的部分空間を$X_\hoeqchoice$と表す. 
%   $U : X_\hoeqchoice \to X_\hoequiv : (f,g,h) \mapsto f$をホモトピー逆射を忘れる射とする. 
%   このとき, $U$は次の図式を可換にする自明なKanファイブレーションである. 
%   % https://q.uiver.app/#q=WzAsMyxbMCwwLCJYX1xcaG9lcXVpdiJdLFsxLDAsIlhfXFxob2VxY2hvaWNlIl0sWzEsMSwiWF8xIl0sWzEsMCwiVSIsMl0sWzEsMl0sWzAsMiwiIiwyLHsic3R5bGUiOnsidGFpbCI6eyJuYW1lIjoiaG9vayIsInNpZGUiOiJ0b3AifX19XV0=
%   \[\begin{tikzcd}
%     {X_\hoequiv} & {X_\hoeqchoice} \\
%     & {X_1}
%     \arrow["U"', from=1-2, to=1-1]
%     \arrow[from=1-2, to=2-2]
%     \arrow[hook, from=1-1, to=2-2]
%   \end{tikzcd}\]
%   また, 次のpullbackが存在する.
%   % https://q.uiver.app/#q=WzAsNSxbMCwwLCJYX1xcaG9lcXVpdiJdLFsxLDAsIlhfXFxob2VxY2hvaWNlIl0sWzEsMSwiWF8xIl0sWzIsMCwiWF8zIl0sWzIsMSwiWF8xIFxcdGltZXNfe1hfMH1YXzEgXFx0aW1lc197WF8wfSBYXzEiXSxbMSwwLCJcXGNvbmciLDJdLFsxLDJdLFswLDIsIiIsMix7InN0eWxlIjp7InRhaWwiOnsibmFtZSI6Imhvb2siLCJzaWRlIjoidG9wIn19fV0sWzEsMywiIiwyLHsic3R5bGUiOnsidGFpbCI6eyJuYW1lIjoiaG9vayIsInNpZGUiOiJ0b3AifX19XSxbMyw0LCIoZF8xZF8zLGRfMGRfMyxkXzFkXzApIl0sWzIsNCwiKHNfMGRfMCxcXGlkX3t4XzF9LHNfMGRfMSkiLDJdLFsxLDQsIiIsMCx7InN0eWxlIjp7Im5hbWUiOiJjb3JuZXIifX1dXQ==
%   \[\begin{tikzcd}
%     {X_\hoequiv} & {X_\hoeqchoice} & {X_3} \\
%     & {X_1} & {X_1 \times_{X_0}X_1 \times_{X_0} X_1}
%     \arrow["\cong"', from=1-2, to=1-1]
%     \arrow[from=1-2, to=2-2]
%     \arrow[hook, from=1-1, to=2-2]
%     \arrow[hook, from=1-2, to=1-3]
%     \arrow["{(d_1d_3,d_0d_3,d_1d_0)}", from=1-3, to=2-3]
%     \arrow["{(s_0d_0,\id_{x_1},s_0d_1)}"', from=2-2, to=2-3]
%     \arrow["\lrcorner"{anchor=center, pos=0.125}, draw=none, from=1-2, to=2-3]
%   \end{tikzcd}\] 
% \end{lemma}

% ホモトピー同値を用いて定義される空間として, 次の2つが重要である. 

% \begin{definition}[Segal空間亜群]
%   $X$の任意の射がホモトピー同値のとき, $X$をSegal空間亜群(Segal space groupoid)という. 
% \end{definition}

% 局所的なホモトピー同値の定義を与える. 

% \begin{definition}[$x$と$y$の間のホモトピー同値の空間]
%   $x,y$を$X$の対象とする. 
%   単体的集合$\hoequiv_X(x,y)$を次のpullbackで定義し, $x$と$y$の間のホモトピー同値の空間(space of homotopy equivalences between two objects $x$ and $y$)という. 
%   % https://q.uiver.app/#q=WzAsNCxbMCwwLCJcXGhvZXF1aXZfWCh4LHkpIl0sWzEsMCwiWF9cXGhvZXF1aXYiXSxbMSwxLCJYXzAgXFx0aW1lcyBYXzAiXSxbMCwxLCJcXERlbHRhXjAiXSxbMCwxXSxbMSwyLCIoZF8wLGRfMSkiXSxbMCwzXSxbMywyLCIoeCx5KSIsMl0sWzAsMiwiIiwxLHsic3R5bGUiOnsibmFtZSI6ImNvcm5lciJ9fV1d
%   \[\begin{tikzcd}
%     {\hoequiv_X(x,y)} & {X_\hoequiv} \\
%     {\Delta[0]} & {X_0 \times X_0}
%     \arrow[from=1-1, to=1-2]
%     \arrow["{(d_0,d_1)}", from=1-2, to=2-2]
%     \arrow[from=1-1, to=2-1]
%     \arrow["{(x,y)}"', from=2-1, to=2-2]
%     \arrow["\lrcorner"{anchor=center, pos=0.125}, draw=none, from=1-1, to=2-2]
%   \end{tikzcd}\]
% \end{definition}

% \begin{example} \label{eg:homotopy_equivalence_is_isomorphism_in_NC}
%   $\C$を通常の圏とする. 
%   脈体$N\C$におけるホモトピー同値は$\C$における同型射である. 
% \end{example}

% \begin{proof}
%   $N\C$において射$f$がホモトピー同値であることと, $\Ho(N\C)$において射$[f]$が同型射であることは同値である. 
%   \cref{thrm:C_is_HoNC}より, $\Ho(N\C) = \C$である. 
%   よって, $N\C$において射$f$がホモトピー同値であることと, $\C$において射$[f]$が同型射であることは同値である. 
% \end{proof}

% \subsection{完備Segal空間} \label{subsec:complete_segal_space}

% \cref{sec:reedy_ファイブラント}と\cref{sec:segal_space}でSegal空間がホモトピー論と圏論の2つの性質を持つことを見た. 
% しかし, Segal空間のホモトピー論と圏論は次で見るように整合性がない.

% \begin{definition}
%   $I(1)$を2つの対象$x,y$とその間の可逆な射からなる圏とする. 
%   \footnote{
%     恒等射はもちろん存在するが, 恒等射は明記しない. 
%   }
%   % https://q.uiver.app/#q=WzAsMixbMCwwLCJ4Il0sWzEsMCwieSJdLFswLDEsImYiLDAseyJvZmZzZXQiOi0xfV0sWzEsMCwiZl57LTF9IiwwLHsib2Zmc2V0IjotMX1dXQ==
%   \[\begin{tikzcd}
%     x & y
%     \arrow["f", shift left, from=1-1, to=1-2]
%     \arrow["{f^{-1}}", shift left, from=1-2, to=1-1]
%   \end{tikzcd}\]
% \end{definition}

% \begin{remark} \label{rem:E(1)}
%   Segal空間$i_F^\ast(NI(1))$を$E(1)$と表す. 
%   $E(1)$は離散単体的空間なので, 任意の$n$に対して, 
%   \begin{align*}
%     E(1)_n = \{x,y\}^n
%   \end{align*}
%   である. 
%   具体的には, $E(1)_n$は集合$\{0,\cdots,n-1\}$から$\{x,y\}$への射の集まりである. 
%   よって, $E(1)_n$は$2^{n+1}$個の元を持つ.
%   例えば, $E(1)_0$は$\{x,y\}$である. 
%   また, $E(1)_1$は$\{\id_x,\id_y, f, f^{-1}\}$である. 
% \end{remark}

% \begin{lemma}
%   圏$I(1)$と$\{\ast\}$は圏同値である. 
%   しかし, $E(1) := i_F^\ast(NI(1))$と$F(1) := i_F^\ast(N\{\ast\})$はSegal空間として圏同値ではない.
% \end{lemma}

% \begin{proof}
%   $E(1)$は各次元で可縮ではないが, $F(1)$は各次元で可縮であることから, 明らかにSegal空間として圏同値でない.
% \end{proof}

% Segal空間のホモトピー論と圏論に整合性を持たせる条件が完備性である. 

% \begin{remark}
%   単体的空間$X$に対して, ホモトピー同値の空間$X_\hoequiv \subset X_1$が存在する. 
%   恒等射の定義
%   \begin{align*}
%     s_0 : X_0 \to X_1 : x \mapsto \id_x
%   \end{align*}
%   において, \cref{eg:id_is_homotopy_equivalence}より, 恒等射はホモトピー同値である. 
%   よって, $s_0$は$X_\hoequiv$を経由して分解される. 
%   % https://q.uiver.app/#q=WzAsMyxbMCwwLCJYXzAiXSxbMiwwLCJYXzEiXSxbMSwxLCJYX1xcaG9lcXVpdiJdLFswLDEsInNfMCJdLFswLDJdLFsyLDEsIiIsMix7InN0eWxlIjp7InRhaWwiOnsibmFtZSI6Imhvb2siLCJzaWRlIjoidG9wIn19fV1d
% \[\begin{tikzcd}
% 	{X_0} && {X_1} \\
% 	& {X_\hoequiv}
% 	\arrow["{s_0}", from=1-1, to=1-3]
% 	\arrow[from=1-1, to=2-2]
% 	\arrow[hook, from=2-2, to=1-3]
% \end{tikzcd}\]
% \end{remark}

% 単体的空間の射$s_0 : X_0 \to X_1$は単射ではあるが, 全射ではない. 
% ($X_1$の任意の同型射が恒等射のみであるとき, $s_0$は全射である.)
% よって, 完備Segal空間は次のように定義される. 

% \begin{definition}[完備Segal空間]
%   $X$をSegal空間とする.
%   単体的集合の射
%   \begin{align*}
%     s_0 : X_0 \to X_\hoequiv
%   \end{align*}
%   が弱同値のとき, $X$を完備Segal空間(complete Segal space)という. 
% \end{definition}

% \begin{example}
%   単体的空間$E(1)$は完備Segal空間ではない.
% \end{example}

% \begin{proof}
%   \cref{rem:E(1)}より, 
%   \begin{align*}
%     E(1)_0 &= \{x,y\} \\
%     E(1)_1 &= \{\id_x,\id_y, f, f^{-1}\} \\
%     E(1)_\hoequiv &= \{\id_x,\id_y, f, f^{-1}\} = E(1)_1
%   \end{align*}
%   であるが, 
%   \begin{align*}
%     E(1)_0 \to E(1)_\hoequiv = \{x,y\} \to \{\id_x,\id_y, f, f^{-1}\}
%   \end{align*}
%   は弱同値ではない. 
%   よって, $E(1)$は完備Segal空間ではない.
% \end{proof}

% 完備Segal空間の定義と同値な条件がある. 

% \begin{theorem}
%   単体的空間$X$に対して, 次は同値である. 
%   \begin{enumerate}
%     \item $X$は完備Segal空間である. 
%     \item 次の図式はホモトピーpullbackである. 
%     % https://q.uiver.app/#q=WzAsNCxbMCwwLCJYXzAiXSxbMSwwLCJYXzMiXSxbMCwxLCJYXzEiXSxbMSwxLCJYXzEgXFx0aW1lc197WF8wfSBYXzEgXFx0aW1lc197WF8wfSBYXzEiXSxbMCwxXSxbMCwyXSxbMSwzXSxbMiwzXSxbMCwzLCIiLDEseyJzdHlsZSI6eyJuYW1lIjoiY29ybmVyIn19XV0=
%     \[\begin{tikzcd}
%       {X_0} & {X_3} \\
%       {X_1} & {X_1 \times_{X_0} X_1 \times_{X_0} X_1}
%       \arrow[from=1-1, to=1-2]
%       \arrow[from=1-1, to=2-1]
%       \arrow[from=1-2, to=2-2]
%       \arrow[from=2-1, to=2-2]
%       \arrow["\lrcorner"{anchor=center, pos=0.125}, draw=none, from=1-1, to=2-2]
%     \end{tikzcd}\]
%     \item 次の単体的空間の射は弱同値である. 
%     \begin{align*}
%       \Map_\sSpace(E(1),X) \to \Map_\sSpace(F(0),X)
%     \end{align*}
%     \item 任意の対象$x,y \in X$に対して, $x$と$y$の間のホモトピー同値の空間$\hoequiv_X(x,y)$は$x$から$y$への$X_0$における道の空間と弱同値である. 
%   \end{enumerate}
% \end{theorem}

% 完備Segal空間の条件をまとめる. 

% \begin{remark}
%   完備Segal空間は次の条件を満たす単体的空間である. 
%   \begin{description}
%     \item[(Reedyファイブラント条件)] 垂直な軸はホモトピー論の性質をもつ. 
%     \item[(Segal条件)] 水平な軸は圏論的な性質をもつ. 
%     \item[(完備性)] ホモトピー論の性質と圏論の性質は整合的である.  
%   \end{description}
% \end{remark}

% Segal空間亜群の完備性も定義できる. 

% \begin{definition}[完備Segal空間亜群]
%   $X$を完備Segal空間とする. 
%   $X$の任意の射がホモトピー同値のとき, $X$を完備Segal空間亜群(complete Segal space groupoid)という.
% \end{definition}

% \begin{theorem} \label{thrm:NC_is_complete_equals_has_non_trivial_morphism}
%   $\C$を通常の圏とする. 
%   脈体$N\C$が完備Segal空間であることと, $\C$が恒等射以外の同型射を持たないことは同値である. 
% \end{theorem}

% \begin{proof}
%   ($\Rightarrow$) : $N\C$が完備Segal空間であるとする. 
%   このとき, 単体的空間の射
%   \begin{align*}
%     N\C_0 \to N^C_\hoequiv
%   \end{align*}
%   は弱同値である. 
%   $N\C_0$と$NC_\hoequiv$は離散的なので, 
%   \begin{align*}
%     N\C_0 \to NC_\hoequiv
%   \end{align*}
%   は集合の全単射である. 
%   よって, 恒等射以外のホモトピー同値は存在しない. 
%   \cref{eg:homotopy_equivalence_is_isomorphism_in_NC}より, $N\C$におけるホモトピー同値は$\C$における同型射である.
%   よって, $\C$は恒等射以外の同型射を持たない.

%   ($\Leftarrow$) : $\C$が恒等射以外の同型射を持たないとすると, 
%   \begin{align*}
%     N\C_0 \to N\C_\hoequiv = \Iso\C : x \mapsto \id_x
%   \end{align*}
%   は集合の全単射である. 
%   よって, ホモトピー同値である. 
%   従って, 脈体$N\C$は完備Segal空間である.
% \end{proof}

% \cref{thrm:NC_is_complete_equals_has_non_trivial_morphism}より, 通常の圏$\C$が恒等射以外の同型射を持つとき, 単体的集合の脈体ではうまくいかない.
% そのため, 圏のホモトピー論を考慮するような脈体として, 圏の分類図式と呼ばれる概念を定義する. 

% % \begin{definition}[核]
  
% % \end{definition}

% \begin{definition}[分類図式]
%   $\C$を通常の圏とする. 
%   単体的空間$\N\C$を任意の$n,l \geq 0$に対して次のように定義し, $\C$の分類図式(classifying diagram)という. 
%   \begin{align*}
%     \N\C_{n,l} &:= \Hom_\Cat([n] \times I(l), \C) \\
%     \N\C_n &:= N(\Fun([n],\C)^\core)
%   \end{align*}
% \end{definition}

% \begin{remark}
%   $\C$の分類図式において, $n=0$のとき, 
%   \begin{align*}
%     \N\C_0 = N(\Fun([0],\C)^\core) = N(\C^\core)
%   \end{align*}
% \end{remark}

% \begin{remark}
%   $\C$の分類図式は次のように表せる. 
%   % https://q.uiver.app/#q=WzAsMjQsWzAsMCwiXFxOXFxDX3swLDB9Il0sWzEsMCwiXFxOXFxDX3sxLDB9Il0sWzAsMSwiXFxOXFxDX3swLDF9Il0sWzEsMSwiXFxOXFxDX3sxLDF9Il0sWzAsMiwiXFxOXFxDX3swLDJ9Il0sWzEsMiwiXFxOXFxDX3sxLDJ9Il0sWzIsMCwiXFxOXFxDX3syLDB9Il0sWzIsMSwiXFxOXFxDX3syLDF9Il0sWzIsMiwiXFxOXFxDX3syLDJ9Il0sWzMsMCwiXFxjZG90cyJdLFs0LDAsIlxcTlxcQ197LSwwfSJdLFszLDEsIlxcY2RvdHMiXSxbNCwxLCJcXE5cXENfey0sMX0iXSxbMywyLCJcXGNkb3RzIl0sWzQsMiwiXFxOXFxDX3stLDJ9Il0sWzAsMywiXFx2ZG90cyJdLFsxLDMsIlxcdmRvdHMiXSxbMiwzLCJcXHZkb3RzIl0sWzAsNCwiXFxDXlxcY29yZSJdLFsxLDQsIlxcRnVuKFsxXSxcXEMpXlxcY29yZSJdLFsyLDQsIlxcRnVuKFsyXSxcXEMpXlxcY29yZSJdLFs0LDMsIlxcdmRvdHMiXSxbMyw0LCJcXGNkb3RzIl0sWzMsMywiXFxkZG90cyJdLFswLDFdLFsxLDAsIiIsMix7Im9mZnNldCI6MX1dLFsxLDAsIiIsMSx7Im9mZnNldCI6LTF9XSxbMCwyXSxbMiwzXSxbMSwzXSxbMiwwLCIiLDEseyJvZmZzZXQiOjF9XSxbMiwwLCIiLDEseyJvZmZzZXQiOi0xfV0sWzMsMSwiIiwxLHsib2Zmc2V0IjoxfV0sWzMsMSwiIiwxLHsib2Zmc2V0IjotMX1dLFszLDIsIiIsMSx7Im9mZnNldCI6MX1dLFszLDIsIiIsMSx7Im9mZnNldCI6LTF9XSxbNCwyXSxbNCwyLCIiLDEseyJvZmZzZXQiOi0yfV0sWzUsM10sWzIsNCwiIiwxLHsib2Zmc2V0IjoxfV0sWzIsNCwiIiwxLHsib2Zmc2V0IjotMX1dLFs0LDIsIiIsMSx7Im9mZnNldCI6Mn1dLFs0LDVdLFs1LDQsIiIsMSx7Im9mZnNldCI6LTF9XSxbMyw1LCIiLDEseyJvZmZzZXQiOjF9XSxbMyw1LCIiLDEseyJvZmZzZXQiOi0xfV0sWzUsMywiIiwxLHsib2Zmc2V0IjoyfV0sWzUsMywiIiwxLHsib2Zmc2V0IjotMn1dLFs2LDFdLFsxLDYsIiIsMSx7Im9mZnNldCI6LTF9XSxbMSw2LCIiLDEseyJvZmZzZXQiOjF9XSxbNiwxLCIiLDEseyJvZmZzZXQiOjJ9XSxbNiwxLCIiLDEseyJvZmZzZXQiOi0yfV0sWzcsM10sWzMsNywiIiwxLHsib2Zmc2V0IjoxfV0sWzMsNywiIiwxLHsib2Zmc2V0IjotMX1dLFs3LDMsIiIsMSx7Im9mZnNldCI6Mn1dLFs3LDMsIiIsMSx7Im9mZnNldCI6LTJ9XSxbNiw3XSxbNyw2LCIiLDEseyJvZmZzZXQiOjF9XSxbNyw2LCIiLDEseyJvZmZzZXQiOi0xfV0sWzgsN10sWzcsOCwiIiwxLHsib2Zmc2V0IjoxfV0sWzcsOCwiIiwxLHsib2Zmc2V0IjotMX1dLFs4LDcsIiIsMSx7Im9mZnNldCI6Mn1dLFs4LDcsIiIsMSx7Im9mZnNldCI6LTJ9XSxbOCw1XSxbNSw4LCIiLDEseyJvZmZzZXQiOjF9XSxbNSw4LCIiLDEseyJvZmZzZXQiOi0xfV0sWzgsNSwiIiwxLHsib2Zmc2V0IjoyfV0sWzgsNSwiIiwxLHsib2Zmc2V0IjotMn1dLFsxOCwxOV0sWzE5LDE4LCIiLDEseyJvZmZzZXQiOjF9XSxbMjAsMTldLFsxOSwyMCwiIiwxLHsib2Zmc2V0IjoxfV0sWzE5LDE4LCIiLDEseyJvZmZzZXQiOi0xfV0sWzE5LDIwLCIiLDEseyJvZmZzZXQiOi0xfV0sWzIwLDE5LCIiLDEseyJvZmZzZXQiOjJ9XSxbMjAsMTksIiIsMSx7Im9mZnNldCI6LTJ9XSxbMTAsMTJdLFsxMiwxMCwiIiwxLHsib2Zmc2V0IjoxfV0sWzEyLDEwLCIiLDEseyJvZmZzZXQiOi0xfV0sWzE0LDEyXSxbMTIsMTQsIiIsMSx7Im9mZnNldCI6MX1dLFsxMiwxNCwiIiwxLHsib2Zmc2V0IjotMX1dLFsxNCwxMiwiIiwxLHsib2Zmc2V0IjoyfV0sWzE0LDEyLCIiLDEseyJvZmZzZXQiOi0yfV0sWzYsOV0sWzksNiwiIiwxLHsib2Zmc2V0IjoxfV0sWzksNiwiIiwxLHsib2Zmc2V0IjotMX1dLFs2LDksIiIsMSx7Im9mZnNldCI6Mn1dLFs2LDksIiIsMSx7Im9mZnNldCI6LTJ9XSxbOSw2LCIiLDEseyJvZmZzZXQiOjN9XSxbOSw2LCIiLDEseyJvZmZzZXQiOi0zfV0sWzcsMTFdLFsxMSw3LCIiLDEseyJvZmZzZXQiOi0xfV0sWzExLDcsIiIsMSx7Im9mZnNldCI6MX1dLFs3LDExLCIiLDEseyJvZmZzZXQiOjJ9XSxbNywxMSwiIiwxLHsib2Zmc2V0IjotMn1dLFsxMSw3LCIiLDEseyJvZmZzZXQiOjN9XSxbMTEsNywiIiwxLHsib2Zmc2V0IjotM31dLFs4LDEzXSxbMTMsOCwiIiwxLHsib2Zmc2V0IjoxfV0sWzEzLDgsIiIsMSx7Im9mZnNldCI6LTF9XSxbOCwxMywiIiwxLHsib2Zmc2V0IjoyfV0sWzgsMTMsIiIsMSx7Im9mZnNldCI6LTJ9XSxbMTMsOCwiIiwxLHsib2Zmc2V0IjozfV0sWzEzLDgsIiIsMSx7Im9mZnNldCI6LTN9XSxbOSwxMCwiIiwxLHsic3R5bGUiOnsiaGVhZCI6eyJuYW1lIjoibm9uZSJ9fX1dLFsxMCw5LCIiLDEseyJvZmZzZXQiOjEsInN0eWxlIjp7ImhlYWQiOnsibmFtZSI6Im5vbmUifX19XSxbMTEsMTIsIiIsMSx7InN0eWxlIjp7ImhlYWQiOnsibmFtZSI6Im5vbmUifX19XSxbMTIsMTEsIiIsMSx7Im9mZnNldCI6MSwic3R5bGUiOnsiaGVhZCI6eyJuYW1lIjoibm9uZSJ9fX1dLFsxMywxNCwiIiwxLHsic3R5bGUiOnsiaGVhZCI6eyJuYW1lIjoibm9uZSJ9fX1dLFsxNCwxMywiIiwxLHsib2Zmc2V0IjoxLCJzdHlsZSI6eyJoZWFkIjp7Im5hbWUiOiJub25lIn19fV0sWzQsMTVdLFsxNSw0LCIiLDEseyJvZmZzZXQiOjF9XSxbMTUsNCwiIiwxLHsib2Zmc2V0IjotMX1dLFs0LDE1LCIiLDEseyJvZmZzZXQiOjJ9XSxbNCwxNSwiIiwxLHsib2Zmc2V0IjotMn1dLFsxNSw0LCIiLDEseyJvZmZzZXQiOi0zfV0sWzUsMTZdLFsxNiw1LCIiLDEseyJvZmZzZXQiOjF9XSxbMTYsNSwiIiwxLHsib2Zmc2V0IjotMX1dLFs1LDE2LCIiLDEseyJvZmZzZXQiOjJ9XSxbNSw0LCIiLDEseyJvZmZzZXQiOjF9XSxbNSwxNiwiIiwxLHsib2Zmc2V0IjotMn1dLFsxNiw1LCIiLDEseyJvZmZzZXQiOjN9XSxbMTYsNSwiIiwxLHsib2Zmc2V0IjotM31dLFsxNSw0LCIiLDEseyJvZmZzZXQiOjN9XSxbMTUsMTgsIiIsMSx7InN0eWxlIjp7ImhlYWQiOnsibmFtZSI6Im5vbmUifX19XSxbMTUsMTgsIiIsMSx7Im9mZnNldCI6MSwic3R5bGUiOnsiaGVhZCI6eyJuYW1lIjoibm9uZSJ9fX1dLFsxNiwxOSwiIiwxLHsic3R5bGUiOnsiaGVhZCI6eyJuYW1lIjoibm9uZSJ9fX1dLFsxNiwxOSwiIiwxLHsib2Zmc2V0IjoxLCJzdHlsZSI6eyJoZWFkIjp7Im5hbWUiOiJub25lIn19fV0sWzE3LDIwLCIiLDEseyJzdHlsZSI6eyJoZWFkIjp7Im5hbWUiOiJub25lIn19fV0sWzE3LDIwLCIiLDEseyJvZmZzZXQiOjEsInN0eWxlIjp7ImhlYWQiOnsibmFtZSI6Im5vbmUifX19XSxbOCwxN10sWzE3LDgsIiIsMSx7Im9mZnNldCI6MX1dLFsxNyw4LCIiLDEseyJvZmZzZXQiOi0xfV0sWzgsMTcsIiIsMSx7Im9mZnNldCI6Mn1dLFs4LDE3LCIiLDEseyJvZmZzZXQiOi0yfV0sWzE3LDgsIiIsMSx7Im9mZnNldCI6M31dLFsxNyw4LCIiLDEseyJvZmZzZXQiOi0zfV0sWzIwLDIyXSxbMjMsMjIsIiIsMSx7InN0eWxlIjp7ImhlYWQiOnsibmFtZSI6Im5vbmUifX19XSxbMjMsMjIsIiIsMSx7Im9mZnNldCI6MSwic3R5bGUiOnsiaGVhZCI6eyJuYW1lIjoibm9uZSJ9fX1dLFsyMywyMSwiIiwxLHsic3R5bGUiOnsiaGVhZCI6eyJuYW1lIjoibm9uZSJ9fX1dLFsyMywyMSwiIiwxLHsib2Zmc2V0IjotMSwic3R5bGUiOnsiaGVhZCI6eyJuYW1lIjoibm9uZSJ9fX1dLFsxNCwyMV0sWzIxLDE0LCIiLDEseyJvZmZzZXQiOjF9XSxbMjEsMTQsIiIsMSx7Im9mZnNldCI6LTF9XSxbMTQsMjEsIiIsMSx7Im9mZnNldCI6Mn1dLFsxNCwyMSwiIiwxLHsib2Zmc2V0IjotMn1dLFsyMSwxNCwiIiwxLHsib2Zmc2V0IjozfV0sWzIxLDE0LCIiLDEseyJvZmZzZXQiOi0zfV0sWzIyLDIwLCIiLDEseyJvZmZzZXQiOjF9XSxbMjIsMjAsIiIsMSx7Im9mZnNldCI6LTF9XSxbMjAsMjIsIiIsMSx7Im9mZnNldCI6Mn1dLFsyMCwyMiwiIiwxLHsib2Zmc2V0IjotMn1dLFsyMiwyMCwiIiwxLHsib2Zmc2V0IjozfV0sWzIyLDIwLCIiLDEseyJvZmZzZXQiOi0zfV1d
%   \[\begin{tikzcd}
%     {\N\C_{0,0}} & {\N\C_{1,0}} & {\N\C_{2,0}} & \cdots & {\N\C_{-,0}} \\
%     {\N\C_{0,1}} & {\N\C_{1,1}} & {\N\C_{2,1}} & \cdots & {\N\C_{-,1}} \\
%     {\N\C_{0,2}} & {\N\C_{1,2}} & {\N\C_{2,2}} & \cdots & {\N\C_{-,2}} \\
%     \vdots & \vdots & \vdots & \ddots & \vdots \\
%     {\C^\core} & {\Fun([1],\C)^\core} & {\Fun([2],\C)^\core} & \cdots
%     \arrow[from=1-1, to=1-2]
%     \arrow[shift right, from=1-2, to=1-1]
%     \arrow[shift left, from=1-2, to=1-1]
%     \arrow[from=1-1, to=2-1]
%     \arrow[from=2-1, to=2-2]
%     \arrow[from=1-2, to=2-2]
%     \arrow[shift right, from=2-1, to=1-1]
%     \arrow[shift left, from=2-1, to=1-1]
%     \arrow[shift right, from=2-2, to=1-2]
%     \arrow[shift left, from=2-2, to=1-2]
%     \arrow[shift right, from=2-2, to=2-1]
%     \arrow[shift left, from=2-2, to=2-1]
%     \arrow[from=3-1, to=2-1]
%     \arrow[shift left=2, from=3-1, to=2-1]
%     \arrow[from=3-2, to=2-2]
%     \arrow[shift right, from=2-1, to=3-1]
%     \arrow[shift left, from=2-1, to=3-1]
%     \arrow[shift right=2, from=3-1, to=2-1]
%     \arrow[from=3-1, to=3-2]
%     \arrow[shift left, from=3-2, to=3-1]
%     \arrow[shift right, from=2-2, to=3-2]
%     \arrow[shift left, from=2-2, to=3-2]
%     \arrow[shift right=2, from=3-2, to=2-2]
%     \arrow[shift left=2, from=3-2, to=2-2]
%     \arrow[from=1-3, to=1-2]
%     \arrow[shift left, from=1-2, to=1-3]
%     \arrow[shift right, from=1-2, to=1-3]
%     \arrow[shift right=2, from=1-3, to=1-2]
%     \arrow[shift left=2, from=1-3, to=1-2]
%     \arrow[from=2-3, to=2-2]
%     \arrow[shift right, from=2-2, to=2-3]
%     \arrow[shift left, from=2-2, to=2-3]
%     \arrow[shift right=2, from=2-3, to=2-2]
%     \arrow[shift left=2, from=2-3, to=2-2]
%     \arrow[from=1-3, to=2-3]
%     \arrow[shift right, from=2-3, to=1-3]
%     \arrow[shift left, from=2-3, to=1-3]
%     \arrow[from=3-3, to=2-3]
%     \arrow[shift right, from=2-3, to=3-3]
%     \arrow[shift left, from=2-3, to=3-3]
%     \arrow[shift right=2, from=3-3, to=2-3]
%     \arrow[shift left=2, from=3-3, to=2-3]
%     \arrow[from=3-3, to=3-2]
%     \arrow[shift right, from=3-2, to=3-3]
%     \arrow[shift left, from=3-2, to=3-3]
%     \arrow[shift right=2, from=3-3, to=3-2]
%     \arrow[shift left=2, from=3-3, to=3-2]
%     \arrow[from=5-1, to=5-2]
%     \arrow[shift right, from=5-2, to=5-1]
%     \arrow[from=5-3, to=5-2]
%     \arrow[shift right, from=5-2, to=5-3]
%     \arrow[shift left, from=5-2, to=5-1]
%     \arrow[shift left, from=5-2, to=5-3]
%     \arrow[shift right=2, from=5-3, to=5-2]
%     \arrow[shift left=2, from=5-3, to=5-2]
%     \arrow[from=1-5, to=2-5]
%     \arrow[shift right, from=2-5, to=1-5]
%     \arrow[shift left, from=2-5, to=1-5]
%     \arrow[from=3-5, to=2-5]
%     \arrow[shift right, from=2-5, to=3-5]
%     \arrow[shift left, from=2-5, to=3-5]
%     \arrow[shift right=2, from=3-5, to=2-5]
%     \arrow[shift left=2, from=3-5, to=2-5]
%     \arrow[from=1-3, to=1-4]
%     \arrow[shift right, from=1-4, to=1-3]
%     \arrow[shift left, from=1-4, to=1-3]
%     \arrow[shift right=2, from=1-3, to=1-4]
%     \arrow[shift left=2, from=1-3, to=1-4]
%     \arrow[shift right=3, from=1-4, to=1-3]
%     \arrow[shift left=3, from=1-4, to=1-3]
%     \arrow[from=2-3, to=2-4]
%     \arrow[shift left, from=2-4, to=2-3]
%     \arrow[shift right, from=2-4, to=2-3]
%     \arrow[shift right=2, from=2-3, to=2-4]
%     \arrow[shift left=2, from=2-3, to=2-4]
%     \arrow[shift right=3, from=2-4, to=2-3]
%     \arrow[shift left=3, from=2-4, to=2-3]
%     \arrow[from=3-3, to=3-4]
%     \arrow[shift right, from=3-4, to=3-3]
%     \arrow[shift left, from=3-4, to=3-3]
%     \arrow[shift right=2, from=3-3, to=3-4]
%     \arrow[shift left=2, from=3-3, to=3-4]
%     \arrow[shift right=3, from=3-4, to=3-3]
%     \arrow[shift left=3, from=3-4, to=3-3]
%     \arrow[no head, from=1-4, to=1-5]
%     \arrow[shift right, no head, from=1-5, to=1-4]
%     \arrow[no head, from=2-4, to=2-5]
%     \arrow[shift right, no head, from=2-5, to=2-4]
%     \arrow[no head, from=3-4, to=3-5]
%     \arrow[shift right, no head, from=3-5, to=3-4]
%     \arrow[from=3-1, to=4-1]
%     \arrow[shift right, from=4-1, to=3-1]
%     \arrow[shift left, from=4-1, to=3-1]
%     \arrow[shift right=2, from=3-1, to=4-1]
%     \arrow[shift left=2, from=3-1, to=4-1]
%     \arrow[shift left=3, from=4-1, to=3-1]
%     \arrow[from=3-2, to=4-2]
%     \arrow[shift right, from=4-2, to=3-2]
%     \arrow[shift left, from=4-2, to=3-2]
%     \arrow[shift right=2, from=3-2, to=4-2]
%     \arrow[shift right, from=3-2, to=3-1]
%     \arrow[shift left=2, from=3-2, to=4-2]
%     \arrow[shift right=3, from=4-2, to=3-2]
%     \arrow[shift left=3, from=4-2, to=3-2]
%     \arrow[shift right=3, from=4-1, to=3-1]
%     \arrow[no head, from=4-1, to=5-1]
%     \arrow[shift right, no head, from=4-1, to=5-1]
%     \arrow[no head, from=4-2, to=5-2]
%     \arrow[shift right, no head, from=4-2, to=5-2]
%     \arrow[no head, from=4-3, to=5-3]
%     \arrow[shift right, no head, from=4-3, to=5-3]
%     \arrow[from=3-3, to=4-3]
%     \arrow[shift right, from=4-3, to=3-3]
%     \arrow[shift left, from=4-3, to=3-3]
%     \arrow[shift right=2, from=3-3, to=4-3]
%     \arrow[shift left=2, from=3-3, to=4-3]
%     \arrow[shift right=3, from=4-3, to=3-3]
%     \arrow[shift left=3, from=4-3, to=3-3]
%     \arrow[from=5-3, to=5-4]
%     \arrow[no head, from=4-4, to=5-4]
%     \arrow[shift right, no head, from=4-4, to=5-4]
%     \arrow[no head, from=4-4, to=4-5]
%     \arrow[shift left, no head, from=4-4, to=4-5]
%     \arrow[from=3-5, to=4-5]
%     \arrow[shift right, from=4-5, to=3-5]
%     \arrow[shift left, from=4-5, to=3-5]
%     \arrow[shift right=2, from=3-5, to=4-5]
%     \arrow[shift left=2, from=3-5, to=4-5]
%     \arrow[shift right=3, from=4-5, to=3-5]
%     \arrow[shift left=3, from=4-5, to=3-5]
%     \arrow[shift right, from=5-4, to=5-3]
%     \arrow[shift left, from=5-4, to=5-3]
%     \arrow[shift right=2, from=5-3, to=5-4]
%     \arrow[shift left=2, from=5-3, to=5-4]
%     \arrow[shift right=3, from=5-4, to=5-3]
%     \arrow[shift left=3, from=5-4, to=5-3]
%   \end{tikzcd}\]
% \end{remark}

% 圏の分類図式が完備Segal空間であることを示す. 

% \begin{lemma}
%   $\C$を通常の圏とする. 
%   圏の分類図式$\N\C$はReedyファイブラントである.
% \end{lemma}

% \begin{lemma}
%   $\C$を通常の圏とする. 
%   圏の分類図式$\N\C$はSegal空間である.
% \end{lemma}

% \begin{proof}
%   圏の脈体の定義より, 任意の$n \geq 0$に対して, $N\C_n =\Fun([n],\C)$である. 
%   \cref{eg:NC_is_Segal_space}より, $N\C$はSegal空間である. 
%   pullbackをとる操作は核をとる操作と脈体をとる操作で保たれる. 
%   よって, 圏の分類図式$\N\C$はSegal空間である.
% \end{proof}

% \begin{theorem}
%   $\C$を通常の圏とする. 
%   圏の分類図式$\N\C$は完備Segal空間である.
% \end{theorem}

% \begin{proof}
%   完備性を示す. 
%   これは
%   \begin{align*}
%     \N\C_0 = N(\C^\core) \cong N(\Fun(I(1),\C)^\core) \cong N(\Iso(\C^\core)) = \N\C_\hoequiv
%   \end{align*}
%   から従う.  
% \end{proof}

\newpage


\section{完備Segal空間について} \label{sec:CSS}

\subsection{ボックス積}

記法は\cite{JT07}のchapter 2に従う. 

\begin{definition}[ボックス積]
  $A,B$を単体的集合とする. 
  このとき, 単体的空間$A \Box B$を任意の$m,n \geq 0$に対して次のように定義し, $A \Box B$を$A$と$B$のボックス積(box product)という. 
  \begin{align*}
    (A \Box B)_{m,n} := A_m \times B_n
  \end{align*}
\end{definition}

\begin{remark}
  構成$(A,B) \mapsto A \Box B$は双関手$- \Box ? : \sSet \times \sSet \to \sSpace$を定める. 
  特に, 任意の$m,n \geq 0$に対して次が成立する. 
  \begin{align*}
    \Delta[m] \Box \Delta[n] 
    &= \Hom_\sSet(\Delta[-],\Delta[m]) \times \Hom_\sSet(\Delta[?],\Delta[m]) \\
    &\cong \Hom_{\sSet \times \sSet}(\Delta[-] \times \Delta[?],\Delta[m] \times \Delta[n]) \\
    &\cong \Hom_{\Delta \times \Delta}(- \times ?, m \times n)
  \end{align*}
\end{remark}

ボックス積を与える双関手は両方の変数に対して右随伴を持つ. 

\begin{definition}
  $A$を単体的集合, $X$を単体的空間とする. 
  このとき, 単体的集合$A \backslash X$を任意の$n \geq 0$に対して次のように定義する.
  \begin{align*}
    (A \backslash X)_n := \Hom_\sSpace(A \Box \Delta[n],X)
  \end{align*}
\end{definition}

\begin{definition}
  $B$を単体的集合, $X$を単体的空間とする. 
  このとき, 単体的集合$X/B$を任意の$n \geq 0$に対して次のように定義する.
  \begin{align*}
    (X/B)_n := \Hom_\sSpace(\Delta[n] \Box B, X)
  \end{align*}
\end{definition}

\begin{remark}
  構成$(A,X) \mapsto A \backslash X$は双関手$- \backslash ? : \sSet^\myop \times \sSpace \to \sSet$を定める. 
  特に, 任意の$m \geq 0$に対して次が成立する.
  \begin{align*}
    \Delta[m] \backslash X = \Hom_\sSpace(\Delta[m] \Box \Delta[-],X) \cong X_{m,-}
  \end{align*}

  同様に, 構成$(X,B) \mapsto X/B$は双関手$? / - : \sSpace \times \sSet^\myop \to \sSet$を定める. 
  特に, 任意の$n \geq 0$に対して次が成立する.
  \begin{align*}
    X / \Delta[n] = \Hom_\sSpace(\Delta[-] \Box \Delta[n], X) \cong X_{-,n} 
  \end{align*}
\end{remark}

\begin{remark}
  $A,B$を単体的集合, $X$を単体的空間とする. 
  このとき, 次の同型が存在する. 
  \begin{align*}
    \Hom_\sSpace(A \Box B,X) \cong \Hom_\sSet(B,A \backslash X) \cong \Hom_\sSet(A, X/B)
  \end{align*}
  つまり, 次の随伴が存在する. 
  \begin{align*}
    (- \Box ?) \dashv (- \backslash ?), ~~ (- \Box ?) \dashv (- / ?)
  \end{align*}
\end{remark}

ボックス積を与える操作はarrow category上の双関手(とその随伴)を定める. 

\begin{notation}
  圏$\C$に対して, $\C$上のarrow categoryを$\C^I$と表す. 
\end{notation}

\begin{definition}
  $u : A \to B, v : S \to T$を単体的集合の射とする. 
  このとき, 次のpushoutの普遍性から定まる単体的空間の射を次のように表す. 
  \begin{align*}
    u \Box' v := A \Box T \sqcup_{A \Box S} B \Box S \to B \Box T
  \end{align*}
  % https://q.uiver.app/#q=WzAsNSxbMCwwLCJBIFxcQm94IFMiXSxbMCwxLCJBIFxcQm94IFQiXSxbMSwwLCJCIFxcQm94IFMiXSxbMSwxLCJBIFxcQm94IFQgXFxzcWN1cF97QSBcXEJveCBTfSBCIFxcQm94IFMiXSxbMiwyLCJCIFxcQm94IFQiXSxbMCwxLCJcXGlkX1MgXFxCb3ggdiIsMl0sWzAsMiwidSBcXEJveCBcXGlkX1MiXSxbMiwzXSxbMSwzXSxbMywwLCIiLDEseyJzdHlsZSI6eyJuYW1lIjoiY29ybmVyIn19XSxbMiw0LCIiLDAseyJjdXJ2ZSI6LTJ9XSxbMSw0LCIiLDAseyJjdXJ2ZSI6Mn1dLFszLDQsInUgXFxCb3gnIHYiLDAseyJzdHlsZSI6eyJib2R5Ijp7Im5hbWUiOiJkYXNoZWQifX19XV0=
  \[\begin{tikzcd}
    {A \Box S} & {B \Box S} \\
    {A \Box T} & {A \Box T \sqcup_{A \Box S} B \Box S} \\
    && {B \Box T}
    \arrow["{\id_S \Box v}"', from=1-1, to=2-1]
    \arrow["{u \Box \id_S}", from=1-1, to=1-2]
    \arrow[from=1-2, to=2-2]
    \arrow[from=2-1, to=2-2]
    \arrow["\lrcorner"{anchor=center, pos=0.125, rotate=180}, draw=none, from=2-2, to=1-1]
    \arrow[curve={height=-12pt}, from=1-2, to=3-3]
    \arrow[curve={height=12pt}, from=2-1, to=3-3]
    \arrow["{u \Box' v}", dashed, from=2-2, to=3-3]
  \end{tikzcd}\]
\end{definition}

\begin{remark}
  構成$(u,v) \mapsto u \Box' v$は双関手$ - \Box' ? : \sSet^I \times \sSet^I \to \sSpace^I$を定める. 
\end{remark}

この双関手は両方の変数に対して右随伴を持つ. 

\begin{definition}
  $u : A \to B$を単体的集合の射, $f : X \to Y$を単体的空間の射とする. 
  このとき, 次のpullbackの普遍性から定まる単体的集合の射を次のように表す. 
  \begin{align*}
    \angle{u \backslash f} := B \backslash X \to B \backslash Y \times_{A \backslash Y} A \backslash X
  \end{align*}
  % https://q.uiver.app/#q=WzAsNSxbMSwxLCJCIFxcYmFja3NsYXNoIFkgXFx0aW1lc197QSBcXGJhY2tzbGFzaCBZfSBBIFxcYmFja3NsYXNoIFgiXSxbMSwyLCJCIFxcYmFja3NsYXNoIFkiXSxbMiwxLCJBIFxcYmFja3NsYXNoIFgiXSxbMiwyLCJBIFxcYmFja3NsYXNoIFkiXSxbMCwwLCJCIFxcYmFja3NsYXNoIFgiXSxbMCwxXSxbMCwyXSxbMiwzLCJcXGlkX0EgXFxiYWNrc2xhc2ggZiJdLFsxLDMsInUgXFxiYWNrc2xhc2ggXFxpZF9ZIiwyXSxbMCwzLCIiLDEseyJzdHlsZSI6eyJuYW1lIjoiY29ybmVyIn19XSxbNCwwLCJcXGFuZ2xle3UgXFxiYWNrc2xhc2ggZn0iLDAseyJzdHlsZSI6eyJib2R5Ijp7Im5hbWUiOiJkYXNoZWQifX19XSxbNCwxLCIiLDEseyJjdXJ2ZSI6Mn1dLFs0LDIsIiIsMSx7ImN1cnZlIjotMn1dXQ==
  \[\begin{tikzcd}
    {B \backslash X} \\
    & {B \backslash Y \times_{A \backslash Y} A \backslash X} & {A \backslash X} \\
    & {B \backslash Y} & {A \backslash Y}
    \arrow[from=2-2, to=3-2]
    \arrow[from=2-2, to=2-3]
    \arrow["{\id_A \backslash f}", from=2-3, to=3-3]
    \arrow["{u \backslash \id_Y}"', from=3-2, to=3-3]
    \arrow["\lrcorner"{anchor=center, pos=0.125}, draw=none, from=2-2, to=3-3]
    \arrow["{\angle{u \backslash f}}", dashed, from=1-1, to=2-2]
    \arrow[curve={height=12pt}, from=1-1, to=3-2]
    \arrow[curve={height=-12pt}, from=1-1, to=2-3]
  \end{tikzcd}\]
\end{definition}

\begin{definition}
  $v : S \to T$を単体的集合の射, $f : X \to Y$を単体的空間の射とする. 
  このとき, 次のpullbackの普遍性から定まる単体的集合の射を次のように表す.
  \begin{align*}
    \angle{f/v} := X/T \to Y/T \times_{Y/S} X/S
  \end{align*}
  % https://q.uiver.app/#q=WzAsNSxbMSwxLCJZL1QgXFx0aW1lc197WS9TfSBYL1MiXSxbMSwyLCJZL1QiXSxbMiwxLCJYL1MiXSxbMiwyLCJZL1MiXSxbMCwwLCJYL1QiXSxbMCwxXSxbMCwyXSxbMiwzLCJmL1xcaWRfUyJdLFsxLDMsIlxcaWRfWS92IiwyXSxbMCwzLCIiLDEseyJzdHlsZSI6eyJuYW1lIjoiY29ybmVyIn19XSxbNCwwLCJcXGFuZ2xle2Yvdn0iLDAseyJzdHlsZSI6eyJib2R5Ijp7Im5hbWUiOiJkYXNoZWQifX19XSxbNCwxLCIiLDEseyJjdXJ2ZSI6Mn1dLFs0LDIsIiIsMSx7ImN1cnZlIjotMn1dXQ==
  \[\begin{tikzcd}
    {X/T} \\
    & {Y/T \times_{Y/S} X/S} & {X/S} \\
    & {Y/T} & {Y/S}
    \arrow[from=2-2, to=3-2]
    \arrow[from=2-2, to=2-3]
    \arrow["{f/\id_S}", from=2-3, to=3-3]
    \arrow["{\id_Y/v}"', from=3-2, to=3-3]
    \arrow["\lrcorner"{anchor=center, pos=0.125}, draw=none, from=2-2, to=3-3]
    \arrow["{\angle{f/v}}", dashed, from=1-1, to=2-2]
    \arrow[curve={height=12pt}, from=1-1, to=3-2]
    \arrow[curve={height=-12pt}, from=1-1, to=2-3]
  \end{tikzcd}\]
\end{definition}

\begin{remark}
  構成$(u,f) \mapsto \angle{u \backslash f}$は双関手$(\sSet^\myop)^I \times \sSpace^I \to \sSet^I$を定める. 
  特に, 任意の単体的集合$A$に対して, 次が成立する.
  \begin{align*}
    \angle{(\emptyset \to A) \backslash f} \cong A \backslash f
  \end{align*}
\end{remark}

\begin{remark}
  構成$(f,v) \mapsto \angle{f/v}$は双関手$\sSpace^I \times (\sSet^\myop)^I \to \sSet^I$を定める. 
  特に, 任意の単体的空間$X$に対して, 次が成立する. 
  \begin{align*}
    \angle{(X \to 1)/v} \cong X/v
  \end{align*}
\end{remark}

\begin{remark}
  % $u : A \to B , v : S \to T$を単体的集合の射, $f : X \to Y$を単体的空間の射とする. 
  次の随伴が存在する. 
  \begin{align*}
    (- \Box' ?) \dashv \angle{- \backslash ?}, ~~ (- \Box' ?) \dashv \angle{-/?}
  \end{align*}
\end{remark}

\subsection{$\sSpace$上の垂直Reedyモデル構造}

\begin{notation}
  任意の$n \geq 0$に対して, 単体的集合の包含$\partial \Delta[n] \hookrightarrow \Delta[n]$を$\delta_n$と表す. 

  任意の$n \geq 0$に対して, 包含$\Lambda[n,k] \hookrightarrow \Delta[n]$を$h^k_n$と表す. 
\end{notation}

\begin{proposition} \label{prop:mono_in_sSpace}
  $\sSpace$におけるmono射は次の射の集合の飽和クラスである.
  \begin{align*}
    \{\delta_m \Box' \delta_n : (\partial \Delta[m]  \Box \Delta[n]) \cup (\Delta[m] \Box \partial \Delta[n]) \hookrightarrow \Delta[m] \cup \Delta[n] ~|~ m,n \geq 0\}
  \end{align*} 
\end{proposition}

\begin{definition}[自明なファイブレーション]
  任意の$m,n \geq 0$において, $\delta_m \Box' \delta_n$に対してRLPを持つ射を単体的空間の自明な垂直ファイブレーション(trivial vertical fibration)という.
\end{definition}

\begin{proposition} \label{prop:trivial_fibration}
  単体的空間の射$f : X \to Y$に対して, 次は全て同値である. 
  \begin{enumerate}
    \item $f$は単体的空間の自明な垂直ファイブレーションである.
    \item 任意の$m \geq 0$に対して, $\angle{\delta_m \backslash f}$はKan trivial fibraionである. 
    \item 任意の単体的集合のmono射$u$に対して, $\angle{u \backslash f}$はKan trivial fibraionである. 
    \item 任意の$n \geq 0$に対して, $\angle{f/\delta_n}$はKan trivial fibraionである. 
    \item 任意の単体的集合のmono射$v$に対して, $\angle{f/v}$はKan trivial fibraionである.
  \end{enumerate}
\end{proposition}

$\sSpace$上の垂直Reedyモデル構造におけるweak equivalenceを定義する. 

\begin{definition}[各列Kan weak equivalence]
  $f : X \to Y$を単体的空間の射とする.
  任意の$m \geq 0$に対して, $\Delta[m] \backslash f = f_{m,-} : X_{m,-} \to Y_{m,-}$がKan weak equivalenceのとき, $f$を各列Kan weak equivalence(column-wise Kan weak equivalence)という.
\end{definition}

$\sSpace$上の垂直Reedyモデル構造におけるfibrationを定義する. 

\begin{definition}[垂直ファイブレーション]
  $f : X \to Y$を単体的空間の射とする.
  任意の$m \geq 0$に対して, $\angle{\delta_m \backslash f}$がKan fibraionのとき, $f$を垂直ファイブレーション(vertical fibration)という. 
\end{definition}

\begin{proposition} \label{prop:v-fibration}
  単体的空間の射$f : X \to Y$に対して, 次は全て同値である. 
  \begin{enumerate}
    \item $f$は垂直ファイブレーションである.
    \item 任意の$m \geq 0$に対して, $\angle{\delta_m \backslash f}$はKan fibraionである. 
    \item 任意の単体的集合のmono射$u$に対して, $\angle{u \backslash f}$はKan fibraionである. 
    \item 任意の$n \geq 0$と$0 \leq k \leq n$に対して, $\angle{f/h^k_n}$はKan trivial fibraionである. 
    \item 任意の単体的集合の緩射$v$に対して, $\angle{f/v}$はKan trivial fibraionである.
  \end{enumerate}
\end{proposition}

\begin{definition}[垂直ファイブラント]
  $X$を単体的空間とする. 
  単体的空間の射$X \to \Delta[0]^t$が垂直ファイブレーションのとき, $X$を垂直ファイブラント(vertical fibrant)という. 
\end{definition}

\cref{prop:trivial_fibration}と同様に, 垂直ファイブレーションを特徴づけることができる. 

% \begin{lemma} \label{prop:vfib_is_angle_Kanfib} % Rie08のThorem 4.2が参考になりそう.
%   $f$を単体的空間の射とする.
%   $f$が垂直ファイブレーションのとき, 任意の単体的集合のmono射$u$に対して, $\angle{u \backslash f}$はKan fibraionである.
% \end{lemma}

\subsection{$\sSpace$上の水平Reedyモデル構造}

\begin{definition}[圏的定値]
  $X$を単体的空間とする. 
  任意の$n \geq 0$に対して, $\Delta$における射$[n] \to [0]$が定める単体的集合の射$X_{-,0} \to X_{-,n}$がJoyal weak equivalenceのとき, $X$は圏的定値(categorically constant)であるという.
\end{definition}

\begin{example}
  任意の垂直ファイブラントは圏的定値である. 
\end{example}

\begin{proof}
  $X$を垂直ファイブラントな単体的空間とする. 
  このとき, 単体的空間の射$X \to \Delta[0]$は垂直ファイブレーションである.
  任意の$n \geq 0$に対して, $i : \Delta[0] \hookrightarrow \Delta[n]$はKan trivial cofibraionである.
  \cref{prop:v-fibration}より, $X / i : X / \Delta[n] \to X / \Delta[0]$はKan trivial fibrationである. 
  一意な射$t : \Delta[n] \to \Delta[0]$は$ti=\id_{\Delta[0]}$を満たす. 
  このとき, $(X/i) \circ (X/t) = \id_{X / \Delta[0]}$である. 
  2-out-of-3より, $X/t$もKan weak equivalenceである. 
  ここで, $X / \Delta[n] \cong X_{-,n}$である. 
  つまり, $X_{-,n} \to X_{-,n}$はKan weak equivalenceである. (ことまでしか言えない気がする.)
\end{proof}

$\sSpace$上の水平Reedyモデル構造におけるweak equivalenceを定義する. 

\begin{definition}[各行Joyal weak equivalence]
  $f : X \to Y$を単体的空間の射とする. 
  任意の$n \geq 0$に対して, $\Delta[n] \backslash f = f_{-,n} : X_{-,n} \to Y_{-,n}$がJoyal weak equivalenceのとき, $f$を各行Joyal weak equivalence(row-wise Joyal weak equivalence)という. 
\end{definition}

$\sSpace$上の水平Reedyモデル構造におけるfibrationを定義する. 

\begin{definition}[水平ファイブレーション]
  $f : X \to Y$を単体的空間の射とする. 
  任意の$n \geq 0$に対して, $\angle{f / \delta_n}$がJoyal fibrationのとき, $f$を水平ファイブレーション(horizontal fibration)という. 
\end{definition}

\begin{definition}[水平ファイブラント]
  $X$を単体的空間とする. 
  単体的空間の射$X \to \Delta[0]^t$が水平ファイブレーションのとき, $X$を水平ファイブラント(horizontal fibrant)という. 
\end{definition}

\subsection{補足 : Segal空間と完備Segal空間について}

\begin{notation}
  任意の$n \geq 0$に対して, $n$鎖を$I_n \subset \Delta[n]$と表す. 
\end{notation}

\begin{remark} \label{rem:InX_is}
  任意の単体的空間$X$に対して, 任意の$n \geq 0$において, 次の単体的空間の同型が存在する. 
  \begin{align*}
    I_n \backslash X = X_1 \times_{X_0} \cdots \times_{X_0} X_1
  \end{align*}
\end{remark}

\begin{lemma} \label{prop:complete_is}
  Segal空間$X$に対して, 次の3つは同値である.
  \begin{enumerate}
    \item $X$は完備である. 
    \item $t : J \to \{0\}$から定まる単体的集合の射$t \backslash X : 1 \backslash X \to J \backslash X$はKan weak equivalenceである. 
    \item 包含$u_0 : \{0\} \hookrightarrow J$から定まる単体的集合の射$u_0 \backslash X : J \backslash X \to 1 \backslash X$はKan trivial fibrationである.
  \end{enumerate}  
\end{lemma}

\begin{proof}
  (2)と(3)の同値性を示す. 
  $t : J \to \{0\}$と$u_0 : \{0\} \hookrightarrow J$は$tu_0 = \id_{\{0\}}$を満たすので, $(u_0 \backslash X) \circ (t \backslash X) = \id_{1 \backslash X}$である. 
  2-out-of-3より, $t \backslash X$がKan weak equivalenceであることと, $u_0 \backslash X$がKan weak equivalenceであることは同値である.
  また, \cref{prop:v-fibration}より, $u_0 \backslash X$はKan fibraionである. 
\end{proof}



\newpage


% \section{Segal前圏について}

% \subsection{Segal前圏の定義}

% \begin{definition}[Segal前圏]
%   $X$を単体的空間とする.
%   $X_0$が離散単体的集合のとき, $X$をSegal前圏(Segal precategory)という. 
% \end{definition}

% \begin{definition}[Segal圏]
%   $X$をSegal前圏とする. 
%   任意の$n \geq 2$に対して, Segal写像
%   \begin{align*}
%     \varphi_n : X_n \to X_1 \times_{X_0} \cdots \times_{X_0} X_1 
%   \end{align*}
%   が単体的集合のweak equivalenceのとき, $X$をSegal圏(Segal category)という.
% \end{definition}

% \begin{remark}
%   Segal圏とSegal空間は似ているが, 次の意味で異なっている. 
%   \begin{itemize}
%     \item Segal圏はSegal空間と異なり, Reedyファイブラントであることを課していない. 
%     よって, Segal圏がSegal空間であるとは限らない. 
%     \item Segal空間はSegal圏と異なり, $X_0$が離散単体的集合であることを課していない.
%     よって, Segal空間がSegal圏であるとは限らない. 
%   \end{itemize}
% \end{remark}

% \begin{notation}
%   Segal前圏のなす圏を$\PreSeCat$と表す. 
% \end{notation}

% \begin{definition}[簡約関手]
  
% \end{definition}

% \newpage


% \section{単体的圏について} \label{sec:simplicial_category}

% 単体的圏(simplicial category)とは$\sSet$豊穣圏のことである. 
% weak equivalenceの集まりを持つ圏に対して, 単体的局所化(simplicial localization)とハンモック局所化(hammock localization)という2つの方法によって単体的圏が得られる. 
% このような圏としてモデル圏を考えると, 単体的局所化の$0$次ホモトピー群をとることで, モデル圏のホモトピー圏が得られることが分かる. 
% よって, 単体的圏のなす圏のモデル構造は「ホモトピー圏のホモトピー圏」のモデルとして考えられ, 単体的圏は$(\infty,1)$圏のモデルとみなせる. (\cref{rem:homotopy_theory_of_homotopy_theory})
% 単体的圏のなす圏のモデル構造は\cref{sec:model_stru_in_SC}で見る. 
% \cref{sec:simplicial_category}では, 単体的圏の基本的な事柄を復習する. 

% \subsection{単体的圏と$O$対象固定な単体的圏}

% 単体圏の基本的なことについて復習する. 

% \begin{definition}[単体的圏]
%   $\sSet$豊穣圏を単体的圏(simplicial category)という. 
% \end{definition}

% 単体的圏の間の関手を定義する. 

% \begin{definition}[単体的関手]
%   $\sSet$豊穣関手を単体的関手(simplicial functor)という. 
% \end{definition}

% \begin{notation}
%   単体的圏のなす圏を$\sSetCat$と表す. 
% \end{notation}

% \begin{proposition}
%   $\sSetCat$は有限直積と有限余直積を持つ. 
% \end{proposition}

% 単体的圏の間のweak equivalenceとしてDwyer-Kan同値がある. 

% \begin{definition}[Dwyer-Kan同値]
%   $\C,\D$を単体的圏, $F : \C \to \D$を単体的関手とする. 
%   $F$が次の条件を満たすとき, $F$をDwyer-Kan同値(Dwyer-Kan equivalence)という.
%   \begin{enumerate}
%     \item $\C$の任意の対象$x,y$に対して, $\Map_\C(x,y) \to \Map_\D(Fx,Fy)$は単体的集合のweak equivalenceである. 
%     \item 誘導される関手$\pi_0F : \pi_0\C \to \pi_0\D$は圏同値である. 
%   \end{enumerate}
% \end{definition} 

% \begin{remark}
%   Dwyer-Kan同値の条件(W1)より, (W2)は$\pi_0F$が本質的全射であるとしてもよい. 
% \end{remark}

% 全ての単体的圏と単体的関手のなす圏$\sSetCat$を考えることが目標である. 
% 簡単のためにまずは, 固定した集合の集まりが対象である単体的圏と, その集合に対して恒等的に作用する単体的関手のなす$\sSetCat$の部分圏を考える. 

% \begin{definition}[$O$対象固定な単体的圏, 単体的関手]
%   $O$を集合とする. 
%   対象が$O$である単体的圏を$O$対象固定な単体的圏($O$-fixed-object simplicial category)という. 
%   対象に関して恒等的に作用する$O$対象固定な単体的圏の間の単体的関手を対象固定な単体的関手($O$-fixed-object simplicial functor)という. 
% \end{definition}

% \begin{notation}
%   $O$を集合とする. 
%   $O$対象固定な単体的圏のなす圏を$\sSetCat_O$と表す. 
% \end{notation}

% $\sSetCat$における極限や余極限は対象の集合を保たない. 
% 例えば, $\C, \D$を2点対象の単体的圏とする. 
% このとき, 直積$\C \times \D$は4点対象の単体的圏である. 
% そのため, $\sSetCat_O$が(余)直積で閉じるためには, (余)直積の定義を修正する必要がある.

% \begin{definition}[自由積]
  
% \end{definition}

% \begin{definition}[自由射]
%   $F : \C \to \D$を$O$対象固定な単体的関手とする. 
%   $F$が次の条件を満たすとき, $F$を自由射(free map)という. 
%   \begin{itemize}
%     \item $F$は射空間上においてmono射である. 
%     \item 任意の次数$k$に対して, 圏$\D_k$は
%   \end{itemize}
% \end{definition}

% \subsection{単体的局所化とハンモック局所化}

% 圏$\C$と弱同値の集まり$W$に対して, $W$による$\C$の局所化$\C[W^{-1}]$が得られる. 
% モデル圏$\M$とweak equivalenceの集まり$W$に対して, 局所化$\M[W^{-1}]$は$\M$のホモトピー圏$\Ho(\M)$に一致する. 
% 一般には, このモデル圏の局所化は極限や余極限を保たない. 
% よって, ホモトピー圏ではなく, ホモトピーの情報を保った圏の局所化を考える必要がある. 
% この問題を考えるために2つの方法がある. 
% 1つは単体的局所化(simplicial localization), もう1つはハンモック局所化(hammock localization)である. 

% % \begin{definition}[自由圏]
  
% % \end{definition}

% % \begin{definition}[standard simplicial resolution]
% %   $\C$を圏, $F^n\C$を$\C$の$n$重自由圏とする. 

% % \end{definition}

% 次にハンモック局所化を見る. 

% \begin{notation}
%   $(\C,W)$をweak equivalenceの集まりを持つ圏とする. 
%   任意の$n \geq 0$に対して, 圏$\D_n$を次のように定義する. 
%   \begin{itemize}
%     \item $\D_n$の対象は長さ$n$の次のように表される$\C$のzigzag射
%     % https://q.uiver.app/#q=WzAsNSxbMCwwLCJYIl0sWzEsMCwiQ18wIl0sWzIsMCwiQ18xIl0sWzMsMCwiXFxjZG90cyJdLFs0LDAsIlkiXSxbMSwwLCJcXHNpbSIsMl0sWzEsMl0sWzMsMiwiXFxzaW0iLDJdLFszLDRdXQ==
%     \[\begin{tikzcd}
%       X & {C_0} & {C_1} & \cdots & Y
%       \arrow["\sim"', from=1-2, to=1-1]
%       \arrow[from=1-2, to=1-3]
%       \arrow["\sim"', from=1-4, to=1-3]
%       \arrow[from=1-4, to=1-5]
%     \end{tikzcd}\]
%     ここで, 左向きの射はweak equivalenceであり, 隣接する射は逆向きであり, 恒等射を含まない. 
%     \item $\D_n$の射は次のように表される$\C$の射の集まり (これをハンモック図式という.)
%     % https://q.uiver.app/#q=WzAsMTEsWzAsMSwiWCJdLFsxLDAsIkNfezAsMX0iXSxbMiwwLCJDX3sxLDJ9Il0sWzMsMCwiXFxjZG90cyJdLFs0LDEsIlkiXSxbMSwxLCJcXGNkb3RzIl0sWzIsMSwiXFxjZG90cyJdLFszLDEsIlxcY2RvdHMiXSxbMSwyLCJDX3trLDF9Il0sWzIsMiwiQ197aywyfSJdLFszLDIsIlxcY2RvdHMiXSxbMSwwLCJcXHNpbSIsMl0sWzEsMl0sWzMsMiwiXFxzaW0iLDJdLFsxLDUsIlxcc2ltIl0sWzIsNiwiXFxzaW0iXSxbMyw3LCJcXHNpbSJdLFs1LDgsIlxcc2ltIl0sWzYsOSwiXFxzaW0iXSxbMTAsOSwiXFxzaW0iXSxbNywxMCwiXFxzaW0iXSxbNSw2XSxbNyw2LCJcXHNpbSJdLFs1LDAsIlxcc2ltIl0sWzgsMCwiXFxzaW0iXSxbMyw0XSxbNyw0XSxbMTAsNF1d
%     \[\begin{tikzcd}
%       & {C_{0,1}} & {C_{1,2}} & \cdots \\
%       X & \cdots & \cdots & \cdots & Y \\
%       & {C_{k,1}} & {C_{k,2}} & \cdots
%       \arrow["\sim"', from=1-2, to=2-1]
%       \arrow[from=1-2, to=1-3]
%       \arrow["\sim"', from=1-4, to=1-3]
%       \arrow["\sim", from=1-2, to=2-2]
%       \arrow["\sim", from=1-3, to=2-3]
%       \arrow["\sim", from=1-4, to=2-4]
%       \arrow["\sim", from=2-2, to=3-2]
%       \arrow["\sim", from=2-3, to=3-3]
%       \arrow["\sim", from=3-4, to=3-3]
%       \arrow["\sim", from=2-4, to=3-4]
%       \arrow[from=2-2, to=2-3]
%       \arrow["\sim", from=2-4, to=2-3]
%       \arrow["\sim", from=2-2, to=2-1]
%       \arrow["\sim", from=3-2, to=2-1]
%       \arrow[from=1-4, to=2-5]
%       \arrow[from=2-4, to=2-5]
%       \arrow[from=3-4, to=2-5]
%     \end{tikzcd}\]
%     ここで, 垂直な射はweak equivalenceであり, 同じ列の水平な射は全て同じ向きである. 
%   \end{itemize}
% \end{notation}

% \begin{notation}
%   圏$\D_n$に対して, 単体的集合$L^\h\C(X,Y)$を次のように定義する. 
%   \begin{itemize}
%     \item 任意の$n \geq 0$に対して, $L^\h\C(X,Y)_n$は$\D_n$の脈体
%     \item 面写像と退化写像は通常の脈体と同様
%   \end{itemize}
% \end{notation}

% \begin{remark}
%   $\C$の任意の対象$X,Y,Z$に対して, 単体的集合の射
%   \begin{align*}
%     L^\h\C(X,Y) \times L^\h\C(Y,Z) \to L^\h\C(X,Z)
%   \end{align*}
%   がそれぞれのハンモック図式を水平に繋げることで定まる. 
% \end{remark}

% \begin{definition}[ハンモック局所化]
%   $(\C,W)$をweak equivalenceの集まりを持つ圏とする. 
%   このとき, 単体的圏$L^H(\C,W)$を次のように定義し, $L^H(\C,W)$を$W$による$\C$のハンモック局所化(hammock localization)という. 
%   \begin{itemize}
%     \item $L^H(\C,W)$の対象は$\C$と同じ
%     \item $L^H(\C,W)$の任意の対象$X,Y$に対して, 射空間は単体的集合$L^\h\C(X,Y)$
%   \end{itemize}
% \end{definition}

% 単体的圏の2つの局所化は単体的圏として同値(Dwyer-Kan同値)である. 
% ここでは単体的モデル圏についての主張を述べる. 

% \begin{proposition}
%   $\M$をモデル圏, $W$を$\M$のweak equivalenceの集まりとする. 
%   このとき, $\M$の単体的局所化$L(\M,W)$とハンモック局所化$L^H(\M,W)$はDwyer-Kan同値である.
% \end{proposition}

% よって, 特に単体的モデル圏$\M$に対して, ハンモック局所化(または単体的局所化)により, $\M$の単体的構造を復元することができる. 

% \begin{proposition}
%   $\M$を単体的モデル圏, $W$を$\M$のweak equivalenceの集まりとする. 
%   このとき, $\M$と$\M$のハンモック局所化$L^H(\M,W)$はDwyer-Kan同値である.
% \end{proposition}

% 更に, 次の命題が成立する.

% \begin{proposition}
%   任意の単体的圏はDwyer-Kan同値の違いを除いて, weak equivalenceの集まりを持つ圏の単体的局所化(またはハンモック局所化)として得られる. 
% \end{proposition}

% \begin{remark} \label{rem:homotopy_theory_of_homotopy_theory}  
%   「ホモトピー論」をweak equivalenceを持つ圏であって, ホモトピー圏をとる操作が定義されている圏と思うとき, 単体的圏はホモトピー論の1つのモデルとして見れる. 
%   モデル圏に対して, 単体的局所化をとり$0$次ホモトピー群をとることで, ホモトピー圏が得られることを見た. 
%   つまり, 単体的圏のなす圏のモデル構造を考えることは, 「ホモトピー論のホモトピー論」とみなすことができる. 
%   よって, 単体的圏は圏論的に定義されたものであったが, ホモトピー論としての側面も持っているので, 単体的圏は$(\infty,1)$圏のモデルとして考えられる.  
% \end{remark}

% \subsection{単体的脈体とホモトピー連接脈体}

% 単体的圏に対する脈体を考える. 

% \begin{definition}[単体的脈体]
%   $\C$を単体的圏とし, 関手$\Delta^\myop \to \Cat$とみなす. 
%   \footnote{
%     通常の脈体$N$は$N : \Cat \to \sSet : \C \mapsto \Hom_\Cat([-],\C)$である. 
%     また, $\C : \Delta^\myop \to \Cat : [n] \mapsto C_n$である. 
%   }
%   単体的空間$SN(\C)$を任意の$n \geq 0$に対して次のように定義し, $\C$の単体的脈体(simplicial nerve)という. 
%   \begin{align*}
%     SN(\C)_{-,n} := N(\C_n) = \Hom_\Cat([-],\C_n)
%   \end{align*}
% \end{definition}

% \begin{remark}
%   任意の$n \geq 0$に対して, 
%   \begin{align*}
%     SN(\C)_{0,n} = \Ob(\C_n) = \Ob(\C)
%   \end{align*}
%   よって, 任意の単体的空間$\C$に対して, $SN(\C)_0$は離散的である. 
% \end{remark}

% \begin{remark}
%   $\C$の合成は単体的集合の同型
%   \begin{align*}
%     SN(\C)_n \cong SN(\C)_1 \times_{SN(\C)_0} \cdots \times_{SN(\C)_0} SN(\C)_1
%   \end{align*}
%   を定める. 
% \end{remark}

% 単体的脈体は$\C : \Delta^\myop \to \Cat : [n] \mapsto C_n$を用いて定義された. 
% この$[n]$をfree resolution $F_\ast[n]$に置き換えることで, 単体圏の情報を単体的集合に与えることができる. 

% \begin{definition}[ホモトピー連接脈体]
%   $\C$を単体的圏とする. 
%   単体的集合$\tilde{N}(\C)$を任意の$n \geq 0$に対して次のように定義し, $\tilde{N}(\C)$を$\C$のホモトピー連接脈体(homotopy coherence nerve)という. 
%   \begin{align*}
%     \tilde{N}(\C)_n := \Hom_\sSetCat(F_\ast[n],\C)
%   \end{align*}
%   ここで, $F_\ast[n]$は圏$[n]$のfree resolutionである. 
% \end{definition}

% 単体的圏の単体的脈体やホモトピー連接脈体はそれぞれ, Segal空間と擬圏のモデルを考えるときに重要である. (\cref{sec:quillen_equiv_sSetJoyal_and_SCBergner}を参照)

% \newpage


\section{相対圏について} \label{sec:relative_category}

ホモトピー論の視点から考えると, $(\infty,1)$圏は弱同値を構造として持つことが要請される. 
弱同値を持つような圏を相対圏(relative category)といい, 相対圏のなす圏にモデル構造を入れることを考える. 
相対圏はBarwickとKan \cite{BK11}により, ホモトピー論のホモトピー論のモデルとして導入された. 

\subsection{相対圏の定義}

相対圏は「weak equivalenceの集まりを持つ圏」であり, すでに様々なところで出てきたが, 改めて相対圏を定義する. 

\begin{definition}[相対圏]
  $\C$を圏, $W$を$\C$の全ての対象を含む$\C$の部分圏とする. 
  このとき, 2つ組$(\C,W)$を相対圏(relative category)という. 
  また, $\C$をunderlying category, $\W$の射をweak equivalenceという. 
  相対圏$(\C,W)$を単に$\C$だけで表すこともある. 
\end{definition}

圏$\C$が与えられたとき, 2つの極端な相対圏の構造が考えられる. 
1つは$\C$の全ての射が$W$の射である場合, もう1つは恒等射のみが$W$の射である場合である. 

\begin{definition}[極大と極小]
  $(\C,W)$を相対圏とする. 
  $W=\C$のとき, $(\C,W)$は極大(maximal)であるという. 
  $W$が恒等射以外の射を含まないとき, $(\C,W)$は極小(minimal)であるという.
  圏$\C$に対する極大相対圏を$\C_\max$, 極小相対圏を$\C_\min$と表す. 
\end{definition}

相対圏の間の関手を定義する. 

\begin{definition}[相対関手]
  $(\C,W), (\D,V)$を相対圏とする. 
  関手$F : \C \to \D$が$F(W) \subset V$を満たすとき, $F : (\C,W) \to (\D,V)$を相対関手(relative functor)という. 
\end{definition}

\begin{definition}[相対包含]
  $i : (\C,W) \to (\D,V)$を相対関手とする. 
  関手$i : \C \to \D$が通常の包含かつ, $W= V \cap \C$を満たすとき, $i$を相対包含(relative inclusion)という. 
\end{definition}

\begin{notation}
  相対圏と相対関手のなす圏を$\RelCat$と表す. 
\end{notation}

相対圏の直積を定義する. 

\begin{definition}[直積]
  相対圏$\C,\D$に対して, 相対圏$\C \times \D$を次のように定義し, $\C$と$\D$の直積(product)という. 
  \begin{itemize}
    \item $\C \times \D$の対象は$\C$の対象と$\D$の対象の組
    \item $\C \times \D$の射は$\C$の射と$\D$の射の組
    \item $\C \times \D$のweak equivalenceは$\C$のweak equivalenceと$\D$のweak equivalenceの組
  \end{itemize}
\end{definition}

相対圏のべき対象を定義する. 

\begin{definition}[べき対象]
  相対圏$\C,\D$に対して, 相対圏$\D^\C$を次のように定義し, $\D$のべき対象(exponential object)という. 
  \begin{itemize}
    \item $\D^\C$の対象は相対関手$\C \to \D$
    \item $\D^\C$の射は相対関手$\C \times [1]_\min \to \D$
    \item $\D^\C$のweak equivalenceは相対関手$\C \times [1]_\max \to \D$
  \end{itemize}
\end{definition}

\begin{proposition}
  $\RelCat$はCartesian閉である.
\end{proposition}

\begin{proof}
  定義より, $\RelCat$は有限直積を持つ. 
  $\C$を相対圏とする. 
  直積をとる関手$- \times \C : \RelCat \to \RelCat$が右随伴$(-)^\C : \RelCat \to \RelCat$を持つことを示せばよい. 
  これは通常の圏の直積とべきの随伴性から従う. 
\end{proof}

相対関手の間の強ホモトピーを定義する. 

\begin{definition}[強ホモトピー]
  $F,G : \C \to \D$を相対関手とする. 
  相対関手$H : \C \times [1]_\max \to \D$が$\C$の任意の対象$C$と射$f$に対して, 
  \begin{align*}
    & H(C,0)=F(C),~ H(C,1)=G(C) \\
    & H(f,0)=F(f),~ H(f,1)=G(f) 
  \end{align*}
  を満たすとき, $H$を$F$から$G$への強ホモトピー(strong homotopy)という. 
  このとき, $F$と$G$は強ホモトピック(strong homotopic)であるという. 
\end{definition}

\begin{definition}[ホモトピー同値]
  $F: \C \to \D$を相対関手とする. 
  ある相対関手$G : \D \to \C$が存在して, $GF$が$\Id_\C$と強ホモトピックかつ$FG$が$\Id_\D$と強ホモトピックであるとき, $F$をホモトピー同値(homotopy equivalence)という. 
\end{definition}

強ホモトピーは$\RelCat$のCartesian構造と整合性がある. 

\begin{proposition} \label{prop:homotopic_compatible_power_in_RelCat}
  $F,G : \C \to \D$を強ホモトピックな相対関手とする. 
  このとき, 任意の相対圏$\E$に対して, 相対関手$F^\ast,G^\ast : \E^\D \to \E^\C$は強ホモトピックである. 
\end{proposition}

\begin{proof}
  $H : \C \times [1]_\max \to \D$を$F$から$G$への強ホモトピーとする. 
  このとき, $H$は相対関手
  \begin{align*}
    H^\ast : \E^\D \to \E^{\C \times [1]_\max} \cong (\E^\C)^{[1]_\max}
  \end{align*}
  を定める. 
  $H^\ast$から, 積とべきの随伴で与えられる相対関手
  \begin{align*}
    \E^\D \times [1]_\max \to \E^\C
  \end{align*}
  が得られる. 
  この$H^\ast$は$\E^\D$から$\E^\C$への強ホモトピーである. 
  よって, $F^\ast$と$G^\ast$は強ホモトピックである. 
\end{proof}

\begin{corollary} \label{prop:homotopy_equiv_compatible_power}
  $F : \C \to \D$をホモトピー同値とする.
  このとき, 任意の相対圏$\E$に対して, 相対関手$F^\ast : \E^\D \to \E^\C$はホモトピー同値である.
\end{corollary}

相対関手の強ホモトピーを用いて, 相対包含の強分解レトラクトを定義する. 
強分解レトラクトは$\RelCat$上のモデル構造におけるcofibrationの定義に用いる. 

\begin{definition}[強分解レトラクト]
  $i : \C \to \D$を相対包含とする. 
  このとき, $ri = \Id_\C$を満たす相対関手$r : \D \to \C$と$ir$から$\Id_\D$への強ホモトピー$S$の組$(r,S)$を$\D$から$\C$への強分解レトラクト(strong deformation retraction)という. 
  単に, $D$を$C$の強分解レトラクトということもある. 
  % https://q.uiver.app/#q=WzAsNCxbMCwwLCJcXEMiXSxbMSwwLCJcXEQiXSxbMiwwLCJcXEMiXSxbMywwLCJcXEQiXSxbMCwxLCJpIiwwLHsic3R5bGUiOnsidGFpbCI6eyJuYW1lIjoiaG9vayIsInNpZGUiOiJ0b3AifX19XSxbMSwyLCJyIl0sWzIsMywiaSIsMCx7InN0eWxlIjp7InRhaWwiOnsibmFtZSI6Imhvb2siLCJzaWRlIjoidG9wIn19fV0sWzAsMiwiXFxJZF9cXEMiLDAseyJjdXJ2ZSI6LTN9XSxbMSwzLCJcXElkX1xcRCIsMix7ImN1cnZlIjozfV0sWzIsOCwiUyIsMCx7InNob3J0ZW4iOnsidGFyZ2V0IjoyMH19XSxbNywxLCIiLDAseyJzaG9ydGVuIjp7InNvdXJjZSI6MjB9LCJzdHlsZSI6eyJoZWFkIjp7Im5hbWUiOiJub25lIn19fV1d
  \[\begin{tikzcd}
    \C & \D & \C & \D
    \arrow["i", hook, from=1-1, to=1-2]
    \arrow["r", from=1-2, to=1-3]
    \arrow["i", hook, from=1-3, to=1-4]
    \arrow[""{name=0, anchor=center, inner sep=0}, "{\Id_\C}", curve={height=-18pt}, from=1-1, to=1-3]
    \arrow[""{name=1, anchor=center, inner sep=0}, "{\Id_\D}"', curve={height=18pt}, from=1-2, to=1-4]
    \arrow["S", shorten >=1pt, Rightarrow, from=1-3, to=1]
    \arrow[shorten <=1pt, Rightarrow, no head, from=0, to=1-2]
  \end{tikzcd}\]
\end{definition}

\begin{definition}[(余)ふるい]
  $\D$を圏, $\C$を$\D$の部分圏とする. 
  \begin{itemize}
    \item $\D$の任意の射$f : d \to c$に対して, $c$が$\C$の対象ならば対象$d$と射$f$も$\C$の対象であるとき, $\C$を$\D$におけるふるい(sieve)という. 
    \item $\D$の任意の射$f : c \to d$に対して, $c$が$\C$の対象ならば対象$d$と射$f$も$\C$の対象であるとき, $\C$を$\D$における余ふるい(cosieve)という. 
  \end{itemize}
\end{definition}

(余)ふるいは相対圏の言葉を用いて次のように表せる. 

\begin{remark}
  $\C$が$\D$におけるふるいであることと, ある関手$\alpha : \D \to [1]_\max$が存在して$\alpha^{-1}(0)=\C$を満たすことは同値である. 
  相対的に, $\C$が$\D$における余ふるいであることと, ある関手$\beta : \D \to [1]_\max$が存在して$\beta^{-1}(1)=\C$を満たすことは同値である. 
\end{remark}

$\RelCat$上のモデル構造におけるfibrationの定義に用いるDwyer包含を定義する.

\begin{definition}[Dwyer包含]
  $i : (\C,W) \hookrightarrow (\D,V)$を相対包含とする. 
  $i$が次の条件を満たすとき, $i$をDwyer包含(Dwyer inclusion)という.
  \begin{itemize}
    \item $C$は$D$のふるい 
    \item $C$は$D$における$C$を含む最小の余ふるい$ZC$の強分解レトラクト 
  \end{itemize}
\end{definition}

\begin{definition}[Dwyer射]
  $F : (\C,W) \to (\D,V)$を相対関手とする. 
  $F$が圏同型$F' : (\C,W) \to (\C',W')$とDwyer包含$i : (\C',W') \hookrightarrow (D,V)$を用いて, $F=i \circ F'$と(一意に)分解できるとき, $F$をDwyer射(Dwyer map)という. 
  % https://q.uiver.app/#q=WzAsMyxbMCwwLCIoXFxDLFcpIl0sWzIsMCwiKFxcRCxWKSJdLFsxLDEsIihcXEMnLFcnKSJdLFswLDEsIkYiXSxbMCwyLCJGJyIsMix7ImxldmVsIjoyLCJzdHlsZSI6eyJoZWFkIjp7Im5hbWUiOiJub25lIn19fV0sWzIsMSwiaSIsMix7InN0eWxlIjp7InRhaWwiOnsibmFtZSI6Imhvb2siLCJzaWRlIjoidG9wIn19fV1d
  \[\begin{tikzcd}
    {(\C,W)} && {(\D,V)} \\
    & {(\C',W')}
    \arrow["F", from=1-1, to=1-3]
    \arrow["{F'}"', Rightarrow, no head, from=1-1, to=2-2]
    \arrow["i"', hook, from=2-2, to=1-3]
  \end{tikzcd}\]
\end{definition}

Dwyer射の性質をいくつか述べる. 

\begin{lemma}
  Dwyer射はレトラクトで閉じる.
\end{lemma}

\begin{lemma}
  Dwyer射は超限合成で閉じる.
\end{lemma}

\subsection{細分化関手}

underlying categoryが半順序集合(poset)である相対圏を相対半順序集合(relative poset)という. 
相対半順序集合に対する細分化関手(subdivision functor)の構成は$\RelCat$上のモデル構造を定義する上で非常に重要である. 

\begin{definition}[相対半順序集合]
  相対圏$(\P,W)$のunderlying categoryが半順序集合のとき, $(\P,W)$を相対半順序集合(relative poset)という. 
\end{definition}

\begin{notation}
  相対半順序集合のなす$\RelCat$の充満部分圏を$\RelPos$と表す. 
\end{notation}

\begin{definition}[終細分化]
  $\P$を相対半順序集合とする. 
  このとき, 相対半順序集合$\xi_t\P$を次のように定義し, $\P$の終細分化(terminal subdivision)という. 
  \begin{itemize}
    \item $\xi_t\P$の対象は$\RelPos$におけるmono射$x : [n]_\min \to \P$ ($n \geq 0$)
    \item $\xi_t\P$の射は次の図式を可換にする射$[n_1]_\min \to [n_2]_\min$
    % https://q.uiver.app/#q=WzAsMyxbMCwwLCJbbl8xXV9cXG1pbiJdLFsyLDAsIltuXzJdX1xcbWluIl0sWzEsMSwiXFxQIl0sWzAsMV0sWzAsMiwieF8xIiwyXSxbMiwxLCJ4XzIiLDJdXQ==
    \[\begin{tikzcd}
      {[n_1]_\min} && {[n_2]_\min} \\
      & \P
      \arrow[from=1-1, to=1-3]
      \arrow["{x_1}"', from=1-1, to=2-2]
      \arrow["{x_2}", from=1-3, to=2-2]
    \end{tikzcd}\]
    \item $\xi_t\P$のweak equivalenceは上の図式が誘導する射$x_1(n_1) \to x_2(n_2)$が$\P$におけるweak equivalenceである上の可換図式における射
  \end{itemize}
\end{definition}

相対半順序集合の終細分化は自然な射影を持つ. 

\begin{definition}[終射影関手]
  $\P$を相対半順序集合, $\xi_t\P$を$\P$の終細分化とする. 
  このとき, 相対関手$\pi_t : \xi_t\P \to \P$を次のように定義し, $\pi_t$を終射影関手(terminal projection functor)という.
  \begin{itemize}
    \item $\xi_t\P$の対象$x : [n]_\min \to \P$に対して, $\pi_t(x) := x(n)$
    \item $\xi_t\P$の射$[n_1]_\min \to [n_2]_\min$に対して, $\pi_t([n_1]_\min \to [n_2]_\min) := x_1(n_1) \to x_2(n_2)$ 
  \end{itemize}
\end{definition}

\begin{remark}
  $\xi_t\P$における射がweak equivalenceであることと, $\pi_t$の像が$\P$におけるweak equivalenceであることは同値である. 
\end{remark}

終細分化をとる操作は$\RelPos$上の関手を定める.

\begin{definition}[終細分化関手]
  関手$\xi_t : \RelPos \to \RelPos$を次のように定義し, $\xi_t$を終細分化関手(terminal subdivision functor)という. 
  \begin{itemize}
    \item $\RelPos$の対象$\P$に対して, $\xi_t(\P) := \xi_t\P$
    \item $\RelPos$の射$F : \P \to \Q$に対して, $\xi_t(F) : \xi_t\P \to \xi_t\Q$を次のように定める. 
    \begin{itemize}
      \item $\xi_t\P$の対象$x : [n]_\min \to \P$に対して, $\xi_t(F)(x)$は$[n]_\min \to [n]_\min$がepi射であり, 次の図式を可換にするような一意なmono射$x' : [m]_\min \to \Q$
      % https://q.uiver.app/#q=WzAsNCxbMCwwLCJbbl1fXFxtaW4iXSxbMSwwLCJbbV1fXFxtaW4iXSxbMSwxLCJcXFEiXSxbMCwxLCJcXFAiXSxbMCwxLCIiLDAseyJzdHlsZSI6eyJoZWFkIjp7Im5hbWUiOiJlcGkifX19XSxbMSwyLCJ4JyJdLFswLDMsIngiLDJdLFszLDIsIkYiLDJdXQ==
      \[\begin{tikzcd}
        {[n]_\min} & {[m]_\min} \\
        \P & \Q
        \arrow[two heads, from=1-1, to=1-2]
        \arrow["{x'}", from=1-2, to=2-2]
        \arrow["x"', from=1-1, to=2-1]
        \arrow["F"', from=2-1, to=2-2]
      \end{tikzcd}\]
      \item $\xi_t\P$の射$[n_1]_\min \to [n_2]_\min$に対して, $\xi_t(F)([n_1]_\min \to [n_2]_\min)$は次の図式を可換にするような射$[m_1]_\min \to [m_2]_\min$
      % https://q.uiver.app/#q=WzAsNixbMCwwLCJbbl8xXV9cXG1pbiJdLFsyLDAsIltuXzJdX1xcbWluIl0sWzEsMSwiXFxQIl0sWzAsMiwiW21fMV1fXFxtaW4iXSxbMSwzLCJcXFEiXSxbMiwyLCJbbV8yXV9cXG1pbiJdLFswLDFdLFswLDIsInhfMSIsMl0sWzEsMiwieF8yIl0sWzAsMywiIiwxLHsic3R5bGUiOnsiaGVhZCI6eyJuYW1lIjoiZXBpIn19fV0sWzMsNCwieF8xJyIsMl0sWzEsNSwiIiwxLHsic3R5bGUiOnsiaGVhZCI6eyJuYW1lIjoiZXBpIn19fV0sWzUsNCwieF8yJyJdLFszLDVdLFsyLDRdXQ==
      \[\begin{tikzcd}
        {[n_1]_\min} && {[n_2]_\min} \\
        & \P \\
        {[m_1]_\min} && {[m_2]_\min} \\
        & \Q
        \arrow[from=1-1, to=1-3]
        \arrow["{x_1}"', from=1-1, to=2-2]
        \arrow["{x_2}", from=1-3, to=2-2]
        \arrow[two heads, from=1-1, to=3-1]
        \arrow["{x_1'}"', from=3-1, to=4-2]
        \arrow[two heads, from=1-3, to=3-3]
        \arrow["{x_2'}", from=3-3, to=4-2]
        \arrow[from=3-1, to=3-3]
        \arrow[from=2-2, to=4-2]
      \end{tikzcd}\]
    \end{itemize}
  \end{itemize}
\end{definition}

双対的に, 相対半順序集合の始細分化が定義される. 

\begin{definition}[始細分化]
  $\P$を相対半順序集合とする. 
  このとき, 相対半順序集合$\xi_i\P$を次のように定義し, $\P$の始細分化(initial subdivision)という. 
  \begin{itemize}
    \item $\xi_i\P$の対象は$\RelPos$におけるmono射$x : [n]_\min \to \P$ ($n \geq 0$)
    \item $\xi_i\P$の射は次の図式を可換にする射$[n_2]_\min \to [n_1]_\min$
    % https://q.uiver.app/#q=WzAsMyxbMCwwLCJbbl8yXV9cXG1pbiJdLFsyLDAsIltuXzFdX1xcbWluIl0sWzEsMSwiXFxQIl0sWzAsMV0sWzAsMiwieF8yIiwyXSxbMSwyLCJ4XzEiXV0=
    \[\begin{tikzcd}
      {[n_2]_\min} && {[n_1]_\min} \\
      & \P
      \arrow[from=1-1, to=1-3]
      \arrow["{x_2}"', from=1-1, to=2-2]
      \arrow["{x_1}", from=1-3, to=2-2]
    \end{tikzcd}\]
    \item $\xi_i\P$のweak equivalenceは上の図式が誘導する射$x_2(0) \to x_1(0)$が$\P$におけるweak equivalenceである上の可換図式における射
  \end{itemize}
\end{definition}

相対半順序集合の始細分化は自然な射影を持つ. 

\begin{definition}[始射影関手]
  $\P$を相対半順序集合, $\xi_i\P$を$\P$の始細分化とする. 
  このとき, 相対関手$\pi_i : \xi_i\P \to \P$を次のように定義し, $\pi_i$を始射影関手(initial projection functor)という.
  \begin{itemize}
    \item $\xi_i\P$の対象$x : [n]_\min \to \P$に対して, $\pi_t(x) := x(0)$
    \item $\xi_i\P$の射$[n_2]_\min \to [n_1]_\min$に対して, $\pi_t([n_2]_\min \to [n_1]_\min) := x_2(0) \to x_1(0)$ 
  \end{itemize}
\end{definition}

\begin{remark}
  $\xi_i\P$における射がweak equivalenceであることと, $\pi_i$の像が$\P$におけるweak equivalenceであることは同値である. 
\end{remark}

始細分化をとる操作は$\RelPos$上の関手を定める. (省略)

\begin{example}
  相対半順序集合$\P = [2]$に対して, $\P$の終細分化(左)と始細分化(右)はそれぞれ次のようになる. 
  ここで, $\xrightarrow{\sim}$はweak equivalenceを表す. 
  % https://q.uiver.app/#q=WzAsMTQsWzEsMCwiMSJdLFsxLDEsIjAsMSwyIl0sWzAsMiwiMCJdLFsyLDIsIjIiXSxbMSwyLCIwLDIiXSxbMiwxLCIxLDIiXSxbMCwxLCIwLDEiXSxbNCwyLCIwIl0sWzQsMSwiMCwxIl0sWzUsMSwiMCwxLDIiXSxbNiwxLCIxLDIiXSxbNiwyLCIyIl0sWzUsMiwiMCwyIl0sWzUsMCwiMSJdLFswLDFdLFsyLDFdLFszLDEsIlxcc2ltIiwyXSxbMiw0XSxbMyw0LCJcXHNpbSJdLFswLDVdLFszLDUsIlxcc2ltIiwyXSxbMCw2LCJcXHNpbSIsMl0sWzIsNl0sWzQsMSwiXFxzaW0iLDJdLFs2LDFdLFs1LDEsIlxcc2ltIiwyXSxbNyw4LCJcXHNpbSJdLFs4LDksIlxcc2ltIl0sWzEwLDldLFsxMSwxMF0sWzExLDldLFs3LDksIlxcc2ltIl0sWzcsMTIsIlxcc2ltIiwyXSxbMTEsMTJdLFsxMiw5LCJcXHNpbSJdLFsxMyw5XSxbMTMsOF0sWzEzLDEwLCJcXHNpbSJdXQ==
  \[\begin{tikzcd}
    & 1 &&&& 1 \\
    {0,1} & {0,1,2} & {1,2} && {0,1} & {0,1,2} & {1,2} \\
    0 & {0,2} & 2 && 0 & {0,2} & 2
    \arrow[from=1-2, to=2-2]
    \arrow[from=3-1, to=2-2]
    \arrow["\sim"', from=3-3, to=2-2]
    \arrow[from=3-1, to=3-2]
    \arrow["\sim", from=3-3, to=3-2]
    \arrow[from=1-2, to=2-3]
    \arrow["\sim"', from=3-3, to=2-3]
    \arrow["\sim"', from=1-2, to=2-1]
    \arrow[from=3-1, to=2-1]
    \arrow["\sim"', from=3-2, to=2-2]
    \arrow[from=2-1, to=2-2]
    \arrow["\sim"', from=2-3, to=2-2]
    \arrow["\sim", from=3-5, to=2-5]
    \arrow["\sim", from=2-5, to=2-6]
    \arrow[from=2-7, to=2-6]
    \arrow[from=3-7, to=2-7]
    \arrow[from=3-7, to=2-6]
    \arrow["\sim", from=3-5, to=2-6]
    \arrow["\sim"', from=3-5, to=3-6]
    \arrow[from=3-7, to=3-6]
    \arrow["\sim", from=3-6, to=2-6]
    \arrow[from=1-6, to=2-6]
    \arrow[from=1-6, to=2-5]
    \arrow["\sim", from=1-6, to=2-7]
  \end{tikzcd}\]
\end{example}

相対半順序集合の始細分化と終細分化を両方とった相対半順序集合を定義する. 

\begin{definition}[2重細分化]
  $\P$を相対半順序集合とする. 
  $\xi\P := \xi_t\xi_i\P$を$\P$の2重細分化(two-fold subdivision)という. 
\end{definition}

相対半順序集合の2重細分化は自然な射影を持つ. 

\begin{definition}[射影関手]
  $\P$を相対半順序集合, $\xi\P$を$\P$の2重細分化とする. 
  このとき, 相対関手$\pi : \xi\P \to \P$を
  \begin{align*}
    \pi := \pi_i \circ \pi_t : \xi_t\xi_i\P \to \xi_i\P \to \P
  \end{align*}
  で定義し, $\pi$を射影関手(projection functor)という. 
\end{definition}

\begin{lemma}
  終(始)細分化関手はdomainが有限相対半順序集合である射の間の強ホモトピーを保つ. 
\end{lemma}

\begin{proposition} \label{prop:all_maps_is_homotopy_equivalence_in_RelCat}
  任意の$m,n \geq 0$に対して, 次の可換図式における全ての射はホモトピー同値である. 
  % https://q.uiver.app/#q=WzAsNixbMCwwLCJcXHhpKFtuXV9cXG1pbiBcXHRpbWVzIFttXV9cXG1heCkiXSxbMSwwLCJcXHhpX2koW25dX1xcbWluIFxcdGltZXMgW21dX1xcbWF4KSJdLFsyLDAsIltuXV9cXG1pbiBcXHRpbWVzIFttXV9cXG1heCJdLFsyLDEsIltuXV9cXG1pbiJdLFsxLDEsIlxceGlfaVtuXV9cXG1pbiJdLFswLDEsIlxceGlbbl1fXFxtaW4iXSxbMCwxLCJcXHBpX3RcXHhpX2kiXSxbMSwyLCJcXHBpX2kiXSxbMiwzXSxbMSw0XSxbMCw1XSxbNSw0LCJcXHBpX3RcXHhpX2kiLDJdLFs0LDMsIlxccGlfaSIsMl1d
  \[\begin{tikzcd}
    {\xi([n]_\min \times [m]_\max)} & {\xi_i([n]_\min \times [m]_\max)} & {[n]_\min \times [m]_\max} \\
    {\xi[n]_\min} & {\xi_i[n]_\min} & {[n]_\min}
    \arrow["{\pi_t\xi_i}", from=1-1, to=1-2]
    \arrow["{\pi_i}", from=1-2, to=1-3]
    \arrow[from=1-3, to=2-3]
    \arrow[from=1-2, to=2-2]
    \arrow[from=1-1, to=2-1]
    \arrow["{\pi_t\xi_i}"', from=2-1, to=2-2]
    \arrow["{\pi_i}"', from=2-2, to=2-3]
  \end{tikzcd}\]
  ここで, 垂直な射は射影$[n]_\min \times [m]_\max \to [n]_\min$から定まる射である. 
\end{proposition}

\begin{proposition} \label{prop:sieve_induce_dwyer_map}
  $\Q$を相対半順序集合, $\P$を$\Q$におけるふるい(または余ふるい), $F : \P \hookrightarrow \Q$を相対半順序集合の相対包含とする. 
  このとき, 誘導される$\xi_t : \xi_t\P \hookrightarrow \xi_t\Q$はDwyer射である. 
\end{proposition}

\newpage


\section{位相空間の圏に入るモデル構造} \label{sec:model_stru_in_Top}

位相空間の圏にモデル構造を入れるとき, weak equivalenceとして弱ホモトピー同値(weak homotopy equivalence)とホモトピー同値(homotopy equivalence)の2つが考えられる.
実際, 弱ホモトピー同値をweak equivalenceとするモデル構造としてQuillenモデル構造が, ホモトピー同値をweak equivalenceとするモデル構造としてStr{\o}mモデル構造がある. 
Quillenモデル構造は\cite{Qui67}で, Str{\o}mモデル構造は\cite{Str72}でそれぞれ証明された. 

\subsection{Quillenモデル構造} \label{subsec:quillen_model_stru_on_Top}

\begin{definition}[Quillenモデル構造]
  $\Top$には次のモデル構造が存在する. 
  これを$\Top$上のQuillenモデル構造
  \footnote{
    古典的(classical)モデル構造やQuillen-Serreモデル構造, qモデル構造と呼ばれることもある. 
  }
  といい, $\TopQuillen$と表す. 
  \begin{itemize}
    \item weak equivalenceは位相空間の弱ホモトピー同値
    \item fibrationはSerreファイブレーション
    \item cofibrationはrelative cell complexのレトラクト
  \end{itemize}
\end{definition}

\begin{remark}
  $\TopQuillen$において, 任意の対象(位相空間)はファイブラントであり, 任意のCW複体のレトラクトはコファイブラントである. 
\end{remark}

\begin{remark}
  $\TopQuillen$は 
  \begin{align*}
    I &:= \{S^{n-1} \hookrightarrow D^n ~|~ n \geq 0\}, \\
    J &:= \{D^n \times \{0\} \hookrightarrow D^n \times I ~|~ n \geq 0\}
  \end{align*}
  をそれぞれgenerating cofibration, generating trivial cofibrationの集合とするコファイブラント生成なモデル圏である. 
\end{remark}

\subsection{Str{\o}mモデル構造}

$\Top$上のQuillenモデル構造におけるweak equivalenceは弱ホモトピー同値であるが, Str{\o}mモデル構造はホモトピー同値をweak equivalenceとするようなモデル構造である. 

\begin{definition}[Str{\o}mモデル構造]
  $\Top$には次のモデル構造が存在する. 
  これを$\Top$上のStr{\o}mモデル構造
  \footnote{
    Hurewiczモデル構造やhモデル構造と呼ばれることもある.
  }
  といい, $\TopStrom$と表す. 
  \begin{itemize}
    \item weak equivalenceは位相空間のホモトピー同値
    \item fibrationはHurewiczファイブレーション
    \item cofibrationは閉Hurewiczコファイブレーション
  \end{itemize}
\end{definition}

\begin{remark}
  $\TopStrom$において, 任意の対象(位相空間)はファイブラントかつコファイブラントである.
\end{remark}

\begin{remark}
  $\TopStrom$はコファイブラント生成なモデル圏ではない. 
\end{remark}

\subsection{$\TopQuillen$と$\TopStrom$のQuillen随伴}

恒等関手による$\TopQuillen$と$\TopStrom$のQuillen随伴が定まる. 

\begin{proposition} \label{prop:quillen_adj_TopQuillen_TopStrom}
  恒等関手$\Id : \TopQuillen \to \TopStrom$と恒等関手$\Id : \TopStrom \to \TopQuillen$は, $\TopQuillen$と$\TopStrom$のQuillen随伴を定める. 
  \begin{align*}
    \Id : \TopQuillen \rightleftarrows \TopStrom : \Id
  \end{align*}
\end{proposition}

\begin{proof}
  右随伴がweak equivalenceとfibrationを保つことを示す. 

  まず, 任意のホモトピー同値($\TopStrom$におけるweak equivalence)は弱ホモトピー同値($\TopQuillen$におけるweak equivalence)である. 

  次に, 任意のHurewiczファイブレーション($\TopStrom$におけるfibration)はSerreファイブレーション($\TopQuillen$におけるfibration)である.
\end{proof}

\begin{remark}
  \cref{prop:quillen_adj_TopQuillen_TopStrom}のQuillen随伴$\Id : \TopQuillen \rightleftarrows \TopStrom : \Id$はQuillen同値ではない. 
\end{remark}

\begin{proof}
  \cref{prop:quillen_adj_TopQuillen_TopStrom}のQuillen随伴がQuillen同値であると仮定する.

  このとき, 任意のCW複体のレトラクト($\TopQuillen$におけるコファイブラント) $X$と位相空間($\TopStrom$におけるファイブラント) $Y$に対して, $X \to Y$が弱ホモトピー同値($\TopQuillen$におけるweak equivalence)であることと, ホモトピー同値($\TopStrom$におけるweak equivalence)であることは同値である. 
  しかし, CW複体とホモトピー同値であるが弱ホモトピー同値ではない位相空間は存在するので矛盾する.
\end{proof}

\subsection{$\sSetKan$と$\TopQuillen$のQuillen同値} \label{sec:quillen_equiv_sSetKan_and_TopQuillen}

モデル圏$\TopQuillen$のホモトピー圏はCW複体上の古典的なホモトピー圏と一致するので, $\TopQuillen$はCW複体のホモトピー論を表していると思える. 
$\TopQuillen$と$\sSetKan$の間の特異単体と幾何学的実現はQuillen同値を定める. (\cref{prop:quillen_equiv_sSetKan_and_TopQuillen})
これはKan複体のホモトピー仮説(homotopy hypothesis)の主張(の一部)である. 
これは, $(\infty,1)$圏論において$\TopQuillen$と$\sSetKan$が$(\infty,0)$圏のなす$(\infty,1)$圏のモデルとみなせることを意味している. 

\cref{sec:quillen_equiv_sSetKan_and_TopQuillen}の目標は次の\cref{prop:quillen_equiv_sSetKan_and_TopQuillen}を証明することである. 

\begin{proposition} \label{prop:quillen_equiv_sSetKan_and_TopQuillen}
  幾何学的実現$|-| : \sSet \to \Top$と特異単体$\Sing : \Top \to \sSet$は, $\sSetKan$と$\TopQuillen$のQuillen同値 
  \begin{align*}
    |-| : \sSetKan \rightleftarrows \TopQuillen : \Sing
  \end{align*}
  を定める. 
\end{proposition}

証明のために, いくつか準備をする. 
$\sSetKan$はコファイブラント生成なモデル圏なので, generating (trivial) cofibrationについて考えればよいが, より広いクラスに対して成立する命題についてはそれを証明する.  

まず, 右随伴がfibrationを保つことを示す. 

\begin{lemma}[\href{https://kerodon.net/tag/021V}{Tag 021V kerodon}] \label{prop:Serre_fib_is_Kan_fib}
  特異単体はSerreファイブレーションをKanファイブレーションにうつす. 
\end{lemma}

\begin{proof}
  $f : X \to Y$をSerreファイブレーションとする. 
  このとき, 単体的集合の射$\Sing(f) : \Sing(X) \to \Sing(Y)$がKanファイブレーションであることを示す.
  $\Sing(f) : \Sing(X) \to \Sing(Y)$が任意の$n > 0$と$0 \geq i \geq n$において, $\Lambda[n,i] \hookrightarrow \Delta[n]$に対してRLPを持つことを示せばよい. 
  随伴性より, $f : X \to Y$が$i : |\Lambda[n,i]| \hookrightarrow |\Delta[n]|$に対してRLPを持つことを示せばよい.
  連続写像$c : |\Delta[n]| \to [0,1]$を
  \begin{align*}
    c(t_0,\cdots,t_n) := \min\{t_0,\cdots,t_{i-1},t_{i+1},\cdots,t_n\}
  \end{align*}
  で定める. 
  次に, $h : [0,1] \times |\Delta[n]| \to |\Delta[n]|$を 
  \begin{align*}
    h(s, (t_0,\cdots,t_n)) &:= (t_0-\lambda, \cdots, t_{i-1}-\lambda,t_i+n\lambda,t_{i+1}-\lambda,\cdots,t_n-\lambda) \\
    \lambda &:= \max\{0,c(t_0,\cdots,t_n)-s\}
  \end{align*}
  で定める. 
  構成より, 合成
  \begin{align*}
    |\Delta^n| \xrightarrow{(c,\id)}  [0,1] \times |\Delta[n]| \xrightarrow{h} |\Delta[n]|
  \end{align*}
  は恒等射$\id : |\Delta^n| \to |\Delta^n|$に一致する. 
  また, $|\Lambda[n,i]| \subset |\Delta[n]|$に対して, (途中)
\end{proof}

より強く, 次のことが言える. 

\begin{lemma} 
  位相空間の連続写像$f : X \to Y$がSerreファイブレーションであることと, 単体的集合の射$\Sing(f) : \Sing(X) \to \Sing(Y)$がKanファイブレーションであることは同値である.
\end{lemma}

\begin{proof}
  \cref{prop:Serre_fib_is_Kan_fib}の逆を示す. 
  $\Sing(f) : \Sing(X) \to \Sing(Y)$をKanファイブレーションとする. 
  任意の$n \geq 0$に対して, $\Sing(f)$は緩射
  \begin{align*}
    \{0\} \times \Delta[n] \hookrightarrow \Delta[1] \times \Delta[n]
  \end{align*}
  に対してRLPを持つ. 
  よって, 位相空間の連続写像$f : X \to Y$は
  \begin{align*}
    |\{0\} \times \Delta[n]| \hookrightarrow |\Delta[1] \times \Delta[n]|
  \end{align*}
  に対してRLPを持つ. 
  幾何学的実現は有限直積と交換し, $|\partial \Delta[n]| \cong S^{n-1}$かつ$|\Delta[n]| \cong D^n$である. 
  よって, この射は$\Top$における射
  \begin{align*}
    \{0\} \times D^n \hookrightarrow [0,1] \times D^n
  \end{align*}
  と同一視できる. 
  よって, $f$はSerreファイブレーションである.  
\end{proof}

右随伴がgenerating cofibrationを保つことを示す. 

\begin{lemma} \label{prop:geometric_realization_of_simplicial_set_is_CW_complex}
  幾何学的実現は$\sSetKan$におけるgenerating cofibrationを$\TopQuillen$におけるgenerating cofibrationにうつす. 
\end{lemma}

\begin{proof}
  $i : \partial \Delta[n] \hookrightarrow \Delta[n]$を$\sSetKan$におけるgenerating cofibrationとする. 
  $i$の幾何学的実現をとる. 
  $|\partial \Delta[n]| \cong S^{n-1}$かつ$|\Delta[n]| \cong D^n$である. 
  このとき, $|i| : S^{n-1} \hookrightarrow D^n$は$\TopQuillen$におけるgenerating cofibrationである.
\end{proof}

最後に, Quillen同値を示すために必要な命題を示す. 

\begin{lemma} \label{prop:X_to_Sing|X|}
  $X$を単体的集合とする. 
  このとき, 随伴$(|-| \dashv \Sing)$の単位射
  \begin{align*}
    \eta_X : X \to \Sing(|X|)
  \end{align*}
  は単体的集合の弱ホモトピー同値である. 
\end{lemma}

\begin{corollary} \label{prop:|SingX|_to_X}
  $X$を位相空間とする. 
  このとき, 随伴$(|-| \dashv \Sing)$の余単位射
  \begin{align*}
    \mu_X : |\Sing(X)| \to X 
  \end{align*}
  は位相空間の弱ホモトピー同値である. 
\end{corollary}

最後に, \cref{prop:quillen_equiv_sSetKan_and_TopQuillen}を示す. % Daniel-Robert Prop2.34

\begin{proof}[\cref{prop:quillen_equiv_sSetKan_and_TopQuillen}の証明]
  %  まず, $|-| : \sSetKan \rightleftarrows \TopQuillen : \Sing$がQuillen随伴であることを示す. 
  %  つまり, $\Sing$がSerreファイブレーションをKanファイブレーションにうつし, $|-|$が単体的集合のmono射をrelative cell complexのレトラクトにうつすことを見ればよい. 
  %  前者は\cref{prop:Serre_fib_is_Kan_fib}, 後者は\cref{prop:geometric_realization_of_simplicial_set_is_CW_complex}から従う. 
  Quillen随伴であることは, \cref{prop:Serre_fib_is_Kan_fib}と\cref{prop:geometric_realization_of_simplicial_set_is_CW_complex}から従う.
  Quillen同値であることは, \cref{prop:X_to_Sing|X|}と\cref{prop:|SingX|_to_X}から従う. 
\end{proof}

\newpage


\section{小圏の圏に入るモデル構造} \label{sec:model_stru_in_Cat}

小圏の圏にモデル構造を入れることにより, 圏論(の一部)をホモトピー論の枠組みで考えられるようになる. 
まず, 圏同値(categorical equivalence)をweak equivalenceとするようなモデル構造が存在する. 
他にも, 圏の脈体の間の射がKan weak equivalenceであるようなThomasonモデル構造が存在する. 
前者の存在は\cite{Rezk96}で, 後者の存在は\cite{Thom80}でそれぞれ証明された. 

\subsection{Naturalモデル構造}

小圏の圏にモデル構造を入れるとき, まずweak equivalenceとして圏同値(categorical equivalence)が考えられる. 
選択公理を仮定すると, weak equivalenceを圏同値とするような$\Cat$上のモデル構造は一意である. 
(Chris Schommer-Priesの\href{https://sbseminar.wordpress.com/2012/11/16/the-canonical-model-structure-on-cat/}{The canonical model structure on Cat}を参照)

\begin{definition}[擬ファイブレーション]
  $\C,\D$を圏, $F : \C \to \D$を関手とする. 
  任意の対象$c \in \Ob(\C)$とdomainが$F(c)$の同型射$g \in \D$に対して, domainが$c$のある同型射$f \in \C$が存在して, $F(f)=g$を満たすとき, $F$を擬ファイブレーション(quasi-fibration)という. 
\end{definition}

擬ファイブレーションはリフトを用いて特徴づけることができる. 

\begin{remark} \label{rem:quasi_fibration_has_RLP}
  $\C,\D$を圏, $F : \C \to \D$を関手とする. 
  このとき, $F$が擬ファイブレーションであることと, 次の四角がリフトを持つことは同値である. 
  % https://q.uiver.app/#q=WzAsNCxbMCwwLCIwIl0sWzEsMCwiXFxDIl0sWzEsMSwiXFxEIl0sWzAsMSwiSSJdLFsxLDIsIkYiXSxbMCwzXSxbMywyXSxbMCwxXSxbMywxLCIiLDEseyJzdHlsZSI6eyJib2R5Ijp7Im5hbWUiOiJkYXNoZWQifX19XV0=
  \[\begin{tikzcd}
    {\{0\}} & \C \\
    I & \D
    \arrow["F", from=1-2, to=2-2]
    \arrow[from=1-1, to=2-1]
    \arrow[from=2-1, to=2-2]
    \arrow[from=1-1, to=1-2]
    \arrow[dashed, from=2-1, to=1-2]
  \end{tikzcd}\]
  ここで, $\{0\}$は1点圏, $I$は2点対象とその間の一意な同型射からなる圏とする. 
\end{remark}

\begin{definition}[対象上のmono関手]
  $\C,\D$を圏, $F : \C \to \D$を関手とする. 
  $\Ob(F) : \Ob(\C) \to \Ob(\D)$がmono射のとき, $F$を対象上のmono関手(monic on objects)という. 
\end{definition}

\begin{definition}[naturalモデル構造]
  $\Cat$には次のモデル構造が存在する. 
  これを$\Cat$上のnaturalモデル構造
  \footnote{
    自明な(trivial)モデル構造や圏的(categorical)モデル構造と呼ばれることもある.
    このとき, $\Catnat$におけるfibrationをisofibraion, cofibrationをisocofibrationということもある. 
  }
  といい, $\Catnat$と表す. 
  \begin{itemize}
    \item weak equivalenceは通常の圏同値
    \item fibrationは擬ファイブレーション
    \item cofibrationは対象上のmono関手
  \end{itemize}
\end{definition}

\begin{remark}
  $\Catnat$において, 任意の対象(小圏)はファイブラントかつコファイブラントである. 
\end{remark}

\begin{remark}
  $\Catnat$は
  \begin{align*}
    I &:= \{\emptyset \to \{0\}, \{0\} \sqcup \{1\} \to \{0 \to 1\}, \{0 \rightrightarrows 1\} \to \{0 \to 1\}\} \\
    J &:= \{\{0\} \to \{0 \leftrightarrow 1\}\}
  \end{align*}
  をそれぞれgenerating cofibration, generating trivial cofibrationの集合とするコファイブラント生成なモデル圏である. 
\end{remark}

\begin{proof}
  まず, $I$に関して考える. 
  $u : \emptyset \to \{0\}, v : \{0\} \sqcup \{1\} \to \{0 \to 1\}, w : \{0 \rightrightarrows 1\} \to \{0 \to 1\}$とする. 
  まず, $u,v,w$がcofibrationであることは明らかである. 
  よって, trivial fibrationは$u,v,w$に対してRLPを持つ. 

  逆に, 関手$F : \C \to \D$が$u,v,w$に対してRLPを持つとする. 
  $u$に対してRLPを持つとき, $F$は対象上のepi射である. 
  $v$に対してRLPを持つとき, $G$は充満である. 
  $w$に対してRLPを持つとき, $G$は忠実である.
  \cref{rem:trifib_is_in_Cat}より, $F$はtrivial fibrationである. 
  
  $J$に関しては\cref{rem:quasi_fibration_has_RLP}より従う. 
\end{proof}

$\Catnat$におけるtrivial fibrationとtrivial cofibrationは簡単に表すことができる.

\begin{remark} \label{rem:trifib_is_in_Cat}
  $\C,\D$を圏, $F : \C \to \D$を関手とする. 
  このとき, $F$がtrivial fibrationであることと, $F$が圏同値かつ対象上のepi関手
  \footnote{
    対象上のepi関手は対象上のmono関手と同様に定義される. 
    $\C,\D$を圏, $F : \C \to \D$を関手とする. 
    $\Ob(\C) \to \Ob(\D)$がepi射のとき, $F$を対象上のepi関手(epic on objects)という. 
  }
  であることは同値である.
\end{remark}

\begin{remark} \label{rem:tricof_is_in_Cat}
  $\C,\D$を圏, $F : \C \to \D$を関手とする. 
  このとき, $F$がtrivial cofibrationであることと, $\C$が$\D$と圏同値であるような$\D$の部分圏であることは同値である. 
\end{remark}

$\Cat$上にnaturalモデル構造が存在することを示すために, いくつか準備をする.

まず, $\Catnat$がリフト性質を満たすことを示す. 

\begin{lemma} \label{prop:Cat_has_lift}
  次の図式において, $F$を対象上のmono関手, $G$を擬ファイブレーションとする. 
  % https://q.uiver.app/#q=WzAsNCxbMCwwLCJcXEMiXSxbMCwxLCJcXEQiXSxbMSwwLCJcXEUiXSxbMSwxLCJcXEYiXSxbMCwxLCJGIiwyXSxbMCwyLCJVIl0sWzIsMywiRyJdLFsxLDMsIlYiLDJdLFsxLDIsIkgiLDIseyJzdHlsZSI6eyJib2R5Ijp7Im5hbWUiOiJkYXNoZWQifX19XV0=
  \[\begin{tikzcd}
    \C & \E \\
    \D & \F
    \arrow["F"', from=1-1, to=2-1]
    \arrow["U", from=1-1, to=1-2]
    \arrow["G", from=1-2, to=2-2]
    \arrow["V"', from=2-1, to=2-2]
    \arrow["H"', dashed, from=2-1, to=1-2]
  \end{tikzcd}\]
  更に, $F$か$G$が圏同値のとき, この四角はリフト$H$を持つ. 
\end{lemma}

\begin{proof}
  まず, $G$が圏同値のときを考える. 
  このとき, $\Ob(F) : \Ob(\C) \to \Ob(\D)$はmono射, $\Ob(G) : \Ob(\E) \to \Ob(\F)$はepi射である. 
  $\D$の対象$d$が$F(\C)$に属するとき, $d=F(c)$を満たす一意な$\C$の対象$c$を用いて, $H(d) := U(c)$とする. 
  $d$が$F(\C)$に属さないとき, $V(d)=G(e)$を満たす$e \in \Ob(\E)$を用いて, $H(d):=e$とする.
  \footnote{
    ここで選択公理を用いている.
    実は, $F$が圏同値のときも同様の議論で示すことができる. 
    nlabの\href{https://ncatlab.org/nlab/show/canonical+model+structure+on+Cat}{Canonical model structure on Cat}のPropositin 1.2を参照. 
    このとき, $\Ob(F)$はepi射なので, 選択公理は用いない. 
    本文中の証明は\cite{Rezk96} Theorem 3.1を参考にした. 
  }
  $G$は圏同値かつ擬ファイブレーションなので, \cref{rem:trifib_is_in_Cat}より, 任意の$f : d \to d' \in \D$に対して, 
  \begin{align*}
    G : \Hom_\E(H(d),H(d')) \to  \Hom_\N(GH(d),GH(d')) = \Hom_\N(V(d),V(d'))
  \end{align*}
  は同型である. 
  よって, 射の対応$H : \C \to \D$はこの同型を用いて定める. 
  このとき, 求める四角の可換性はすぐに示すことができる. 

  次に, $F$が圏同値のときを考える. 
  \cref{rem:tricof_is_in_Cat}より, ある関手$F' : \D \to \C$が存在して, $F'F=\Id_\C$かつ$\alpha : FF' \to \Id_\D$は自然同型である. 
  更に, $\alpha$を$F$の像に制限すると, $\alpha|_{F(\C)} = \Id_{F(\C)}$である. 
  まず, 対象の対応$\Ob(H) : \Ob(\C) \to \Ob(\D)$を次のように定義する.
  $G$は擬ファイブレーションなので, $UF'(d) \in \Ob(\E)$とdomainが$GUF'(d) = VFF'(d)$である同型射$V(\alpha_d) : VFF'(d) \to V(d)$に対して, ある同型射$\beta_d : UF'(d) \to x$が存在して, $G(\beta_d) = V(\alpha_d)$かつ$G(x) = V(d)$となる. 
  よって, 任意の$d \in \Ob(D)$に対して, $\Ob(H)(d) := x$とする. 
  % https://q.uiver.app/#q=WzAsNCxbMCwwLCJVRicoZCkiXSxbMCwxLCJcXE9iKEgpKGQpIDo9eCJdLFsyLDAsIlZGRicoZCk9R1VGJyhkKSJdLFsyLDEsIlYoZCkiXSxbMCwxLCJcXGJldGFfZCIsMl0sWzIsMywiVihcXGFscGhhX2QpIl0sWzQsNSwiRyIsMCx7InNob3J0ZW4iOnsic291cmNlIjoyMCwidGFyZ2V0IjoyMH0sImxldmVsIjoxLCJzdHlsZSI6eyJ0YWlsIjp7Im5hbWUiOiJtYXBzIHRvIn19fV1d
  \[\begin{tikzcd}
    {UF'(d)} && {VFF'(d)=GUF'(d)} \\
    {\Ob(H)(d) :=x} && {V(d)}
    \arrow[""{name=0, anchor=center, inner sep=0}, "{\beta_d}"', from=1-1, to=2-1]
    \arrow[""{name=1, anchor=center, inner sep=0}, "{V(\alpha_d)}", from=1-3, to=2-3]
    \arrow["G", shorten <=22pt, shorten >=22pt, maps to, from=0, to=1]
  \end{tikzcd}\]
  次に, 射の対応$H : \C \to \D$を次のように定める. 
  任意の$\D$の射$f : d \to d'$に対して, 
  \begin{align*}
    H(f) := \beta_{d'} \cdot UF'(f) \cdot \beta_d^{-1} : H(d) \xrightarrow{\beta_d^{-1}} UF'(d) \xrightarrow{UF'(f)} UF'(d') \xrightarrow{\beta_{d'}} H(d')
  \end{align*}
  とする. 
  また, $d \in \Ob(\D)$が$F(\C)$に属する, つまりある対象$c \in \Ob(\C)$が一意に存在して$d=F(c)$と表せるときを考える. 
  $\alpha|_{F(c)} = \Id_{F(c)}$なので, $HF(c) = U(c)$かつ$\beta_{F(c)} = \id_{U(c)}$である. 
  これらのことから, 求める四角の可換性はすぐに示すことができる. 
\end{proof}

$\Catnat$が分解系を持つことを示す. 

\begin{lemma} \label{prop:Cat_has_fs}
  任意の関手$F : \C \to \D$は圏同値かつ対象上のmono射$U : \C \to \C'$と擬ファイブレーション$V : \C' \to \D$を用いて$F=VU$と分解できる. 
  また, 任意の関手$F : \C \to \D$は対象上のmono射$U : \C \to \D'$と圏同値かつ擬ファイブレーション$V : \D' \to \D$を用いて$F=VU$と分解できる. 
\end{lemma}

\begin{proof}
  まず, 任意の関手$F : \C \to \D$が圏同値かつ対象上のmono射$U : \C \to \C'$と擬ファイブレーション$V : \C' \to \D$を用いて$F=VU$と分解できることを示す. 
  圏$\C'$を次のように定義する. 
  まず, $\C'$の対象は 
  \begin{align*}
    \Ob(\C') := \{(c,d,\alpha) ~|~ c \in \Ob(\C), d \in \Ob(\D), \alpha : F(c) \cong d \in \D\}
  \end{align*}
  $\C'$の任意の対象$(c,d,\alpha), (c',d',\alpha')$に対して, 
  \begin{align*}
    \Hom_{\C'}((c,d,\alpha), (c',d',\alpha')) 
    := \Hom_\C(c,c')
  \end{align*}
  このとき, 関手$U : \C \to \C'$を任意の$c \in \Ob(\C)$と$f : c \to c' \in \C$に対して
  \begin{align*}
    U(c) := (c,F(c),\id_{F(c)}) ,~ U(f) := f
  \end{align*}
  とする. 
  関手$V : \C' \to \C$を任意の$(c,d,\alpha) \in \Ob(\C')$と$f : (c,d,\alpha) \to (c',d',\alpha') \in \C'$に対して
  \begin{align*}
    V((c,d,\alpha)) := d ,~ V(f) := \alpha^{-1} \cdot F(f) \cdot \alpha'
  \end{align*}
  とする. 
  このとき, $U$は圏同値かつ対象上のmono射, $V$は擬ファイブレーションである.

  次に, 任意の関手$F : \C \to \D$が対象上のmono射$U : \C \to \D'$と圏同値かつ擬ファイブレーション$V : \D' \to \D$を用いて$F=VU$と分解できることを示す. 
  圏$\D'$を$D' = \C \sqcup \D$で定義する. 
  このとき, 関手$U : \C \to \D'$を任意の$c \in \Ob(\C)$と$f : c \to c' \in \C$に対して
  \begin{align*}
    U(c) := c, ~ U(f) := F(f)
  \end{align*}
  とする. 
  関手$V : \D' \to \D$を任意の$(c,d) \in \Ob(\C \sqcup \D)$に対して, 
  \begin{align*}
    V((c,d)) := (F(c),d)
  \end{align*}
  とする. 
  このとき, $U$は対象上のmono射, $V$は圏同値かつ擬ファイブレーションである. 
\end{proof}

$\Cat$上にnaturalモデル構造が存在することを示す.

\begin{proof}
  まず, $\Cat$は任意の(有限)極限と(有限)余極限を持つ. 

  次に, weak equivalenceが2-out-of-3を満たすことは明らかである. 

  また, weak equivalenceとcofibrationがretractで閉じることは簡単に示すことができる.
  fibrationがretractで閉じることは\cref{rem:quasi_fibration_has_RLP}より, リフトの一般論から示すことができる. 

  リフト性質を満たすことは\cref{prop:Cat_has_lift}で, 分解系を持つことは\cref{prop:Cat_has_fs}で既に示した. 
\end{proof}

\subsection{Thomasonモデル構造}

\subsection{$\sSetJoyal$と$\Catnat$のQuillen随伴} \label{sec:quillen_adj_sSetJoyal_Catnat}

\cref{sec:quillen_adj_sSetJoyal_Catnat}の目標は次の\cref{prop:quillen_adj_sSetJoyal_Catnat}を示すことである. 

\begin{proposition}[\cite{Joy08} Proposition 6.14] \label{prop:quillen_adj_sSetJoyal_Catnat}
  基本圏をとる関手$\tau_1 : \sSetJoyal \to \Catnat$と脈体$N : \Catnat \to \sSetJoyal$は, $\sSetJoyal$と$\Catnat$のQuillen随伴を定める.
  \begin{align*}
    \tau_1 : \sSetJoyal \rightleftarrows \Catnat : N
  \end{align*}
\end{proposition}

\begin{proof}
  まず, 左随伴がcofibrationを保つことを示す.
  任意の単体的集合$X$に対して, $\Ob(\tau_1(X))=X_0$である.
  また, 単体的集合の射$f : X \to Y$がmono射($\sSetJoyal$におけるcofibration)のとき, 特に$f_0 : X_0 \to Y_0$は対象上のmono射($\Catnat$におけるcofibration)である. 
  よって, 左随伴はcofibrationを保つ.

  次に, 左随伴がweak equivalenceを保つことを示す.
  $X$を単体的集合, $\C$を圏とする.
  \cite{Joy08} B.0.16より, 
  \begin{align*}
    \Fun(X,N(\C)) = \Fun(\tau_1(X),N(\C))
  \end{align*}
  である. 
  $\sSet^{\tau_0}$の定義より, 
  \begin{align*}
    \tau_0(X,N(\C))=\tau_0(\tau_1(X),N(\C))
  \end{align*}
  である. 
  従って, 任意の単体的集合の射$f : X \to Y$に対して, 
  \begin{align*}
    \tau_0(f,N(\C))=\tau_0(\tau_1(f),N(\C))
  \end{align*}
  である. 
  任意の圏$\C$に対して$N(\C)$は擬圏である. 
  よって, $f$が弱圏同値($\sSetJoyal$におけるweak equivalence)のとき, $\tau_0(f,N(\C))$は同型である. 
  つまり, $\tau_0(\tau_1(f),N(\C))$も同型である. 
  Yonedaの補題より, $\tau_1(f)$は$\Cat^{\tau_0}$における同型射である. 
  つまり, $\tau_1(f)$は圏同値($\Catnat$におけるweak equivalence)である. 
\end{proof}

\subsection{$\sSetKan$と$\CatThom$のQuillen同値}


\newpage
\bibliographystyle{alpha}
\bibliography{../model_reference}


\end{document}