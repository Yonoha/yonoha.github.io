\RequirePackage{plautopatch}
\documentclass[uplatex, a4paper, 14Q, dvipdfmx]{jsarticle}
\usepackage{docmute}
\usepackage{../new_mypackage}

\title{単体的集合の圏に入るモデル構造}
\author{よの}
\date{\today}

\begin{document}

\maketitle

\begin{abstract}
  単体的集合の圏に入るKan-Quillenモデル構造について説明する. 
\end{abstract}

\tableofcontents

\section{単体的集合の圏に入るモデル構造}

$I$を任意の$n \geq 0$に対して, 包含$\partial \Delta^n \hookrightarrow \Delta^n$のなす射の集まりとする. 
$J$を任意の$n>0$と$0 \leq r \leq n$に対して, 包含$\Lambda^n_i \hookrightarrow \Delta^n$のなす射の集まりとする. 

\begin{definition}
  幾何学実現がホモトピー同値となる$\sSet$の射を弱同値(weak equivalence)という.
  $\LLP(\RLP(I))$に属する$\sSet$の射をコファイブレーション(cofibration)という.
  $\RLP(J)$に属する$\sSet$の射をファイブレーション(fibration)という.
  $\LLP(\RLP(J))$に属する$\sSet$の射を緩射拡大(anodyne extension)という.
\end{definition}

$\sSet$におけるコファイブレーションは次のように簡単に表すことができる.

\begin{proposition}
  $f : K \to L$を単体的集合の射とする. 
  $f$がコファイブレーションであることと, $f$が単射であることは同値である. 
  また, 任意のコファイブレーションは相対$I$胞体単体である.
\end{proposition}

\begin{proof}
  
\end{proof}



\end{document}