\RequirePackage{plautopatch}
\documentclass[uplatex, a4paper, 14Q, dvipdfmx]{jsarticle}
\usepackage{docmute}
\usepackage{mypackage_for_segal}

\title{完備Segal空間について}
\author{よの}
\date{\today}

\begin{document}

\maketitle

\begin{abstract}
  新しい$(\infty,1)$圏のモデルとして, Rezk \cite{Rez00}は完備Segal空間を定義した.
  本稿は\cite{Rez00}のlecture noteである\cite{Ras18}をまとめたものである.
  単体的集合の基本的なことは前提知識として仮定する.  
\end{abstract}

\tableofcontents

\section{単体的空間} \label{sec:simplicial_space}

単体的集合には, 圏の脈体とKan複体という2つの特別なクラスがある.
高次圏論は, 圏論とホモトピー論を同時に一般化した枠組みで考えるものであった. 
Joyalはこれらの拡張条件に着目することで. 共通の一般化として擬圏を定義した.
一方, Rezkは単体的空間の枠組みで2つの単体的集合のクラスを1つにまとめることを考えた. 

\begin{definition}[単体的集合]
  関手$\Delta^\myop \to \Set$を単体的集合(simplicial set)という. 
  \footnote{
    \cite{Rez00}では, 単体的集合を空間(space)と呼んでいるが, 本稿ではこの表記を用いない. 
  }
  単体的集合のなす圏を$\sSet$と表す. 
\end{definition}

\begin{definition}[単体的空間]
  関手$\Delta^\myop \to \sSet$を単体的空間(simplicial space)という. 
  単体的空間のなす圏を$\sSpace$と表す. 
\end{definition}

\begin{remark}
  単体的空間$X : \Delta^\myop \to \sSet : [n] \mapsto X_n$を固定する. 
  $\Delta$における射$d^i : [n-1] \to [n]$と$s^i : [n+1] \to [n]$から定まる単体的集合の射をそれぞれ$d_i : X_n \to X_{n-1}$と$s_i : X_n \to X_{n+1}$と表す. 
\end{remark}

\begin{remark} \label{rem:bisimplicial_set}
  積-Hom随伴より, 圏同値$\sSpace \cong \Fun(\Delta^\myop \times \Delta^\myop, \Set)$が成立する. 
  このことから, 単体的空間は両単体的集合(bisimplicial set)とも呼ばれる. 
\end{remark}

\begin{remark}
  単体的空間$X$は次のように表せる. 
  ここで, $X_{n}$は単体的集合であり, $X_{n,k}$は集合である. % https://q.uiver.app/#q=WzAsMTcsWzEsMCwiWF97MH0iXSxbMiwwLCJYX3sxfSJdLFsxLDEsIlhfezAsMH0iXSxbMiwxLCJYX3sxLDB9Il0sWzEsMiwiWF97MCwxfSJdLFsyLDIsIlhfezEsMX0iXSxbMywwLCJYX3syfSJdLFszLDEsIlhfezIsMH0iXSxbMywyLCJYX3syLDF9Il0sWzQsMCwiXFxjZG90cyJdLFs0LDEsIlxcY2RvdHMiXSxbNCwyLCJcXGNkb3RzIl0sWzEsMywiXFx2ZG90cyJdLFsyLDMsIlxcdmRvdHMiXSxbMywzLCJcXHZkb3RzIl0sWzQsMywiXFxkZG90cyJdLFswLDAsIlgiXSxbMiwzXSxbMywyLCIiLDEseyJvZmZzZXQiOjF9XSxbMywyLCIiLDEseyJvZmZzZXQiOi0xfV0sWzQsMiwic18wIiwxXSxbNSwzXSxbNCw1XSxbMyw1LCIiLDEseyJvZmZzZXQiOjF9XSxbMyw1LCIiLDEseyJvZmZzZXQiOi0xfV0sWzYsMV0sWzEsNiwiIiwxLHsib2Zmc2V0IjotMX1dLFsxLDYsIiIsMSx7Im9mZnNldCI6MX1dLFs2LDEsIiIsMSx7Im9mZnNldCI6Mn1dLFs2LDEsIiIsMSx7Im9mZnNldCI6LTJ9XSxbNywzXSxbMyw3LCIiLDEseyJvZmZzZXQiOjF9XSxbMyw3LCIiLDEseyJvZmZzZXQiOi0xfV0sWzcsMywiIiwxLHsib2Zmc2V0IjoyfV0sWzcsMywiIiwxLHsib2Zmc2V0IjotMn1dLFs4LDddLFs3LDgsIiIsMSx7Im9mZnNldCI6MX1dLFs3LDgsIiIsMSx7Im9mZnNldCI6LTF9XSxbOCw1XSxbNSw4LCIiLDEseyJvZmZzZXQiOjF9XSxbNSw4LCIiLDEseyJvZmZzZXQiOi0xfV0sWzgsNSwiIiwxLHsib2Zmc2V0IjoyfV0sWzgsNSwiIiwxLHsib2Zmc2V0IjotMn1dLFs2LDldLFs5LDYsIiIsMSx7Im9mZnNldCI6MX1dLFs5LDYsIiIsMSx7Im9mZnNldCI6LTF9XSxbNiw5LCIiLDEseyJvZmZzZXQiOjJ9XSxbNiw5LCIiLDEseyJvZmZzZXQiOi0yfV0sWzksNiwiIiwxLHsib2Zmc2V0IjozfV0sWzksNiwiIiwxLHsib2Zmc2V0IjotM31dLFs3LDEwXSxbMTAsNywiIiwxLHsib2Zmc2V0IjotMX1dLFsxMCw3LCIiLDEseyJvZmZzZXQiOjF9XSxbNywxMCwiIiwxLHsib2Zmc2V0IjoyfV0sWzcsMTAsIiIsMSx7Im9mZnNldCI6LTJ9XSxbMTAsNywiIiwxLHsib2Zmc2V0IjozfV0sWzEwLDcsIiIsMSx7Im9mZnNldCI6LTN9XSxbOCwxMV0sWzExLDgsIiIsMSx7Im9mZnNldCI6MX1dLFsxMSw4LCIiLDEseyJvZmZzZXQiOi0xfV0sWzgsMTEsIiIsMSx7Im9mZnNldCI6Mn1dLFs4LDExLCIiLDEseyJvZmZzZXQiOi0yfV0sWzExLDgsIiIsMSx7Im9mZnNldCI6M31dLFsxMSw4LCIiLDEseyJvZmZzZXQiOi0zfV0sWzQsMTJdLFsxMiw0LCIiLDEseyJvZmZzZXQiOjF9XSxbMTIsNCwiIiwxLHsib2Zmc2V0IjotMX1dLFs0LDEyLCIiLDEseyJvZmZzZXQiOjJ9XSxbNCwxMiwiIiwxLHsib2Zmc2V0IjotMn1dLFs1LDEzXSxbMTMsNSwiIiwxLHsib2Zmc2V0IjoxfV0sWzEzLDUsIiIsMSx7Im9mZnNldCI6LTF9XSxbNSwxMywiIiwxLHsib2Zmc2V0IjoyfV0sWzUsNCwiIiwxLHsib2Zmc2V0IjoxfV0sWzUsMTMsIiIsMSx7Im9mZnNldCI6LTJ9XSxbOCwxNF0sWzE0LDgsIiIsMSx7Im9mZnNldCI6MX1dLFsxNCw4LCIiLDEseyJvZmZzZXQiOi0xfV0sWzgsMTQsIiIsMSx7Im9mZnNldCI6Mn1dLFs4LDE0LCIiLDEseyJvZmZzZXQiOi0yfV0sWzAsMiwiIiwwLHsibGV2ZWwiOjIsInN0eWxlIjp7ImhlYWQiOnsibmFtZSI6Im5vbmUifX19XSxbMSwzLCIiLDAseyJsZXZlbCI6Miwic3R5bGUiOnsiaGVhZCI6eyJuYW1lIjoibm9uZSJ9fX1dLFs2LDcsIiIsMCx7ImxldmVsIjoyLCJzdHlsZSI6eyJoZWFkIjp7Im5hbWUiOiJub25lIn19fV0sWzUsNCwiIiwxLHsib2Zmc2V0IjotMX1dLFswLDEsIihzJylfMCIsMV0sWzEsMCwiZF8wIiwyLHsib2Zmc2V0IjoxfV0sWzIsNCwiZF8xIiwwLHsib2Zmc2V0IjotMX1dLFsyLDQsImRfMCIsMix7Im9mZnNldCI6MX1dLFsxLDAsImRfMSIsMCx7Im9mZnNldCI6LTF9XSxbMTYsMCwiIiwwLHsibGV2ZWwiOjIsInN0eWxlIjp7ImhlYWQiOnsibmFtZSI6Im5vbmUifX19XV0=
  \[\begin{tikzcd}
    X & {X_{0}} & {X_{1}} & {X_{2}} & \cdots \\
    & {X_{0,0}} & {X_{1,0}} & {X_{2,0}} & \cdots \\
    & {X_{0,1}} & {X_{1,1}} & {X_{2,1}} & \cdots \\
    & \vdots & \vdots & \vdots & \ddots
    \arrow[from=2-2, to=2-3]
    \arrow[shift right, from=2-3, to=2-2]
    \arrow[shift left, from=2-3, to=2-2]
    \arrow["{s_0}"{description}, from=3-2, to=2-2]
    \arrow[from=3-3, to=2-3]
    \arrow[from=3-2, to=3-3]
    \arrow[shift right, from=2-3, to=3-3]
    \arrow[shift left, from=2-3, to=3-3]
    \arrow[from=1-4, to=1-3]
    \arrow[shift left, from=1-3, to=1-4]
    \arrow[shift right, from=1-3, to=1-4]
    \arrow[shift right=2, from=1-4, to=1-3]
    \arrow[shift left=2, from=1-4, to=1-3]
    \arrow[from=2-4, to=2-3]
    \arrow[shift right, from=2-3, to=2-4]
    \arrow[shift left, from=2-3, to=2-4]
    \arrow[shift right=2, from=2-4, to=2-3]
    \arrow[shift left=2, from=2-4, to=2-3]
    \arrow[from=3-4, to=2-4]
    \arrow[shift right, from=2-4, to=3-4]
    \arrow[shift left, from=2-4, to=3-4]
    \arrow[from=3-4, to=3-3]
    \arrow[shift right, from=3-3, to=3-4]
    \arrow[shift left, from=3-3, to=3-4]
    \arrow[shift right=2, from=3-4, to=3-3]
    \arrow[shift left=2, from=3-4, to=3-3]
    \arrow[from=1-4, to=1-5]
    \arrow[shift right, from=1-5, to=1-4]
    \arrow[shift left, from=1-5, to=1-4]
    \arrow[shift right=2, from=1-4, to=1-5]
    \arrow[shift left=2, from=1-4, to=1-5]
    \arrow[shift right=3, from=1-5, to=1-4]
    \arrow[shift left=3, from=1-5, to=1-4]
    \arrow[from=2-4, to=2-5]
    \arrow[shift left, from=2-5, to=2-4]
    \arrow[shift right, from=2-5, to=2-4]
    \arrow[shift right=2, from=2-4, to=2-5]
    \arrow[shift left=2, from=2-4, to=2-5]
    \arrow[shift right=3, from=2-5, to=2-4]
    \arrow[shift left=3, from=2-5, to=2-4]
    \arrow[from=3-4, to=3-5]
    \arrow[shift right, from=3-5, to=3-4]
    \arrow[shift left, from=3-5, to=3-4]
    \arrow[shift right=2, from=3-4, to=3-5]
    \arrow[shift left=2, from=3-4, to=3-5]
    \arrow[shift right=3, from=3-5, to=3-4]
    \arrow[shift left=3, from=3-5, to=3-4]
    \arrow[from=3-2, to=4-2]
    \arrow[shift right, from=4-2, to=3-2]
    \arrow[shift left, from=4-2, to=3-2]
    \arrow[shift right=2, from=3-2, to=4-2]
    \arrow[shift left=2, from=3-2, to=4-2]
    \arrow[from=3-3, to=4-3]
    \arrow[shift right, from=4-3, to=3-3]
    \arrow[shift left, from=4-3, to=3-3]
    \arrow[shift right=2, from=3-3, to=4-3]
    \arrow[shift right, from=3-3, to=3-2]
    \arrow[shift left=2, from=3-3, to=4-3]
    \arrow[from=3-4, to=4-4]
    \arrow[shift right, from=4-4, to=3-4]
    \arrow[shift left, from=4-4, to=3-4]
    \arrow[shift right=2, from=3-4, to=4-4]
    \arrow[shift left=2, from=3-4, to=4-4]
    \arrow[Rightarrow, no head, from=1-2, to=2-2]
    \arrow[Rightarrow, no head, from=1-3, to=2-3]
    \arrow[Rightarrow, no head, from=1-4, to=2-4]
    \arrow[shift left, from=3-3, to=3-2]
    \arrow["{s_0}"{description}, from=1-2, to=1-3]
    \arrow["{d_0}"', shift right, from=1-3, to=1-2]
    \arrow["{d_1}", shift left, from=2-2, to=3-2]
    \arrow["{d_0}"', shift right, from=2-2, to=3-2]
    \arrow["{d_1}", shift left, from=1-3, to=1-2]
    \arrow[Rightarrow, no head, from=1-1, to=1-2]
  \end{tikzcd}\]
\end{remark}

\begin{definition}[離散単体的空間]
  $X$を単体的空間とする.
  任意の$n \geq 0$に対して$X_n=X_0$であるとき, $X$は離散的(discrete)であるという. 
\end{definition}

離散単体的空間によって, $\sSet$は$\sSpace$に埋め込むことができる. 

\begin{definition}[垂直埋め込み]
  第1成分への射影
  \begin{align*}
    i_F : \Delta \times \Delta \to \Delta : ([n],[m]) \mapsto [n]
  \end{align*}
  は埋め込み
  \begin{align*}
    i_F^\ast : \sSet \to \sSpace : i_F^\ast(X)_{k,l} :=X_k
  \end{align*}
  を定める. 
  この関手を垂直埋め込み(vertical embedding functor)という. 
\end{definition}

標準的単体$\Delta[n]$の一般化として, ($\Delta[n]$を離散単体的集合とみなした)単体的空間$F(n)$を定義する. 

\begin{definition}[$n$次単体的集合関手]
  集合$\Hom_{\Delta}([-],[n])$を離散単体的集合とみなすことで定まる単体的空間$F(n)$を$n$次単体的集合関手($n$-th simplicial set functor)という. 
  つまり, $F(n)$の$k$単体は次のように表せる. 
  \begin{align*}
    F(n)_k = \Delta^\myop \to \sSet : [n] \mapsto \Hom_{\Delta}([k],[n])
  \end{align*}
  垂直埋め込みを用いると, $F(n)=i_F^\ast(\Delta[n])$と表せる. 
\end{definition} 

\begin{remark} \label{rem:yoneda_for_sSpace}
  Yonedaの補題より, 単体的集合の同型$\Map_{\sSpace}(F(n),X) \cong X_n$が成立する. 
\end{remark}

同様に, $n$次単体的集合関手の境界が定義できる.

\begin{definition}[$n$次単体的集合関手の境界]
  任意の$i < j$に対して$d^jd^i = d^id^{j-1}$から定まる次のコイコライザ$\partial F[n]$を$n$次単体的集合関手の境界(boundary of $n$-th simplicial set functor)という. % https://q.uiver.app/#q=WzAsMyxbMCwwLCJcXGNvcHJvZF97MCBcXGxlcSBpIFxcbGVxIGogXFxsZXEgbn0gRihuLTIpIl0sWzEsMCwiXFxjb3Byb2RfezAgXFxsZXEgaSBcXGxlcSBufSBGKG4tMSkiXSxbMiwwLCJcXHBhcnRpYWxcXERlbHRhW25dIl0sWzAsMSwiIiwxLHsib2Zmc2V0IjotMX1dLFsxLDJdLFswLDEsIiIsMSx7Im9mZnNldCI6MX1dXQ==
  \[\begin{tikzcd}
    {\coprod_{0 \leq i \leq j \leq n} F(n-2)} & {\coprod_{0 \leq i \leq n} F(n-1)} & {\partial F[n]}
    \arrow[shift left, from=1-1, to=1-2]
    \arrow[from=1-2, to=1-3]
    \arrow[shift right, from=1-1, to=1-2]
  \end{tikzcd}\]
  垂直埋め込みを用いると, $\partial F(n)=i_F^\ast(\partial\Delta[n])$と表せる. 
\end{definition}

対角関手によって, 単体的空間から単体的集合が定まる. 

\begin{definition}[対角関手]
  $\mathrm{diag} :\sSpace \to \sSet$を単体的空間$X$に対して, $\mathrm{diag}X$の$n$単体を$\mathrm{diag}X_n := X_{n,n}$で定義し, 対角関手(diagonal functor)という. 
\end{definition}

% \begin{remark} \label{rem:diagram_of_simplicial_space}
%   \cref{rem:bisimplicial_set}より, 単体的空間は次のように表せる. 
%   ここで, $X_{i,j}$は集合であり, $X_{i,-}$や$X_{-,j}$は単体的集合である. 
%   % https://q.uiver.app/#q=WzAsMjQsWzAsMCwiWF97MCwwfSJdLFsxLDAsIlhfezEsMH0iXSxbMCwxLCJYX3swLDF9Il0sWzEsMSwiWF97MSwxfSJdLFswLDIsIlhfezAsMn0iXSxbMSwyLCJYX3sxLDJ9Il0sWzIsMCwiWF97MiwwfSJdLFsyLDEsIlhfezIsMX0iXSxbMiwyLCJYX3syLDJ9Il0sWzMsMCwiXFxjZG90cyJdLFs0LDAsIlhfey0sMH0iXSxbMywxLCJcXGNkb3RzIl0sWzQsMSwiWF97LSwxfSJdLFszLDIsIlxcY2RvdHMiXSxbNCwyLCJYX3stLDJ9Il0sWzAsMywiXFx2ZG90cyJdLFsxLDMsIlxcdmRvdHMiXSxbMiwzLCJcXHZkb3RzIl0sWzAsNCwiWF97MCwxfSJdLFsxLDQsIlhfezEsLX0iXSxbMiw0LCJYX3syLC19Il0sWzQsMywiXFx2ZG90cyJdLFszLDQsIlxcY2RvdHMiXSxbMywzLCJcXGRkb3RzIl0sWzAsMV0sWzEsMCwiIiwyLHsib2Zmc2V0IjoxfV0sWzEsMCwiIiwxLHsib2Zmc2V0IjotMX1dLFswLDJdLFsyLDNdLFsxLDNdLFsyLDAsIiIsMSx7Im9mZnNldCI6MX1dLFsyLDAsIiIsMSx7Im9mZnNldCI6LTF9XSxbMywxLCIiLDEseyJvZmZzZXQiOjF9XSxbMywxLCIiLDEseyJvZmZzZXQiOi0xfV0sWzMsMiwiIiwxLHsib2Zmc2V0IjoxfV0sWzMsMiwiIiwxLHsib2Zmc2V0IjotMX1dLFs0LDJdLFs0LDIsIiIsMSx7Im9mZnNldCI6LTJ9XSxbNSwzXSxbMiw0LCIiLDEseyJvZmZzZXQiOjF9XSxbMiw0LCIiLDEseyJvZmZzZXQiOi0xfV0sWzQsMiwiIiwxLHsib2Zmc2V0IjoyfV0sWzQsNV0sWzUsNCwiIiwxLHsib2Zmc2V0IjotMX1dLFszLDUsIiIsMSx7Im9mZnNldCI6MX1dLFszLDUsIiIsMSx7Im9mZnNldCI6LTF9XSxbNSwzLCIiLDEseyJvZmZzZXQiOjJ9XSxbNSwzLCIiLDEseyJvZmZzZXQiOi0yfV0sWzYsMV0sWzEsNiwiIiwxLHsib2Zmc2V0IjotMX1dLFsxLDYsIiIsMSx7Im9mZnNldCI6MX1dLFs2LDEsIiIsMSx7Im9mZnNldCI6Mn1dLFs2LDEsIiIsMSx7Im9mZnNldCI6LTJ9XSxbNywzXSxbMyw3LCIiLDEseyJvZmZzZXQiOjF9XSxbMyw3LCIiLDEseyJvZmZzZXQiOi0xfV0sWzcsMywiIiwxLHsib2Zmc2V0IjoyfV0sWzcsMywiIiwxLHsib2Zmc2V0IjotMn1dLFs2LDddLFs3LDYsIiIsMSx7Im9mZnNldCI6MX1dLFs3LDYsIiIsMSx7Im9mZnNldCI6LTF9XSxbOCw3XSxbNyw4LCIiLDEseyJvZmZzZXQiOjF9XSxbNyw4LCIiLDEseyJvZmZzZXQiOi0xfV0sWzgsNywiIiwxLHsib2Zmc2V0IjoyfV0sWzgsNywiIiwxLHsib2Zmc2V0IjotMn1dLFs4LDVdLFs1LDgsIiIsMSx7Im9mZnNldCI6MX1dLFs1LDgsIiIsMSx7Im9mZnNldCI6LTF9XSxbOCw1LCIiLDEseyJvZmZzZXQiOjJ9XSxbOCw1LCIiLDEseyJvZmZzZXQiOi0yfV0sWzE4LDE5XSxbMTksMTgsIiIsMSx7Im9mZnNldCI6MX1dLFsyMCwxOV0sWzE5LDIwLCIiLDEseyJvZmZzZXQiOjF9XSxbMTksMTgsIiIsMSx7Im9mZnNldCI6LTF9XSxbMTksMjAsIiIsMSx7Im9mZnNldCI6LTF9XSxbMjAsMTksIiIsMSx7Im9mZnNldCI6Mn1dLFsyMCwxOSwiIiwxLHsib2Zmc2V0IjotMn1dLFsxMCwxMl0sWzEyLDEwLCIiLDEseyJvZmZzZXQiOjF9XSxbMTIsMTAsIiIsMSx7Im9mZnNldCI6LTF9XSxbMTQsMTJdLFsxMiwxNCwiIiwxLHsib2Zmc2V0IjoxfV0sWzEyLDE0LCIiLDEseyJvZmZzZXQiOi0xfV0sWzE0LDEyLCIiLDEseyJvZmZzZXQiOjJ9XSxbMTQsMTIsIiIsMSx7Im9mZnNldCI6LTJ9XSxbNiw5XSxbOSw2LCIiLDEseyJvZmZzZXQiOjF9XSxbOSw2LCIiLDEseyJvZmZzZXQiOi0xfV0sWzYsOSwiIiwxLHsib2Zmc2V0IjoyfV0sWzYsOSwiIiwxLHsib2Zmc2V0IjotMn1dLFs5LDYsIiIsMSx7Im9mZnNldCI6M31dLFs5LDYsIiIsMSx7Im9mZnNldCI6LTN9XSxbNywxMV0sWzExLDcsIiIsMSx7Im9mZnNldCI6LTF9XSxbMTEsNywiIiwxLHsib2Zmc2V0IjoxfV0sWzcsMTEsIiIsMSx7Im9mZnNldCI6Mn1dLFs3LDExLCIiLDEseyJvZmZzZXQiOi0yfV0sWzExLDcsIiIsMSx7Im9mZnNldCI6M31dLFsxMSw3LCIiLDEseyJvZmZzZXQiOi0zfV0sWzgsMTNdLFsxMyw4LCIiLDEseyJvZmZzZXQiOjF9XSxbMTMsOCwiIiwxLHsib2Zmc2V0IjotMX1dLFs4LDEzLCIiLDEseyJvZmZzZXQiOjJ9XSxbOCwxMywiIiwxLHsib2Zmc2V0IjotMn1dLFsxMyw4LCIiLDEseyJvZmZzZXQiOjN9XSxbMTMsOCwiIiwxLHsib2Zmc2V0IjotM31dLFs5LDEwLCIiLDEseyJzdHlsZSI6eyJoZWFkIjp7Im5hbWUiOiJub25lIn19fV0sWzEwLDksIiIsMSx7Im9mZnNldCI6MSwic3R5bGUiOnsiaGVhZCI6eyJuYW1lIjoibm9uZSJ9fX1dLFsxMSwxMiwiIiwxLHsic3R5bGUiOnsiaGVhZCI6eyJuYW1lIjoibm9uZSJ9fX1dLFsxMiwxMSwiIiwxLHsib2Zmc2V0IjoxLCJzdHlsZSI6eyJoZWFkIjp7Im5hbWUiOiJub25lIn19fV0sWzEzLDE0LCIiLDEseyJzdHlsZSI6eyJoZWFkIjp7Im5hbWUiOiJub25lIn19fV0sWzE0LDEzLCIiLDEseyJvZmZzZXQiOjEsInN0eWxlIjp7ImhlYWQiOnsibmFtZSI6Im5vbmUifX19XSxbNCwxNV0sWzE1LDQsIiIsMSx7Im9mZnNldCI6MX1dLFsxNSw0LCIiLDEseyJvZmZzZXQiOi0xfV0sWzQsMTUsIiIsMSx7Im9mZnNldCI6Mn1dLFs0LDE1LCIiLDEseyJvZmZzZXQiOi0yfV0sWzE1LDQsIiIsMSx7Im9mZnNldCI6LTN9XSxbNSwxNl0sWzE2LDUsIiIsMSx7Im9mZnNldCI6MX1dLFsxNiw1LCIiLDEseyJvZmZzZXQiOi0xfV0sWzUsMTYsIiIsMSx7Im9mZnNldCI6Mn1dLFs1LDQsIiIsMSx7Im9mZnNldCI6MX1dLFs1LDE2LCIiLDEseyJvZmZzZXQiOi0yfV0sWzE2LDUsIiIsMSx7Im9mZnNldCI6M31dLFsxNiw1LCIiLDEseyJvZmZzZXQiOi0zfV0sWzE1LDQsIiIsMSx7Im9mZnNldCI6M31dLFsxNSwxOCwiIiwxLHsic3R5bGUiOnsiaGVhZCI6eyJuYW1lIjoibm9uZSJ9fX1dLFsxNSwxOCwiIiwxLHsib2Zmc2V0IjoxLCJzdHlsZSI6eyJoZWFkIjp7Im5hbWUiOiJub25lIn19fV0sWzE2LDE5LCIiLDEseyJzdHlsZSI6eyJoZWFkIjp7Im5hbWUiOiJub25lIn19fV0sWzE2LDE5LCIiLDEseyJvZmZzZXQiOjEsInN0eWxlIjp7ImhlYWQiOnsibmFtZSI6Im5vbmUifX19XSxbMTcsMjAsIiIsMSx7InN0eWxlIjp7ImhlYWQiOnsibmFtZSI6Im5vbmUifX19XSxbMTcsMjAsIiIsMSx7Im9mZnNldCI6MSwic3R5bGUiOnsiaGVhZCI6eyJuYW1lIjoibm9uZSJ9fX1dLFs4LDE3XSxbMTcsOCwiIiwxLHsib2Zmc2V0IjoxfV0sWzE3LDgsIiIsMSx7Im9mZnNldCI6LTF9XSxbOCwxNywiIiwxLHsib2Zmc2V0IjoyfV0sWzgsMTcsIiIsMSx7Im9mZnNldCI6LTJ9XSxbMTcsOCwiIiwxLHsib2Zmc2V0IjozfV0sWzE3LDgsIiIsMSx7Im9mZnNldCI6LTN9XSxbMjAsMjJdLFsyMywyMiwiIiwxLHsic3R5bGUiOnsiaGVhZCI6eyJuYW1lIjoibm9uZSJ9fX1dLFsyMywyMiwiIiwxLHsib2Zmc2V0IjoxLCJzdHlsZSI6eyJoZWFkIjp7Im5hbWUiOiJub25lIn19fV0sWzIzLDIxLCIiLDEseyJzdHlsZSI6eyJoZWFkIjp7Im5hbWUiOiJub25lIn19fV0sWzIzLDIxLCIiLDEseyJvZmZzZXQiOi0xLCJzdHlsZSI6eyJoZWFkIjp7Im5hbWUiOiJub25lIn19fV0sWzE0LDIxXSxbMjEsMTQsIiIsMSx7Im9mZnNldCI6MX1dLFsyMSwxNCwiIiwxLHsib2Zmc2V0IjotMX1dLFsxNCwyMSwiIiwxLHsib2Zmc2V0IjoyfV0sWzE0LDIxLCIiLDEseyJvZmZzZXQiOi0yfV0sWzIxLDE0LCIiLDEseyJvZmZzZXQiOjN9XSxbMjEsMTQsIiIsMSx7Im9mZnNldCI6LTN9XSxbMjIsMjAsIiIsMSx7Im9mZnNldCI6MX1dLFsyMiwyMCwiIiwxLHsib2Zmc2V0IjotMX1dLFsyMCwyMiwiIiwxLHsib2Zmc2V0IjoyfV0sWzIwLDIyLCIiLDEseyJvZmZzZXQiOi0yfV0sWzIyLDIwLCIiLDEseyJvZmZzZXQiOjN9XSxbMjIsMjAsIiIsMSx7Im9mZnNldCI6LTN9XV0=
%   \[\begin{tikzcd}
%     {X_{0,0}} & {X_{1,0}} & {X_{2,0}} & \cdots & {X_{-,0}} \\
%     {X_{0,1}} & {X_{1,1}} & {X_{2,1}} & \cdots & {X_{-,1}} \\
%     {X_{0,2}} & {X_{1,2}} & {X_{2,2}} & \cdots & {X_{-,2}} \\
%     \vdots & \vdots & \vdots & \ddots & \vdots \\
%     {X_{0,-}} & {X_{1,-}} & {X_{2,-}} & \cdots
%     \arrow[from=1-1, to=1-2]
%     \arrow[shift right, from=1-2, to=1-1]
%     \arrow[shift left, from=1-2, to=1-1]
%     \arrow[from=1-1, to=2-1]
%     \arrow[from=2-1, to=2-2]
%     \arrow[from=1-2, to=2-2]
%     \arrow[shift right, from=2-1, to=1-1]
%     \arrow[shift left, from=2-1, to=1-1]
%     \arrow[shift right, from=2-2, to=1-2]
%     \arrow[shift left, from=2-2, to=1-2]
%     \arrow[shift right, from=2-2, to=2-1]
%     \arrow[shift left, from=2-2, to=2-1]
%     \arrow[from=3-1, to=2-1]
%     \arrow[shift left=2, from=3-1, to=2-1]
%     \arrow[from=3-2, to=2-2]
%     \arrow[shift right, from=2-1, to=3-1]
%     \arrow[shift left, from=2-1, to=3-1]
%     \arrow[shift right=2, from=3-1, to=2-1]
%     \arrow[from=3-1, to=3-2]
%     \arrow[shift left, from=3-2, to=3-1]
%     \arrow[shift right, from=2-2, to=3-2]
%     \arrow[shift left, from=2-2, to=3-2]
%     \arrow[shift right=2, from=3-2, to=2-2]
%     \arrow[shift left=2, from=3-2, to=2-2]
%     \arrow[from=1-3, to=1-2]
%     \arrow[shift left, from=1-2, to=1-3]
%     \arrow[shift right, from=1-2, to=1-3]
%     \arrow[shift right=2, from=1-3, to=1-2]
%     \arrow[shift left=2, from=1-3, to=1-2]
%     \arrow[from=2-3, to=2-2]
%     \arrow[shift right, from=2-2, to=2-3]
%     \arrow[shift left, from=2-2, to=2-3]
%     \arrow[shift right=2, from=2-3, to=2-2]
%     \arrow[shift left=2, from=2-3, to=2-2]
%     \arrow[from=1-3, to=2-3]
%     \arrow[shift right, from=2-3, to=1-3]
%     \arrow[shift left, from=2-3, to=1-3]
%     \arrow[from=3-3, to=2-3]
%     \arrow[shift right, from=2-3, to=3-3]
%     \arrow[shift left, from=2-3, to=3-3]
%     \arrow[shift right=2, from=3-3, to=2-3]
%     \arrow[shift left=2, from=3-3, to=2-3]
%     \arrow[from=3-3, to=3-2]
%     \arrow[shift right, from=3-2, to=3-3]
%     \arrow[shift left, from=3-2, to=3-3]
%     \arrow[shift right=2, from=3-3, to=3-2]
%     \arrow[shift left=2, from=3-3, to=3-2]
%     \arrow[from=5-1, to=5-2]
%     \arrow[shift right, from=5-2, to=5-1]
%     \arrow[from=5-3, to=5-2]
%     \arrow[shift right, from=5-2, to=5-3]
%     \arrow[shift left, from=5-2, to=5-1]
%     \arrow[shift left, from=5-2, to=5-3]
%     \arrow[shift right=2, from=5-3, to=5-2]
%     \arrow[shift left=2, from=5-3, to=5-2]
%     \arrow[from=1-5, to=2-5]
%     \arrow[shift right, from=2-5, to=1-5]
%     \arrow[shift left, from=2-5, to=1-5]
%     \arrow[from=3-5, to=2-5]
%     \arrow[shift right, from=2-5, to=3-5]
%     \arrow[shift left, from=2-5, to=3-5]
%     \arrow[shift right=2, from=3-5, to=2-5]
%     \arrow[shift left=2, from=3-5, to=2-5]
%     \arrow[from=1-3, to=1-4]
%     \arrow[shift right, from=1-4, to=1-3]
%     \arrow[shift left, from=1-4, to=1-3]
%     \arrow[shift right=2, from=1-3, to=1-4]
%     \arrow[shift left=2, from=1-3, to=1-4]
%     \arrow[shift right=3, from=1-4, to=1-3]
%     \arrow[shift left=3, from=1-4, to=1-3]
%     \arrow[from=2-3, to=2-4]
%     \arrow[shift left, from=2-4, to=2-3]
%     \arrow[shift right, from=2-4, to=2-3]
%     \arrow[shift right=2, from=2-3, to=2-4]
%     \arrow[shift left=2, from=2-3, to=2-4]
%     \arrow[shift right=3, from=2-4, to=2-3]
%     \arrow[shift left=3, from=2-4, to=2-3]
%     \arrow[from=3-3, to=3-4]
%     \arrow[shift right, from=3-4, to=3-3]
%     \arrow[shift left, from=3-4, to=3-3]
%     \arrow[shift right=2, from=3-3, to=3-4]
%     \arrow[shift left=2, from=3-3, to=3-4]
%     \arrow[shift right=3, from=3-4, to=3-3]
%     \arrow[shift left=3, from=3-4, to=3-3]
%     \arrow[no head, from=1-4, to=1-5]
%     \arrow[shift right, no head, from=1-5, to=1-4]
%     \arrow[no head, from=2-4, to=2-5]
%     \arrow[shift right, no head, from=2-5, to=2-4]
%     \arrow[no head, from=3-4, to=3-5]
%     \arrow[shift right, no head, from=3-5, to=3-4]
%     \arrow[from=3-1, to=4-1]
%     \arrow[shift right, from=4-1, to=3-1]
%     \arrow[shift left, from=4-1, to=3-1]
%     \arrow[shift right=2, from=3-1, to=4-1]
%     \arrow[shift left=2, from=3-1, to=4-1]
%     \arrow[shift left=3, from=4-1, to=3-1]
%     \arrow[from=3-2, to=4-2]
%     \arrow[shift right, from=4-2, to=3-2]
%     \arrow[shift left, from=4-2, to=3-2]
%     \arrow[shift right=2, from=3-2, to=4-2]
%     \arrow[shift right, from=3-2, to=3-1]
%     \arrow[shift left=2, from=3-2, to=4-2]
%     \arrow[shift right=3, from=4-2, to=3-2]
%     \arrow[shift left=3, from=4-2, to=3-2]
%     \arrow[shift right=3, from=4-1, to=3-1]
%     \arrow[no head, from=4-1, to=5-1]
%     \arrow[shift right, no head, from=4-1, to=5-1]
%     \arrow[no head, from=4-2, to=5-2]
%     \arrow[shift right, no head, from=4-2, to=5-2]
%     \arrow[no head, from=4-3, to=5-3]
%     \arrow[shift right, no head, from=4-3, to=5-3]
%     \arrow[from=3-3, to=4-3]
%     \arrow[shift right, from=4-3, to=3-3]
%     \arrow[shift left, from=4-3, to=3-3]
%     \arrow[shift right=2, from=3-3, to=4-3]
%     \arrow[shift left=2, from=3-3, to=4-3]
%     \arrow[shift right=3, from=4-3, to=3-3]
%     \arrow[shift left=3, from=4-3, to=3-3]
%     \arrow[from=5-3, to=5-4]
%     \arrow[no head, from=4-4, to=5-4]
%     \arrow[shift right, no head, from=4-4, to=5-4]
%     \arrow[no head, from=4-4, to=4-5]
%     \arrow[shift left, no head, from=4-4, to=4-5]
%     \arrow[from=3-5, to=4-5]
%     \arrow[shift right, from=4-5, to=3-5]
%     \arrow[shift left, from=4-5, to=3-5]
%     \arrow[shift right=2, from=3-5, to=4-5]
%     \arrow[shift left=2, from=3-5, to=4-5]
%     \arrow[shift right=3, from=4-5, to=3-5]
%     \arrow[shift left=3, from=4-5, to=3-5]
%     \arrow[shift right, from=5-4, to=5-3]
%     \arrow[shift left, from=5-4, to=5-3]
%     \arrow[shift right=2, from=5-3, to=5-4]
%     \arrow[shift left=2, from=5-3, to=5-4]
%     \arrow[shift right=3, from=5-4, to=5-3]
%     \arrow[shift left=3, from=5-4, to=5-3]
%   \end{tikzcd}\]
% \end{remark}

% 単体的集合を単体的空間に2種類の方法で埋め込むことができる. 

% \begin{remark}
%   \begin{enumerate}
%     \item 第1成分への射影
%     \begin{align*}
%       i_F : \Delta \times \Delta \to \Delta : ([n],[m]) \mapsto [n]
%     \end{align*}
%     は埋め込み
%     \begin{align*}
%       i_F^\ast : \sSet \to \sSpace : i_F^\ast(X)_{k,l} :=X_k
%     \end{align*}
%     を定める. 
%     $i_F^\ast$を垂直埋め込み(vertical embedding)という. 
%     \item 第2成分への射影
%     \begin{align*}
%       i_\Delta : \Delta \times \Delta \to \Delta : ([n],[m]) \mapsto [m]
%     \end{align*}
%     は埋め込み
%     \begin{align*}
%       i_\Delta^\ast : \sSet \to \sSpace : i_F^\ast(X)_{k,l} := X_l
%     \end{align*}
%     を定める. 
%     $\i_\Delta^\ast$を水平埋め込み(horizontal embedding)という. 
%   \end{enumerate}
% \end{remark}

% 埋め込みを用いて, いくつかの代表的な単体的空間を定義する. 

% \begin{definition}[空間関手と標準的単体]
%   単体的空間$F(n)$を次のように定義し, $n$次空間関手($n$-th space functor)という. 
%   \begin{align*}
%     (F(n))_{k,l} := i_F^\ast(\Delta^n) = \Hom_\Delta([k],[n])
%   \end{align*}
%   単体的空間$\Delta^n$を次のように定義し, 標準的$n$単体(standard $n$-simplex)という.
%   \begin{align*}
%     (\Delta^n)_{k,l} := i_\Delta^\ast(\Delta^n) = \Hom_\Delta([l],[n])
%   \end{align*}
% \end{definition}

% \begin{definition}[空間関手の境界]
%   単体的空間$\partial F(n)$を次のように定義し, $n$次空間関手の境界(boundary of the $n$-th space functor)という. 
%   \begin{align*}
%     \partial F(n) := i_F^\ast(\partial \Delta^n)
%   \end{align*}
% \end{definition}

% $\sSet$と同様に, $\sSpace$はCartesian閉である. 

% \begin{definition}[べき乗]
%   単体的空間$X,Y \in \sSpace$に対して, 単体的空間$Y^X$を次のように定義する. 
%   \begin{align*}
%     (Y^X)_{n,l} := \Hom_\sSpace(F(n) \times \Delta^l \times X,Y)
%   \end{align*}
% \end{definition}

% \begin{lemma}
%   単体的空間の圏$\sSpace$はCartesian閉である. 
%   つまり, 次の同型が存在する. 
%   \begin{align*}
%     \Map_{\sSpace}(X \times Y,Z) \cong \Map_{\sSpace}(X,Z^Y)
%   \end{align*}
% \end{lemma}

\section{Reedyファイブラント空間} \label{sec:reedy_fibrant}

単体的空間の定義の動機づけにあるように, $(\infty,1)$圏のモデルは空間の性質を持つ必要がある.
$\sSet$上のKan-Quillenモデル構造において, Kan複体はファイブラント対象であった. 
Reedyモデル圏の一般論から, $\sSpace$上にReedyモデル構造が誘導される. 
このReedyモデル構造におけるファイブラント条件を用いて, 空間の性質を特徴づける. 
実際, Reedyファイブラント空間$X$に対して, $X_n$はKan複体となる (\cref{lem:Xn_is_Kan_fibrant}).

% \begin{definition}[Reedyファイブレーション]
%   $X \to Y$を単体的空間の射とする. 
%   任意の$n,l \geq 0$と$0 \leq i \leq n$に対して, 次の図式がリフトを持つとき, $f$をReedyファイブレーション(Reedy fibration)という. 
%   % https://q.uiver.app/#q=WzAsNCxbMCwwLCJcXHBhcnRpYWwgRihuKSBcXHRpbWVzIFxcRGVsdGFebCBcXGNvcHJvZF97XFxwYXJ0aWFsIEYobikgXFx0aW1lcyBcXExhbWJkYV5uX2l9IEYobikgXFx0aW1lcyBcXExhbWJkYV5uX2kiXSxbMSwwLCJYIl0sWzEsMSwiWSJdLFswLDEsIkYobikgXFx0aW1lcyBcXERlbHRhXmwiXSxbMCwxXSxbMSwyLCJmIl0sWzAsM10sWzMsMl0sWzMsMSwiIiwxLHsic3R5bGUiOnsiYm9keSI6eyJuYW1lIjoiZGFzaGVkIn19fV1d
%   \[\begin{tikzcd}
%     {\partial F(n) \times \Delta^l \coprod_{\partial F(n) \times \Lambda^l_i} F(n) \times \Lambda^l_i} & X \\
%     {F(n) \times \Delta^l} & Y
%     \arrow[from=1-1, to=1-2]
%     \arrow["f", from=1-2, to=2-2]
%     \arrow[from=1-1, to=2-1]
%     \arrow[from=2-1, to=2-2]
%     \arrow[dashed, from=2-1, to=1-2]
%   \end{tikzcd}\]
% \end{definition}

% \begin{definition}[Reedyファイブラント]
%   単体的空間の射$X \to \Delta^0$がReedyファイブレーションのとき, $X$をReedyファイブラント(Reedy fibrant)という. 
% \end{definition}

% ReedyファイブラントはKanファイブレーションを用いて定義することができる. 

\begin{definition}[Reedyファイブラント]
  任意の$n,l \geq 0$と$0 \leq i \leq n$に対して, 次の単体的集合の射 
  \begin{align*}
    \Map_{\sSpace}(F(n),X) \to \Map_{\sSpace}(\partial F(n),X)
  \end{align*}
  がKanファイブレーションのとき, $X$をReedyファイブラント(Reedy fibrant)
  \footnote{
    $\sSet$上のKan-Quillenモデル構造から定まるReedyモデル構造におけるファイブラント対象であるが, ここではReedyモデル構造について立ち入らない. 
  }
  という.
\end{definition}

\begin{example}
  任意の$n \geq 0$に対して, $F(n)$はReedyファイブラントである. 
\end{example}

\begin{example}
  任意の圏$\C$に対して, 脈体$\nerve(\C)$はReedyファイブラントである. 
\end{example}

% \begin{proof}
%   次の図式のリフト$\alpha : F(k) \times \Delta^l \to F(n)$が存在することを言えばよい. 
%   % https://q.uiver.app/#q=WzAsMyxbMCwwLCJcXHBhcnRpYWwgRihrKSBcXHRpbWVzIFxcRGVsdGFebCBcXGNvcHJvZF97XFxwYXJ0aWFsIEYoaykgXFx0aW1lcyBcXExhbWJkYV5sX2l9IEYoaykgXFx0aW1lcyBcXExhbWJkYV5sX2kiXSxbMCwxLCJGKGspIFxcdGltZXMgXFxEZWx0YV5sIl0sWzEsMCwiRihuKSJdLFswLDFdLFsxLDIsIlxcYWxwaGEiLDIseyJzdHlsZSI6eyJib2R5Ijp7Im5hbWUiOiJkYXNoZWQifX19XSxbMCwyXV0=
%   \[\begin{tikzcd}
%     {\partial F(k) \times \Delta^l \coprod_{\partial F(k) \times \Lambda^l_i} F(k) \times \Lambda^l_i} & {F(n)} \\
%     {F(k) \times \Delta^l}
%     \arrow[from=1-1, to=2-1]
%     \arrow["\alpha"', dashed, from=2-1, to=1-2]
%     \arrow[from=1-1, to=1-2]
%   \end{tikzcd}\]
%   Yonedaの補題より, 
%   \begin{align*}
%     \Map_{\sSpace}(F(k) \times \Delta^l, F(n)) 
%     &\cong \Hom_{\Delta \times \Delta}({k} \times [l], [n] \times [0]) \\
%     &\cong \Hom_{\Delta \times \Delta}([k],[l]) \\
%     &\cong \Map_{\sSpace}(F(k),F(n))
%   \end{align*}
%   次の図式において, $\theta$と$\theta'$は一致するので, $\alpha := \theta \circ p_1$とすればよい. 
%   % https://q.uiver.app/#q=WzAsNSxbMSwwLCJcXHBhcnRpYWwgRihrKSBcXHRpbWVzIFxcRGVsdGFebCBcXGNvcHJvZF97XFxwYXJ0aWFsIEYoaykgXFx0aW1lcyBcXExhbWJkYV5sX2l9IEYoaykgXFx0aW1lcyBcXExhbWJkYV5sX2kiXSxbMSwxLCJGKGspIFxcdGltZXMgXFxEZWx0YV5sIl0sWzIsMCwiRihuKSJdLFsyLDEsIkYoaykiXSxbMCwwLCJGKGspIFxcdGltZXMgXFxMYW1iZGFebF9pIl0sWzAsMV0sWzEsMiwiXFxhbHBoYSIsMix7InN0eWxlIjp7ImJvZHkiOnsibmFtZSI6ImRhc2hlZCJ9fX1dLFswLDJdLFsxLDMsInBfMSIsMl0sWzMsMiwiXFx0aGV0YSIsMl0sWzQsMCwiIiwyLHsic3R5bGUiOnsidGFpbCI6eyJuYW1lIjoiaG9vayIsInNpZGUiOiJ0b3AifX19XSxbNCwyLCJcXHRoZXRhJyIsMCx7ImN1cnZlIjotM31dXQ==
%   \[\begin{tikzcd}
%     {F(k) \times \Lambda^l_i} & {\partial F(k) \times \Delta^l \coprod_{\partial F(k) \times \Lambda^l_i} F(k) \times \Lambda^l_i} & {F(n)} \\
%     & {F(k) \times \Delta^l} & {F(k)}
%     \arrow[from=1-2, to=2-2]
%     \arrow["\alpha"', dashed, from=2-2, to=1-3]
%     \arrow[from=1-2, to=1-3]
%     \arrow["{p_1}"', from=2-2, to=2-3]
%     \arrow["\theta"', from=2-3, to=1-3]
%     \arrow[hook, from=1-1, to=1-2]
%     \arrow["{\theta'}", curve={height=-18pt}, from=1-1, to=1-3]
%   \end{tikzcd}\]
% \end{proof}

\begin{remark}
  $X$を単体的空間とする. 
  任意の$0 \leq i \leq n$に対して, $\Delta$の射
  \begin{align*}
    i : [0] \to [n] : 0 \mapsto i
  \end{align*}
  は単体的集合の射$i^\ast : X_n \to X_0$を定める. 
  よって, $i$は次の単体的集合の射
  \begin{align*}
    (0^\ast, \cdots, n^\ast) : X_n \to X_0 \times \cdots \times X_0 \cong (X_0)^{n+1}
  \end{align*}
  を定める. 
\end{remark}

\begin{lemma} \label{lem:Xn_to_X0^n+1_is_Kan_fibration}
  $X$をReedyファイブラントとする. 
  任意の$n \geq 0$に対して, 単体的集合の射$X_n \to (X_0)^{n+1}$はKanファイブレーションである. 
\end{lemma}

% \begin{proof}
%   次の図式のリフトが存在することを示す. 
%   % https://q.uiver.app/#q=WzAsNCxbMCwwLCJcXExhbWJkYV5rX2kiXSxbMSwwLCJYX24gXFxjb25nIFxcTWFwX1xcc1NwYWNlKEYobiksWCkiXSxbMCwxLCJcXERlbHRhXmsiXSxbMSwxLCIoWF8wKV57bisxfSBcXGNvbmcgXFxNYXBfXFxzU3BhY2UoXFxjb3Byb2Rfe24rMX1GKDApLFgpIl0sWzAsMV0sWzAsMiwiIiwyLHsic3R5bGUiOnsidGFpbCI6eyJuYW1lIjoiaG9vayIsInNpZGUiOiJ0b3AifX19XSxbMiwzXSxbMSwzXSxbMiwxLCIiLDEseyJzdHlsZSI6eyJib2R5Ijp7Im5hbWUiOiJkYXNoZWQifX19XV0=
%   \[\begin{tikzcd}
%     {\Lambda^k_i} & {X_n \cong \Map_{\sSpace}(F(n),X)} \\
%     {\Delta^k} & {(X_0)^{n+1} \cong \Map_{\sSpace}(\coprod_{n+1}F(0),X)}
%     \arrow[from=1-1, to=1-2]
%     \arrow[hook, from=1-1, to=2-1]
%     \arrow[from=2-1, to=2-2]
%     \arrow[from=1-2, to=2-2]
%     \arrow[dashed, from=2-1, to=1-2]
%   \end{tikzcd}\]
%   積-Map随伴より, 次の射が全射であることを示せばよい. 
%   % https://q.uiver.app/#q=WzAsMixbMCwwLCJcXEhvbV9cXHNTZXQoXFxEZWx0YV5rLFxcTWFwX1xcc1NwYWNlKEYobiksWCkpIl0sWzAsMSwiXFxIb21fXFxzU2V0KFxcTGFtYmRhXmtfaSxcXE1hcF9cXHNTcGFjZShGKG4pLFgpKSBcXHRpbWVzX3tcXE1hcF9cXHNTcGFjZShcXExhbWJkYV5rX2ksXFxNYXBfXFxzU3BhY2UoXFxjb3Byb2Rfe24rMX1GKDApLFgpKX0gXFxIb21fXFxzU2V0KFxcRGVsdGFeayxcXE1hcF9cXHNTcGFjZShcXGNvcHJvZF97bisxfUYoMCksWCkpIl0sWzAsMV1d
%   \[\begin{tikzcd}
%     {\Hom_\sSet(\Delta^k,\Map_{\sSpace}(F(n),X))} \\
%     {\Hom_\sSet(\Lambda^k_i,\Map_{\sSpace}(F(n),X)) \times_{\Map_{\sSpace}(\Lambda^k_i,\Map_{\sSpace}(\coprod_{n+1}F(0),X))} \Hom_\sSet(\Delta^k,\Map_{\sSpace}(\coprod_{n+1}F(0),X))}
%     \arrow[from=1-1, to=2-1]
%   \end{tikzcd}\]
%   \cref{rem:bisimplicial_set}より, これは次の射が全射であることと同値である. 
%   % https://q.uiver.app/#q=WzAsMixbMCwwLCJcXEhvbV9cXHNTcGFjZShGKG4pIFxcdGltZXMgXFxEZWx0YV5rLFgpIl0sWzAsMSwiXFxIb21fXFxzU3BhY2UoRihuKSBcXHRpbWVzIFxcTGFtYmRhXmtfaSxYKSBcXHRpbWVzX3tcXE1hcF9cXHNTcGFjZShcXGNvcHJvZF97bisxfUYoMCkgXFx0aW1lcyBcXExhbWJkYV5rX2ksWCl9IFxcSG9tX1xcc1NwYWNlKFxcY29wcm9kX3tuKzF9RigwKSBcXHRpbWVzIFxcRGVsdGFeayxYKSJdLFswLDFdXQ==
%   \[\begin{tikzcd}
%     {\Hom_\sSpace(F(n) \times \Delta^k,X)} \\
%     {\Hom_\sSpace(F(n) \times \Lambda^k_i,X) \times_{\Map_{\sSpace}(\coprod_{n+1}F(0) \times \Lambda^k_i,X)} \Hom_\sSpace(\coprod_{n+1}F(0) \times \Delta^k,X)}
%     \arrow[from=1-1, to=2-1]
%   \end{tikzcd}\]
%   これは次の図式のリフト$F(n) \times \Delta^k \to X$が存在することと同値である. 
%   % https://q.uiver.app/#q=WzAsMyxbMCwwLCIoRihuKSBcXHRpbWVzIFxcTGFtYmRhXmtfaSkgXFxjb3Byb2Rfe1xcY29wcm9kX3tuKzF9RigwKSBcXHRpbWVzIFxcTGFtYmRhXmtfaX0gXFx0aW1lcyBcXERlbHRhXmsiXSxbMCwxLCJGKG4pIFxcdGltZXMgXFxEZWx0YV5rIl0sWzEsMCwiWCJdLFswLDFdLFsxLDIsIiIsMCx7InN0eWxlIjp7ImJvZHkiOnsibmFtZSI6ImRhc2hlZCJ9fX1dLFswLDJdXQ==
%   \[\begin{tikzcd}
%     {(F(n) \times \Lambda^k_i) \coprod_{\coprod_{n+1}F(0) \times \Lambda^k_i} \times \Delta^k} & X \\
%     {F(n) \times \Delta^k}
%     \arrow[from=1-1, to=2-1]
%     \arrow[dashed, from=2-1, to=1-2]
%     \arrow[from=1-1, to=1-2]
%   \end{tikzcd}\]
%   (途中)
% \end{proof}

\begin{lemma} \label{lem:Xn_is_Kan_fibrant}
  $X$をReedyファイブラントとする. 
  任意の$n \geq 0$に対して, $X_n$はKan複体である. 
\end{lemma}

\begin{proof}
  $X$はReedyファイブラントなので, 次の単体的集合の射
  \begin{align*}
    \Map_{\sSpace}(F(n),X) \to \Map_{\sSpace}(\partial F(n),X)
  \end{align*}
  はKanファイブレーションである. 
  $n=0$を考えると, 
  \begin{align*}
    \Map_{\sSpace}(F(0),X) \cong X_0 , ~~\Map_{\sSpace}(\partial F(0),X) \cong \Delta^0
  \end{align*}
  なので, $X_0 \to \Delta^0$はKanファイブレーションである. 
  よって, $X_0$はKan複体である.
  Kan複体は直積で保たれるので, $(X_0)^{n+1}$もKan複体である. 
  \cref{lem:Xn_to_X0^n+1_is_Kan_fibration}より, $X_n \to (X_0)^{n+1}$はKanファイブレーションである.
  Kanファイブレーションのクラスは合成で閉じるので, $X_n$はKan複体である.
\end{proof}

\section{Segal空間} \label{sec:segal_space}

単体的空間のReedyファイブラント条件は単体的空間の$n$単体が空間のようにふるまうことを表していた (\cref{lem:Xn_is_Kan_fibrant}). 
この節では, それらが更に圏のようにもふるまうクラスとしてSegal空間を定義する. 
単体的空間のSegal空間は単体的集合のSegal条件の一般化である. 

\begin{definition}[Segal空間]
  $X$をReedyファイブラントとする. 
  任意の$n \geq 2$に対して, 単体的集合の射
  \begin{align*}
    \varphi_n : X_n \to X_1 \times_{X_0} \cdots \times_{X_0} X_1
  \end{align*}
  がKan弱同値のとき, $X$をSegal空間(Segal space)という. 
\end{definition}

\begin{remark}
  $X$がReedyファイブラントのとき, $\varphi_n$はKanファイブレーションである.
  よって, Segal空間の定義において「$\varphi_n$が自明なKanファイブレーションのとき」としてもよい. 
\end{remark}

単体的集合$X_1 \times_{X_0} \cdots \times_{X_0} X_1$は$F(n)$の単体的部分空間として表すことができる. 

\begin{definition}[椎]
  $F(n)$の単体的部分空間
  \begin{align*}
    G(n) := F(1) \coprod_{F(0)} \cdots \coprod_{F(0)} F(1)
  \end{align*}
  を$F(n)$の椎(spine)という. 
\end{definition}

\begin{remark}
  椎の定義より, 次の同型が成立する. 
  \begin{align*}
    \Map_{\sSpace}(G(n),X) \cong X_1 \times_{X_0} \cdots \times_{X_0} X_1
  \end{align*}
\end{remark}

% \begin{remark} \label{rem:2_segal_condition}
%   $n=2$のSegal空間の条件は$\varphi_2 : X_2 \xrightarrow{\simeq} X_1 \times_{X_0} X_1$である. 
%   $X_2$は2単体の集まりなので, 2単体$\sigma \in X_2$は次のように表せる. 
%   \[
%     \begin{tikzpicture}[auto,->]
%       \node (x) at (0,0) {$x$};
%       \node (z) at (3,0) {$z$};
%       \node (y) at (1.5,2) {$y$};
%       \node (sigma) at (1.5,0.7) {$\sigma$};
%       \draw (x) -- node {$f$} (y);
%       \draw (y) -- node {$g$} (z);
%       \draw (x) -- (z);  
%     \end{tikzpicture}
%   \]
%   同様に, $X_1 \times_{X_0} X_1$は2つの合成可能な射の集まりとみなせて, 次のように表せる. 
%   \[
%     \begin{tikzpicture}[auto,->]
%       \node (x) at (0,0) {$x$};
%       \node (z) at (3,0) {$z$};
%       \node (y) at (1.5,2) {$y$};
%       \draw (x) -- node {$f$} (y);
%       \draw (y) -- node {$g$} (z);
%     \end{tikzpicture}
%   \]
%   Segal条件はこのような図式が次の2単体に拡張できることを意味している. 
%   \[
%     \begin{tikzpicture}[auto,->]
%       \node (x) at (0,0) {$x$};
%       \node (z) at (3,0) {$z$};
%       \node (y) at (1.5,2) {$y$};
%       \node (sigma) at (1.5,0.7) {$\sigma$};
%       \draw (x) -- node {$f$} (y);
%       \draw (y) -- node {$g$} (z);
%       \draw[dashed] (x) -- node[swap] {$h$} (z);  
%     \end{tikzpicture}
%   \]
%   よって, $h$を$gf$と表すことが多いが, $h$は$g$と$f$(と$\sigma$)に対して一意に定まらないことに注意. 
%   しかし, \cref{theorem:composition_is_homotopic}で$h$はホモトピーの違いを除いて一意であることが分かる.  
% \end{remark}

% \begin{remark} \label{rem:3_segal_condition}
%   $n=3$のSegal空間の条件は$\varphi_3 : X_3 \xrightarrow{\simeq} X_1 \times_{X_0} X_1 \times_{X_0} X_1$である. 
%   $X_3$は3単体の集まりで, 3単体は次のように表せる. 
%   \[
%     \begin{tikzpicture}[auto,->]
%       \node (x) at (0,0) {$x$};
%       \node (y) at (4,0.5) {$y$};
%       \node (z) at (3,-1) {$z$};
%       \node (w) at (2,3) {$w$};
%       \node (sigma) at (2.5,-0.3) {$\sigma$};
%       \node (gamma) at (3,1) {$\gamma$};
%       \draw (x) -- node {$f$} (y);
%       \draw (y) -- node {$g$} (z);
%       \draw (z) -- node {$h$} (w);
%       \draw (x) -- (z);
%       \draw (x) -- (w);
%       \draw (y) -- (w);
%     \end{tikzpicture}
%   \]
%   同様に, $X_1 \times_{X_0} X_1 \times_{X_0} X_1$は3つの合成可能な射の集まりとみなせて, 次のように表せる.
%   \[
%     \begin{tikzpicture}[auto,->]
%       \node (x) at (0,0) {$x$};
%       \node (y) at (4,0.5) {$y$};
%       \node (z) at (3,-1) {$z$};
%       \node (w) at (2,3) {$w$};
%       \draw (x) -- node {$f$} (y);
%       \draw (y) -- node {$g$} (z);
%       \draw (z) -- node {$h$} (w);
%     \end{tikzpicture}
%   \]
%   Segal条件はこのような図式が次の3単体に拡張できることを意味している.
%   \[
%     \begin{tikzpicture}[auto,->]
%       \node (x) at (0,0) {$x$};
%       \node (y) at (4,0.5) {$y$};
%       \node (z) at (3,-1) {$z$};
%       \node (w) at (2,3) {$w$};
%       \node (sigma) at (2.5,-0.3) {$\sigma$};
%       \node (gamma) at (3,1) {$\gamma$};
%       \draw (x) -- node {$f$} (y);
%       \draw (y) -- node {$g$} (z);
%       \draw (z) -- node {$h$} (w);
%       \draw[dashed] (x) -- node[swap] {$gf$} (z);
%       \draw[dashed] (x) -- node {$h(gf),~ (hg)f$}  (w);
%       \draw[dashed] (y) -- node[swap] {$hg$} (w);
%     \end{tikzpicture}
%   \]
%   よって, この3単体は合成可能な射の結合性を表している. 
%   2単体において合成は一意ではなかったが, 3単体において$(hg)f$と$h(gf)$は等しいことを意味している. 
% \end{remark}

単体的空間のSegal条件は圏の脈体のSegal条件と非常に似ている. 
実際, 圏の脈体は(離散単体的集合としてみなすと)Segal空間となる. 

\begin{example} \label{eg:NC_is_Segal_space}
  任意の圏$\C$に対して, 脈体$\nerve(\C)$はSegal空間である. 
\end{example}

% \begin{theorem}
%   $X$をReedyファイブラントとする. 
%   $X$がhomotopically constantのとき, $X$はSegal空間である.
% \end{theorem}

% \begin{proof}
%   $X$はReedyファイブラントなので, 次の単体的集合の射
%   \begin{align*}
%     \varphi_n : X_n \to X_1 \times_{X_0} \cdots \times_{X_0} X_1
%   \end{align*}
%   は自明なKanファイブレーションである.
% \end{proof}

\section{Segal空間における射の合成} \label{sec:composition_in_Segal_space}

Segal空間とは, Reedyファイブラント条件とSegal条件を満たすような単体的空間であった. 
Segal空間$X$に対して, $X_0,X_1,X_2,\cdots$はKan複体である. 
圏の脈体と同様に, $X_0$は対象の空間, $X_1$は射の空間, $X_2$は合成を表す空間のように思える. 
この節では, $X$をSegal空間とする. 

\begin{definition}[対象]
  $X_0$の点を$X$の対象(object)という. 
  % $X$の対象の集まりを$\Ob(X)$と表すと, $\Ob(X) = T_{0,0}$である. 
\end{definition}

\begin{definition}[合成空間]
  $X$の対象$x_0,\cdots,x_n$に対して, 単体的集合$\map_X(x_0,\cdots,x_n)$を次のpullbackで定義し, 合成空間(space of composition)という.  
  % https://q.uiver.app/#q=WzAsNCxbMCwwLCJcXG1hcF9YKHhfMCxcXGNkb3RzLHhfbikiXSxbMSwwLCJYX24iXSxbMSwxLCIoWF8wKV57bisxfSJdLFswLDEsIlxcRGVsdGFeMCJdLFswLDFdLFswLDNdLFszLDJdLFsxLDJdLFswLDIsIiIsMSx7InN0eWxlIjp7Im5hbWUiOiJjb3JuZXIifX1dXQ==
  \[\begin{tikzcd}
    {\map_X(x_0,\cdots,x_n)} & {X_n} \\
    {\Delta^0} & {(X_0)^{n+1}}
    \arrow[from=1-1, to=1-2]
    \arrow[from=1-1, to=2-1]
    \arrow["{(x_0,\cdots,x_n)}"', from=2-1, to=2-2]
    \arrow[from=1-2, to=2-2]
    \arrow["\lrcorner"{anchor=center, pos=0.125}, draw=none, from=1-1, to=2-2]
  \end{tikzcd}\]
\end{definition}

\begin{remark}
  \cref{lem:Xn_to_X0^n+1_is_Kan_fibration}より, $X_n \to (X_0)^{n+1}$はKanファイブレーションである.
  Kanファイブレーションのクラスはプルバックで閉じるので, $\map_X(x_0,\cdots,x_n) \to \Delta^0$はKanファイブレーションである.
  つまり, $\map_X(x_0,\cdots,x_n)$はKan複体である.
\end{remark}

\begin{definition}[射]
  $n=1$の合成空間$\map_X(x_0,x_1)$を射空間(mapping space)という. 
  $\map_X(x_0,x_1)$の点を$X$の射(morphism)といい, $f : x_0 \to x_1$と表す. 
\end{definition}

\begin{definition}[恒等射]
  $X$の対象$x$に対して, $s_0 : X_0 \to X_1$による$x$の像$s_0(x)$を$x$の恒等射(identity map)といい, $\id_x$と表す.
\end{definition}

% \begin{remark} \label{rem:map_is_map_times_map}
  
%   % 合成空間の図式は次の図式に拡張できる. 
%   % % https://q.uiver.app/#q=WzAsNSxbMCwwLCJcXG1hcF9YKHhfMCxcXGNkb3RzLHhfbikiXSxbMSwwLCJYX24iXSxbMSwxLCIoWF8wKV57bisxfSJdLFswLDEsIlxcRGVsdGFeMCJdLFsyLDAsIlhfMSBcXHRpbWVzX3tYXzB9IFxcdGltZXMgXFx0aW1lc197WF8wfSBYXzEiXSxbMCwxXSxbMCwzXSxbMywyLCIoeF8wLFxcY2RvdHMseF9uKSIsMl0sWzEsMl0sWzAsMiwiIiwxLHsic3R5bGUiOnsibmFtZSI6ImNvcm5lciJ9fV0sWzEsNCwiXFxjb25nIl0sWzQsMl1d
%   % \[\begin{tikzcd}
%   %   {\map_X(x_0,\cdots,x_n)} & {X_n} & {X_1 \times_{X_0} \cdots \times_{X_0} X_1} \\
%   %   {\Delta^0} & {(X_0)^{n+1}}
%   %   \arrow[from=1-1, to=1-2]
%   %   \arrow[from=1-1, to=2-1]
%   %   \arrow["{(x_0,\cdots,x_n)}"', from=2-1, to=2-2]
%   %   \arrow[from=1-2, to=2-2]
%   %   \arrow["\lrcorner"{anchor=center, pos=0.125}, draw=none, from=1-1, to=2-2]
%   %   \arrow["\simeq", from=1-2, to=1-3]
%   %   \arrow[from=1-3, to=2-2]
%   % \end{tikzcd}\]
% \end{remark}

\begin{lemma} 
  $X$の任意の対象$x_0,\cdots,x_n$に対して, 次の単体的集合の射
  \begin{align*}
    \map_X(x_0,\cdots,x_n) \to \map_X(x_0,x_1) \times \cdots \times \map_X(x_{n-1},x_n)
  \end{align*}
  は自明なKanファイブレーションである.
\end{lemma}

\begin{proof}
  Segal空間$X$に対して, $ X_n \to X_1 \times_{X_0} \cdots \times_{X_0} X_1$は自明なKanファイブレーションである.
  次の図式において, 下と全体の四角はpullbackである. % https://q.uiver.app/#q=WzAsNixbMCwwLCJcXG1hcF9YKHhfMCxcXGNkb3RzLHhfbikiXSxbMSwwLCJYX24iXSxbMSwxLCJYXzEgXFx0aW1lc197WF8wfSBcXGNkb3RzIFxcdGltZXNfe1hfMH0gWF8xIl0sWzEsMiwiKFhfMClee24rMX0iXSxbMCwxLCJcXG1hcF9YKHhfMCx4XzEpIFxcdGltZXMgXFxjZG90cyBcXHRpbWVzIFxcbWFwX1goeF97bi0xfSx4X24pIl0sWzAsMiwiXFxEZWx0YV4wIl0sWzAsMV0sWzEsMiwiXFxzaW0iXSxbMiwzXSxbMCw0XSxbNCw1XSxbNSwzXSxbNCwyXSxbNCwzLCIiLDAseyJzdHlsZSI6eyJuYW1lIjoiY29ybmVyIn19XSxbMCwyLCIiLDAseyJzdHlsZSI6eyJuYW1lIjoiY29ybmVyIn19XV0=
  \[\begin{tikzcd}
    {\map_X(x_0,\cdots,x_n)} & {X_n} \\
    {\map_X(x_0,x_1) \times \cdots \times \map_X(x_{n-1},x_n)} & {X_1 \times_{X_0} \cdots \times_{X_0} X_1} \\
    {\Delta^0} & {(X_0)^{n+1}}
    \arrow[from=1-1, to=1-2]
    \arrow["\sim", from=1-2, to=2-2]
    \arrow[from=2-2, to=3-2]
    \arrow[from=1-1, to=2-1]
    \arrow[from=2-1, to=3-1]
    \arrow[from=3-1, to=3-2]
    \arrow[from=2-1, to=2-2]
    \arrow["\lrcorner"{anchor=center, pos=0.125, rotate=45}, draw=none, from=2-1, to=3-2]
    \arrow["\lrcorner"{anchor=center, pos=0.125, rotate=45}, draw=none, from=1-1, to=2-2]
  \end{tikzcd}\]
  よって, 上の四角もpullbackである. 
  自明なKanファイブレーションのクラスはプルバックで閉じるので, 単体的集合の射
  \begin{align*}
    \map_X(x_0,\cdots,x_n) \to \map_X(x_0,x_1) \times \cdots \times \map_X(x_{n-1},x_n)
  \end{align*}
  は自明なKanファイブレーションである.
\end{proof}

Segal空間において, 射の合成は可縮な空間の違いを除いて一意に定まる. 

\begin{remark}
  $X$の対象$x,y,z$に対して, 合成空間の定義から次の図式が得られる. 
  % https://q.uiver.app/#q=WzAsMyxbMCwwLCJcXG1hcF9YKHgseSx6KSJdLFsxLDAsIlxcbWFwX1goeCx6KSJdLFswLDEsIlxcbWFwX1goeCx5KSBcXHRpbWVzIFxcbWFwX1goeSx6KSJdLFswLDEsImRfMSJdLFswLDIsIlxcY29uZyIsMl1d
  \[\begin{tikzcd}
    {\map_X(x,y,z)} & {\map_X(x,z)} \\
    {\map_X(x,y) \times \map_X(y,z)}
    \arrow["{d_1}", from=1-1, to=1-2]
    \arrow["\sim"', from=1-1, to=2-1]
  \end{tikzcd}\]
  $X$の射$f:x \to y, g : y \to z$に対して, 単体的集合$\Comp(f,g)$を次のプルバックで定義する. % https://q.uiver.app/#q=WzAsNSxbMSwwLCJcXG1hcF9YKHgseSx6KSJdLFsyLDAsIlxcbWFwX1goeCx6KSJdLFsxLDEsIlxcbWFwX1goeCx5KSBcXHRpbWVzIFxcbWFwX1goeSx6KSJdLFswLDAsIlxcQ29tcChmLGcpIl0sWzAsMSwiXFxEZWx0YV4wIl0sWzAsMSwiZF8xIl0sWzAsMiwiXFxzaW0iXSxbMywwLCIiLDIseyJzdHlsZSI6eyJ0YWlsIjp7Im5hbWUiOiJob29rIiwic2lkZSI6InRvcCJ9fX1dLFszLDQsIlxcc2ltIiwyXSxbNCwyLCIoZixnKSIsMl0sWzMsMiwiIiwxLHsic3R5bGUiOnsibmFtZSI6ImNvcm5lciJ9fV1d
  \[\begin{tikzcd}
    {\Comp(f,g)} & {\map_X(x,y,z)} & {\map_X(x,z)} \\
    {\Delta^0} & {\map_X(x,y) \times \map_X(y,z)}
    \arrow["{d_1}", from=1-2, to=1-3]
    \arrow["\sim", from=1-2, to=2-2]
    \arrow[hook, from=1-1, to=1-2]
    \arrow["\sim"', from=1-1, to=2-1]
    \arrow["{(f,g)}"', from=2-1, to=2-2]
    \arrow["\lrcorner"{anchor=center, pos=0.125, rotate=45}, draw=none, from=1-1, to=2-2]
  \end{tikzcd}\]
  つまり, $\Comp(f,g)$は次のように表せる. 
  \begin{align*}
    \Comp(f,g) = \{\sigma \in X_2 ~|~ d_0\sigma = g, ~~ d_2\sigma = f\}
  \end{align*}
  自明なKanファイブレーションのクラスはプルバックで閉じるので, $\Comp(f,g) \to \Delta^0$は自明なKanファイブレーションである. 
  よって, $\Comp(f,g)$は可縮なKan複体である. 
  この意味で, 射の合成は可縮な空間の違いを除いて一意に定まる.
\end{remark}

\begin{example}
  圏$\C$の脈体$\nerve(\C)$の任意の射$f:x \to y, g: y \to z$に対して, $\Comp(f,g) \cong \Delta^0$である. 
\end{example}

\begin{proof}
  次の図式において, 同型射のクラスがプルバックで閉じることから従う. 
  % https://q.uiver.app/#q=WzAsNCxbMCwwLCJcXENvbXAoZixnKSJdLFsxLDAsIk4oXFxDKV8yIl0sWzEsMSwiTihcXEMpXzEgXFx0aW1lc197TihcXEMpXzB9IE4oXFxDKV8xIl0sWzAsMSwiXFxEZWx0YV4wIl0sWzAsMV0sWzEsMiwiXFxjb25nIl0sWzAsM10sWzMsMl0sWzAsMiwiIiwxLHsic3R5bGUiOnsibmFtZSI6ImNvcm5lciJ9fV1d
  \[\begin{tikzcd}
    {\Comp(f,g)} & {\nerve(\C)_2} \\
    {\Delta^0} & {\nerve(\C)_1 \times_{\nerve(\C)_0} \nerve(\C)_1}
    \arrow[from=1-1, to=1-2]
    \arrow["\cong", from=1-2, to=2-2]
    \arrow[from=1-1, to=2-1]
    \arrow[from=2-1, to=2-2]
    \arrow["\lrcorner"{anchor=center, pos=0.125}, draw=none, from=1-1, to=2-2]
  \end{tikzcd}\]
\end{proof}

\section{Segal空間のホモトピー圏}

\cref{sec:composition_in_Segal_space}では, 単体的空間が圏のようにふるまうためにSegal条件を定義した. 
この節では, Segal空間がもつホモトピー論的な性質に着目する.
$X$をSegal空間とする. 

\begin{definition}[ホモトピック]
  $f,g : x \to y$を$X$の射とする.
  単体的集合の射$f,g : \Delta^0 \to \map_X(x,y)$が単体的集合のホモトピックのとき, $f$と$g$はホモトピック(homotopic)であるといい, $f \sim g$と表す.  
\end{definition}

射の合成はホモトピーの違いを除いて結合的かつ単位的である. 

\begin{theorem} \label{theorem:composition_is_homotopic}
  $X$の射$f : w \to x, g : x \to y, h : y \to z$に対して, $h(gf) \sim (hg)f$かつ$f\id_x \sim \id_yf \sim f$が成立する. 
\end{theorem}

\cref{theorem:composition_is_homotopic}より, Segal空間に対して通常の圏が定まる. 

\begin{definition}[ホモトピー圏]
  通常の圏$\Ho(X)$を次のように定義し, $X$のホモトピー圏(homotopy category)という. 
  \begin{itemize}
    \item $\Ho(X)$の対象は$X$の対象と同じ
    \item $\Ho(X)$の任意の対象$x,y$に対して, $\Hom_{\Ho(X)}(x,y)$は射空間のホモトピー類$\pi_0(\map_X(x,y))$
    \item $\Ho(X)$任意の対象$x,y,z$に対して, 合成は$\map_X(x,y) \times \map_X(y,z) \to \map_X(x,z)$から定まる対応
    % \begin{align*}
    %   \map_X(x,y) \times \map_X(y,z) \to \map_X(x,z)
    % \end{align*}
    % から定まる自然な対応
  \end{itemize}
\end{definition}

\begin{theorem} \label{thrm:C_is_HoNC}
  任意の圏$\C$は$\Ho(\nerve(\C))$と圏同型である.
\end{theorem}

\begin{proof}
  対象に関して, 
  \begin{align*}
    \Ob\Ho(\nerve(\C)) 
    = \Ob\nerve(\C) 
    = \nerve(\C)_0 
    = \Ob\C
  \end{align*}
  $\Ho(N\C)$の任意の対象$x,y$に対して, 
  \begin{align*}
    \Hom_{\Ho(\nerve(\C))}(x,y) 
    = \pi_0(\Map_{\nerve(\C)}(x,y)) 
    = \pi_0(\nerve(\C)_1 \times_{\nerve(\C)_0 \times \nerve(\C)_0} \Delta^0)
  \end{align*}
  $\nerve(\C)_1$と$\nerve(\C)_0 \times \nerve(\C)_0$は離散単体的集合なので, 
  \begin{align*}
    \pi_0(\nerve(\C)_1 \times_{\nerve(\C)_0 \times \nerve(\C)_0} \Delta^0) 
    \cong N\C_1 \times_{\nerve(\C)_0 \times \nerve(\C)_0} \Delta^0 
    = \Hom_\C(x,y)
  \end{align*}
  % よって, $\C$と$\Ho(\nerve(\C))$は圏同型である.
\end{proof}

\begin{definition}[ホモトピー同値]
  $X$の射$f \in \map_X(x,y)$に対して, ある射$g \in \map_X(y,x)$が存在して, $gf \sim \id_x$かつ$fg \sim \id_y$
  % \begin{align*}
  %   gf \sim \id_x,~~ fg \sim \id_y
  % \end{align*}
  を満たすとき, $f$をホモトピー同値(homotopy equivalence)という.  
  このとき, $g$を$f$のホモトピー逆射(homotopy inverse)という. 
\end{definition}

\begin{remark}
  ホモトピー逆射はホモトピーの違いを除いて一意である. 
\end{remark}

% \begin{remark}
%   $g$を$f$の左ホモトピー逆射, $h$を$f$の右ホモトピー逆射とする. 
%   \cref{theorem:composition_is_homotopic}より, 
%   \begin{align*}
%     g \sim g \id_y \sim g f h \sim \id_x h \sim h
%   \end{align*}
%   なので, ホモトピー逆射はホモトピーの違いを除いて一意である. 
% \end{remark}

\begin{example} \label{eg:id_is_homotopy_equivalence}
  $X$の任意の対象$x$に対して, 恒等射$\id_x$はホモトピー逆射である.
\end{example}

% ホモトピー同値がホモトピーに関して不変であることを確かめる.

% \begin{definition}
%   $Z(3)$を次の図式の余極限で定まる$F(3)$の単体的部分空間とする. 
%   % https://q.uiver.app/#q=WzAsNSxbMCwwLCJGKDEpIl0sWzEsMSwiRigwKSJdLFsyLDAsIkYoMSkiXSxbMywxLCJGKDApIl0sWzQsMCwiRigxKSJdLFsxLDAsIjEiXSxbMSwyLCIxIiwyXSxbMywyLCIwIl0sWzMsNCwiMCIsMl1d
%   \[\begin{tikzcd}
%     {F(1)} && {F(1)} && {F(1)} \\
%     & {F(0)} && {F(0)}
%     \arrow["1", from=2-2, to=1-1]
%     \arrow["1"', from=2-2, to=1-3]
%     \arrow["0", from=2-4, to=1-3]
%     \arrow["0"', from=2-4, to=1-5]
%   \end{tikzcd}\]
% \end{definition}

% \begin{remark}
%   単体的空間$Z(3)$に対して, $\Map_{\sSpace}(Z(3),X)$は次の極限
%   \begin{align*}
%     \Map_{\sSpace}(Z(3),X) \cong X_1 \times_{X_0} X_1 \times_{X_0} X_1
%   \end{align*}
%   で表され, 次の図式の極限として表せる. 
%   % https://q.uiver.app/#q=WzAsNSxbMCwwLCJYXzEiXSxbMSwxLCJYXzAiXSxbMiwwLCJYXzEiXSxbMywxLCJYXzAiXSxbNCwwLCJYXzEiXSxbMCwxLCJkXzEiLDJdLFsyLDEsImRfMSJdLFsyLDMsImRfMCIsMl0sWzQsMywiZF8wIl1d
%   \[\begin{tikzcd}
%     {X_1} && {X_1} && {X_1} \\
%     & {X_0} && {X_0}
%     \arrow["{d_1}"', from=1-1, to=2-2]
%     \arrow["{d_1}", from=1-3, to=2-2]
%     \arrow["{d_0}"', from=1-3, to=2-4]
%     \arrow["{d_0}", from=1-5, to=2-4]
%   \end{tikzcd}\]
% \end{remark}

\begin{lemma}
  $f,g \in X_1$を$X$の射とする. 
  ある$\gamma : \Delta^1 \to X_1$が存在して, $\gamma(0)=f, \gamma(1)=g$を満たすとする. 
  $g$がホモトピー同値のとき, $f$もホモトピー同値である. 
\end{lemma}

\begin{definition}[ホモトピー同値空間]
  $X$のホモトピー同値のなす$X_1$の単体的部分集合を$X_\hoequiv$と表し, $X$のホモトピー同値空間(space of homotopy equivalences)という. 
\end{definition}

\begin{remark}
  \cref{eg:id_is_homotopy_equivalence}より, 恒等射はホモトピー同値である. 
  よって, $s_0$は$X_\hoequiv$によって分解される. % https://q.uiver.app/#q=WzAsMyxbMCwwLCJYXzAiXSxbMiwwLCJYXzEiXSxbMSwxLCJYX1xcaG9lcXVpdiJdLFswLDEsInNfMCJdLFswLDJdLFsyLDEsIiIsMix7InN0eWxlIjp7InRhaWwiOnsibmFtZSI6Imhvb2siLCJzaWRlIjoidG9wIn19fV1d
  \[\begin{tikzcd}
    {X_0} && {X_1} \\
    & {X_\hoequiv}
    \arrow["{s_0}", from=1-1, to=1-3]
    \arrow[from=1-1, to=2-2]
    \arrow[hook, from=2-2, to=1-3]
  \end{tikzcd}\]
\end{remark}

% \begin{lemma}
%   $X_\hoequiv \hookrightarrow X_1$は単体的集合のmono射である.
% \end{lemma}

% \begin{remark}
%   $X_\hoequiv$の点は$X$の射である.
%   具体的には, $X_\hoequiv$はホモトピー逆射が存在するような$X$の射のなす$X_1$の充満部分空間である.  
% \end{remark}

% ホモトピー同値の空間は可縮である. 
% つまり, 射がホモトピー同値となるような逆射の選択はホモトピーの違いを除いて一意である. 

% \begin{lemma}
%   $X$のホモトピー同値$f$, $f$の左ホモトピー逆射$g$, $f$の右ホモトピー逆射$h$の3つ組$(f,g,h)$で生成される$X_3$の単体的部分空間を$X_\hoeqchoice$と表す. 
%   $U : X_\hoeqchoice \to X_\hoequiv : (f,g,h) \mapsto f$をホモトピー逆射を忘れる射とする. 
%   このとき, $U$は次の図式を可換にする自明なKanファイブレーションである. 
%   % https://q.uiver.app/#q=WzAsMyxbMCwwLCJYX1xcaG9lcXVpdiJdLFsxLDAsIlhfXFxob2VxY2hvaWNlIl0sWzEsMSwiWF8xIl0sWzEsMCwiVSIsMl0sWzEsMl0sWzAsMiwiIiwyLHsic3R5bGUiOnsidGFpbCI6eyJuYW1lIjoiaG9vayIsInNpZGUiOiJ0b3AifX19XV0=
%   \[\begin{tikzcd}
%     {X_\hoequiv} & {X_\hoeqchoice} \\
%     & {X_1}
%     \arrow["U"', from=1-2, to=1-1]
%     \arrow[from=1-2, to=2-2]
%     \arrow[hook, from=1-1, to=2-2]
%   \end{tikzcd}\]
%   また, 次のpullbackが存在する.
%   % https://q.uiver.app/#q=WzAsNSxbMCwwLCJYX1xcaG9lcXVpdiJdLFsxLDAsIlhfXFxob2VxY2hvaWNlIl0sWzEsMSwiWF8xIl0sWzIsMCwiWF8zIl0sWzIsMSwiWF8xIFxcdGltZXNfe1hfMH1YXzEgXFx0aW1lc197WF8wfSBYXzEiXSxbMSwwLCJcXGNvbmciLDJdLFsxLDJdLFswLDIsIiIsMix7InN0eWxlIjp7InRhaWwiOnsibmFtZSI6Imhvb2siLCJzaWRlIjoidG9wIn19fV0sWzEsMywiIiwyLHsic3R5bGUiOnsidGFpbCI6eyJuYW1lIjoiaG9vayIsInNpZGUiOiJ0b3AifX19XSxbMyw0LCIoZF8xZF8zLGRfMGRfMyxkXzFkXzApIl0sWzIsNCwiKHNfMGRfMCxcXGlkX3t4XzF9LHNfMGRfMSkiLDJdLFsxLDQsIiIsMCx7InN0eWxlIjp7Im5hbWUiOiJjb3JuZXIifX1dXQ==
%   \[\begin{tikzcd}
%     {X_\hoequiv} & {X_\hoeqchoice} & {X_3} \\
%     & {X_1} & {X_1 \times_{X_0}X_1 \times_{X_0} X_1}
%     \arrow["\cong"', from=1-2, to=1-1]
%     \arrow[from=1-2, to=2-2]
%     \arrow[hook, from=1-1, to=2-2]
%     \arrow[hook, from=1-2, to=1-3]
%     \arrow["{(d_1d_3,d_0d_3,d_1d_0)}", from=1-3, to=2-3]
%     \arrow["{(s_0d_0,\id_{x_1},s_0d_1)}"', from=2-2, to=2-3]
%     \arrow["\lrcorner"{anchor=center, pos=0.125}, draw=none, from=1-2, to=2-3]
%   \end{tikzcd}\] 
% \end{lemma}

% $\infty$-groupoidと同様に, Segal空間亜群を定義する. 

% \begin{definition}[Segal空間亜群]
%   $X$の任意の射がホモトピー同値のとき, $X$をSegal空間亜群(Segal space groupoid)という. 
% \end{definition}

\begin{definition}[$x$と$y$の間のホモトピー同値空間]
  $x,y$を$X$の対象とする. 
  単体的集合$\hoequiv_X(x,y)$を次のホモトピープルバックで定義し, $x$と$y$の間のホモトピー同値空間(space of homotopy equivalences between two objects $x$ and $y$)という. 
  % https://q.uiver.app/#q=WzAsNCxbMCwwLCJcXGhvZXF1aXZfWCh4LHkpIl0sWzEsMCwiWF9cXGhvZXF1aXYiXSxbMSwxLCJYXzAgXFx0aW1lcyBYXzAiXSxbMCwxLCJcXERlbHRhXjAiXSxbMCwxXSxbMSwyLCIoZF8wLGRfMSkiXSxbMCwzXSxbMywyLCIoeCx5KSIsMl0sWzAsMiwiIiwxLHsic3R5bGUiOnsibmFtZSI6ImNvcm5lciJ9fV1d
  \[\begin{tikzcd}
    {\hoequiv_X(x,y)} & {X_\hoequiv} \\
    {\Delta^0} & {X_0 \times X_0}
    \arrow[from=1-1, to=1-2]
    \arrow["{(d_0,d_1)}", from=1-2, to=2-2]
    \arrow[from=1-1, to=2-1]
    \arrow["{(x,y)}"', from=2-1, to=2-2]
    \arrow["\lrcorner"{anchor=center, pos=0.125}, draw=none, from=1-1, to=2-2]
  \end{tikzcd}\]
\end{definition}

\begin{example} \label{eg:homotopy_equivalence_is_isomorphism_in_NC}
  $\C$を通常の圏とする. 
  $\nerve(\C)$の射がホモトピー同値であることと, $\C$の同型射であることは同値である. 
\end{example}

\section{完備Segal空間} \label{sec:complete_segal_space}

\cref{sec:reedy_fibrant}と\cref{sec:segal_space}でSegal空間が圏論とホモトピー論の性質を持つことを見た. 
しかし, 一般のSegal空間において圏論とホモトピー論の間には整合性がない.

\begin{notation}
  $I(1)$を2つの対象$x,y$とその間の可逆な射からなる圏とする. 
  \begin{align*}
    I(1) := \{x \leftrightarrow y\}
  \end{align*} 
\end{notation}

$I(1)$の脈体$\nerve(I(1))$を$E(1)$と表す.
$E(1)$は離散単体的空間なので, $E(1)_n$は集合$\{0,\cdots,n\}$から$\{x,y\}$への射の集まりとみなせる. 
よって, $E(1)_n$は$2^{n+1}$個の元を持つ.

% \begin{remark} \label{rem:E(1)}
%   $\nerve(I(1))$を$E(1)$と表す. 
%   $E(1)$は離散単体的空間なので, 任意の$n$に対して, $E(1)_n$は集合$\{0,\cdots,n\}$から$\{x,y\}$への射の集まりとみなせる. 
%   よって, $E(1)_n$は$2^{n+1}$個の元を持つ.
%   % 例えば, $E(1)_0$は$\{x,y\}$である. 
%   % また, $E(1)_1$は$\{\id_x,\id_y, x \to y, y \to x\}$と表せる.  
% \end{remark}

\begin{remark}
  $I(1)$と$[0]$は圏同値である. 
  しかし, $E(1)$は各次元で可縮ではない一方で$F(1)$は各次元で可縮である.
  よって, $E(1)$と$F(1)$は同型ではない.
\end{remark}

このように, 通常の圏論では見れない亜群におけるホモトピーの情報をSegal空間では見ることができる. 

\begin{remark}
  離散Segal空間$E(1)$には, 可逆な射$x \to y, y \to x$が存在する. 
  この射の間にホモトピー同値が存在するとき, $x$と$y$は同値である. 
  しかし, Kan複体$E(1)_0$における2つの対象の間には射が存在しないので同値ではない.
\end{remark}

Segal空間$X$において2つの対象がホモトピー同値のとき, Kan複体$X_0$においても同値である条件が必要である. 
このような圏論とホモトピー論の間に整合性を持たせる条件が完備性である. 

% 単体的空間の射$s_0 : X_0 \to X_1$は単射ではあるが, 全射ではない. 
% ($X_1$の任意の同型射が恒等射のみであるとき, $s_0$は全射である.)
% よって, 完備Segal空間は次のように定義される. 

\begin{definition}[完備Segal空間]
  単体的集合の射$s_0 : X_0 \to X_\hoequiv$
  % \begin{align*}
  %   s_0 : X_0 \to X_\hoequiv
  % \end{align*}
  がKan弱同値のとき, $X$は完備(complete)であるという. 
\end{definition}

\begin{example}
  Segal空間$E(1)$は完備でない.
\end{example}

\begin{proof}
  $E(1)$の定義より, 
  \begin{align*}
    E(1)_0 &= \{x,y\} \\
    E(1)_1 &= E(1)_\hoequiv= \{\id_x,\id_y,x \to y,y \to x\} 
  \end{align*}
  であるが, 
  \begin{align*}
    E(1)_0 \to E(1)_\hoequiv : \{x,y\} \to \{\id_x,\id_y,x \to y,y \to x\}
  \end{align*}
  はKan弱同値ではない. 
\end{proof}

\begin{remark}
  Segal空間$X$の射$x \to y$から包含$i : F(1) \hookrightarrow E(1)$が定まる. 
  この包含は単体的集合の射$\Map_{\sSpace}(E(1),X) \to \Map_{\sSpace}(F(1),X)$が定まる. 
\end{remark}

\begin{theorem} \label{prop.to_hoequiv_is_weq}
  $X$をSegal空間とする. 
  単体的集合の射$\Map_{\sSpace}(E(1),X) \to \Map_{\sSpace}(F(1),X)$は$X_\hoequiv$によって分解し, $\Map_{\sSpace}(E(1),X) \to X_\hoequiv$はKanファイブレーションである. 
\end{theorem}

\begin{theorem}
  Segal空間$X$に対して, 次は全て同値である. 
  \begin{enumerate}
    \item $X$は完備である. 
    \item 単体的集合の射$\Map_{\sSpace}(E(1),X) \to \Map_{\sSpace}(F(0),X)$はKan弱同値である. 
    \item 次の図式はホモトピープルバックである. % https://q.uiver.app/#q=WzAsNCxbMCwwLCJYXzAiXSxbMSwwLCJYXzMiXSxbMCwxLCJYXzEiXSxbMSwxLCJYXzEgXFx0aW1lc197WF8wfSBYXzEgXFx0aW1lc197WF8wfSBYXzEiXSxbMCwxXSxbMCwyXSxbMSwzXSxbMiwzXSxbMCwzLCIiLDEseyJzdHlsZSI6eyJuYW1lIjoiY29ybmVyIn19XV0=
    \[\begin{tikzcd}
      {X_0} & {X_3} \\
      {X_1} & {X_1 \times_{X_0} X_1 \times_{X_0} X_1}
      \arrow[from=1-1, to=1-2]
      \arrow[from=1-1, to=2-1]
      \arrow[from=1-2, to=2-2]
      \arrow[from=2-1, to=2-2]
      \arrow["\lrcorner"{anchor=center, pos=0.125}, draw=none, from=1-1, to=2-2]
    \end{tikzcd}\]
    \item $X$の任意の対象に対して, $\hoequiv_X(x,y)$は$\map_X(x,y)$とKan弱同値である. 
  \end{enumerate}
\end{theorem}
 
% \begin{remark}
%   完備Segal空間は次の条件を満たすような単体的空間である. 
%   \begin{description}
%     \item[(Reedyファイブラント条件)] 垂直成分はホモトピー論の性質をもつ. 
%     \item[(Segal条件)] 水平成分は圏論の性質をもつ. 
%     \item[(完備性)] 圏論とホモトピー論の性質は整合的である.  
%   \end{description}
% \end{remark}

% Segal空間亜群の完備性も定義できる. 

% \begin{definition}[完備Segal空間亜群]
%   $X$を完備Segal空間とする. 
%   $X$の任意の射がホモトピー同値のとき, $X$を完備Segal空間亜群(complete Segal space groupoid)という.
% \end{definition}

\begin{theorem} \label{thrm:NC_is_complete_equals_has_non_trivial_morphism}
  $\C$を通常の圏とする. 
  脈体$\nerve(\C)$が完備であることと, $\C$が恒等射以外の同型射を持たないことは同値である. 
\end{theorem}

\begin{proof}
  ($\Rightarrow$) : $\nerve(\C)$が完備Segal空間であるとする. 
  このとき, 単体的集合の射$\nerve(\C)_0 \to \nerve(\C)_\hoequiv$はKan弱同値である. 
  $\nerve(\C)_0$と$\nerve(\C)_\hoequiv$は離散的単体的集合なので, この射は集合の全単射である. 
  よって, 恒等射以外のホモトピー同値は存在しない. 
  \cref{eg:homotopy_equivalence_is_isomorphism_in_NC}より, $\nerve(\C)$におけるホモトピー同値は$\C$における同型射である.
  よって, $\C$は恒等射以外の同型射を持たない.

  ($\Leftarrow$) : $\C$が恒等射以外の同型射を持たないとき, $\nerve(\C)_\hoequiv = \Iso(\C)$である. 
  よって, $\nerve(\C)_0 \to \nerve(\C)_\hoequiv$は集合の全単射である. 
  つまり, 単体的集合の射としてKan弱同値である. 
  従って, 脈体$\nerve(\C)$である.
\end{proof}

\section{圏の分類図式}

\cref{thrm:NC_is_complete_equals_has_non_trivial_morphism}より, $\C$が非自明な同型射を持つとき, 通常の脈体はうまくいかない.
圏のホモトピー論を考慮するような脈体として, 圏の分類図式を定義する. 
分類図式は圏の可逆な射の情報を持っているような脈体である. 

\begin{notation}
  圏$\C$に対して, 圏$\C$に含まれる最大の部分亜群を$\Iso(\C)$と表す. 
\end{notation}

\begin{definition}[分類図式]
  圏$\C$に対して, 単体的空間$\N(\C)$を次のように定義し, $\C$の分類図式(classifying diagram)という. 
  \begin{align*}
    \N(\C)_n := \nerve(\Iso(\C^{[n]}))
  \end{align*}
\end{definition}

\begin{remark}
  $n=0$のとき, $\N(\C)_0 = \nerve(\Iso(\C))$より, $\N(\C)_0$は$\C$の最大の部分亜群の脈体である.   
\end{remark}

圏の分類図式が完備Segal空間であることを示す. 

\begin{lemma}
  圏$\C$に対して, $\N(\C)$はReedyファイブラントである.
\end{lemma}

\begin{lemma}
  圏$\C$に対して, $\N(\C)$はSegal空間である.
\end{lemma}

\begin{proof}
  プルバックは最大の部分亜群をとる操作と脈体をとる操作で保たれる. 
  \cref{eg:NC_is_Segal_space}より, $\nerve(\C)$はSegal空間である. 
  よって, 圏の分類図式$\N\C$はSegal空間である.
\end{proof}

\begin{theorem}
  圏$\C$に対して, $\N(\C)$は完備Segal空間である.
\end{theorem}

% \begin{proof}
%   完備性は
%   \begin{align*}
%     \N(\C)_0 
%     = \nerve(\Iso(\C)) 
%     = \N(\C)_\hoequiv
%   \end{align*}
%   から従う.  
% \end{proof}

\bibliographystyle{alpha}
\bibliography{../cf_kerodon}

\end{document}