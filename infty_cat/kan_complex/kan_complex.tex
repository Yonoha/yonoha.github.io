\RequirePackage{plautopatch}
\documentclass[uplatex, a4paper, 14Q, dvipdfmx]{jsreport}
\usepackage{docmute}
\usepackage{../new_mypackage}

\setcounter{secnumdepth}{4}
\title{Kan複体について}
\author{よの}
\date{\today}

\begin{document}

\maketitle

\begin{abstract}

  本稿は\cite{kerodon}のChapter3: Kan Complexesを和訳したものである. 
  行間は適宜補うようにするが, 和訳と特に見分けのつかないように書く. 
  Variantは定義や注意などに表記を変えている. 

  \cite{kerodon}のChapter1の和訳は\url{https://yonoha.github.io/infty_cat/kerodon/kerodon.pdf}を参照. 
  2章に関する例などは(筆者が勉強できていないので)詳しくは書いていない.

  特に断らない限り, $n$を$0$より大きい整数とする. 
\end{abstract}

\setcounter{tocdepth}{2}
\tableofcontents

\setcounter{chapter}{2}
\chapter{Kan複体について}

単体的集合$X$が任意$0 \leq i \leq n$に対して, $\sigma_0 : \Lambda^n_i \to X$が拡張$\sigma : \Delta^n \to X$を持つとき, $X$はKan複体と呼ばれていた.
% https://q.uiver.app/#q=WzAsMyxbMCwwLCJcXExhbWJkYV5uX2kiXSxbMSwwLCJYIl0sWzAsMSwiXFxEZWx0YV5uIl0sWzAsMSwiXFxzaWdtYV8wIl0sWzAsMiwiIiwyLHsic3R5bGUiOnsidGFpbCI6eyJuYW1lIjoiaG9vayIsInNpZGUiOiJ0b3AifX19XSxbMiwxLCJcXHNpZ21hIiwyLHsic3R5bGUiOnsiYm9keSI6eyJuYW1lIjoiZGFzaGVkIn19fV1d
\[\begin{tikzcd}
	{\Lambda^n_i} & X \\
	{\Delta^n}
	\arrow["{\sigma_0}", from=1-1, to=1-2]
	\arrow[hook, from=1-1, to=2-1]
	\arrow["\sigma"', dashed, from=2-1, to=1-2]
\end{tikzcd}\]

Kan複体は$\infty$圏論において, 次の3つの理由で重要である. 

\begin{itemize}
  \item 任意のKan複体は$\infty$圏である. (例1.3.0.3)
  逆に, 任意の$\infty$圏$\C$は$\C$に含まれる最大のKan複体$\C^\simeq$を持つ. (構成4.4.3.1)
  この$\C^\simeq$は$\C$の重要な不変量である.
  そして, Kan複体のホモトピー圏を理解することは, $\infty$圏のホモトピー圏を理解することへの第一歩である.
  \item $\C$を$\infty$圏とする. 
  任意の点$X,Y \in \C$に対して, Kan複体$\Hom_\C(X,Y)$を考えることができる. 
  この$\Hom_\C(X,Y)$は$X$から$Y$への射の空間と呼ばれる. (構成4.6.1.1)
  この射空間は$\C$の構造を理解するために不可欠である.
  $\infty$圏の関手$F : \C \to \D$がホモトピー逆射をもつことと, ホモトピー圏のレベルで本質的全射であって, 任意の$X,Y \in \C$に対して, $\Hom_\C(X,Y) \to \Hom_\C(FX,FY)$がホモトピー同値であることは同値である. (定理4.6.2.17)
  \item 執筆中
\end{itemize}

この章の目標はKan複体のホモトピー論を理解することである.
1節では, 単体的ホモトピー論の基礎について学ぶ.
特に, Kanファイブレーション(定義3.1.1.1)と緩射(定義3.1.2.1), 単体的集合の(弱)ホモトピー同値(定義3.1.6.1と定義3.1.6.12)を定義し, 基本的な性質を見る.

\newpage
\section{Kan複体のホモトピー論}

$X,Y$を単体的集合, $f_0,f_1 : X \to Y$を単体的集合の射とする. 
$f_0$から$f_1$へのホモトピー(homotopy)とは, $f_0 = h|_{\{0\} \times X}$と$f_1 = h|_{\{1\} \times X}$を満たす単体的集合の射$h : \Delta^1 \times X \to Y$である. (定義3.1.5.2)
しかし, このホモトピーは誤解を招く用語である. 
例えば, 一般の単体的集合に対して, $f_0$から$f_1$へのホモトピーが存在しても, $f_1$から$f_0$へのホモトピーが存在するとは限らない. 
しかし, $Y$がKan複体のときはこの状況は少し改善される. 
一般に, $X$から$Y$への単体的集合の射は単体的集合$\Fun(X,Y)$の点, ホモトピーは$\Fun(X,Y)$の辺とみなせる. 
1章3節で, $Y$がKan複体のとき, $\Fun(X,Y)$もKan複体であることを示す. (系3.1.3.4)

1章1節で, 単体的集合のKanファイブレーション(Kan fibration)を定義する. 
大雑把に言うと, Kanファイブレーション$f : X \to S$は$S$でパラメタづけられたKan複体の族とみなせる. 
特に, $f$がKanファイブレーションのとき, 各ファイバー$X_s := \{s\} \times_S X$はKan複体である. (注意3.1.1.9)
1章3節で, 系3.1.3.4がKan複体のべきに関するより一般的な命題(定理3.1.3.1)から従うことを見る. 
1章2節で, 緩射(anodyne morphism)におけるGabriel-Zisman計算を復習する. 

Kan複体の射$f : X \to Y$の$\hkan$における像が同型射のとき, $f$をホモトピー同値(homotopy equivalence)という. 
つまり, $f$がホモトピー逆射$g : Y \to X$を持つときである. 
これは一般の単体的集合に対して定義することができるが, 使い道が限られている. (定義3.1.6.1)
Kan複体でない単体的集合に対しては, 弱ホモトピー同値(weak homotopy equivalence)を考えた方がよい. (定義3.1.6.12)
これらの概念は1章6節で紹介する. 
1章7節では, 任意の単体的集合$X$に対して, あるKan複体$Q$と緩射$f : X \to Q$が存在することを見る. (系3.1.7.2)
これはQuillenの小対象論法(small object argument)の一例である. 


\subsection{Kanファイブレーション}

単体的集合$X$が次の条件を満たすとき, $X$はKan複体と呼ばれていた.
\begin{itemize}
  \item 任意$0 \leq i \leq n$に対して, $\sigma_0 : \Lambda^n_i \to X$が拡張$\sigma : \Delta^n \to X$を持つ. 
\end{itemize} 
% https://q.uiver.app/#q=WzAsMyxbMCwwLCJcXExhbWJkYV5uX2kiXSxbMSwwLCJYIl0sWzAsMSwiXFxEZWx0YV5uIl0sWzAsMSwiXFxzaWdtYV8wIl0sWzAsMiwiIiwyLHsic3R5bGUiOnsidGFpbCI6eyJuYW1lIjoiaG9vayIsInNpZGUiOiJ0b3AifX19XSxbMiwxLCJcXHNpZ21hIiwyLHsic3R5bGUiOnsiYm9keSI6eyJuYW1lIjoiZGFzaGVkIn19fV1d
\[\begin{tikzcd}
	{\Lambda^n_i} & X \\
	{\Delta^n}
	\arrow["{\sigma_0}", from=1-1, to=1-2]
	\arrow[hook, from=1-1, to=2-1]
	\arrow["\sigma"', dashed, from=2-1, to=1-2]
\end{tikzcd}\]
このKan複体の定義を単体的集合の射においても考えることは非常に有用である.2

\begin{definition}[Kanファイブレーション]
  $f : X \to S$を単体的集合の射とする.
  任意の$0 \leq i \leq n$に対して, 次のリフト問題が解決(図中のドットで描かれる矢印)をもつとき, $f$をKanファイブレーション(Kan fibration)という. 
  % https://q.uiver.app/#q=WzAsNCxbMCwwLCJcXExhbWJkYV5uX2kiXSxbMSwwLCJYIl0sWzAsMSwiXFxEZWx0YV5uIl0sWzEsMSwiUyJdLFswLDEsIlxcc2lnbWFfMCJdLFswLDIsIiIsMix7InN0eWxlIjp7InRhaWwiOnsibmFtZSI6Imhvb2siLCJzaWRlIjoidG9wIn19fV0sWzIsMSwiXFxzaWdtYSIsMix7InN0eWxlIjp7ImJvZHkiOnsibmFtZSI6ImRhc2hlZCJ9fX1dLFsyLDMsIlxcYmFye1xcc2lnbWF9IiwyXSxbMSwzLCJmIl1d
  \[\begin{tikzcd}
    {\Lambda^n_i} & X \\
    {\Delta^n} & S
    \arrow["{\sigma_0}", from=1-1, to=1-2]
    \arrow[hook, from=1-1, to=2-1]
    \arrow["\sigma"', dashed, from=2-1, to=1-2]
    \arrow["{\bar{\sigma}}"', from=2-1, to=2-2]
    \arrow["f", from=1-2, to=2-2]
  \end{tikzcd}\]
  つまり, 任意の単体的集合の射$\sigma_0 : \Lambda^n_i \to X$と$\bar{\sigma} : \Delta^n \to S$に対して, $\sigma_0$を$f \circ \sigma = \bar{\sigma}$を満たす$n$単体$\sigma : \Delta^n \to X$に拡張できるときである. 
\end{definition}

\begin{example}
  $X$を単体的集合とする.
  射影$X \to \Delta^0$がKanファイブレーションであることと, $X$がKan複体であることは同値である. 
\end{example}

\begin{proof}
  射影$X \to \Delta^0$がKanファイブレーションであるとき, 次の図式は可換である.
  \[\begin{tikzcd}
    {\Lambda^n_i} & X \\
    {\Delta^n} & {\Delta^0}
    \arrow["{\sigma_0}", from=1-1, to=1-2]
    \arrow[hook, from=1-1, to=2-1]
    \arrow["\sigma"', dashed, from=2-1, to=1-2]
    \arrow["{\bar{\sigma}}"', from=2-1, to=2-2]
    \arrow["f", from=1-2, to=2-2]
  \end{tikzcd}\]
  このとき, 左上の三角はKan複体の定義そのものである. \\
  逆に, $X$がKan複体であるとき, 次の図式は可換である. 
  \[\begin{tikzcd}
    {\Lambda^n_i} & X \\
    {\Delta^n}
    \arrow["{\sigma_0}", from=1-1, to=1-2]
    \arrow[hook, from=1-1, to=2-1]
    \arrow["\sigma"', dashed, from=2-1, to=1-2]
  \end{tikzcd}\]
  $\Delta^0$は$\sset$における終対象なので, 次の図式は可換である. 
  \[\begin{tikzcd}
    {\Lambda^n_i} & X \\
    {\Delta^n} & {\Delta^0}
    \arrow["{\sigma_0}", from=1-1, to=1-2]
    \arrow[hook, from=1-1, to=2-1]
    \arrow["\sigma"', dashed, from=2-1, to=1-2]
    \arrow["{\bar{\sigma}}"', from=2-1, to=2-2]
    \arrow["f", from=1-2, to=2-2]
  \end{tikzcd}\]
\end{proof}

Kanファイブレーションは包含$\Lambda^n_i \hookrightarrow \Delta^n$に対してRLPを持つ射であるので, RLPに関する一般論からいくつかの命題(例3.1.1.3, 注意3.1.1.5, 注意3.1.1.6, 注意3.1.1.8, 注意3.1.1.10)を示すことができる. 
詳細は\cite[\href{https://kerodon.net/tag/006B}{Tag 006B}]{kerodon}を参照. 

\begin{example}
  任意の単体的集合の同型射はKanファイブレーションである. 
\end{example}

\begin{example}
  $q : X \to S$を単体的集合の射とする. 
  $q$が$X$と$S$の直和因子の同型を定めるとき, $q$はKanファイブレーションである. 
\end{example}

\begin{remark}
  Kanファイブレーションの集まりはレトラクトで閉じる.
\end{remark}

\begin{remark}
  Kanファイブレーションの集まりはpullbackで閉じる.
\end{remark}

注意3.1.1.6の逆は次のような形で成立する. 

\begin{remark}
  $f : X \to S$を単体的集合の射とする.
  任意の$n$単体$\Delta^n \to S$に対して, 射影$\Delta^n \times_S X \to \Delta^n$はKanファイブレーションであるとする.
  このとき, $f$はKanファイブレーションである. 
  更に, 次の図式がpullbackであり, $g$がepi射で$f'$がKanファイブレーションであるとき, $f$もKanファイブレーションである. 
  % https://q.uiver.app/#q=WzAsNCxbMCwwLCJYJyJdLFsxLDAsIlgiXSxbMSwxLCJTIl0sWzAsMSwiUyciXSxbMCwxXSxbMSwyLCJmIl0sWzAsMywiZiciLDJdLFszLDIsImciLDJdXQ==
  \[\begin{tikzcd}
    {X'} & X \\
    {S'} & S
    \arrow[from=1-1, to=1-2]
    \arrow["f", from=1-2, to=2-2]
    \arrow["{f'}"', from=1-1, to=2-1]
    \arrow["g"', from=2-1, to=2-2]
  \end{tikzcd}\]
\end{remark}

\begin{proof}
  前半を示す. 
  射影$\Delta^n \times_S X \to \Delta^n$はKanファイブレーションなので, 次の図式が存在する.
  % https://q.uiver.app/#q=WzAsNixbMCwwLCJcXExhbWJkYV5uX2kiXSxbMSwwLCJcXERlbHRhXm4gXFx0aW1lc19TIFgiXSxbMCwxLCJcXERlbHRhXm4iXSxbMSwxLCJcXERlbHRhXm4iXSxbMiwwLCJYIl0sWzIsMSwiUyJdLFswLDEsIlxcc2lnbWFfMCJdLFswLDIsIiIsMix7InN0eWxlIjp7InRhaWwiOnsibmFtZSI6Imhvb2siLCJzaWRlIjoidG9wIn19fV0sWzIsMSwiXFxzaWdtYSIsMix7InN0eWxlIjp7ImJvZHkiOnsibmFtZSI6ImRhc2hlZCJ9fX1dLFsyLDMsIlxcaWQiLDJdLFsxLDMsInBfMiJdLFsxLDQsInBfMSJdLFs0LDUsImYiXSxbMyw1LCJcXGJhcntcXHNpZ21hfSIsMl1d
  \[\begin{tikzcd}
    {\Lambda^n_i} & {\Delta^n \times_S X} & X \\
    {\Delta^n} & {\Delta^n} & S
    \arrow["{\sigma_0}", from=1-1, to=1-2]
    \arrow[hook, from=1-1, to=2-1]
    \arrow["\sigma"', dashed, from=2-1, to=1-2]
    \arrow["\id"', from=2-1, to=2-2]
    \arrow["{p_2}", from=1-2, to=2-2]
    \arrow["{p_1}", from=1-2, to=1-3]
    \arrow["f", from=1-3, to=2-3]
    \arrow["{\bar{\sigma}}"', from=2-2, to=2-3]
  \end{tikzcd}\]
  ここで, 右側の四角はpullbackである. 
  pullbackの普遍性より, $\sigma' := p_1 \circ \sigma : \Delta^n \to X$は$f$のリフトである. 

  後半を示す.
  前半の主張より, 任意の$n$に対して, 射影$\Delta^n \times_S X \to \Delta^n$がKanファイブレーションであることを示せばよい. 
  $g$はepi射なので, 任意の$n$に対して, $g_n : S'_n \to S_n$は全射である.
  よって, $\sigma' : \Delta^n \to S'$が存在して, 次の図式は可換である. 
  % https://q.uiver.app/#q=WzAsMyxbMCwwLCJcXERlbHRhXm4iXSxbMSwwLCJTJyJdLFsxLDEsIlMiXSxbMCwxLCJcXHNpZ21hJyJdLFswLDIsIlxcc2lnbWEiLDJdLFsxLDIsImciXV0=
  \[\begin{tikzcd}
    {\Delta^n} & {S'} \\
    & S
    \arrow["{\sigma'}", from=1-1, to=1-2]
    \arrow["\sigma"', from=1-1, to=2-2]
    \arrow["g", from=1-2, to=2-2]
  \end{tikzcd}\] 
  このとき, 次の図式において, 左と右の四角はpullbackである. 
  % https://q.uiver.app/#q=WzAsNixbMCwxLCJcXERlbHRhXm4iXSxbMSwwLCJYJyJdLFsxLDEsIlMnIl0sWzAsMCwiXFxEZWx0YV5uIFxcdGltZXNfe1MnfSBYJyJdLFsyLDAsIlgiXSxbMiwxLCJTIl0sWzEsMiwiZiciXSxbMCwyLCJcXHNpZ21hJyJdLFszLDBdLFszLDFdLFsxLDRdLFsyLDUsImciXSxbNCw1LCJmIl0sWzAsNSwiXFxzaWdtYSIsMix7ImN1cnZlIjoyfV1d
  \[\begin{tikzcd}
    {\Delta^n \times_{S'} X'} & {X'} & X \\
    {\Delta^n} & {S'} & S
    \arrow["{f'}", from=1-2, to=2-2]
    \arrow["{\sigma'}", from=2-1, to=2-2]
    \arrow[from=1-1, to=2-1]
    \arrow[from=1-1, to=1-2]
    \arrow[from=1-2, to=1-3]
    \arrow["g", from=2-2, to=2-3]
    \arrow["f", from=1-3, to=2-3]
    \arrow["\sigma"', curve={height=12pt}, from=2-1, to=2-3]
  \end{tikzcd}\]
  よって, 外の四角もpullbackである. 
  つまり, 同型$\Delta^n \times_S X \cong \Delta^n \times_{S'} X'$が成立する. 
  よって, $\Delta^n \times_S X \to \Delta^n$はKanファイブレーションである. 
\end{proof}

\begin{remark}
  Kanファイブレーションの集まりはフィルター付き余極限で閉じる
\end{remark}

\begin{remark}
  $f : X \to S$をKanファイブレーションとする. 
  任意の点$s \in S$に対して, pullback $\{s\} \times_S X$はKan複体である.
\end{remark}

\begin{proof}
  次の図式は可換である. 
  % https://q.uiver.app/#q=WzAsNCxbMCwwLCJcXHtzXFx9IFxcdGltZXNfUyBYIl0sWzEsMCwiWCJdLFsxLDEsIlMiXSxbMCwxLCJcXHtzXFx9Il0sWzAsMV0sWzEsMiwiZiJdLFszLDJdLFswLDNdLFswLDIsIiIsMSx7InN0eWxlIjp7Im5hbWUiOiJjb3JuZXIifX1dXQ==
  \[\begin{tikzcd}
    {\{s\} \times_S X} & X \\
    {\{s\}} & S
    \arrow[from=1-1, to=1-2]
    \arrow["f", from=1-2, to=2-2]
    \arrow[from=2-1, to=2-2]
    \arrow[from=1-1, to=2-1]
  \end{tikzcd}\]
  $\{s\} \cong \Delta^0$であるので, 注意3.1.1.6と例3.1.1.2より従う. 
\end{proof}

\begin{remark}
  Kanファイブレーションの集まりは合成で閉じる.
\end{remark}

\begin{remark}
  $f : X \to Y$をKanファイブレーションとする. 
  $Y$がKan複体であるとき, $X$もKan複体である. 
\end{remark}

\begin{proof}
  例3.1.1.2より, $Y$がKan複体であるとき, $g : Y \to \Delta^0$はKanファイブレーションである. 
  注意3.1.1.10より, $fg : X \to \Delta^0$もKanファイブレーションである. 
  例3.1.1.2より, $X$はKan複体である. 
\end{proof}

\subsection{緩射}

Kanファイブレーションは包含$\Lambda^n_i \hookrightarrow \Delta^n$に対してRLPを持つ射であるが, より広い射のクラスに対してRLPを持つことが分かる.

\begin{definition}[緩射]
  $T$を次の条件を満たす$\sset$の最小の射の集まりとする. 
  \begin{itemize}
    \item 任意の$0 \leq i \leq n$に対して, 包含$\Lambda^n_i \hookrightarrow \Delta^n$は$T$に属する.
    \item $T$は弱飽和である. 
  \end{itemize}
  $T$に属する単体的集合の射を緩射(anodyne morphism)という. 
\end{definition}

\begin{remark}
  緩射の集まりは\cite{GZ}により導入された. 
\end{remark}

\begin{remark}
  任意の緩射は単体的集合のmono射である. 
\end{remark}

\begin{proof}
  任意の$0 \leq i \leq n$に対して, 包含$\Lambda^n_i \hookrightarrow \Delta^n$は単体的集合のmono射である.
  命題1.4.5.13より, 単体的集合のmono射の集まりは弱飽和である. 
  $T$の最小性より, 任意の緩射は単体的集合のmono射である.
\end{proof}

\begin{example}
  任意の内緩射は緩射である. 
  逆は一般には成立しない.
\end{example}

\begin{proof}
  前半は内緩射の定義からすぐに分かる. 
  逆は, 包含$\Lambda^n_0 \hookrightarrow \Delta^n$と$\Lambda^n_n \hookrightarrow \Delta^n$は緩射ではあるが, 内緩射ではないことから分かる. 
\end{proof}

\begin{example}
  任意の$0 \leq i \leq n$に対して, 包含$\{i\} \to \Delta^n$は緩射である. 
\end{example}

\begin{proof}
  $n$単体の椎を$\spine[n]$と表すと, 例1.4.7.7より, 包含$\spine[n] \to \Delta^n$は内緩射である. 
  よって, 例3.1.2.7より, 包含$\{i\} \to \spine[n]$が緩射であることを示せばよいがこれは明らかである(らしい). 
\end{proof}

\begin{remark}
  緩射の集まりは弱飽和なので, 特に次が成立する. 
  \begin{itemize}
    \item 緩射の集まりは同型射を含む.
    \item 緩射の集まりは合成で閉じる. 
    \item 緩射の集まりはpushoutで閉じる. 
    \item 緩射の集まりはレトラクトで閉じる. 
  \end{itemize}
\end{remark}

次の命題は, 単体的集合の射がKanファイブレーションであるか確かめるときに有用である.
これは自明なKanファイブレーションであるか確かめる命題1.4.5.4の類似である.  

\begin{remark}
  $f : X \to S$を単体的集合の射とする.
  このとき, 次の2つは同値である.
  \begin{enumerate}
    \item $f$はKanファイブレーションである.
    \item 任意の緩射$i : A \to B$に対して, 次のリフト問題は解決(図中のドットで描かれる矢印)を持つ. 
    \[\begin{tikzcd}
      A & X \\
      B & S
      \arrow[from=1-1, to=1-2]
      \arrow["f", from=1-2, to=2-2]
      \arrow[from=2-1, to=2-2]
      \arrow["i"', from=1-1, to=2-1]
      \arrow[dashed, from=2-1, to=1-2]
    \end{tikzcd}\] 
  \end{enumerate}
\end{remark}

\begin{proof}
  (2)から(1)を示す. 
  包含$\Lambda^n_i \hookrightarrow \Delta^n$は緩射なので, $i : \Lambda^n_i \hookrightarrow \Delta^n$とすると, Kanファイブレーションの定義に一致する. 

  (1)から(2)を示す.
  命題1.4.4.16より, $f$に対してLLPを持つ射の集まりは弱飽和である. 
  $T$の最小性より, (2)は従う. 
\end{proof}

次の命題は, 緩射の集まりに対する安定性を表している.

\begin{proposition}
  $f : A \hookrightarrow B$と$f' : A' \hookrightarrow B'$を単体的集合のmono射とする.
  $f$か$f'$が緩射であるとき, 誘導される射
  \begin{align*}
    (A \times B') \coprod_{A \times A'} (B \times A') \hookrightarrow B \times B'
  \end{align*}
  は緩射である.
\end{proposition}

この命題の証明は, この節の最後に与える. 

\begin{lemma}
  任意の$0 < i \leq n$に対して, 包含$\Lambda^n_i \hookrightarrow \Delta^n$は射
  \begin{align*}
    (\Delta^1 \times \Lambda^n_i) \coprod_{\{1\} \times \Lambda^n_i} (\{1\} \times \Delta^n) \hookrightarrow \Delta^1 \times \Delta^n
  \end{align*}
  のレトラクトである.
\end{lemma}

\begin{proof}
  ある射$f$が存在して, 次の図式が可換になることを示せばよい. (らしい)
  % https://q.uiver.app/#q=WzAsNixbMCwwLCJcXHswXFx9IFxcdGltZXMgXFxMYW1iZGFebl9pIl0sWzEsMCwiKFxcRGVsdGFebiBcXHRpbWVzIFxcTGFtYmRhXm5faSkgXFxhbWFsZyAoXFx7MVxcfSBcXHRpbWVzIFxcRGVsdGFebikiXSxbMiwwLCJcXExhbWJkYV5uX2kiXSxbMiwxLCJcXERlbHRhXm4iXSxbMSwxLCJcXERlbHRhXjEgXFx0aW1lcyBcXERlbHRhXm4iXSxbMCwxLCJcXHswXFx9IFxcdGltZXMgXFxEZWx0YV5uIl0sWzAsMV0sWzEsMl0sWzIsMywiZl8wIl0sWzEsNCwiZiIsMCx7InN0eWxlIjp7ImJvZHkiOnsibmFtZSI6ImRhc2hlZCJ9fX1dLFswLDUsImZfMCJdLFs1LDRdLFs0LDNdLFswLDIsIlxcaWQiLDAseyJjdXJ2ZSI6LTN9XSxbNSwzLCJcXGlkIiwyLHsiY3VydmUiOjN9XV0=
  \[\begin{tikzcd}
    {\{0\} \times \Lambda^n_i} & {(\Delta^n \times \Lambda^n_i) \amalg (\{1\} \times \Delta^n)} & {\Lambda^n_i} \\
    {\{0\} \times \Delta^n} & {\Delta^1 \times \Delta^n} & {\Delta^n}
    \arrow[from=1-1, to=1-2]
    \arrow[from=1-2, to=1-3]
    \arrow["{f_0}", from=1-3, to=2-3]
    \arrow["f", dashed, from=1-2, to=2-2]
    \arrow["{f_0}", from=1-1, to=2-1]
    \arrow[from=2-1, to=2-2]
    \arrow["r", from=2-2, to=2-3]
    \arrow["\id", curve={height=-18pt}, from=1-1, to=1-3]
    \arrow["\id"', curve={height=18pt}, from=2-1, to=2-3]
  \end{tikzcd}\]
\end{proof}

\begin{lemma}
  $n$を$0$以上の整数とする.
  このとき, 次の条件を満たす単体的部分集合の列が存在する. 
  \begin{align*}
    X(0) \subset X(1) \subset \cdots \subset X(n) \subset X(n+1) = \Delta^1 \times \Delta^n
  \end{align*}
  \begin{enumerate}
    \item $X(0)$は$\Delta^1 \times \partial \Delta^n$と$\{1\} \times \Delta^n$の直和
    \item 任意の$0 \leq i \leq n$に対して, 包含$X(i) \hookrightarrow X(i+1)$は次のpushoutで与えられる. 
    % https://q.uiver.app/#q=WzAsNCxbMCwwLCJcXExhbWJkYV57bisxfV97aSsxfSJdLFsxLDAsIlgoaSkiXSxbMSwxLCJYKGkrMSkiXSxbMCwxLCJcXERlbHRhXntuKzF9Il0sWzAsMV0sWzEsMl0sWzAsM10sWzMsMl1d
    \[\begin{tikzcd}
      {\Lambda^{n+1}_{i+1}} & {X(i)} \\
      {\Delta^{n+1}} & {X(i+1)}
      \arrow[from=1-1, to=1-2]
      \arrow[from=1-2, to=2-2]
      \arrow[from=1-1, to=2-1]
      \arrow[from=2-1, to=2-2]
    \end{tikzcd}\]
  \end{enumerate}
\end{lemma}

\begin{proof}
  任意の$0 \leq i \leq n$に対して, $\sigma_i : \Delta^{n+1} \to \Delta^1 \times \Delta^n$の点の対応を次のように定義する.
  任意の$0 \leq j \leq n+1$に対して, 
  \begin{align*}
    \sigma_i(j) := 
    \begin{cases}
      (0,j) & (j \leq i) \\
      (1,j-1) & (j > i)
    \end{cases}
  \end{align*}
  このとき, $\Delta^1 \times \Delta^n$の単体的部分集合$X(i)$を次のように定義する.
  \begin{align*}
    X(0) &:= (\Delta^1 \times \partial \Delta^n) \amalg (\{1\} \times \Delta^n) \\
    X(i+1) &:= X(i) \amalg \myim(\sigma_i)
  \end{align*}
  ここで, $\Delta^1 \times \Delta^n$は$\{\myim(\sigma_i)\}_{0 \leq i \leq n}$と同一視でき, $X(n+1)$に等しいことが分かる.
  このとき, 条件(1)と(2)を満たすことをみる.
  まず, (1)は$X(0)$の定義から従うことが分かる.
  (2)は任意の$0 \leq i \leq n$に対して, $\sigma_i^{-1}(X(i))$が$\Delta^{n+1}_{i+1}$に等しいことを見ればよい.

\end{proof}

\begin{proof}{命題 3.1.2.8}
  $f' : A' \hookrightarrow B'$を単体的集合のmono射, $T$を任意の射$f : A \to B$から誘導される射
  \begin{align*}
    (\Delta^1 \times \Lambda^n_i) \coprod_{\{1\} \times \Lambda^n_i} (\{1\} \times \Delta^n) \hookrightarrow \Delta^1 \times \Delta^n
  \end{align*}
  が緩射となるような射$f$の集まりを$T$とする.
  このとき, 任意の緩射が$T$に属すことを示せばよい.
  注意3.1.2.6より, $T$は弱飽和である.
  よって, 任意の$n>0$に対して, 包含$\Lambda^n_i \hookrightarrow \Delta^n$が$T$に属すことを示せばよい.
  補題3.1.2.9より, $f$は 
  \begin{align*}
    g : (\Delta^1 \times \Lambda^n_i) \coprod_{\{1\} \times \Lambda^n_i} (\{1\} \times \Delta^n) \hookrightarrow \Delta^1 \times \Delta^n
  \end{align*}
  のレトラクトである.
  $T$は弱飽和なので, $g$が$T$に属することを示せばよい.
  (執筆中)
\end{proof}


\subsection{Kanファイブレーションのべき乗}

$B$と$X$を単体的集合とする.
$X$が$\infty$圏であるとき, 単体的集合$\Fun(B,X)$は$\infty$圏である. (定理1.4.3.7)
更に, $X$がKan複体であるとき, 単体的集合$\Fun(B,X)$はKan複体であることが言える. (系3.1.3.4)
これは次の命題の系である. 

\begin{theorem}
  $f : X \to S$をKanファイブレーション, $i : A \hookrightarrow B$を単体的集合のmono射とする.
  このとき, 誘導される射
  \begin{align*}
    \Fun(B,X) \to \Fun(B,S) \times_{\Fun(A,S)} \Fun(A,X)
  \end{align*}
  はKanファイブレーションである.
\end{theorem}

\begin{proof}
  注意3.1.2.7より, 任意の緩射$i' : A' \hookrightarrow B'$に対して, 次のリフト問題が解決を持つことを示せばよい. 
  % https://q.uiver.app/#q=WzAsNCxbMCwwLCJBJyJdLFsxLDAsIlxcZnVuKEIsWCkiXSxbMSwxLCJcXGZ1bihCLFMpIFxcdGltZXNfe1xcZnVuKEEsUyl9IFxcZnVuKEEsWCkiXSxbMCwxLCJCJyJdLFswLDFdLFsxLDIsImYiXSxbMywyXSxbMCwzLCJpJyIsMl0sWzMsMSwiIiwxLHsic3R5bGUiOnsiYm9keSI6eyJuYW1lIjoiZGFzaGVkIn19fV1d
  \[\begin{tikzcd}
    {A'} & {\Fun(B,X)} \\
    {B'} & {\Fun(B,S) \times_{\Fun(A,S)} \Fun(A,X)}
    \arrow[from=1-1, to=1-2]
    \arrow[from=1-2, to=2-2]
    \arrow[from=2-1, to=2-2]
    \arrow["{i'}"', from=1-1, to=2-1]
    \arrow[dashed, from=2-1, to=1-2]
  \end{tikzcd}\]
  これは次の図式のリフト問題の解決と同値である. 
  % https://q.uiver.app/#q=WzAsNCxbMCwwLCIoQSBcXHRpbWVzIEInKSBcXGNvcHJvZF97QSBcXHRpbWVzIEEnfSAoQiBcXHRpbWVzIEEnKSJdLFsxLDAsIlgiXSxbMSwxLCJTIl0sWzAsMSwiQiBcXHRpbWVzIEInIl0sWzAsMV0sWzEsMiwiZiJdLFswLDNdLFszLDJdLFszLDEsIiIsMCx7InN0eWxlIjp7ImJvZHkiOnsibmFtZSI6ImRhc2hlZCJ9fX1dXQ==
  \[\begin{tikzcd}
    {(A \times B') \coprod_{A \times A'} (B \times A')} & X \\
    {B \times B'} & S
    \arrow[from=1-1, to=1-2]
    \arrow["f", from=1-2, to=2-2]
    \arrow[from=1-1, to=2-1]
    \arrow[from=2-1, to=2-2]
    \arrow[dashed, from=2-1, to=1-2]
  \end{tikzcd}\]
  命題3.1.2.8より, $(A \times B') \coprod_{A \times A'} (B \times A') \hookrightarrow B \times B'$は緩射である. 
  注意3.1.2.7より, 求める射はKanファイブレーションである.
\end{proof}

\begin{corollary}
  $f : X \to S$をKanファイブレーションとする.
  このとき, 任意の単体的集合$B$に対して, $f$の誘導する射$\Fun(B,X) \to \Fun(B,S)$はKanファイブレーションである.
\end{corollary}

\begin{proof}
  定理3.1.3.1において, $A = \emptyset$とすればよい.
\end{proof}

\begin{corollary}
  $i : A \hookrightarrow B$を単体的集合のmono射, $X$をKan複体とする.
  このとき, 制限$\Fun(B,X) \to \Fun(A,X)$はKanファイブレーションである.
\end{corollary}

\begin{proof}
  定理3.1.3.1において, $S = \Delta^0$とすればよい.
\end{proof}

\begin{corollary}
  $X$をKan複体, $B$を単体的集合とする.
  このとき, $\Fun(B,X)$はKan複体である.
\end{corollary}

\begin{proof}
  定理3.1.3.1において, $S = \Delta^0$とすると, $\Fun(B,X) \to \Delta^0$はKanファイブレーションである. 
  例3.1.1.2より, $\Fun(B,X)$はKan複体である.
\end{proof}

自明なKanファイブレーションに対して定理3.1.3.1と同様の結果がある. 
定理3.1.3.1では$i$はmono射であったが, 定理3.1.3.5では緩射である. 

\begin{theorem}
  $f : X \to S$をKanファイブレーション, $i : A \hookrightarrow B$を緩射とする.
  このとき, 誘導される射
  \begin{align*}
    \Fun(B,X) \to \Fun(B,S) \times_{\Fun(A,S)} \Fun(A,X)
  \end{align*}
  は自明なKanファイブレーションである.
\end{theorem}

\begin{proof}
  定理3.1.3.1と同様に証明できる.
  命題1.4.5.4より, 任意のmono射$i' : A' \hookrightarrow B'$に対して, 次のリフト問題が解決を持つことを示せばよい. 
  % https://q.uiver.app/#q=WzAsNCxbMCwwLCJBJyJdLFsxLDAsIlxcZnVuKEIsWCkiXSxbMSwxLCJcXGZ1bihCLFMpIFxcdGltZXNfe1xcZnVuKEEsUyl9IFxcZnVuKEEsWCkiXSxbMCwxLCJCJyJdLFswLDFdLFsxLDIsImYiXSxbMywyXSxbMCwzLCJpJyIsMl0sWzMsMSwiIiwxLHsic3R5bGUiOnsiYm9keSI6eyJuYW1lIjoiZGFzaGVkIn19fV1d
  \[\begin{tikzcd}
    {A'} & {\Fun(B,X)} \\
    {B'} & {\Fun(B,S) \times_{\Fun(A,S)} \Fun(A,X)}
    \arrow[from=1-1, to=1-2]
    \arrow[from=1-2, to=2-2]
    \arrow[from=2-1, to=2-2]
    \arrow["{i'}"', from=1-1, to=2-1]
    \arrow[dashed, from=2-1, to=1-2]
  \end{tikzcd}\]
  これは次の図式のリフト問題の解決と同値である. 
  % https://q.uiver.app/#q=WzAsNCxbMCwwLCIoQSBcXHRpbWVzIEInKSBcXGNvcHJvZF97QSBcXHRpbWVzIEEnfSAoQiBcXHRpbWVzIEEnKSJdLFsxLDAsIlgiXSxbMSwxLCJTIl0sWzAsMSwiQiBcXHRpbWVzIEInIl0sWzAsMV0sWzEsMiwiZiJdLFswLDNdLFszLDJdLFszLDEsIiIsMCx7InN0eWxlIjp7ImJvZHkiOnsibmFtZSI6ImRhc2hlZCJ9fX1dXQ==
  \[\begin{tikzcd}
    {(A \times B') \coprod_{A \times A'} (B \times A')} & X \\
    {B \times B'} & S
    \arrow[from=1-1, to=1-2]
    \arrow["f", from=1-2, to=2-2]
    \arrow[from=1-1, to=2-1]
    \arrow[from=2-1, to=2-2]
    \arrow[dashed, from=2-1, to=1-2]
  \end{tikzcd}\]
  命題3.1.2.8より, $(A \times B') \coprod_{A \times A'} (B \times A') \hookrightarrow B \times B'$は緩射である. 
  例3.1.2.3より, この射はmono射である. 
  命題1.4.5.4より, 求める射はKanファイブレーションである.
\end{proof}

\begin{corollary}
  $i : A \hookrightarrow B$を緩射, $X$をKan複体とする.
  このとき, 制限$\Fun(B,X) \to \Fun(A,X)$は自明なKanファイブレーションである. 
\end{corollary}

\begin{proof}
  系3.1.3.3と同様
\end{proof}

\begin{construction}
  $B,X$を単体的集合, $\Fun(B,X)$を$B$から$X$への関手単体的集合とする. (定義1.4.3.1)
  \begin{itemize}
    \item $A$を単体的集合, $i : A \to B, f : A \to X$を単体的集合の射とする. 
    このとき, 点$f \in \Fun(A,X)$上の$i$の前合成$\Fun(B,X) \to \Fun(A,X)$のファイバーを$\Fun_{A /}(B,X) \subset \Fun(B,X)$と表す. 
    % https://q.uiver.app/#q=WzAsNCxbMCwwLCJcXEZ1bl97QSAvfShCLFgpIl0sWzEsMCwiXFxGdW4oQixYKSJdLFswLDEsIlxce2ZcXH0iXSxbMSwxLCJcXEZ1bihBLFgpIl0sWzAsMV0sWzAsMl0sWzIsMywiIiwyLHsic3R5bGUiOnsidGFpbCI6eyJuYW1lIjoiaG9vayIsInNpZGUiOiJ0b3AifX19XSxbMSwzLCItIFxcY2lyYyBpIl0sWzAsMywiIiwxLHsic3R5bGUiOnsibmFtZSI6ImNvcm5lciJ9fV1d
    \[\begin{tikzcd}
      {\Fun_{A /}(B,X)} & {\Fun(B,X)} \\
      {\{f\}} & {\Fun(A,X)}
      \arrow[from=1-1, to=1-2]
      \arrow[from=1-1, to=2-1]
      \arrow[hook, from=2-1, to=2-2]
      \arrow["{- \circ i}", from=1-2, to=2-2]
    \end{tikzcd}\]
    \item $S$を単体的集合, $g : B \to S, q : X \to S$を単体的集合の射とする. 
    このとき, 点$g \in \Fun(B,S)$上の$q$の後合成$\Fun(B,X) \to \Fun(B,S)$のファイバーを$\Fun_{/ S}(B,X) \subset \Fun(B,X)$と表す. 
    % https://q.uiver.app/#q=WzAsNCxbMCwwLCJcXEZ1bl97LyBTfShCLFgpIl0sWzEsMCwiXFxGdW4oQixYKSJdLFswLDEsIlxce3FcXH0iXSxbMSwxLCJcXEZ1bihCLFMpIl0sWzAsMV0sWzAsMl0sWzIsMywiIiwyLHsic3R5bGUiOnsidGFpbCI6eyJuYW1lIjoiaG9vayIsInNpZGUiOiJ0b3AifX19XSxbMSwzLCJxIFxcY2lyYyAtIl0sWzAsMywiIiwxLHsic3R5bGUiOnsibmFtZSI6ImNvcm5lciJ9fV1d
    \[\begin{tikzcd}
      {\Fun_{/ S}(B,X)} & {\Fun(B,X)} \\
      {\{q\}} & {\Fun(B,S)}
      \arrow[from=1-1, to=1-2]
      \arrow[from=1-1, to=2-1]
      \arrow[hook, from=2-1, to=2-2]
      \arrow["{q \circ -}", from=1-2, to=2-2]
    \end{tikzcd}\]
    \item 次の可換図式
    % https://q.uiver.app/#q=WzAsNCxbMCwwLCJBIl0sWzEsMCwiWCJdLFsxLDEsIlMiXSxbMCwxLCJCIl0sWzAsMSwiZiJdLFsxLDIsInEiXSxbMCwzLCJpIiwyXSxbMywyLCJnIiwyXV0=
    \[\begin{tikzcd}
      A & X \\
      B & S
      \arrow["f", from=1-1, to=1-2]
      \arrow["q", from=1-2, to=2-2]
      \arrow["i"', from=1-1, to=2-1]
      \arrow["g"', from=2-1, to=2-2]
    \end{tikzcd}\]   
    が存在するとき, 共通部分$\Fun_{A /}(B,X) \cap \Fun_{/ S}(B,X)$を$\Fun_{A/ / S}(B,X) \subset \Fun(B,X)$と表す.  
  \end{itemize}
\end{construction}

\begin{remark}
  $B,X$を単体的集合とし, $\Fun(B,X)$の点$\Delta^0 \times B \to X$を単体的集合の射$\bar{f} : B \to X$と同一視する.
  このとき, 次の3つが成立する. 
  \begin{itemize}
    \item $A$を単体的集合, $i : A \to B, f : A \to X$を単体的集合の射とする. 
    単体的集合$\Fun_{A /}(B,X)$の点は射$g = \bar{f} \circ i$を満たす射$\bar{f}$と同一視できる. 
    \item $S$を単体的集合, $g : B \to S, q : X \to S$を単体的集合の射とする. 
    単体的集合$\Fun_{/ S}(B,X)$の点は射$g = q \circ \bar{f}$を満たす射$\bar{f}$と同一視できる.
    \item 次の可換図式
    % https://q.uiver.app/#q=WzAsNCxbMCwwLCJBIl0sWzEsMCwiWCJdLFsxLDEsIlMiXSxbMCwxLCJCIl0sWzAsMSwiZiJdLFsxLDIsInEiXSxbMCwzLCJpIiwyXSxbMywyLCJnIiwyXV0=
    \[\begin{tikzcd}
      A & X \\
      B & S
      \arrow["f", from=1-1, to=1-2]
      \arrow["q", from=1-2, to=2-2]
      \arrow["i"', from=1-1, to=2-1]
      \arrow["g"', from=2-1, to=2-2]
    \end{tikzcd}\]   
    が存在するとき, $\Fun_{A/ / S}(B,X)$の点は上の図式の解決$\bar{f} : B \to X$と同一視できる. 
  \end{itemize}
\end{remark}

\begin{remark}
  次の図式
  % https://q.uiver.app/#q=WzAsNCxbMCwwLCJBIl0sWzEsMCwiWCJdLFsxLDEsIlMiXSxbMCwxLCJCIl0sWzAsMSwiZiJdLFsxLDIsInEiXSxbMCwzLCJpIiwyXSxbMywyLCJnIiwyXV0=
  \[\begin{tikzcd}
    A & X \\
    B & S
    \arrow["f", from=1-1, to=1-2]
    \arrow["q", from=1-2, to=2-2]
    \arrow["i"', from=1-1, to=2-1]
    \arrow["g"', from=2-1, to=2-2]
  \end{tikzcd}\]   
  が可換でなくても, 構成3.1.3.7の$\Fun_{A/ / S}(B,X)$は定義できるが, これは空集合となる. 
\end{remark}

\begin{remark}
  次の可換図式
  % https://q.uiver.app/#q=WzAsNCxbMCwwLCJBIl0sWzEsMCwiWCJdLFsxLDEsIlMiXSxbMCwxLCJCIl0sWzAsMSwiZiJdLFsxLDIsInEiXSxbMCwzLCJpIiwyXSxbMywyLCJnIiwyXSxbMywxLCJcXGJhcntmfSIsMix7InN0eWxlIjp7ImJvZHkiOnsibmFtZSI6ImRhc2hlZCJ9fX1dXQ==
  \[\begin{tikzcd}
    A & X \\
    B & S
    \arrow["f", from=1-1, to=1-2]
    \arrow["q", from=1-2, to=2-2]
    \arrow["i"', from=1-1, to=2-1]
    \arrow["g"', from=2-1, to=2-2]
    \arrow["{\bar{f}}"', dashed, from=2-1, to=1-2]
  \end{tikzcd}\]
  が存在するとき, 次の3つが成立する. 
  \begin{itemize}
    \item $S \cong \Delta^0$のとき, $\Fun_{A/ / S}(B,X) = \Fun_{A/ }(B,X)$である. 
    \item $A \cong \emptyset$のとき, $\Fun_{A/ / S}(B,X) = \Fun_{/ S}(B,X)$である.
    \item $S \cong \Delta^0, A \cong \emptyset$のとき, $\Fun_{A/ / S}(B,X) = \Fun_(B,X)$である.
  \end{itemize}
\end{remark}

\begin{proof}
  それぞれ, $\Delta^0$が単体的集合の圏における終対象, $\emptyset$が単体的集合の圏における始対象であることから従う. 
\end{proof}

\begin{remark}
  次の可換図式
  % https://q.uiver.app/#q=WzAsNCxbMCwwLCJBIl0sWzEsMCwiWCJdLFsxLDEsIlMiXSxbMCwxLCJCIl0sWzAsMSwiZiJdLFsxLDIsInEiXSxbMCwzLCJpIiwyXSxbMywyLCJnIiwyXSxbMywxLCJcXGJhcntmfSIsMix7InN0eWxlIjp7ImJvZHkiOnsibmFtZSI6ImRhc2hlZCJ9fX1dXQ==
  \[\begin{tikzcd}
    A & X \\
    B & S
    \arrow["f", from=1-1, to=1-2]
    \arrow["q", from=1-2, to=2-2]
    \arrow["i"', from=1-1, to=2-1]
    \arrow["g"', from=2-1, to=2-2]
    \arrow["{\bar{f}}"', dashed, from=2-1, to=1-2]
  \end{tikzcd}\]
  が存在するとき, $\Fun_{A/ / S}(B,X)$は次の射
  \begin{align*}
    \Fun(B,X) \to \Fun(A,X) \times_{\Fun(A,S)} \Fun(B,S)
  \end{align*}
  のファイバーと同一視できる. 
\end{remark}

\begin{example}
  $q : X \to S$を単体的集合の射とする. 
  任意の点$s \in S$に対して, $\Fun_{/ S}(\{s\},X)$はpullback $X_s=\{s\} \times_S X$と同一視できる. 
  % {s}は1点集合でFun({s},X) = Xとみなせるので 
\end{example}

\begin{proposition}
  $i : A \to B$を単体的集合のmono射, $q : X \to S$をKanファイブレーションとする. 
  次の可換図式
  % https://q.uiver.app/#q=WzAsNCxbMCwwLCJBIl0sWzEsMCwiWCJdLFsxLDEsIlMiXSxbMCwxLCJCIl0sWzAsMSwiZiJdLFsxLDIsInEiXSxbMCwzLCJpIiwyXSxbMywyLCJnIiwyXSxbMywxLCJcXGJhcntmfSIsMix7InN0eWxlIjp7ImJvZHkiOnsibmFtZSI6ImRhc2hlZCJ9fX1dXQ==
  \[\begin{tikzcd}
    A & X \\
    B & S
    \arrow["f", from=1-1, to=1-2]
    \arrow["q", from=1-2, to=2-2]
    \arrow["i"', from=1-1, to=2-1]
    \arrow["g"', from=2-1, to=2-2]
    \arrow["{\bar{f}}"', dashed, from=2-1, to=1-2]
  \end{tikzcd}\]
  が存在するとき, $\Fun_{A/ / S}(B,X)$はKan複体である. 
  更に$i$が緩射のとき, $\Fun_{A/ / S}(B,X)$は可縮なKan複体である.
\end{proposition}

\begin{proof}
  注意3.1.3.11より, $\Fun_{A/ / S}(B,X)$は次の制限 
  \begin{align*}
    \theta : \Fun(B,X) \to \Fun(A,X) \times_{\Fun(A,S)} \Fun(B,S)
  \end{align*}
  のファイバーと同一視できる. 
  定理3.1.3.1より, $\theta$はKanファイブレーションである. 
  注意3.1.1.9より, このpullbackはKan複体である. 
  $i$が緩射のとき, 定理3.1.3.5より, $\theta$は自明なKanファイブレーションである. 
  注意1.4.5.10より, このpullbackは可縮なKan複体である. 
\end{proof}

\begin{corollary}
  $B$を単体的集合, $A$を$B$の単体的部分集合, $f : A \to X$を単体的集合の射とする. 
  $X$がKan複体のとき, $\Fun_{A/}(B,X)$はKan複体である. 
  更に, 包含$A \to B$が緩射のとき, $\Fun_{A/}(B,X)$は可縮である.
\end{corollary}

\begin{proof}
  命題3.1.3.13において$S=\Delta^0$とすればよい. 
\end{proof}

\begin{corollary}
  $q : X \to S$をKanファイブレーション, $g : B \to S$を単体的集合の射とする. 
  このとき, $\Fun_{/ S}(B,X)$はKan複体である.
\end{corollary}

\begin{proof}
  命題3.1.3.13において$A=\emptyset$とすればよい. 
\end{proof}


\subsection{被覆射}

Kanファイブレーションの定義において, リフトの解決が一意であるような射を被覆射という. 

\begin{definition}[被覆射]
  $f : X \to S$を単体的集合の射とする.
  任意の$0 \leq i \leq n$に対して, 次のリフト問題が一意な解決(図中のドットで描かれる矢印)をもつとき, $f$を被覆射(covering map)という. 
  % https://q.uiver.app/#q=WzAsNCxbMCwwLCJcXExhbWJkYV5uX2kiXSxbMSwwLCJYIl0sWzAsMSwiXFxEZWx0YV5uIl0sWzEsMSwiUyJdLFswLDEsIlxcc2lnbWFfMCJdLFswLDIsIiIsMix7InN0eWxlIjp7InRhaWwiOnsibmFtZSI6Imhvb2siLCJzaWRlIjoidG9wIn19fV0sWzIsMSwiXFxzaWdtYSIsMix7InN0eWxlIjp7ImJvZHkiOnsibmFtZSI6ImRhc2hlZCJ9fX1dLFsyLDMsIlxcYmFye1xcc2lnbWF9IiwyXSxbMSwzLCJmIl1d
  \[\begin{tikzcd}
    {\Lambda^n_i} & X \\
    {\Delta^n} & S
    \arrow["{\sigma_0}", from=1-1, to=1-2]
    \arrow[hook, from=1-1, to=2-1]
    \arrow["\sigma"', dashed, from=2-1, to=1-2]
    \arrow["{\bar{\sigma}}"', from=2-1, to=2-2]
    \arrow["f", from=1-2, to=2-2]
  \end{tikzcd}\]
  つまり, 任意の単体的集合の射$\sigma_0 : \Lambda^n_i \to X$と$\bar{\sigma} : \Delta^n \to S$に対して, $\sigma_0$を$f \circ \sigma = \bar{\sigma}$を満たす$n$単体$\sigma : \Delta^n \to X$に一意に拡張できるときである. 
\end{definition}

\begin{remark}
  $f : X \to S$を単体的集合の射とする.
  $f$が被覆射であることと, $f^\myop : X^\myop \to S^\myop$が被覆射であることは同値である. 
\end{remark}

\begin{remark}
  $f : X \to S$を単体的集合の射, $\delta : X \to X \times_S X$を$f$のrelative diagonalとする.
  このとき, $f$が被覆射であることと, $f$と$\delta$がともにKanファイブレーションであることは同値である.
  特に, 任意の被覆射はKanファイブレーションである. 
\end{remark}

\begin{proof}
  $\Rightarrow$を示す.
  まず, $f$が被覆射のとき, Kanファイブレーションであることは明らかである. 
  次の図式を考えると, pullbackの一意性より, 左上の2つの三角は可換である.
  % https://q.uiver.app/#q=WzAsNyxbMCwwLCJcXExhbWJkYV5uX2kiXSxbMSwwLCJYIl0sWzEsMSwiWCBcXHRpbWVzX1MgWCJdLFswLDEsIlxcRGVsdGFebiJdLFsyLDEsIlgiXSxbMiwyLCJTIl0sWzEsMiwiWCJdLFswLDFdLFsxLDIsIlxcZGVsdGEiXSxbMCwzXSxbMywyXSxbMiw0XSxbNCw1LCJmIl0sWzIsNl0sWzYsNV0sWzIsNSwiIiwxLHsic3R5bGUiOnsibmFtZSI6ImNvcm5lciJ9fV0sWzMsMSwiIiwxLHsic3R5bGUiOnsiYm9keSI6eyJuYW1lIjoiZGFzaGVkIn19fV1d
  \[\begin{tikzcd}
    {\Lambda^n_i} & X \\
    {\Delta^n} & {X \times_S X} & X \\
    & X & S
    \arrow[from=1-1, to=1-2]
    \arrow["\delta", from=1-2, to=2-2]
    \arrow[from=1-1, to=2-1]
    \arrow[from=2-1, to=2-2]
    \arrow[from=2-2, to=2-3]
    \arrow["f", from=2-3, to=3-3]
    \arrow[from=2-2, to=3-2]
    \arrow[from=3-2, to=3-3]
    % \arrow["\lrcorner"{anchor=center, pos=0.125}, draw=none, from=2-2, to=3-3]
    \arrow[dashed, from=2-1, to=1-2]
  \end{tikzcd}\]
  よって, $\delta$はKanファイブレーションである. 

  $\Leftarrow$を示す.
  $f$はkanファイブレーションなので, リフト問題は解決を持つ. 
  よって, この解決が一意であることを示せばよいが, これはpullbackの一意性から従う. 
\end{proof}

\begin{remark}
  被覆射の集まりはpullbackで閉じる. 
\end{remark}

\begin{remark}
  $f : X \to S$を単体的集合の射, $g : Y \to Z$を被覆射とする.
  このとき, $g$が被覆射であることと, $gf$が被覆射であることは同値である.
  特に, 被覆射の集まりは合成で閉じる. 
\end{remark}

\begin{remark}
  $f : X \to S$を単体的集合の射とする.
  このとき, 次の2つは同値である.
  \begin{enumerate}
    \item $f$は被覆射である.
    \item 任意の緩射$i : A \to B$に対して, 次のリフト問題は一意な解決(図中のドットで描かれる矢印)を持つ. 
    \[\begin{tikzcd}
      A & X \\
      B & S
      \arrow[from=1-1, to=1-2]
      \arrow["f", from=1-2, to=2-2]
      \arrow[from=2-1, to=2-2]
      \arrow["i"', from=1-1, to=2-1]
      \arrow[dashed, from=2-1, to=1-2]
    \end{tikzcd}\] 
  \end{enumerate}
\end{remark}

\begin{proof}
  注意3.1.2.7と注意3.1.4.3より従う. 
\end{proof}

\begin{proposition}
  $f : X \to S$を被覆射, $i : A \hookrightarrow B$を単体的集合のmono射とする.
  このとき, 誘導される射
  \begin{align*}
    \Fun(B,X) \to \Fun(B,S) \times_{\Fun(A,S)} \Fun(A,X)
  \end{align*}
  は被覆射である.
\end{proposition}

\begin{proof}
  注意3.1.4.6より, 任意の緩射$i' : A' \hookrightarrow B'$に対して, 次のリフト問題が一意な解決を持つことを示せばよい. 
  % https://q.uiver.app/#q=WzAsNCxbMCwwLCJBJyJdLFsxLDAsIlxcZnVuKEIsWCkiXSxbMSwxLCJcXGZ1bihCLFMpIFxcdGltZXNfe1xcZnVuKEEsUyl9IFxcZnVuKEEsWCkiXSxbMCwxLCJCJyJdLFswLDFdLFsxLDIsImYiXSxbMywyXSxbMCwzLCJpJyIsMl0sWzMsMSwiIiwxLHsic3R5bGUiOnsiYm9keSI6eyJuYW1lIjoiZGFzaGVkIn19fV1d
  \[\begin{tikzcd}
    {A'} & {\Fun(B,X)} \\
    {B'} & {\Fun(B,S) \times_{\Fun(A,S)} \Fun(A,X)}
    \arrow[from=1-1, to=1-2]
    \arrow[from=1-2, to=2-2]
    \arrow[from=2-1, to=2-2]
    \arrow["{i'}"', from=1-1, to=2-1]
    \arrow[dashed, from=2-1, to=1-2]
  \end{tikzcd}\]
  これは次の図式のリフト問題の解決と同値である. 
  % https://q.uiver.app/#q=WzAsNCxbMCwwLCIoQSBcXHRpbWVzIEInKSBcXGNvcHJvZF97QSBcXHRpbWVzIEEnfSAoQiBcXHRpbWVzIEEnKSJdLFsxLDAsIlgiXSxbMSwxLCJTIl0sWzAsMSwiQiBcXHRpbWVzIEInIl0sWzAsMV0sWzEsMiwiZiJdLFswLDNdLFszLDJdLFszLDEsIiIsMCx7InN0eWxlIjp7ImJvZHkiOnsibmFtZSI6ImRhc2hlZCJ9fX1dXQ==
  \[\begin{tikzcd}
    {(A \times B') \coprod_{A \times A'} (B \times A')} & X \\
    {B \times B'} & S
    \arrow[from=1-1, to=1-2]
    \arrow["f", from=1-2, to=2-2]
    \arrow[from=1-1, to=2-1]
    \arrow[from=2-1, to=2-2]
    \arrow[dashed, from=2-1, to=1-2]
  \end{tikzcd}\]
  命題3.1.2.8より, $(A \times B') \coprod_{A \times A'} (B \times A') \hookrightarrow B \times B'$は緩射である. 
  注意3.1.4.6より, 求める射は被覆射である.
\end{proof}

\begin{corollary}
  $f : X \to S$を被覆射とする.
  このとき, 任意の単体的集合$B$に対して, $f$の誘導する射$\Fun(B,X) \to \Fun(B,S)$は被覆射である.
\end{corollary}

\begin{proof}
  定理3.1.3.2と同様に, 命題3.1.4.7において, $A = \emptyset$とすればよい.
\end{proof}

\begin{proposition}
  $f : X \to S$を位相空間の被覆とする.
  このとき, $\Sing(f) : \Sing(X) \to \Sing(S)$は定義3.1.4.1の意味の被覆射である.
\end{proposition}

\begin{proof}
  $\delta : X \to X \times_S X$を$f$のrelative diagonalとする.
  (途中)
\end{proof}

被覆射は単純な局所的な構造を持っている. 

\begin{proposition}
  $f : X \to S$を単体的集合の射とする. 
  このとき, 次の3つは同値である. 
  \begin{enumerate}
    \item $f$は被覆射である. 
    \item 任意の標準的単体の射$u : \Delta^m \to \Delta^n$に対して, $u$の合成は同型$X_n \to X_m \times_{S_m} S_n$を定める. 
    \item 任意の$n$単体$\sigma : \Delta^n \to S$に対して, 射影$\Delta^n \times_S X \to \Delta^n$は$\Delta^n \times_S X$の各連結成分における同型を定める. 
  \end{enumerate}
\end{proposition}

\begin{proof}
  (1)から(2)を示す. 
  $u : \Delta^m \to \Delta^n$を単体的集合の射とする. 
  点$v : \Delta^0 \to \Delta^m$をとると, 例3.1.2.5より$v$は緩射である. 
  注意3.1.1.10より, $u \circ v$も緩射である. 
  注意3.1.4.6より, 次の図式を可換にするような射$X_m : \Delta^m \to X$が一意に存在する. 
  % https://q.uiver.app/#q=WzAsNCxbMCwwLCJcXERlbHRhXjAiXSxbMSwwLCJYIl0sWzEsMSwiUyJdLFswLDEsIlxcRGVsdGFebSJdLFswLDEsIlhfMCJdLFsxLDIsImYiXSxbMCwzLCJ2IiwyXSxbMywyLCJTX20iLDJdLFszLDEsIlhfbSIsMix7InN0eWxlIjp7ImJvZHkiOnsibmFtZSI6ImRhc2hlZCJ9fX1dXQ==
  \[\begin{tikzcd}
    {\Delta^0} & X \\
    {\Delta^m} & S
    \arrow["{X_0}", from=1-1, to=1-2]
    \arrow["f", from=1-2, to=2-2]
    \arrow["v"', from=1-1, to=2-1]
    \arrow["{S_m}"', from=2-1, to=2-2]
    \arrow["{X_m}"', dashed, from=2-1, to=1-2]
  \end{tikzcd}\]
  よって, 次の図式はpullbackである. 
  % https://q.uiver.app/#q=WzAsNCxbMCwwLCJYX20iXSxbMSwwLCJYXzAiXSxbMSwxLCJTXzAiXSxbMCwxLCJTX20iXSxbMCwxLCItIFxcY2lyYyB2Il0sWzEsMiwiZiBcXGNpcmMgLSJdLFswLDMsImYgXFxjaXJjIC0iLDJdLFszLDIsIi0gXFxjaXJjIHYiLDJdXQ==
  \[\begin{tikzcd}
    {X_m} & {X_0} \\
    {S_m} & {S_0}
    \arrow["{- \circ v}", from=1-1, to=1-2]
    \arrow["{f \circ -}", from=1-2, to=2-2]
    \arrow["{f \circ -}", from=1-1, to=2-1]
    \arrow["{- \circ v}"', from=2-1, to=2-2]
  \end{tikzcd}\]
  同様に, 次の図式の外側の四角も可換である. 
  % https://q.uiver.app/#q=WzAsNixbMSwwLCJYX20iXSxbMiwwLCJYXzAiXSxbMiwxLCJTXzAiXSxbMSwxLCJTX20iXSxbMCwwLCJYX24iXSxbMCwxLCJTX24iXSxbMCwxLCItIFxcY2lyYyB2Il0sWzEsMiwiZiBcXGNpcmMgLSJdLFswLDMsImYgXFxjaXJjIC0iXSxbMywyLCItIFxcY2lyYyB2IiwyXSxbNCwwLCItIFxcY2lyYyB1IiwyXSxbNCw1LCJmIFxcY2lyYyAtIl0sWzUsMywiLSBcXGNpcmMgdSJdXQ==
  \[\begin{tikzcd}
    {X_n} & {X_m} & {X_0} \\
    {S_n} & {S_m} & {S_0}
    \arrow["{- \circ v}", from=1-2, to=1-3]
    \arrow["{f \circ -}", from=1-3, to=2-3]
    \arrow["{f \circ -}", from=1-2, to=2-2]
    \arrow["{- \circ v}", from=2-2, to=2-3]
    \arrow["{- \circ u}", from=1-1, to=1-2]
    \arrow["{f \circ -}", from=1-1, to=2-1]
    \arrow["{- \circ u}", from=2-1, to=2-2]
  \end{tikzcd}\]
  よって, 左の四角も可換である. 
  つまり, $X_n \to X_m \times_{S_m} S_n$は同型である. 

  (2)から(3)を示す. 

\end{proof}

\begin{example}
  $X$を単体的集合とする. 
  一意な射$X \to \Delta^0$が被覆射であることと, $X$が離散(定義1.1.4.9)であることは同値である. 
\end{example}

\begin{corollary}
  $f : X \to S$を単体的集合のmono射とする. 
  このとき, 次の3つは同値である. 
  \begin{enumerate}
    \item $f$は$X$を$S$の直和因子にうつす. 
    \item $f$は被覆射である. 
    \item $f$はKanファイブレーションである. 
  \end{enumerate}
\end{corollary}

\begin{proof}
  (1)から(2)と(2)から(3)は明らかである. 

\end{proof}


\subsection{Kan複体のホモトピー圏}

単体的集合の圏は良いホモトピーの概念を備えている.

1.3.6節では$\infty$圏におけるホモトピーを定義した.
この節では, 一般の単体的集合におけるホモトピーを定義する. 

\begin{definition}[ホモトピック]
  $X,Y$を単体的集合, 単体的集合の射$f,g : X \to Y$を単体的集合$\Fun(X,Y)$の点と同一視する.
  $f$と$g$が$\Fun(X,Y)$において同じ連結成分に属しているとき, $f$と$g$はホモトピック(homotopic)であるという.
\end{definition}

定義3.1.5.1を具体的に書き下す.

\begin{definition}[ホモトピー]
  $X,Y$を単体的集合, $f_0,f_1 : X \to Y$を単体的集合の射とする.
  $\Fun(X,Y)$における射$h : \Delta^1 \times X \to Y$が$f_0 = h|_{\{0\} \times X}$と$f_1 = h|_{\{1\} \times X}$を満たすとき, $h$を$f_0$から$f_1$へのホモトピー(homotopy)という.
\end{definition}

\begin{remark}[ホモトピー拡張リフト性質]
  $f : X \to S$をKanファイブレーションとする. 

\end{remark}

\begin{proposition}
  $X,Y$を単体的集合, $f,g : X \to Y$を単体的集合の射とする.
  このとき, 次の2つが成立する.
  \begin{enumerate}
    \item $f$と$g$がホモトピックであることと, $X$から$Y$への射の列$=_0,_1,\cdots,f_n=g$が存在して, 任意の$0 \leq i \leq n$に対して, $f_{i-1}$から$f_i$へのホモトピーか$f_i$から$f_{i-1}$へのホモトピーが存在することは同値である. 
    \item $Y$をKan複体とする. 
    このとき, $f$と$g$がホモトピックであることと, $f$から$g$へのホモトピーが存在することは同値である. 
  \end{enumerate}
\end{proposition}

\begin{proof}
  まず, (1)を示す.
  注意1.1.6.23において, $S_\bullet = \Fun(X,Y)$とすればよい(らしい). 

  次に(2)を示す.
  $Y$がKan複体であるとき, 系3.1.3.4より, $\Fun(X,Y)$もKan複体である.
  命題1.1.9.10において, $S_\bullet = \Fun(X,Y)$とする. 
  $\Fun(X,Y)$の点は$f,g$と同一視できる.
  定義より, $f$と$g$が$\Fun(X,Y)$において同じ連結成分に属するとき, $f$と$g$がホモトピックであることは同値である. 
  また, ある射$h : \Delta^1 \times X \to Y$が存在して, $d_0(h) = f$かつ$d_1(h) = g$を満たすことと, $f_0$から$f_1$へのホモトピーが存在することは同値である. 
\end{proof}

\begin{example}
  $X$を単体的集合, $Y$を位相空間とする. 
  随伴性より, 連続関数$f_0,f_1 : |X| \to Y$に対応する単体的集合の射$f_0',f_0' : X \to \Sing(Y)$が存在する. 
  $h : [0,1] \times |X| \to Y$を$f_0=h|_{\{0\} \times |X|}$かつ$f_1=h|_{\{1\} \times |X|}$を満たす連続関数(つまり, $h$は位相空間の圏における$f_0$から$f_1$へのホモトピー)とする. 
  このとき,  
  \begin{align*}
    |\Delta^1 \times X| \xrightarrow{\theta} |\Delta^1| \times |X| \cong [0,1] \times |X| \xrightarrow{h} Y 
  \end{align*}
  は随伴性より, $f_0'$から$f_1'$への(定義3.1.5.2の意味の)ホモトピー$h' : \Delta^1 \times X \to \Sing(Y)$を定める. 
  ここで, $\theta$が同相であることは系3.5.5.2で示す. 
  以上より, 構成$h \mapsto h'$は, $f_0$から$f_1$への(連続関数としての)ホモトピーの集まりと$f_0'$から$f_1'$への(単体的集合の射としての)ホモトピーの集まりの間の全単射を定める. 
\end{example}

\begin{example}
  例3.1.5.5において$X=\Sing(X)$とする. 
  $X,Y$を位相空間, $f_0,f_1 : |X| \to Y$を連続関数, $h : [0,1] \times X \to Y$を$f_0$から$f_1$へのホモトピーとする. 
  このとき, $h$は単体的集合の射$\Sing(f_0),\Sing(f_1) : \Sing(X) \to \Sing(Y)$の間のホモトピーを定める. 
\end{example}

\begin{example}
  $\C,\D$を圏, $F,G : \C \to \D$を関手とする. 
  この関手を単体的集合の射$N(F),N(G) : N(\C) \to N(\D)$と同一視する. 
  このとき, $N(F)$から$N(G)$へのホモトピーは単体的集合の射
  \begin{align*}
    h : \Delta^1 \times N(\C) \cong N([1] \times \C) \to N(\D)
  \end{align*}
  である. 
  命題1.2.2.1 ($N$は忠実充満)なので, $h$の持つ情報は関手$H : [1] \times \C \to \D$と等しい. 
  ここで, $F$から$G$へのホモトピーは圏の圏における自然変換である. 
  以上より, $F$から$G$への自然変換の集まりは$N(F)$から$N(G)$へのホモトピーの集まりの間の全単射を定める. 
\end{example}

\begin{example}
  EM空間について
\end{example}

\begin{notation}
  $f : X \to Y$を単体的集合の射とする. 
  このとき, $f$のホモトピー類を$[f]$と表す. 
  つまり, $[f]$は$f$の集合$\pi_0\Fun(X,Y)$の像である. 
\end{notation}

\begin{construction}[Kan複体のホモトピー圏]
  圏$\hkan$を次のように定義する. 
  \begin{itemize}
    \item $\hkan$の対象はKan複体
    \item 任意の$X,Y \in \hkan$に対して, $\Hom_\hkan(X,Y):=[X,Y] = \pi_0(\Fun(X,Y))$
    \item 任意の$X,Y,Z \in \hkan$に対して, 合成 
    \begin{align*}
      \circ : \Hom_\hkan(Y,Z) \times \Hom_\hkan(X,Y) \to \Hom_\hkan(X,Z)
    \end{align*}
    はホモトピー類の合成$[g] \circ [f] = [g \circ f]$により定める.
  \end{itemize}
  $\hkan$をKan複体のホモトピー圏(the homotopy category of Kan complexes)という.
\end{construction}

\begin{remark}
  $\kan$をKan複体のなす$\sset$の充満部分圏, $\C$を圏とする. 
  $\D$を次の条件を満たす関手$\kan \to \C$のなす$\Fun(\kan,\C)$の充満部分圏とする. 
  \begin{itemize}
    \item $X,Y$をKan複体, $u_0,u_1 : X \to Y$をホモトピックな射とする. 
    このとき, $\Hom_\C(FX,FY)$において, $Fu_0=Fu_1$である. 
  \end{itemize}
  このとき, 商関手$\kan \to \hkan$の前合成は関手圏の圏同型$\Fun(\hkan,\C) \to \D$を定める. 
\end{remark}

\begin{remark}
  局所Kan単体的集合について
\end{remark}

\begin{construction}
  Kan複体のホモトピー2圏について
\end{construction}

\begin{remark}
  
\end{remark}

\begin{remark}
  
\end{remark}


\subsection{ホモトピー同値と弱ホモトピー同値}

$f : X \to Y$をKan複体の射とする. 
$f$のホモトピー類$[f]$がホモトピー圏$\hkan$における同型射のとき, $f$をホモトピー同値(homotopy equivalence)という.
この定義は一般の単体的集合に拡張することができる.

\begin{definition}[ホモトピー同値]
  $f : X \to Y$を単体的集合の射とする. 
  単体的集合の射$g : Y \to X$に対して, $gf$と$fg$がそれぞれ$\id_X$と$\id_Y$と(定義3.1.5.1の意味で)ホモトピックであるとき, $g$を$f$の単体的ホモトピー逆射(simplicial homotopy inverse)という. 
  $X,Y$がKan複体のとき, 特に$g$を$f$のホモトピー逆射(homotopy inverse)という. 
  $f$が単体的ホモトピー逆射をもつとき, $f$をホモトピー同値(homotopy equivalence)という.
\end{definition}

\begin{remark}
  $f : X \to Y$を単体的集合の射とする. 
  単体的集合の射$g : Y \to X$に対して, $gf$と$fg$がそれぞれ$\id_X$と$\id_Y$と(定義3.1.5.1の意味で)ホモトピックであるとき, $g$を$f$のホモトピー逆射(homotopy inverse)ということが多い.
  しかし, $X,Y$が$\infty$圏のとき, ホモトピー同値は$gf$と$fg$がそれぞれ$\id_X$と$\id_Y$と$\infty$圏$\Fun(X,X), \Fun(Y,Y)$の対象として同型(isomorphic)であるときに使われる. (定義4.5.1.10と注意4.5.1.14)
  このため, 定義3.1.6.1では単体的ホモトピー逆射という言葉を使っている. 
  $X,Y$がKan複体のとき, この用語の衝突は問題がない. 
\end{remark}

\begin{remark}
  $f : X \to Y$を位相空間のホモトピー同値とする. 
  例3.1.5.6より, 誘導される単体的集合の射$\Sing(f) : \Sing(X) \to \Sing(Y)$は(定義3.1.6.1の意味で)ホモトピー同値である. 
\end{remark}

\begin{proof}
  $h : [0,1] \times X \to X$を$gf$から$\id_X$へのホモトピーとする. 
  
\end{proof}

\begin{remark}
  
\end{remark}

\begin{remark}
  $f : X \to Y$をKan複体の射とする.
  $f$がホモトピー同値のとき, 誘導される基本亜群の射$\pi_{\leq 1}(f) : \pi_{\leq 1}(X) \to \pi_{\leq 1}(Y)$は圏同値である. 
  特に, $f$は全単射$\pi_0(f) : \pi_0(X) \to \pi_0(Y)$を定める. 
\end{remark}

\begin{remark}
  $f : X \to Y$を単体的集合の射とする.
  このとき, 次の3つは同値である. 
  \begin{itemize}
    \item $f$はホモトピー同値である. 
    \item 任意の単体的集合$Z$に対して, $f$の後合成が定める$\pi_0\Fun(Y,Z) \to \pi_0\Fun(X,Z)$は全単射である. 
    \item 任意の単体的集合$W$に対して, $f$の後合成が定める$\pi_0\Fun(W,X) \to \pi_0\Fun(W,Y)$は全単射である. 
  \end{itemize}
  特に, $W=\Delta^0$とすると, $f$がホモトピー同値のとき, $\pi_0(f) : \pi_0(X) \to \pi_0(Y)$は全単射である. 
\end{remark}

\begin{remark}(2-out-of-3)
  $f : X \to Y, g : Y \to Z$を単体的集合の射とする.
  $f,g,gf$のうち2つがホモトピー同値のとき, 残り1つもホモトピー同値である. 
\end{remark}

\begin{remark}
  直積について
\end{remark}

ホモトピー同値の例を挙げる. 

\begin{proposition}
  $F : \C \to \D$を関手とする. 
  $F$が左随伴か右随伴を持つとき, $N(F) : N(\C) \to N(\D)$はホモトピー同値である.
\end{proposition}

\begin{proof}
  $F$が右随伴$G : \D \to \C$を持つとする. 
  $F$が左随伴を持つときも同様に示せる.
  このとき, 自然変換$u : \id_\C \to GF, v : FG \to \id_\D$が存在する. 
  例3.1.5.7より, $N(F)$は$N(G)$を単体的ホモトピー逆射に持つホモトピー同値である. 
\end{proof}

\begin{proposition}
  $f : X \to S$を自明なKanファイブレーションとする. 
  このとき, $f$はホモトピー同値である. 
\end{proposition}

\begin{proof}
  命題1.4.5.4より, 次のリフト問題は解決$g$を持つ. 
  % https://q.uiver.app/#q=WzAsNCxbMCwwLCJcXGVtcHR5c2V0Il0sWzEsMCwiWCJdLFsxLDEsIlMiXSxbMCwxLCJTIl0sWzAsMV0sWzEsMiwiZiJdLFswLDNdLFszLDIsIlxcaWRfUyIsMl0sWzMsMSwiZyIsMix7InN0eWxlIjp7ImJvZHkiOnsibmFtZSI6ImRhc2hlZCJ9fX1dXQ==
  \[\begin{tikzcd}
    \emptyset & X \\
    S & S
    \arrow[from=1-1, to=1-2]
    \arrow["f", from=1-2, to=2-2]
    \arrow[from=1-1, to=2-1]
    \arrow["{\id_S}"', from=2-1, to=2-2]
    \arrow["g"', dashed, from=2-1, to=1-2]
  \end{tikzcd}\]
  このとき, $g$は$f$の切断である. 
  この$g$が$f$の単体的ホモトピー逆射であることを示す. 
  $fg$と$\id_S$がホモトピー同値であることは$g$の定義から従う. 
  よって, $\id_X$から$gf$へのホモトピーが存在することを示す. 
  命題1.4.5.4より, 次のリフト問題は解決$h$を持つ. 
  % https://q.uiver.app/#q=WzAsNCxbMCwwLCJcXHswLDFcXH0gXFx0aW1lcyBYIl0sWzIsMCwiWCJdLFsyLDEsIlMiXSxbMCwxLCJcXERlbHRhXjEgXFx0aW1lcyBYIl0sWzAsMSwiKFxcaWRfWCxnZikiXSxbMSwyLCJmIl0sWzAsM10sWzMsMiwiZiIsMl0sWzMsMSwiaCIsMix7InN0eWxlIjp7ImJvZHkiOnsibmFtZSI6ImRhc2hlZCJ9fX1dXQ==
  \[\begin{tikzcd}
    {\{0,1\} \times X} && X \\
    {\Delta^1 \times X} && S
    \arrow["{(\id_X,gf)}", from=1-1, to=1-3]
    \arrow["f", from=1-3, to=2-3]
    \arrow[from=1-1, to=2-1]
    \arrow["f"', from=2-1, to=2-3]
    \arrow["h"', dashed, from=2-1, to=1-3]
  \end{tikzcd}\]
この$h$は$\id_X$から$gf$へのホモトピーである.
\end{proof}

\begin{example}
  2章の内容
\end{example}

\begin{definition}[弱ホモトピー同値]
  $f : X \to Y$を単体的集合の射とする. 
  任意のKan複体$Z$に対して, $f$の前合成が全単射$\pi_0\Fun(Y,Z) \to \pi_0\Fun(X,Z)$を定めるとき, $f$を弱ホモトピー同値(weak homotopy equivalence)という.
\end{definition}

\begin{proposition}
  $f : X \to Y$を単体的集合の射とする. 
  $f$がホモトピー同値のとき, $f$は弱ホモトピー同値である. 
  逆は, $X$と$Y$がともにKan複体であるときに成立する.
\end{proposition}

\begin{proof}
  前半は例3.1.6.6の(1)と(2)の同値性より従う. 
  後半を示す. 
  $f : X \to Y$を弱ホモトピー同値とする. 
  定義3.1.6.12において$Z=X$とすると, $f$は全単射$\pi_0\Fun(Y,X) \to \pi_0\Fun(X,X)$を定める.
  よって, ある単体的集合の射$g : Y \to X$が存在して, $gf$は$\id_X$とホモトピックである.
  このとき, $fgf$は$f(=\id_Y)f$とホモトピックである. 
  定義3.1.6.12において$Z=Y$とすると, $f$は全単射$\pi_0\Fun(Y,Y) \to \pi_0\Fun(X,Y)$を定める.
  特に単射であるので, $fg$は$\id_Y$とホモトピックである. 
  よって, $g$は$f$のホモトピー逆射である.
\end{proof}

\begin{proposition}
  $i : A \to B$を緩射とする. 
  このとき, $i$は弱ホモトピー同値である. 
\end{proposition}

\begin{proof}
  $X$をKan複体の射とする. 
  系3.1.3.6より, $\theta : \Fun(B,X) \to \Fun(A,X)$は自明なKanファイブレーションである. 
  命題3.1.6.10より, $\theta$はホモトピー同値である. (途中)
\end{proof}

\begin{remark}
  命題3.1.6.14の逆は部分的に成立する. 
  $f : A \to B$が単体的集合のmono射のとき, $f$は弱ホモトピー同値である. 
  よって, $f$は緩射である. (系3.1.6.15)
\end{remark}

\begin{remark}(2-out-of-3)
  $f : X \to Y, g : Y \to Z$を単体的集合の射とする.
  $f,g,gf$のうち2つが弱ホモトピー同値のとき, 残り1つも弱ホモトピー同値である. 
\end{remark}

\begin{proposition}
  normalized chain comlexについて
\end{proposition}

\begin{remark}
  命題3.1.6.17の部分的な逆について
\end{remark}


\subsection{ファイブラント置換}

Kan複体について調べることは, 単体的集合のホモトピー論を理解するために非常に有用である. 
しかし, Kan複体のホモトピー論を調べようとすると, より一般の単体的集合について考える必要がある. 
例えば, Kan複体の射$f_0,f_1 : S \to T$に対して, $f_0$から$f_1$へのホモトピーは単体的集合の射$h : \Delta^1 \times S \to T$として定義されるが, $\Delta^1$も$\Delta^1 \times S$もKan複体ではない. 
($S=\emptyset$という自明な場合を除いてである. 練習1.1.9.2を参照)
Kan複体でない単体的集合$X$を調べるとき, $X$を同じホモトピー型を持つKan複体に置き換える(replace)ことが有用である. 
この置き換えは常に存在する. 
より正確にいうと, $X$に対して, 単体的集合の射$X \to Q$が弱同値となるようなKan複体$Q$が存在する. (系3.1.7.2)
この節の目標は, この結果のfiberwiseな場合(命題3.1.7.1)を証明することである. 

\begin{proposition}
  $f : X \to Y$を単体的集合の射とする.
  このとき, ある緩射(つまり弱ホモトピー同値) $f' : X \to Q(f)$とKanファイブレーション$f'' : Q(f) \to Y$が存在して, 次の図式は可換である. 
  % https://q.uiver.app/#q=WzAsMyxbMCwwLCJYIl0sWzIsMCwiWSJdLFsxLDEsIlEoZikiXSxbMCwxLCJmIl0sWzAsMiwiZiciLDJdLFsyLDEsImYnJyIsMl1d
  \[\begin{tikzcd}
    X && Y \\
    & {Q(f)}
    \arrow["f", from=1-1, to=1-3]
    \arrow["{f'}"', from=1-1, to=2-2]
    \arrow["{f''}"', from=2-2, to=1-3]
  \end{tikzcd}\]
  更に, 単体的集合$Q(f)$と$f',f''$は$f$の関手性による. 
  このとき, 関手 
  \begin{align*}
    \Fun([1],\sset) \to \sset : (f : X \to Y) \mapsto Q(f)
  \end{align*}
  はフィルター付き余極限と交換する.
\end{proposition}

この命題はQuillenの小対象論法(small object argument)の系である. 

\begin{proof}
  $f : X \to Y$を単体的集合の射とする.
  単体的集合の列$\{X(m)\}_{m \geq 0}$と単体的集合の射$f(m) : X(m) \to Y$を帰納的に定義する. 
  まず, $X(0) := X, f(0) := f$とする. 
  定義された単体的集合の射$f(m) : X(m) \to Y$に対して, $S(m)$を次の可換図式$\sigma$の集まりとする. 
  % https://q.uiver.app/#q=WzAsNCxbMCwwLCJcXExhbWJkYV5uX2kiXSxbMCwxLCJcXERlbHRhXm4iXSxbMSwwLCJYKG0pIl0sWzEsMSwiWSJdLFswLDEsIiIsMSx7InN0eWxlIjp7InRhaWwiOnsibmFtZSI6Imhvb2siLCJzaWRlIjoidG9wIn19fV0sWzAsMl0sWzIsMywiZihtKSJdLFsxLDMsInVfXFxzaWdtYSIsMl1d
  \[\begin{tikzcd}
    {\Lambda^n_i} & {X(m)} \\
    {\Delta^n} & Y
    \arrow[hook, from=1-1, to=2-1]
    \arrow[from=1-1, to=1-2]
    \arrow["{f(m)}", from=1-2, to=2-2]
    \arrow["{u_\sigma}"', from=2-1, to=2-2]
  \end{tikzcd}\]
  ここで, $0 \leq i \leq n, n > 0$とする. 
  図式$\sigma \in S(m)$に対して, $X(m+1)$を次のpushoutで定義する. 
  % https://q.uiver.app/#q=WzAsNCxbMCwwLCJcXGNvcHJvZF97XFxzaWdtYSBcXGluIFMobSl9IFxcTGFtYmRhXm5faSJdLFswLDEsIlxcY29wcm9kX3tcXHNpZ21hIFxcaW4gUyhtKX0gXFxEZWx0YV5uIl0sWzEsMCwiWChtKSJdLFsxLDEsIlgobSsxKSJdLFswLDEsIiIsMSx7InN0eWxlIjp7InRhaWwiOnsibmFtZSI6Imhvb2siLCJzaWRlIjoidG9wIn19fV0sWzAsMl0sWzIsM10sWzEsM10sWzMsMCwiIiwxLHsic3R5bGUiOnsibmFtZSI6ImNvcm5lciJ9fV1d
  \[\begin{tikzcd}
    {\coprod_{\sigma \in S(m)} \Lambda^n_i} & {X(m)} \\
    {\coprod_{\sigma \in S(m)} \Delta^n} & {X(m+1)}
    \arrow[hook, from=1-1, to=2-1]
    \arrow[from=1-1, to=1-2]
    \arrow[from=1-2, to=2-2]
    \arrow[from=2-1, to=2-2]
    \arrow["\lrcorner"{anchor=center, pos=0.125, rotate=180}, draw=none, from=2-2, to=1-1]
  \end{tikzcd}\]
  また, $f(m+1) : X(m+1) \to Y$を$X(m)$への制限が$f(m)$に等しく, 各$\Delta^n$への制限が$u_\sigma$に等しい一意な射とする. 
  $X(m)$の定義より, 次の緩射の列が存在する.
  \begin{align*}
    X=X(0) \hookrightarrow X(1) \hookrightarrow X(2) \hookrightarrow \cdots
  \end{align*}
  ここで, $Q(f) := \colim_{m} X(m)$とする. 
  緩射の集まりは超限合成で閉じるので, 自然な射$f' : X \to Q(f)$も緩射である. 
  また, ある射$f'' : Q(f) \to Y$が一意に存在して, 次の図式を可換にする. 
  % https://q.uiver.app/#q=WzAsNSxbMCwwLCJYPVgoMCkiXSxbMSwwLCJYKDEpIl0sWzIsMCwiXFxjZG90cyJdLFszLDAsIlEoZik9IFxcY29saW1fbSBYKG0pIl0sWzIsMSwiWSJdLFswLDEsIiIsMix7InN0eWxlIjp7InRhaWwiOnsibmFtZSI6Imhvb2siLCJzaWRlIjoidG9wIn19fV0sWzEsMl0sWzIsM10sWzAsNCwiZiIsMl0sWzMsNCwiZicnIl0sWzAsMywiZiciLDAseyJjdXJ2ZSI6LTJ9XV0=
  \[\begin{tikzcd}
    {X=X(0)} & {X(1)} & \cdots & {Q(f)= \colim_m X(m)} \\
    && Y
    \arrow[hook, from=1-1, to=1-2]
    \arrow[from=1-2, to=1-3]
    \arrow[from=1-3, to=1-4]
    \arrow["f"', from=1-1, to=2-3]
    \arrow["{f''}", from=1-4, to=2-3]
    \arrow["{f'}", curve={height=-12pt}, from=1-1, to=1-4]
  \end{tikzcd}\]
  構成より, $f \mapsto Q(f)$は関手的であり, フィルター付き余極限と交換する. 
  後は, $f'' : Q(f) \to Y$がKanファイブレーションであることを示せばよい. 
  つまり, 次の図式が解決を持つことを示せばよい. 
  % https://q.uiver.app/#q=WzAsNCxbMCwwLCJcXExhbWJkYV5uX2kiXSxbMSwwLCJRKGYpIl0sWzEsMSwiWSJdLFswLDEsIlxcRGVsdGFebiJdLFswLDEsInYiXSxbMSwyLCJmJyciXSxbMCwzLCIiLDAseyJzdHlsZSI6eyJ0YWlsIjp7Im5hbWUiOiJob29rIiwic2lkZSI6InRvcCJ9fX1dLFszLDJdLFszLDEsIiIsMSx7InN0eWxlIjp7ImJvZHkiOnsibmFtZSI6ImRhc2hlZCJ9fX1dXQ==
  \[\begin{tikzcd}
    {\Lambda^n_i} & {Q(f)} \\
    {\Delta^n} & Y
    \arrow["v", from=1-1, to=1-2]
    \arrow["{f''}", from=1-2, to=2-2]
    \arrow[hook, from=1-1, to=2-1]
    \arrow[from=2-1, to=2-2]
    \arrow[dashed, from=2-1, to=1-2]
  \end{tikzcd}\]
  $\Lambda^n_i$は有限単体的集合なので, $v$の像は十分大きな$m \gg 0$に対して, $X(m)$の像に含まれる. 
  つまり, 次の図式は可換である. 
  % https://q.uiver.app/#q=WzAsMyxbMCwwLCJcXExhbWJkYV5uX2kiXSxbMiwwLCJRKGYpIl0sWzEsMSwiWChtKSJdLFswLDEsInYiXSxbMCwyLCJ2JyIsMl0sWzIsMV1d
  \[\begin{tikzcd}
    {\Lambda^n_i} && {Q(f)} \\
    & {X(m)}
    \arrow["v", from=1-1, to=1-3]
    \arrow["{v'}"', from=1-1, to=2-2]
    \arrow[from=2-2, to=1-3]
  \end{tikzcd}\]
  次の図式は$S(m)$の元である. 
  % https://q.uiver.app/#q=WzAsNCxbMCwwLCJcXExhbWJkYV5uX2kiXSxbMSwwLCJYKG0rMSkiXSxbMSwxLCJZIl0sWzAsMSwiXFxEZWx0YV5uIl0sWzAsMSwidiJdLFsxLDIsImYobSsxKSJdLFswLDMsIiIsMCx7InN0eWxlIjp7InRhaWwiOnsibmFtZSI6Imhvb2siLCJzaWRlIjoidG9wIn19fV0sWzMsMl0sWzMsMSwiIiwxLHsic3R5bGUiOnsiYm9keSI6eyJuYW1lIjoiZGFzaGVkIn19fV1d
  \[\begin{tikzcd}
    {\Lambda^n_i} & {X(m)} \\
    {\Delta^n} & Y
    \arrow["{v'}", from=1-1, to=1-2]
    \arrow["{f(m)}", from=1-2, to=2-2]
    \arrow[hook, from=1-1, to=2-1]
    \arrow[from=2-1, to=2-2]
    \arrow[dashed, from=2-1, to=1-2]
  \end{tikzcd}\]
  よって, 次の図式は可換である. 
  % https://q.uiver.app/#q=WzAsNyxbMCwwLCJcXExhbWJkYV5uX2kiXSxbMSwwLCJcXGNvcHJvZF97XFxzaWdtYSBcXGluIFMobSl9IFxcTGFtYmRhXm5faSJdLFswLDIsIlxcRGVsdGFebiJdLFsyLDAsIlgobSkiXSxbMywwLCJYKG0rMSkiXSxbMiwxLCJcXGNvcHJvZF97XFxzaWdtYSBcXGluIFMobSl9IFxcRGVsdGFebiJdLFszLDIsIlkiXSxbMCwxLCIiLDAseyJzdHlsZSI6eyJ0YWlsIjp7Im5hbWUiOiJob29rIiwic2lkZSI6InRvcCJ9fX1dLFswLDIsIiIsMix7InN0eWxlIjp7InRhaWwiOnsibmFtZSI6Imhvb2siLCJzaWRlIjoidG9wIn19fV0sWzEsM10sWzMsNF0sWzIsNSwiIiwyLHsic3R5bGUiOnsidGFpbCI6eyJuYW1lIjoiaG9vayIsInNpZGUiOiJ0b3AifX19XSxbNSw0XSxbNCw2LCJmKG0rMSkiXSxbMiw2XSxbMSw1XV0=
  \[\begin{tikzcd}
    {\Lambda^n_i} & {\coprod_{\sigma \in S(m)} \Lambda^n_i} & {X(m)} & {X(m+1)} \\
    && {\coprod_{\sigma \in S(m)} \Delta^n} \\
    {\Delta^n} &&& Y
    \arrow[hook, from=1-1, to=1-2]
    \arrow[hook, from=1-1, to=3-1]
    \arrow[from=1-2, to=1-3]
    \arrow[from=1-3, to=1-4]
    \arrow[hook, from=3-1, to=2-3]
    \arrow[from=2-3, to=1-4]
    \arrow["{f(m+1)}", from=1-4, to=3-4]
    \arrow[from=3-1, to=3-4]
    \arrow[from=1-2, to=2-3]
  \end{tikzcd}\]
  pushoutの普遍性より, この図式は可換であり, 解決を持つことが分かる. 
  よって, 元の図式も解決を持つ. 
\end{proof}

$Y = \Delta^0$とすると, 次の系を得る. 

\begin{corollary}
  $X$を単体的集合とする. 
  このとき, あるKan複体$Q$と緩射$f : X \to Q$が存在する.
\end{corollary}
  
\begin{remark}
  系3.1.7.2において, Kan複体$Q$と緩射$f'$は関手性による. (命題3.1.7.1の証明から分かる.)
  この命題はほかの方法で示すこともできる. 
  例えば, $Q$を構成3.3.6.1で得られる$\Ex^\infty(X)$や特異単体的集合$\Sing(|X|)$とすることができる. (命題3.3.6.9と定理3.5.4.1)
  この構成は関手$X \mapsto \Ex(X), X \mapsto \Sing(|X|)$を定め, ともに有限極限と交換する. 
\end{remark}

\begin{corollary}
  $f : X \to Y$を単体的集合の射とする.
  このとき, 次の2つは同値である. 
  \begin{enumerate}
    \item $f$は緩射である. 
    \item $f$はKanファイブレーションに対してLLPを持つ.
  \end{enumerate}
\end{corollary}

\begin{proof}
  (1)から(2)は注意3.1.2.7より従う. 

  (2)から(1)を示す.
  命題3.1.7.1より, 緩射$f' : X \to Q$とKanファイブレーション$f'' : Q \to Y$を用いて, $f$は$X \xrightarrow{f'} Q \xrightarrow{f''} Y$と分解できる. 
  $f$が(2)を満たすとき, 次のリフト問題は解決$h : Y \to Q$を持つ.
  % https://q.uiver.app/#q=WzAsNCxbMCwwLCJYIl0sWzEsMCwiUSJdLFsxLDEsIlkiXSxbMCwxLCJZIl0sWzAsMSwiZiciXSxbMSwyLCJmJyciXSxbMCwzLCJmIiwyXSxbMywyLCJcXGlkX1kiLDJdLFszLDEsImgiLDIseyJzdHlsZSI6eyJib2R5Ijp7Im5hbWUiOiJkYXNoZWQifX19XV0=
  \[\begin{tikzcd}
    X & Q \\
    Y & Y
    \arrow["{f'}", from=1-1, to=1-2]
    \arrow["{f''}", from=1-2, to=2-2]
    \arrow["f"', from=1-1, to=2-1]
    \arrow["{\id_Y}"', from=2-1, to=2-2]
    \arrow["h"', dashed, from=2-1, to=1-2]
  \end{tikzcd}\]
  このとき, 次の図式は可換なので, $f$は$f'$のレトラクトである. 
  % https://q.uiver.app/#q=WzAsNixbMCwwLCJYIl0sWzEsMCwiWCJdLFsyLDAsIlgiXSxbMiwxLCJZIl0sWzAsMSwiWSJdLFsxLDEsIlEiXSxbMCwxLCJcXGlkX1giXSxbMSwyLCJcXGlkX1giXSxbMiwzLCJmIl0sWzAsNCwiZiJdLFs0LDUsImgiLDJdLFs1LDMsImYnJyIsMl0sWzEsNSwiZiciXV0=
  \[\begin{tikzcd}
    X & X & X \\
    Y & Q & Y
    \arrow["{\id_X}", from=1-1, to=1-2]
    \arrow["{\id_X}", from=1-2, to=1-3]
    \arrow["f", from=1-3, to=2-3]
    \arrow["f", from=1-1, to=2-1]
    \arrow["h"', from=2-1, to=2-2]
    \arrow["{f''}"', from=2-2, to=2-3]
    \arrow["{f'}", from=1-2, to=2-2]
  \end{tikzcd}\]
  緩射の集まりはレトラクトで閉じるので, $f$は緩射である. 
\end{proof}

\begin{corollary}
  $f : X \to Y$を単体的集合の射, $Z$をKan複体とする.
  $f$が弱同値のとき, $f$の合成はホモトピー同値$\Fun(Y,Z) \to \Fun(X,Z)$を定める.
\end{corollary}

% \begin{proof}
%   注意3.1.6.6より, 任意の単体的集合$A$に対して, $\theta : \Fun(A,\Fun(Y,Z)) \to \Fun(A,\Fun(X,Z))$が連結成分において全単射を定めることを示せばよい.
%   直積-Fun随伴より, $\theta$は$\Fun(Y,\Fun(A,Z)) \to \Fun(X,\Fun(A,Z))$と同一視できる. 
%   系3.1.3.4より, $Z$がKan複体のとき, $\Fun(A,Z)$はKan複体である. 
%   よって, $f$は弱同値である. 
% \end{proof}

構成3.1.5.10のKan複体のホモトピー圏$\hkan$はKan複体の圏$\kan$の(ホモトピックな射を同一視する)商圏で定義された. 
しかし, $\hkan$は$\kan$の弱同値による局所化としても定義できる. (6章3節)

\begin{proposition}
  $\C$を圏, $F : \kan \to \C$を関手とする. 
  このとき, 次の2つは同値である. 
  \begin{enumerate}
    \item $X,Y$をKan複体, $u_0,u_1 : X \to Y$をホモトピックな射とする. 
    このとき, $\Hom_\C(FX,FY)$において, $Fu_0=Fu_1$である. 
    \item Kan複体のホモトピー同値$u : X \to Y$に対して, $\C$における射$Fu : FX \to FY$は同型射である. 
  \end{enumerate}
\end{proposition}

\begin{proof}
  (2)から(1)を示す. 
  $X,Y$をKan複体, $u_0,u_1 : X \to Y$をホモトピックな射とする. 
  $u_0,u_1$を$\Fun(X,Y)$における点とみなす. 
  $u_0$と$u_1$はホモトピックなので, ある辺$e : \Delta^1 \to \Fun(X,Y)$が存在して, $e(0) = u_0, e(1) = u_1$を満たす. 
  命題3.1.7.1より, 緩射$e' : \Delta^1 \to Q$とKanファイブレーション$e'' : Q \to \Fun(X,Y)$が存在して, $e$は$\Delta^1 \xrightarrow{e'} Q \xrightarrow{e''} \Fun(X,Y)$と分解できる. 
  $Y$はKan複体なので, 命題3.1.3.4より, $\Fun(X,Y)$もKan複体である. 
  $Q$もKan複体なので, $e''$はKan複体の射$h : Q \times X \to Y$と同一視できる.
  ここで, 包含$i_0 : X \to Q \times X$を$\id_X$と包含$\{e'(0)\} \to Q$の直積とする. 
  同様に, 包含$i_1 : X \to Q \times X$を$\id_X$と包含$\{e'(1)\} \to Q$の直積とする. 
  % https://q.uiver.app/#q=WzAsNixbMCwwLCJcXERlbHRhXjEiXSxbMiwwLCJcXEZ1bihYLFkpIl0sWzEsMSwiUSJdLFszLDAsIlgiXSxbNSwwLCJZIl0sWzQsMSwiUSBcXHRpbWVzIFgiXSxbMCwxLCJlIl0sWzAsMiwiZSciLDJdLFsyLDEsImUnJyIsMl0sWzMsNCwidSJdLFszLDUsImkiLDAseyJvZmZzZXQiOi0xfV0sWzUsNCwiaCIsMl0sWzUsMywicCIsMCx7Im9mZnNldCI6LTF9XV0=
  \[\begin{tikzcd}
    {\Delta^1} && {\Fun(X,Y)} & X && Y \\
    & Q &&& {Q \times X}
    \arrow["e", from=1-1, to=1-3]
    \arrow["{e'}"', from=1-1, to=2-2]
    \arrow["{e''}"', from=2-2, to=1-3]
    \arrow["u", from=1-4, to=1-6]
    \arrow["i", shift left, from=1-4, to=2-5]
    \arrow["h"', from=2-5, to=1-6]
    \arrow["p", shift left, from=2-5, to=1-4]
  \end{tikzcd}\]
  $\Lambda^1_0 \cong \Delta^0$なので, 緩射の定義から, 制限$e'|_{\{0\}}, e'|_{\{1\}}$は緩射である. 
  命題3.1.6.14より, $e'|_{\{0\}}, e'|_{\{1\}}$は弱ホモトピー同値である. 
  命題3.6.1.13より, $\Delta^0$と$Q$はKan複体なので, これらはホモトピー同値である. 
  仮定より, $F(i_0)$と$F(i_1)$は$\C$における同型射である. 
  $i_0$と$i_1$は射影$\pi : Q \times X \to Q$の左ホモトピーなので, $F(\pi)$は$\C$における同型射である(らしい). 
  よって, 
  \begin{align*}
    F(u_0) 
    = F(h) \circ F(i_0)
    = F(h) \circ F(\pi)^{-1}
    = F(h) \circ F(i_1) 
    = F(u_1)
  \end{align*}
\end{proof}

\begin{corollary}
  $\C$を圏, $\E$をKan複体のホモトピー同値を$\C$における同型射にうつす関手$F : \kan \to \C$の貼る$\Fun(\kan,\C)$の充満部分圏とする. 
  商関手$\kan \to \hkan$の前合成は圏同型$\Fun(\hkan,\C) \to \E$を定める. 
\end{corollary}

\begin{proof}
  命題3.1.7.6と注意3.1.5.11より従う. 
\end{proof}

\begin{proposition}
  $\C$を圏, $\E'$を単体的集合の弱ホモトピー同値を$\C$における同型射にうつす関手$F : \sset \to \C$の貼る$\Fun(\sset,\C)$の充満部分圏とする. 
  このとき, 次の2つが成立する. 
  \begin{enumerate}
    \item 任意の関手$F \in \E'$に対して, 制限$F|_{\kan}$は$\kan \to \hkan \to \xrightarrow{\bar{F}} \C$に分解できる. 
    \item 構成$F \mapsto \bar{F}$は圏同値$\E' \to \Fun(\hkan,\C)$を定める. 
  \end{enumerate}
\end{proposition}

\begin{proof}
  $\E$をKan複体のホモトピー同値を$\C$における同型射にうつす関手$F : \kan \to \C$の貼る$\Fun(\sset,\C)$の充満部分圏とする. 
  系3.1.7.7より, 制限$F \mapsto F'$が圏同値$\E \to \E$を定めることを示せばよい. 
  命題3.1.7.1より, ファイブラント置換$Q$と自然変換$u : \id_{\sset} \to Q$をとる. 
  任意の単体的集合の射$f : X \to Y$に対して, 次の図式は可換である. 
  % https://q.uiver.app/#q=WzAsNCxbMCwwLCJYIl0sWzEsMCwiWSJdLFsxLDEsIlEoWSkiXSxbMCwxLCJRKFgpIl0sWzAsMSwiZiJdLFsxLDIsInVfWSJdLFswLDMsInVfWCIsMl0sWzMsMiwiUShmKSIsMl1d
  \[\begin{tikzcd}
    X & Y \\
    {Q(X)} & {Q(Y)}
    \arrow["f", from=1-1, to=1-2]
    \arrow["{u_Y}", from=1-2, to=2-2]
    \arrow["{u_X}"', from=1-1, to=2-1]
    \arrow["{Q(f)}"', from=2-1, to=2-2]
  \end{tikzcd}\]
 で, 垂直な射は緩射である. 
  命題3.1.6.14より, これは弱ホモトピー同値である. 
  $f$が弱ホモトピー同値のとき, 注意3.1.6.16より, $Q(f)$も弱ホモトピー同値である. 
  $Q(X), Q(Y)$はKan複体なので, 命題3.1.6.13より, $Q(f)$はホモトピー同値である. 
  つまり, $Q$は単体的集合の弱ホモトピー同値をKan複体のホモトピー同値にうつす. 
  よって, $Q$の前合成は関手$\theta : \E \to \E'$を定める. 
  $\theta$が制限$\E \to \E$のホモトピー逆関手であることを示せばよい. 
  (途中)
\end{proof}

\begin{remark}
  系3.1.7.7と系3.1.7.8は次のように表すことができる.
  \begin{itemize}
    \item Kan複体のホモトピー圏$\hkan$はKan複体の圏$\kan$に形式的にホモトピー同値を逆射として付け加えることでできる.
    \item Kan複体のホモトピー圏$\hkan$は単体的集合の圏$\sset$に形式的に弱ホモトピー同値を逆射として付け加えることでできる.
  \end{itemize}
  これらは$\hkan$と圏同値の違いを除いて特徴づける.
  実際, 系3.1.7.7は$\hkan$と圏同型の違いを除いて特徴づけている. 
\end{remark}

命題3.1.7.1の別証明を与える. 

\begin{example}
  $f : X \to Y$をKan複体の射とする. 
  ファイバー積$X \times_{\Fun(\{0\},Y)} \Fun(\Delta^1,Y)$を$P(f)$と表す. 
  このとき, $f$は$X \xrightarrow{f'} P(f) \xrightarrow{f''} Y$と分解する.
  ここで, $f'$は$\id_X : X \to X$と合成$X \xrightarrow{f} Y \xrightarrow{\delta} \Fun(X,Y)$の組である.
  また, $f''$は$\{1\} \subset \Delta^1$での評価射で与えられる. 
  更に, 次が成立する. (途中)
\end{example}

\section{ホモトピー群}

この章の目的は次の疑問に答えることである. 

\begin{question}
  $f : X \to Y$をKan複体の射とする. 
  $f$はいつホモトピー逆射$g : Y \to X$を持つだろうか.
\end{question}

まず, 部分的に疑問3.2.0.1に答えることができる. 
任意のKan複体$X$に対して, $X$の基本亜群を$\pi_{\leq 1}(X)$と表す. (定義1.3.6.12)
任意の点$x \in X$に対して, 自己同型群$\Aut_{\pi_{\leq 1}(X)}(x) = \Hom_{\pi_{\leq 1}(X)}(x,x)$を$\pi_1(X,x)$と表し, $X$の(点$x$に対する)基本群(fundamental group)という.
任意のKan複体の射$f : X \to Y$は関手$\pi_{\leq 1}(f) : \pi_{\leq 1}(X) \to \pi_{\leq 1}(Y)$を定める. 
更に, $f$がホモトピー同値のとき, $\pi_{\leq 1}(f)$は圏同値である. (注意3.1.6.5)
つまり, 任意のホモトピー同値$f : X \to Y$に対して, 次が成立する. 
(途中)


\subsection{基点付きKan複体}

3.1.5でKan複体の集まりから, Kan複体のホモトピー圏$\hkan$を構成した. (構成3.1.5.10)
$\hkan$の射はKan複体の射のホモトピー類である.
この節では, 基点付きKan複体に対して同様の概念を考える. 

\begin{definition}[基点付き単体的集合]
  $X$を単体的集合, $x$を$X$の点とする. 
  $X$と$x$の組$(X,x)$を基点付き単体的集合(pointed simplicial set)という. 
  $X$がKan複体のとき, 組$(X,x)$を基点付きKan複体(pointed Kan complex)という. 
  $(X,x), (Y,y)$を基点付き単体的集合とする. 
  単体的集合の射$f : X \to Y$が$f(x) = y$を満たすとき, $f : (X,x) \to (Y,y)$を基点付き射(pointed map)という. 
  基点付きKan複体を対象, 基点付き射を射とする圏を$\kan_\ast$と表す.
\end{definition}

\begin{remark}
  基点付き単体的集合$(X,x)$に対して, $X$を基点付き単体的集合, $x$を基点(base point)ということもある. 
\end{remark}

\begin{definition}[基点付きホモトピック]
  $(X,x), (Y,y)$を基点付き単体的集合とする.
  基点付き射$f,g : X \to Y$を単体的集合$\Fun(X,Y) \times_{\Fun(\{x\},Y)} \{y\}$の点とみなす. 
  $f$と$g$が$\Fun(X,Y) \times_{\Fun(\{x\},Y)} \{y\}$の同じ連結成分に属するとき, $f$と$g$は基点付きホモトピック(pointed homotopic)であるという. 
\end{definition}

\begin{definition}
  $(X,x), (Y,y)$を基点付き単体的集合, $f,g : X \to Y$を基点付き射とする. 
  単体的集合の射$h : \Delta^1 \times X \to Y$が$f = h|_{\{0\} \times X}, g = h|_{\{1\} \times Y}$かつ, $h|_{\Delta^1 \times \{x\}}$が$y$の退化する辺であるとする.
  このとき, $h$を$f$から$g$への基点付きホモトピー(pointed homotopy)という. 
\end{definition}

\begin{proposition}
  $(X,x), (Y,y)$を基点付き単体的集合, $f,g : X \to Y$を基点付き射とする. 
  このとき, 次が成立する. 
  \begin{enumerate}
    \item $f$と$g$がホモトピックであることと, 基点付き射の列$(f=f_0,f_1,\cdots,f_n=g)$が存在して, 任意の$1 \leq i \leq n$に対して, $f_{i-1}$から$f_i$への基点付きホモトピーか$f_i$から$f_{i-1}$への基点付きホモトピーが存在する. 
    \item $Y$をKan複体とする. 
    $f$と$g$が基点付きホモトピックであることと, $f$から$g$への基点付きホモトピーが存在することは同値である.
  \end{enumerate}
\end{proposition}

\begin{proof}
  (1)は注意1.1.6.23において, $\Fun(X,Y) \times_{\Fun(\{x\},Y)} \{y\}$を考えればよい.
  (2)を示す. 
  $Y$をKan複体とする. 
  系3.1.3.3より, 評価射$\Fun(X,Y) \to \Fun(\{x\},Y)$はKanファイブレーションである. 
  注意3.1.1.9より, $\Fun(X,Y) \times_{\Fun(\{x\},Y)} \{y\}$はKan複体である. 
  よって, 命題1.1.9.10において, $\Fun(X,Y) \times_{\Fun(\{x\},Y)} \{y\}$を考えればよい. 
\end{proof}

\begin{example}
  $(X,x)$を基点付き単体的集合, $(Y,y)$を基点付き位相空間, $f_0,f_1 : |X| \to Y$を$x$を$y$にうつす連続写像とする. 
  $f_0',f_1' : X \to \Sing(Y)$を随伴で対応する基点付き射とする. 
  $h : [0,1] \times X \to Y$を$f_0$から$f_1$への基点を保つホモトピーとする. 
  このとき, 合成
  \begin{align*}
    |\Delta^1 \times X| \xrightarrow{\theta} |\Delta^1| \times |X| \cong [0,1] \times |X| \to Y
  \end{align*}
  は随伴性から$f_0'$から$f_1'$への(定義3.2.1.4の意味の)基点付きホモトピー$h' : \Delta^1 \times X \to \Sing(Y)$を定める. 
  系3.5.2.2より, $\theta$は同相写像なので, 任意の$f_0$から$f_1$への基点を保つホモトピーは$f_0'$から$f_1'$への基点付きホモトピーを定める.
\end{example}


\section{$\Ex^\infty$関手}


\section{ホモトピープルバックとホモトピープッシュアウト}

単体的集合の圏は任意の極限と余極限を持つ. (注意1.1.1.13)
特に, 単体的集合の図式$X_0 \to X \gets X_1$に対して, ファイバー積$X_0 \times_X X_1$が得られる. 
しかし, この構成は弱ホモトピー同値によって保たれない. 

\begin{remark}
  次の単体的集合の可換図式を考える. 
  % https://q.uiver.app/#q=WzAsNixbMCwwLCJYXzAiXSxbMSwwLCJYIl0sWzIsMCwiWF8xIl0sWzEsMSwiWSJdLFswLDEsIllfMCJdLFsyLDEsIllfMSJdLFswLDFdLFsyLDFdLFsxLDMsIlxcc2ltIl0sWzAsNCwiXFxzaW0iXSxbNCwzXSxbNSwzXSxbMiw1LCJcXHNpbSJdXQ==
  \[\begin{tikzcd}
    {X_0} & X & {X_1} \\
    {Y_0} & Y & {Y_1}
    \arrow[from=1-1, to=1-2]
    \arrow[from=1-3, to=1-2]
    \arrow["\sim", from=1-2, to=2-2]
    \arrow["\sim", from=1-1, to=2-1]
    \arrow[from=2-1, to=2-2]
    \arrow[from=2-3, to=2-2]
    \arrow["\sim", from=1-3, to=2-3]
  \end{tikzcd}\]
  ここで, 垂直な射は弱同値とする. 
  このとき, ファイバー積の間の射
  \begin{align*}
    X_0 \times_X X_1 \to Y_0 \times_Y Y_1
  \end{align*}
  が弱ホモトピー同値とは限らない. 
  例えば, 次の単体的集合の可換図式を考える. 
  % https://q.uiver.app/#q=WzAsNixbMCwwLCJcXHswXFx9Il0sWzEsMCwiXFxEZWx0YV4xIl0sWzIsMCwiXFx7MVxcfSJdLFsxLDEsIlxcRGVsdGFeMCJdLFswLDEsIlxcezBcXH0iXSxbMiwxLCJcXHsxXFx9Il0sWzAsMV0sWzIsMV0sWzEsMywiXFxzaW0iXSxbMCw0LCIiLDAseyJsZXZlbCI6Miwic3R5bGUiOnsiaGVhZCI6eyJuYW1lIjoibm9uZSJ9fX1dLFs0LDMsIlxcY29uZyIsMl0sWzUsMywiXFxjb25nIl0sWzIsNSwiIiwwLHsibGV2ZWwiOjIsInN0eWxlIjp7ImhlYWQiOnsibmFtZSI6Im5vbmUifX19XV0=
  \[\begin{tikzcd}
    {\{0\}} & {\Delta^1} & {\{1\}} \\
    {\{0\}} & {\Delta^0} & {\{1\}}
    \arrow[from=1-1, to=1-2]
    \arrow[from=1-3, to=1-2]
    \arrow["\sim", from=1-2, to=2-2]
    \arrow[Rightarrow, no head, from=1-1, to=2-1]
    \arrow["\cong"', from=2-1, to=2-2]
    \arrow["\cong", from=2-3, to=2-2]
    \arrow[Rightarrow, no head, from=1-3, to=2-3]
  \end{tikzcd}\]
  上の図式のファイバー積は空集合だが, 下の図式のファイバー積は$\Delta^0$と同型である. 
  空集合と$\Delta^0$は弱ホモトピー同値でない. 
\end{remark}

条件を緩めることで, 注意3.4.0.2を回避することができる. 

\begin{proposition}
  次の単体的集合の可換図式を考える. 
  % https://q.uiver.app/#q=WzAsNixbMCwwLCJYXzAiXSxbMSwwLCJYIl0sWzIsMCwiWF8xIl0sWzEsMSwiWSJdLFswLDEsIllfMCJdLFsyLDEsIllfMSJdLFswLDEsImYiXSxbMiwxXSxbMSwzLCJcXHNpbSJdLFswLDQsIlxcc2ltIl0sWzQsMywiZiciLDJdLFs1LDNdLFsyLDUsIlxcc2ltIl1d
  \[\begin{tikzcd}
    {X_0} & X & {X_1} \\
    {Y_0} & Y & {Y_1}
    \arrow["f", from=1-1, to=1-2]
    \arrow[from=1-3, to=1-2]
    \arrow["\sim", from=1-2, to=2-2]
    \arrow["\sim", from=1-1, to=2-1]
    \arrow["{f'}"', from=2-1, to=2-2]
    \arrow[from=2-3, to=2-2]
    \arrow["\sim", from=1-3, to=2-3]
  \end{tikzcd}\]
  ここで, 垂直な射は弱同値とする.
  $f,f'$がKanファイブレーションのとき, 誘導される射$X_0 \times_X X_1 \to Y_0 \times_Y Y_1$は弱ホモトピー同値である. 
\end{proposition}

\begin{proof}
  Kanファイブレーションはプルバックで閉じるので, 次の図式において, 垂直な射はKanファイブレーションである. 
  % https://q.uiver.app/#q=WzAsNCxbMCwwLCJYXzAgXFx0aW1lc19YIFhfMSJdLFsxLDAsIllfMCBcXHRpbWVzX1kgWV8xIl0sWzEsMSwiWV8xIl0sWzAsMSwiWF8xIl0sWzAsMV0sWzEsMl0sWzAsM10sWzMsMl1d
  \[\begin{tikzcd}
    {X_0 \times_X X_1} & {Y_0 \times_Y Y_1} \\
    {X_1} & {Y_1}
    \arrow[from=1-1, to=1-2]
    \arrow[from=1-2, to=2-2]
    \arrow[from=1-1, to=2-1]
    \arrow["\sim", from=2-1, to=2-2]
  \end{tikzcd}\]
  命題3.3.7.1より, 任意の$X$の点$x$に対して, ファイバー積の間の誘導される射
  \begin{align*}
    X_0 \times_X \{x\} \to Y_0 \times_Y \{y\}
  \end{align*}
  がKan複体のホモトピー同値であることを示せばよい. 
  次の図式において, 命題3.3.7.1より, $X_0 \times_X \{x\} \to Y_0 \times_Y \{y\}$はホモトピー同値である. 
  % https://q.uiver.app/#q=WzAsNCxbMCwwLCJYXzAiXSxbMSwwLCJZXzAiXSxbMSwxLCJZIl0sWzAsMSwiWCJdLFswLDEsIlxcc2ltIl0sWzEsMiwiZiciXSxbMCwzLCJmIiwyXSxbMywyLCJcXHNpbSIsMl1d
  \[\begin{tikzcd}
    {X_0} & {Y_0} \\
    X & Y
    \arrow["\sim", from=1-1, to=1-2]
    \arrow["{f'}", from=1-2, to=2-2]
    \arrow["f"', from=1-1, to=2-1]
    \arrow["\sim"', from=2-1, to=2-2]
  \end{tikzcd}\]
\end{proof}

疑問3.4.0.1をより一般的に考えるために, ファイバー積をホモトピーによって保たれる対象に変えることを考える. 

\begin{construction}[ホモトピーファイバー積]
  $X$をKan複体, $f_0 : X_0 \to X, f_1 : X_1 \to X$を単体的集合の射とする. 
  単体的集合 
  \begin{align*}
    X_0 \times_X^h X_1 
    := X_0 \times_{\Fun(\{0\},X)} \Fun(\Delta^1,X) \times_{\Fun(\{1\},X)} X_1
  \end{align*}
  を$X$上の$X_0$と$X_1$のホモトピーファイバー積(homotopy fiber product)という.
\end{construction}

\begin{remark}
  単体的集合の図式$X_0 \to X \gets X_1$に対して, 単体的集合$X_0 \times_{\Fun(\{0\},X)} \Fun(\Delta^1,X) \times_{\Fun\{1\},X} X_1$はwell-definedである. 
  しかし, $X$がKan複体のときのみ, これを$X_0 \times_X^h X_1$と表し, ホモトピーファイバー積という. 
  $X$が一般の単体的集合のとき, これを$X_0 \tilde{\times}_X X_1$と表し, 向き付きファイバー積という. (定義4.6.4.1)
  $X$が$\infty$圏のとき, ホモトピーファイバー積の定義はわずかに異なる. (構成4.5.2.1)
\end{remark}

\begin{example}
  $f_0 : X_0 \to X, f_1 : X_1 \to X$を位相空間の連続写像とする. 
  $x_0$を$X_0$の点, $x_1$を$X_1$の点, $p : [0,1] \to X$を$p(0)=f_0(x_0)$と$p(1)=f_1(x_1)$を満たす連続写像とする. 
  この3つ組$(x_0,x_1,p)$を$X_0 \times_X^h X_1$と表し, 位相空間の$X$上の$X_0$と$X_1$のホモトピーファイバー積(homotopy fiber product)という. 
  特異単体$\Sing$は右随伴であり, 極限と交換するので,
  \begin{align*}
    \Sing(X_0 \times_X^h X_1) \cong \Sing(X_0) \times_{\Sing(X)}^h \Sing(X_1)
  \end{align*}
  となり, 右辺は構成3.4.0.3の意味のKan複体のホモトピーファイバー積である.
\end{example}

\begin{remark}
  $f : X \to Y$をKan複体の射とする. 
  このとき, $f$がホモトピー同値であることと, 任意の$y \in Y$に対してホモトピーファイバー積$X \times_X^h \{y\}$が可縮なKan複体であることは同値である.
  実際, 例3.1.7.10より, $f$は次のように分解できる. ($f_0=f : X \to Y, f1=\id_Y : Y \to Y$とすればよい.)
  \begin{align*}
    X \xrightarrow{\delta} X \times_Y^h \xrightarrow{\pi} Y 
  \end{align*}
  ここで, $\delta$はホモトピー同値, $\pi$はKanファイブレーションである. 
  2-out-of-3より, $f$がホモトピー同値であることと, $\pi$がホモトピー同値であることは同値である. 
  命題3.3.7.4より, これは各ファイバー積$\pi^{-1}\{y\}=X \times_X^h Y$が可縮なKan複体であることは同値である. 
\end{remark}

構成3.4.0.3において, ホモトピーファイバー積の定義より, 包含
\begin{align*}
  X \hookrightarrow \Fun(\Delta^1,X) : x \mapsto \id_x
\end{align*}
は通常のファイバー積$X_0 \times_X X_1$からホモトピーファイバー積$X_0 \times_X^h X_1$へのmono射を定める. 

\begin{proposition}
  $f_0 : X_0 \to X, f_1 : X_1 \to X$を単体的集合の射とする.
  $X$をKan複体, $f_0$か$f_1$のいずれかがKanファイブレーションであるとする. 
  このとき, $X_0 \times_X X_1 \to X_0 \times_X^h X_1$は弱ホモトピー同値である.
\end{proposition}

\begin{proof}
  $f_0$がKanファイブレーションであるとする. 
  系3.1.3.6より, 評価射$\ev_0 : \Fun(\Delta^1,X) \to \Fun(\{1\},X)$は自明なKanファイブレーションである. 
  自明なKanファイブレーションはpullbackで閉じるので, $q : \Fun(\Delta^1,X) \times_{\Fun(\{1\},X)} X_1 \to X_1$も自明なKanファイブレーションである.
  対角射$\delta : X \hookrightarrow \Fun(\Delta^1,X)$は$q$の切断$s : X_1 \to \Fun(\Delta^1,X) \times_{\Fun(\{1\},X)} X_1$を定める. 
  弱ホモトピー同値は切断で閉じるので, $s$も弱ホモトピー同値である. 
  次の図式において, 命題3.4.0.2を用いると, $X_0 \times_X X_1 \to X_0 \times_X^h X_1$は弱ホモトピー同値であることが分かる.
  % https://q.uiver.app/#q=WzAsNixbMCwwLCJYXzAiXSxbMSwwLCJYIl0sWzEsMSwiWCJdLFswLDEsIlhfMCJdLFsyLDAsIlhfMSJdLFsyLDEsIlxcRnVuKFxcRGVsdGFeMSxYKSBcXHRpbWVzX3tcXEZ1bihcXHsxXFx9LFgpfSBYXzEiXSxbMCwxLCJmXzAiXSxbMSwyLCIiLDAseyJsZXZlbCI6Miwic3R5bGUiOnsiaGVhZCI6eyJuYW1lIjoibm9uZSJ9fX1dLFswLDMsIiIsMix7ImxldmVsIjoyLCJzdHlsZSI6eyJoZWFkIjp7Im5hbWUiOiJub25lIn19fV0sWzMsMiwiZl8wIiwyXSxbNCwxXSxbNCw1LCJzIl0sWzUsMl1d
  \[\begin{tikzcd}
    {X_0} & X & {X_1} \\
    {X_0} & X & {\Fun(\Delta^1,X) \times_{\Fun(\{1\},X)} X_1}
    \arrow["{f_0}", from=1-1, to=1-2]
    \arrow[Rightarrow, no head, from=1-2, to=2-2]
    \arrow[Rightarrow, no head, from=1-1, to=2-1]
    \arrow["{f_0}"', from=2-1, to=2-2]
    \arrow[from=1-3, to=1-2]
    \arrow["s", from=1-3, to=2-3]
    \arrow[from=2-3, to=2-2]
  \end{tikzcd}\]
\end{proof}

\begin{remark}
  命題3.4.0.7において, $f_0$か$f_1$のいずれかがKanファイブレーションであるという仮定は必要である. 
  例えば, 次のような反例がある. 
  $x,y$を$X$を点とする. 
  $x \neq y$のとき, ファイバー積$\{x\} \times_X \{y\}$は空集合である. 
  一方, ホモトピーファイバー積$\{x\} \times_X^h \{y\}$は空であるとは限らない. 
  $\{x\} \times_X^h \{y\}$の点として, domainが$x$, codomainが$y$である$\Fun(\Delta^1,X)$辺$p : x \to y$がある. 
\end{remark}

一般に, mono射$X_0 \times_X X_1 \hookrightarrow X_0 \times_X^h X_1$が弱同値とならないのは, ホモトピーファイバー積のデメリットではなく, 特徴と考えるべきである. 
ホモトピー論の視点からすると, ホモトピーファイバー積は通常のファイバー積より良いふるまいをする. 

\begin{proposition}
  $X,Y$をKan複体とする. 
  次の単体的集合の可換図式を考える. 
  % https://q.uiver.app/#q=WzAsNixbMCwwLCJYXzAiXSxbMSwwLCJYIl0sWzIsMCwiWF8xIl0sWzEsMSwiWSJdLFswLDEsIllfMCJdLFsyLDEsIllfMSJdLFswLDEsImYiXSxbMiwxXSxbMSwzLCJcXHNpbSJdLFswLDQsIlxcc2ltIl0sWzQsMywiZiciLDJdLFs1LDNdLFsyLDUsIlxcc2ltIl1d
  \[\begin{tikzcd}
    {X_0} & X & {X_1} \\
    {Y_0} & Y & {Y_1}
    \arrow[from=1-1, to=1-2]
    \arrow[from=1-3, to=1-2]
    \arrow["\sim", from=1-2, to=2-2]
    \arrow["\sim", from=1-1, to=2-1]
    \arrow[from=2-1, to=2-2]
    \arrow[from=2-3, to=2-2]
    \arrow["\sim", from=1-3, to=2-3]
  \end{tikzcd}\]
  ここで, 垂直な射は弱同値とする.
  このとき, 誘導される射$X_0 \times^h_X X_1 \to Y_0 \times^h_Y Y_1$は弱ホモトピー同値である. 
\end{proposition}

\begin{proof}
  次の図式を考える. 
  % https://q.uiver.app/#q=WzAsNixbMCwwLCJcXEZ1bihcXERlbHRhXjEsWCkiXSxbMSwwLCJcXEZ1bihcXHBhcnRpYWwgXFxEZWx0YV4xLFgpIl0sWzIsMCwiWF8wIFxcdGltZXMgWF8xIl0sWzEsMSwiXFxGdW4oXFxwYXJ0aWFsIFxcRGVsdGFeMSxZKSJdLFswLDEsIlxcRnVuKFxcRGVsdGFeMSxZKSJdLFsyLDEsIllfMCBcXHRpbWVzIFlfMSJdLFswLDFdLFsyLDFdLFsxLDMsIlxcc2ltIl0sWzAsNCwiXFxzaW0iXSxbNCwzXSxbNSwzXSxbMiw1LCJcXHNpbSJdXQ==
  \[\begin{tikzcd}
    {\Fun(\Delta^1,X)} & {\Fun(\partial \Delta^1,X)} & {X_0 \times X_1} \\
    {\Fun(\Delta^1,Y)} & {\Fun(\partial \Delta^1,Y)} & {Y_0 \times Y_1}
    \arrow[from=1-1, to=1-2]
    \arrow[from=1-3, to=1-2]
    \arrow["\sim", from=1-2, to=2-2]
    \arrow["\sim", from=1-1, to=2-1]
    \arrow[from=2-1, to=2-2]
    \arrow[from=2-3, to=2-2]
    \arrow["\sim", from=1-3, to=2-3]
  \end{tikzcd}\]
  系3.1.3.3より, 左の水平な射はKanファイブレーションである. 
  垂直な射は弱同値なので, 命題3.4.0.2より, $X_0 \times^h_X X_1 \to Y_0 \times^h_Y Y_1$は弱ホモトピー同値である. 
\end{proof}

\begin{remark}
  $X$をKan複体, $f_0 : X_0 \to X, f_1 : X_1 \to X$を単体的集合の射とする.
  一般に, 2つのホモトピーファイバー積$X_0 \times^h_X X_1$と$X_1 \times^h_X X_0$は単体的集合として同型ではない. 
  実際, 次の同型は存在する.
  \begin{align*}
    (X_1 \times^h_X X_0)^\myop
    \cong X_0^\myop \times^h_{X_0^\myop} X_1^\myop
  \end{align*}
  しかし, $X_0 \times^h_X X_1$と$X_1 \times^h_X X_0$は同じホモトピー型を持たない. 
  命題3.1,7,1より, $f_0$は弱同値$w : X_0 \to X_0'$とKanファイブレーション$f_0' : X_0' \to X$を用いて, $X_0 \xrightarrow{w} X_0' \xrightarrow{f_0'} X$のように分解できる. 
  命題3.4.0.7より, 
  \begin{align*}
    X_0' \times_X X_1 \hookrightarrow X_0' \times^h_X X_1,~ X_1 \times_X X_0' \hookrightarrow X_1 \times^h_X X_0'
  \end{align*}
  は弱同値である. 
  命題3.4.0.9より, 
  \begin{align*}
    X_0 \times_X X_1 \to  X_0' \times_X X_1,~ X_1 \times^h_X X_0 \to X_1 \times^h_X X_0'
  \end{align*}
  は弱同値である.
  よって, 
  \begin{align*}
    X_0 \times_X X_1 \xrightarrow{\sim} X_0' \times_X X_1 \hookleftarrow X_0' \times_X X_1 \cong X_1 \times_X X_0' \hookrightarrow X_1 \times^h_X X_0' \xleftarrow{\sim} X_1 \times^h_X X_0
  \end{align*} 
  は弱同値である. 
  つまり, $X_0 \times^h_X X_1 \to X_1 \times^h_X X_0$は弱同値である. 
\end{remark}

ホモトピーファイバー積は単体的集合として定義されたが, 図式の持つ性質として定義した方が都合がよい. 

\subsection{ホモトピープルバック図式}

一般の単体的集合に対して, ホモトピープルバック図式を定義する. 
命題3.1.7.1の分解を特に明記せずに用いる. 

\begin{definition}[ホモトピープルバック図式]
  次の単体的集合の可換図式を考える. 
  % https://q.uiver.app/#q=WzAsNCxbMCwwLCJYX3swMX0iXSxbMSwwLCJYXzAiXSxbMSwxLCJYIl0sWzAsMSwiWF8xIl0sWzAsMV0sWzEsMl0sWzAsM10sWzMsMl1d
  \[\begin{tikzcd}
    {X_{01}} & {X_0} \\
    {X_1} & X
    \arrow[from=1-1, to=1-2]
    \arrow["q"', from=1-2, to=2-2]
    \arrow[from=1-1, to=2-1]
    \arrow[from=2-1, to=2-2]
  \end{tikzcd}\]
  $q$を弱同値$w : X_0 \to X_0'$とKanファイブレーション$q' : X_0' \to X$を用いて$q=q'w$と分解する. 
  誘導される射$X_{0,1} \to X_0' \times_X X_1$が弱ホモトピー同値のとき, 上の図式をホモトピープルバック図式(homotopy pullback square)という.
  % https://q.uiver.app/#q=WzAsNixbMCwwLCJYX3swMX0iXSxbMiwxLCJYXzAiXSxbMywyLCJYIl0sWzEsMiwiWF8xIl0sWzEsMSwiWF8wJyBcXHRpbWVzX1ggWF8xIl0sWzMsMSwiWF8wJyJdLFswLDEsIiIsMCx7ImN1cnZlIjotMn1dLFswLDMsIiIsMCx7ImN1cnZlIjoyfV0sWzMsMl0sWzEsMiwicSIsMl0sWzQsMV0sWzEsNSwidyJdLFs1LDIsInEnIl0sWzQsM10sWzAsNF0sWzQsMiwiIiwwLHsic3R5bGUiOnsibmFtZSI6ImNvcm5lciJ9fV1d
  \[\begin{tikzcd}
    {X_{01}} \\
    & {X_0' \times_X X_1} & {X_0} & {X_0'} \\
    & {X_1} && X
    \arrow[curve={height=-12pt}, from=1-1, to=2-3]
    \arrow[curve={height=12pt}, from=1-1, to=3-2]
    \arrow[from=3-2, to=3-4]
    \arrow["q"', from=2-3, to=3-4]
    \arrow[from=2-2, to=2-3]
    \arrow["w", from=2-3, to=2-4]
    \arrow["{q'}", from=2-4, to=3-4]
    \arrow[from=2-2, to=3-2]
    \arrow[from=1-1, to=2-2]
    \arrow["\lrcorner"{anchor=center, pos=0.125}, draw=none, from=2-2, to=3-4]
  \end{tikzcd}\]
\end{definition}

定義3.4.0.1の条件を確かめるためには, 1つの分解$q=q'w$についてのみ考えればよい. 

\begin{proposition}
  定義3.4.0.1において, 弱同値$w' : X_0 \to X_0'$を用いて, $q=q'w'$と分解されるとする. 
  このとき, 定義3.4.0.1の図式がホモトピープルバック図式であることと, 誘導される射$\rho' : X_{0,1} \to X_0' \times_X X_1$が弱ホモトピー同値であることは同値である. 
\end{proposition}

\begin{proof}
  $q$が弱同値$w'' : X_0 \to X_0''$とKanファイブレーション$q'' : X_0'' \to X$を用いて$q=q''w''$と分解されるとする. 
  $\rho'$が弱同値であることと, 誘導される射$\rho'' : X_{01} \to X_0'' \times_X X_1$が弱同値であることが同値であることを示せばよい. 
  命題3.1.7.1より, $w'$は緩射である. 
  注意3.1.2.7より, 次のリフト問題は解決$u : X_0' \to X_0''$を持つ. 
  % https://q.uiver.app/#q=WzAsNCxbMCwwLCJYXzAiXSxbMSwwLCJYXzAnJyJdLFsxLDEsIlgiXSxbMCwxLCJYXzAnIl0sWzAsMSwidycnIl0sWzEsMiwicScnIl0sWzAsMywidyciLDJdLFszLDIsInEnIiwyXSxbMywxLCJ1IiwyLHsic3R5bGUiOnsiYm9keSI6eyJuYW1lIjoiZGFzaGVkIn19fV1d
  \[\begin{tikzcd}
    {X_0} & {X_0''} \\
    {X_0'} & X
    \arrow["{w''}", from=1-1, to=1-2]
    \arrow["{q''}", from=1-2, to=2-2]
    \arrow["{w'}"', from=1-1, to=2-1]
    \arrow["{q'}"', from=2-1, to=2-2]
    \arrow["u"', dashed, from=2-1, to=1-2]
  \end{tikzcd}\]
  $w',w''$は弱同値なので, 2-out-of-3より, $u$も弱同値である. 
  次の図式を考えると, 命題3.4.0.2より, $X_0' \times_X X_1 \to X_0'' \times_X X_1$は弱ホモトピー同値である. 
  % https://q.uiver.app/#q=WzAsNixbMCwwLCJYXzAnIl0sWzEsMCwiWCJdLFsyLDAsIlhfMSJdLFsxLDEsIlgiXSxbMiwxLCJYXzEiXSxbMCwxLCJYXzAnJyJdLFswLDEsInEnIl0sWzIsMV0sWzEsMywiIiwwLHsibGV2ZWwiOjIsInN0eWxlIjp7ImhlYWQiOnsibmFtZSI6Im5vbmUifX19XSxbNCwzXSxbNSwzLCJxJyciLDJdLFsyLDQsIiIsMix7ImxldmVsIjoyLCJzdHlsZSI6eyJoZWFkIjp7Im5hbWUiOiJub25lIn19fV0sWzAsNSwiXFxzaW0iXV0=
  \[\begin{tikzcd}
    {X_0'} & X & {X_1} \\
    {X_0''} & X & {X_1}
    \arrow["{q'}", from=1-1, to=1-2]
    \arrow[from=1-3, to=1-2]
    \arrow[Rightarrow, no head, from=1-2, to=2-2]
    \arrow[from=2-3, to=2-2]
    \arrow["{q''}"', from=2-1, to=2-2]
    \arrow[Rightarrow, no head, from=1-3, to=2-3]
    \arrow["\sim", from=1-1, to=2-1]
  \end{tikzcd}\]
  2-out-of-3より, $\rho'$が弱同値であることと, $\rho''$が弱同値であることは同値である.
\end{proof}

\begin{example}
  次の単体的集合の可換図式を考える.
  % https://q.uiver.app/#q=WzAsNCxbMCwwLCJYX3swMX0iXSxbMSwwLCJYXzAiXSxbMSwxLCJYIl0sWzAsMSwiWF8xIl0sWzAsMV0sWzEsMl0sWzAsM10sWzMsMl1d
  \[\begin{tikzcd}
    {X_{01}} & {X_0} \\
    {X_1} & X
    \arrow[from=1-1, to=1-2]
    \arrow["q"', from=1-2, to=2-2]
    \arrow[from=1-1, to=2-1]
    \arrow[from=2-1, to=2-2]
  \end{tikzcd}\]
  $q$をKanファイブレーションとする. 
  命題3.4.1.2において, $q$の分解を$q=q\id_{X_0}$とおく. 
  この図式がホモトピープルバック図式であることと, 誘導される射$X_{01} \to X_0' \times X_1$が弱ホモトピー同値であることは同値である.
  よって, この図式がプルバック図式のとき, ホモトピープルバック図式である.  
  これは$q$がKanファイブレーションでないときには成立しない. 
\end{example}

\begin{example}
  次の単体的集合の可換図式を考える.
  % https://q.uiver.app/#q=WzAsNCxbMCwwLCJYJyJdLFsxLDAsIlgiXSxbMSwxLCJTIl0sWzAsMSwiUyciXSxbMCwxXSxbMSwyLCJxIl0sWzAsMywicSciXSxbMywyXV0=
  \[\begin{tikzcd}
    {X'} & X \\
    {S'} & S
    \arrow[from=1-1, to=1-2]
    \arrow["q", from=1-2, to=2-2]
    \arrow["{q'}", from=1-1, to=2-1]
    \arrow[from=2-1, to=2-2]
  \end{tikzcd}\]
  $q,q'$をKanファイブレーションとする. 
  命題3.3.7.1より, この図式がホモトピープルバック図式であることと, 任意の$S'$の点$s'$に対して$S$の点$s$が像として存在し, ファイバー積の間の誘導される射$X_{s'}' \to X_s$がKan複体のホモトピー同値であることは同値である. 
\end{example}

\begin{corollary}
  次の単体的集合の可換図式を考える.
  % https://q.uiver.app/#q=WzAsNCxbMCwwLCJYX3swMX0iXSxbMSwwLCJYXzAiXSxbMSwxLCJYIl0sWzAsMSwiWF8xIl0sWzAsMV0sWzEsMiwicSJdLFswLDMsInEnIl0sWzMsMl1d
  \[\begin{tikzcd}
    {X_{01}} & {X_0} \\
    {X_1} & X
    \arrow[from=1-1, to=1-2]
    \arrow["q", from=1-2, to=2-2]
    \arrow["{q'}", from=1-1, to=2-1]
    \arrow[from=2-1, to=2-2]
  \end{tikzcd}\]
  $q$を弱ホモトピー同値とする. 
  このとき, この図式がホモトピープルバック図式であることと, $q'$が弱ホモトピー同値であることは同値である. 
\end{corollary}

\begin{proof}
  $q$の分解として, $q = \id_X q$を考えればよい. 
  % https://q.uiver.app/#q=WzAsNixbMCwwLCJYX3swMX0iXSxbMiwxLCJYXzAiXSxbMywyLCJYIl0sWzEsMiwiWF8xIl0sWzEsMSwiWCBcXHRpbWVzX1ggWF8xIl0sWzMsMSwiWCJdLFswLDEsIiIsMCx7ImN1cnZlIjotMn1dLFswLDMsIiIsMCx7ImN1cnZlIjoyfV0sWzMsMl0sWzEsMiwicSIsMl0sWzQsMV0sWzEsNSwicSJdLFs1LDIsIlxcaWRfWCJdLFs0LDMsIiIsMCx7ImxldmVsIjoyLCJzdHlsZSI6eyJoZWFkIjp7Im5hbWUiOiJub25lIn19fV0sWzAsNF0sWzQsMiwiIiwwLHsic3R5bGUiOnsibmFtZSI6ImNvcm5lciJ9fV1d
  \[\begin{tikzcd}
    {X_{01}} \\
    & {X \times_X X_1} & {X_0} & X \\
    & {X_1} && X
    \arrow[curve={height=-12pt}, from=1-1, to=2-3]
    \arrow[curve={height=12pt}, from=1-1, to=3-2]
    \arrow[from=3-2, to=3-4]
    \arrow["q"', from=2-3, to=3-4]
    \arrow[from=2-2, to=2-3]
    \arrow["q", from=2-3, to=2-4]
    \arrow["{\id_X}", from=2-4, to=3-4]
    \arrow[Rightarrow, no head, from=2-2, to=3-2]
    \arrow[from=1-1, to=2-2]
    \arrow["\lrcorner"{anchor=center, pos=0.125}, draw=none, from=2-2, to=3-4]
  \end{tikzcd}\]
\end{proof}

\begin{corollary}
  次の単体的集合の可換図式を考える.
  % https://q.uiver.app/#q=WzAsNCxbMCwwLCJYX3swMX0iXSxbMSwwLCJYXzAiXSxbMSwxLCJYIl0sWzAsMSwiWF8xIl0sWzAsMV0sWzEsMiwicSJdLFswLDMsInEnIl0sWzMsMl1d
  \[\begin{tikzcd}
    {X_{01}} & {X_0} \\
    {X_1} & X
    \arrow[from=1-1, to=1-2]
    \arrow["q", from=1-2, to=2-2]
    \arrow["{q'}", from=1-1, to=2-1]
    \arrow[from=2-1, to=2-2]
  \end{tikzcd}\]
  $X$をKan複体とする. 
  この図式がホモトピープルバック図式であることと, 誘導される射 
  \begin{align*}
    \theta : X_{01} \to X_0 \times_X X_1 \hookrightarrow X_0 \times^h_X X_1
  \end{align*}
  が弱ホモトピー同値であることは同値である. 
\end{corollary}

\begin{proof}
  弱同値$w : X_0 \to X_0'$とKanファイブレーション$q' : X_0' \to X$を用いて$q=q'w$と分解する. 
  次の図式を考える. 
  % https://q.uiver.app/#q=WzAsNCxbMCwwLCJYX3swMX0iXSxbMCwxLCJYXzAnIFxcdGltZXNfWCBYXzEiXSxbMSwxLCJYXzAnIFxcdGltZXNeaF9YIFhfMSJdLFsxLDAsIlhfMCBcXHRpbWVzXmhfWCBYXzEiXSxbMCwxLCJcXHJobyIsMl0sWzEsMiwiXFxzaW0iLDJdLFswLDMsIlxcdGhldGEiXSxbMywyLCJcXHNpbSJdXQ==
  \[\begin{tikzcd}
    {X_{01}} & {X_0 \times^h_X X_1} \\
    {X_0' \times_X X_1} & {X_0' \times^h_X X_1}
    \arrow["\rho"', from=1-1, to=2-1]
    \arrow["\sim"', from=2-1, to=2-2]
    \arrow["\theta", from=1-1, to=1-2]
    \arrow["\sim", from=1-2, to=2-2]
  \end{tikzcd}\]
  命題3.4.0.7より, 下の水平な射は弱同値である. 
  命題3.4.0.9より, 右の垂直な射は弱同値である. 
  2-out-of-3より, $\theta$が弱同値であることと, $\rho$が弱同値であることは同値である. 
  命題3.4.1.2より, 求める図式がホモトピープルバック図式であることは同値である.
\end{proof}

\begin{remark}
  次の単体的集合の可換図式を考える.
  % https://q.uiver.app/#q=WzAsNCxbMCwwLCJYX3swMX0iXSxbMSwwLCJYXzAiXSxbMSwxLCJYIl0sWzAsMSwiWF8xIl0sWzAsMV0sWzEsMiwicSJdLFswLDMsInEnIl0sWzMsMl1d
  \[\begin{tikzcd}
    {X_{01}} & {X_0} \\
    {X_1} & X
    \arrow[from=1-1, to=1-2]
    \arrow[from=1-2, to=2-2]
    \arrow[from=1-1, to=2-1]
    \arrow[from=2-1, to=2-2]
  \end{tikzcd}\]
  がホモトピープルバック図式であることと, 次の図式
  % https://q.uiver.app/#q=WzAsNCxbMCwwLCJYX3swMX1eXFxteW9wIl0sWzEsMCwiWF8wXlxcbXlvcCJdLFsxLDEsIlheXFxteW9wIl0sWzAsMSwiWF8xXlxcbXlvcCJdLFswLDFdLFsxLDJdLFswLDNdLFszLDJdXQ==
  \[\begin{tikzcd}
    {X_{01}^\myop} & {X_0^\myop} \\
    {X_1^\myop} & {X^\myop}
    \arrow[from=1-1, to=1-2]
    \arrow[from=1-2, to=2-2]
    \arrow[from=1-1, to=2-1]
    \arrow[from=2-1, to=2-2]
  \end{tikzcd}\]
  がホモトピープルバック図式であることは同値である. 
\end{remark}

\begin{remark}
  $f_0 : X_0 \to X, f_1 : X_1 \to X$を単体的集合の射とする.
  任意の単体的集合に対して, ホモトピープルバック図式が存在するとは限らない. 
  例えば, $f_0 : \{0\} \hookrightarrow \Delta^1, f_1 : \{1\} \hookrightarrow \Delta^1$とする. 
  次の可換図式が存在するとき, 単体的集合$X_{01}$は空である. 
  % https://q.uiver.app/#q=WzAsNCxbMCwwLCJYX3swMX0iXSxbMSwwLCJcXHswXFx9Il0sWzEsMSwiXFxEZWx0YV4xIl0sWzAsMSwiXFx7MVxcfSJdLFswLDFdLFsxLDIsImZfMCIsMCx7InN0eWxlIjp7InRhaWwiOnsibmFtZSI6Imhvb2siLCJzaWRlIjoidG9wIn19fV0sWzAsM10sWzMsMiwiZl8xIiwyLHsic3R5bGUiOnsidGFpbCI6eyJuYW1lIjoiaG9vayIsInNpZGUiOiJ0b3AifX19XV0=
  \[\begin{tikzcd}
    {X_{01}} & {\{0\}} \\
    {\{1\}} & {\Delta^1}
    \arrow[from=1-1, to=1-2]
    \arrow["{f_0}", hook, from=1-2, to=2-2]
    \arrow[from=1-1, to=2-1]
    \arrow["{f_1}"', hook, from=2-1, to=2-2]
  \end{tikzcd}\]
\end{remark}

定義3.4.1.1は非対称的である. 
$f_0 : X_0 \to X$をKanファイブレーションに置き換えたとき, $f_1 : X_1 \to X$は何も変えていない. 
しかし, 次の命題から, これは関係ないことがわかる. 

\begin{proposition}[対称性]
  次の単体的集合の可換図式
  % https://q.uiver.app/#q=WzAsNCxbMCwwLCJYX3swMX0iXSxbMSwwLCJYXzAiXSxbMSwxLCJYIl0sWzAsMSwiWF8xIl0sWzAsMV0sWzEsMiwiZl8wIl0sWzAsM10sWzMsMiwiZl8xIiwyXV0=
  \[\begin{tikzcd}
    {X_{01}} & {X_0} \\
    {X_1} & X
    \arrow[from=1-1, to=1-2]
    \arrow["{f_0}", from=1-2, to=2-2]
    \arrow[from=1-1, to=2-1]
    \arrow["{f_1}"', from=2-1, to=2-2]
  \end{tikzcd}\]
  がホモトピープルバック図式であることと, 次の単体的集合の可換図式
  % https://q.uiver.app/#q=WzAsNCxbMCwwLCJYX3swMX0iXSxbMSwwLCJYXzEiXSxbMSwxLCJYIl0sWzAsMSwiWF8wIl0sWzAsMV0sWzEsMiwiZl8xIl0sWzAsM10sWzMsMiwiZl8wIiwyXV0=
  \[\begin{tikzcd}
    {X_{01}} & {X_1} \\
    {X_0} & X
    \arrow[from=1-1, to=1-2]
    \arrow["{f_1}", from=1-2, to=2-2]
    \arrow[from=1-1, to=2-1]
    \arrow["{f_0}"', from=2-1, to=2-2]
  \end{tikzcd}\]
  がホモトピープルバック図式であることは同値である. 
\end{proposition}

\begin{proof}
  弱同値$w_0,w_1$とKanファイブレーション$f_0',f_1'$を用いて$f_0=f_0'w_0, f_1=f_1'w_1$と分解されるとする. 
  \begin{align*}
    X_0 \xrightarrow{w_0} X_0' \xrightarrow{f_0'} X,~ X_1 \xrightarrow{w_1} X_1' \xrightarrow{f_1'} X
  \end{align*}
  次の可換図式が存在する.
  % https://q.uiver.app/#q=WzAsOSxbMCwwLCJYX3swMX0iXSxbMSwwLCJYXzAgXFx0aW1lc19YIFhfMSciXSxbMSwxLCJYXzAnIFxcdGltZXNfWCBYXzEnIl0sWzAsMSwiWF8wJyBcXHRpbWVzX1ggWF8xIl0sWzIsMCwiWF8wIl0sWzAsMiwiWF8xIl0sWzEsMiwiWF8xJyJdLFsyLDEsIlhfMCciXSxbMiwyLCJYIl0sWzAsMSwidSJdLFsxLDIsInYnIl0sWzAsMywidiJdLFszLDIsInUnIiwyXSxbMSw0XSxbMyw1XSxbNSw2LCJ3XzEiLDJdLFsyLDZdLFs0LDcsIndfMCJdLFsyLDddLFs3LDgsImZfMCciXSxbNiw4LCJmXzEnIiwyXV0=
  \[\begin{tikzcd}
    {X_{01}} & {X_0 \times_X X_1'} & {X_0} \\
    {X_0' \times_X X_1} & {X_0' \times_X X_1'} & {X_0'} \\
    {X_1} & {X_1'} & X
    \arrow["u", from=1-1, to=1-2]
    \arrow["{v'}", from=1-2, to=2-2]
    \arrow["v", from=1-1, to=2-1]
    \arrow["{u'}"', from=2-1, to=2-2]
    \arrow[from=1-2, to=1-3]
    \arrow[from=2-1, to=3-1]
    \arrow["{w_1}"', from=3-1, to=3-2]
    \arrow[from=2-2, to=3-2]
    \arrow["{w_0}", from=1-3, to=2-3]
    \arrow[from=2-2, to=2-3]
    \arrow["{f_0'}", from=2-3, to=3-3]
    \arrow["{f_1'}"', from=3-2, to=3-3]
  \end{tikzcd}\]
  系3.3.7.2より, $u'$と$v'$は弱同値である. 
  2-out-of-3より, $u$が弱同値であることと, $v$が弱同値であることは同値である.
  これは2つのホモトピープルバック図式の同値性を示している. 
\end{proof}

\begin{remark}
  
\end{remark}

\begin{proposition}[推移律]
  次の単体的集合の可換図式を考える.
  % https://q.uiver.app/#q=WzAsNixbMCwwLCJaIl0sWzEsMCwiWSJdLFsyLDAsIlgiXSxbMiwxLCJTIl0sWzEsMSwiVCJdLFswLDEsIlUiXSxbMCwxXSxbMSwyXSxbMiwzLCJmIl0sWzEsNCwiZyJdLFswLDUsImgiXSxbNSw0XSxbNCwzXV0=
  \[\begin{tikzcd}
    Z & Y & X \\
    U & T & S
    \arrow[from=1-1, to=1-2]
    \arrow[from=1-2, to=1-3]
    \arrow["f", from=1-3, to=2-3]
    \arrow["g", from=1-2, to=2-2]
    \arrow["h", from=1-1, to=2-1]
    \arrow[from=2-1, to=2-2]
    \arrow[from=2-2, to=2-3]
  \end{tikzcd}\]
  右の四角がホモトピープルバック図式であるとする. 
  このとき, 左の四角がホモトピープルバック図式であることと, 外の四角がホモトピープルバック図式であることは同値である.
\end{proposition}

\begin{proof}
  
\end{proof}

ホモトピープルバック図式は弱同値で保たれる. 

\begin{corollary}[ホモトピー不変性]
  次の単体的集合の可換図式を考える.
  % https://q.uiver.app/#q=WzAsOCxbMCwyLCJYXzEiXSxbMSwzLCJZIl0sWzIsMiwiWCJdLFszLDMsIlkiXSxbMSwxLCJZX3swMX0iXSxbMywxLCJZXzAiXSxbMCwwLCJYX3swMX0iXSxbMiwwLCJYXzAiXSxbMCwxLCJ3XzEiXSxbMCwyXSxbMiwzLCJ3Il0sWzMsMV0sWzQsMV0sWzQsNV0sWzUsM10sWzYsMF0sWzYsNCwid197MDF9Il0sWzYsN10sWzcsNSwid18wIl0sWzcsMl1d
  \[\begin{tikzcd}
    {X_{01}} && {X_0} \\
    & {Y_{01}} && {Y_0} \\
    {X_1} && X \\
    & Y && Y
    \arrow["{w_1}", from=3-1, to=4-2]
    \arrow[from=3-1, to=3-3]
    \arrow["w", from=3-3, to=4-4]
    \arrow[from=4-4, to=4-2]
    \arrow[from=2-2, to=4-2]
    \arrow[from=2-2, to=2-4]
    \arrow[from=2-4, to=4-4]
    \arrow[from=1-1, to=3-1]
    \arrow["{w_{01}}", from=1-1, to=2-2]
    \arrow[from=1-1, to=1-3]
    \arrow["{w_0}", from=1-3, to=2-4]
    \arrow[from=1-3, to=3-3]
  \end{tikzcd}\]
  $w_0,w_1,w$は弱同値とする. 
  次の3つのうち2つが成立するとき, 残りの1つも成立する. 
  \begin{enumerate}
    \item 奥の四角はホモトピープルバック図式である.
    \item 手前の四角はホモトピープルバック図式である.
    \item $w_{01} : X_{01} \to Y_{01}$は弱同値である. 
  \end{enumerate}
\end{corollary}

\begin{proof}
  
\end{proof}





\subsection{ホモトピープッシュアウト図式}

定義3.4.1.1の双対を考える. 








\newpage
\bibliographystyle{alpha}
\bibliography{../cf_kerodon}

\end{document}